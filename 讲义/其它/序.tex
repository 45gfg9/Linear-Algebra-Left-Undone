\chapter*{序}
\addcontentsline{toc}{chapter}{序}

\section*{一些初衷}
我为这本讲义起了一个大胆的标题,它来源于浙江大学竺可桢学院线性代数II(H)课程选用的
教材《线性代数应该这样学》(英文原版名:《Linear Algebra Done Right》)。我们带着半
娱乐性质地将最后两个单词像矩阵求逆一样(见封面设计)进行了颠倒,得到了本书的英文名:
《Linear Algebra Left Undone》。

接下来我们遇到了一个问题:中文名应该是什么呢?郑俊达同学提供了一个可行解:《线性代数:
未竟之美》。转念一想,这一标题不能更契合我们的编写初衷。事实上,我们认为现行的大部分
线性代数或高等代数教材具有如下问题,它们也困扰了笔者和许多读者的学习,我们也给出了
解决的方案:
\begin{enumerate}
    \item 从线性代数的角度来看,它们的讲解顺序不够自然,大部分教材都从行列式起步,
    缺乏引入地给出各种概念,使得读者无法理解线性代数的本质。可以说这些教材应当更名
    ``行列式与矩阵计算'',因为线性代数着重研究的线性空间和线性映射反而成为了边缘内容。
    因此我们采取了更好的讲解思路,更能体现线性代数的美感而非延续高中填鸭式的数学教育——
    事实上那根本称不上数学,那样的讲授思路根本不够``数学'',失去了数学本身的自然之美,
    而且使得读者误解数学、厌恶数学;
    \item 浙江大学竺可桢学院两学期线性代数课程选择的《大学数学:代数与几何》和
    《线性代数应该这样学》教材采用了从抽象空间引入的方式,更贴近本质。但实践过程中
    许多同学会对``为什么要一开始就学习这些抽象内容''缺乏概念,特别是《线性代数应该这样学》
    对于工科同学而言``数学味道太浓'',因此最后可能学习效果还不如填鸭式地灌输解题方法。
    因此我们在讲义中相当于为教材做了很多的注脚,并且优化了整体设计,提供了大量例题习题,
    都是为了能更自然地引入抽象内容,让读者知道我们为何要学习这些内容,这些内容当年在数学家
    眼中最自然的状态是什么,这样才能使得抽象的概念易于被初学者接受;
    \item 我们的例题和习题编排也是精心设计过的,不会出现大部分教材使用过程中``上课讲的
    、作业做的和考试考的脱节''的情况,这一问题不只是很多数学基础课教学的问题,也是国内
    各个专业都存在的教学问题,笔者也深受其害,所以编写例题习题特别注重对概念和定理的理解、
    对方法的掌握,不会出现教材中说什么知识很重要但没有例子体会很重要的这种抽象情况,并且
    大量的习题贴近所学知识也贴近考试,让读者通过习题更好地掌握知识而非反而迷惑不知道自己
    学了什么,才能更好地体会线性代数的美感而非感受到题海的压迫;
    \item 除了自然的美感外,更重要的是还有``未竟''的美感。我们添加了``史海拾遗''一讲,
    在此之前是线性方程组理论的``已竟''理论,而那之后的标准形理论开始,特别是到后续欧式空间
    的部分,我们可以引入到线性代数在更多数学分支乃至其他学科中的应用。之前在某校友圈中
    看见一个投票,说``线性代数是否应该包括抽象代数和泛函分析的内容'',事实上我们应当模糊
    一些边界,这些内容能让我们更好地理解线性代数本身,同时能让我们看到学习线性代数后能
    如何向着人类知识边界更进一步,这才是``未竟''的美感。我们相信我们自然的讲述能够让读者
    体会到数学的自然之美——我们已经尽力避开不自然的思路以及自然引入抽象的概念,在初识
    ``高等数学''的时候就能够建立起一套与高中不一样的更自然美丽的学习逻辑,从而让读者
    不畏惧数学,能够热爱数学,感受到数学家们探索真理的乐趣。
\end{enumerate}

古人有三不朽:立德、立功、立言,著书立说即为立言。虽说我完全不可能因为编写了一本基础课
的讲义而有如此崇高的地位。但在我心里,我已经通过这本讲义将我的热情、我的想法传达给了
不少的读者,这样也无愧于我在浙江大学的本科四年。未来或许这本讲义会淹没在历史的风尘中,
但我想只要它的某行文字曾经给予读者一丝丝的启发,或更实际地帮助了读者得到了心仪的分数,
我想它就是有价值的,我本人的价值也得到了一定的实现。

\section*{本书的面世}
自2021年秋参与浙江大学竺可桢学院学研部(现竺可桢学院学业指导中心)组织的朋辈辅学
活动以来,我即将第三次参与线性代数荣誉课的辅学活动。犹记讲义最初的简陋版本,那时为了
辅学的期中、期末准备的简单复习提要,里面因为本人时间有限甚至缺少了特征值与特征向量的
内容。那时的讲义基本都是知识点的罗列,缺少了许多重要的例题和证明,犹记第一次拿起这个
讲义站上讲台的时候,我深刻体会到了这一讲义的不足,因此那次的授课整体而言较为玄学,
比较干瘪。因此在2022年再次参加辅学授课时,我借着疫情放开考试延迟的机会分了六个大专题
写出了一本相对完整的适合于《大学数学:代数与几何》的复习讲义。里面的讲解比较全面,
习题也十分丰富,可以说在复习资料中已经能算过得去的一版了。

但我并不满足于此,我希望这本讲义能成为一本真正的完整的讲义,能兼具配套学习、考试复习
的功能,并且在保证体系严谨完整的前提下有更优化的讲解逻辑。因此在2023年的暑假,我
基于原先的复习版本进行扩展重排。在这一版本中,我将原先复习资料中的粗略描述都换为了
严谨的完整叙述,并且反复打磨讲解顺序,从而更自然地将另一本教材《线性代数应该这样学》
的内容自然融合,并且添加了大量的remark更适合于初学者学习。更重要的是,我们中间添加了
许多文字叙述,一方面自然引入我们要讲解的内容,这对于初学者而言是很重要的insight,
另一方面反复强调我们的行文逻辑,对推进逻辑做适当总结,使得读者能更快地形成体系,
同时也补充了很多拓展内容,一些是为了方便读者更自然地理解抽象内容,有一些是契合
``未竟之美''的标题,让读者能体会到数学的美感,体会到学习线性代数后我们知识的边界可以
推广到多远。

\section*{致谢}
我或许首先需要感谢2022年疫情放开之下的寒冬,没有这学期线性代数考试的延期,我也不会
有如此充裕的时间整理出本讲义较为完整的底稿,也就没有这一完整讲义的面世。

在编写的过程中,我需要特别感谢下面记为同学对我的讲义提供了直接的支持:梅敏炫同学主编了
讲义内积部分,郑俊达同学主编了解析几何部分,周健均同学编写了行列式计算进阶部分,朱熙哲
和谢集同学负责了讲义习题答案部分。特别感谢李英琦同学全权负责了本讲义的格式设计以及插图,
感谢王鹤翔同学设计了本讲义的封面。

我还需要感谢同级的王和钧同学,感谢他当年push我写出了最初版本的复习提要。我要感谢比我
低一级的郭苗苗同学,感谢她当年反复邀请我走上讲台实现梦想,虽然可能第一次授课效果一般,
但这对于后来我不断打磨授课方式,打磨讲义有非常重要的意义。我也应当感谢竺可桢学院学研部
(现竺可桢学院学业指导中心)给我提供了一个辅学的平台,让我通过讲义将我的热情能传达给
更多的读者。我还需要感谢支持我前面版本,对前面版本无论赞扬或是提出意见的读者们,正是有了
大家的支持才有了接下来越来越好的版本的面世。

我想我也应该特别感谢数学科学学院的吴志祥、谈之奕和刘康生老师,他们在我线性代数入门过程
中做了重要的引路人的工作,讲义中许多讲解思路也来源于他们精彩的授课。我也应该特别感谢
数学科学学院的王晓光老师,他在复变函数课程以及讲义上的热情以及倾注的心血启发我也应该
将我的学习思路和经验通过讲义传达给更多人,并且启发我思考如何从更高的观点、更自然的角度
引导读者学习新知识,享受追求真理的过程。

\section*{参考文献}
本讲义作为浙江大学竺可桢学院线性代数荣誉课的辅学讲义,因此核心思路来源于我们选择的教材
《大学数学:代数与几何(第二版)》(居余马,李海中)、《大学数学:代数与几何学习辅导》
(林翠琴,居余马)、《线性代数应该这样学(第三版)》([美]Sheldon\enspace Axler)。

在编写与修订的过程中,我也参考了其他非常多优质的教材或辅导资料,如《高等代数(第二版)》
(丘维声)、《高等代数:学习指导书》(丘维声)、《线性代数辅导讲义》(汤家凤)、
《高等代数强化讲义》(李扬)以及《高等代数考研:高频真题分类精解300例》等,并参考了部分
历年试题。在复数域的引入部分我也简单参考了王晓光老师的《复变函数讲义(2023版)》。

最后如果读者学完本讲义后对代数学有浓厚的兴趣,非常推荐读者学习后续的抽象代数课程。这里
推荐与我同级的刘泓健同学编写的\href{https://frightenedfoxcn.github.io/notes/series/alg-for-cs/}
{《写给计算机系学生的代数》}作进一步的了解,我们许多高级专题都对这一讲义有引用。

\begin{flushright}
    \kaishu
    吴一航 \\
    浙江大学计算机科学与技术学院 \\
    yhwu\_is@zju.edu.cn \\
    2023 年 8 月
\end{flushright}