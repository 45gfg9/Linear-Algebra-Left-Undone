\chapter*{致读者}
\addcontentsline{toc}{chapter}{致读者}

\section*{本书特色}
自有底稿成书的念头以来,笔者就十分希望本书能够摆脱市面上大部分线性代数或高等代数教材
固有的一些不够友好的编写风格,力图呈现一本能让读者眼前一亮的讲义。因此本讲义在编写过程
中笔者不断创新讲授思路,大胆摒弃传统的编排风格,总体而言本讲义有如下几大特色:
\begin{enumerate}
    \item 本讲义兼具教材、笔记、复习提纲等多种功能:
    \begin{enumerate}
        \item 说它像是教材,因为我们保留了完整的讲授体系,所有的思路都是反复打磨确认过的,
        保证了整体逻辑的完整和自然;
        \item 说它像是笔记,因为这其中我们特别注重一些细节性的内容,这些内容在教材或授课
        中可能会因为太过平凡被忽略,但在初学中是很重要的,例如我们对求解线性空间像与核、
        求解线性映射矩阵表示的很多讨论都是基于笔者在初学时出现的困惑增添了很多的细节,力求
        读者在初学阶段就能减少因为这些细节带来的困惑;
        \item 说它像是复习提纲,因为在编写过程中我们的很多内容都会分条列出,并且笔者特别
        注意了编写的逻辑连贯性,阅读起来思路比一般教材主线更清晰。除此之外每讲最后还有内容总结,
        并且经常会提供思维导图或是文字描述逻辑等便于读者快速掌握完整的思想体系。
    \end{enumerate}
    \item 本讲义提供了丰富的例题和习题,几乎能覆盖到所有重要的概念、定理和方法,同时
    我们也为这些题目提供了详细的解答,考虑了读者的阅读体验。我们的例题编排特别考虑了初学
    者在学习过程中可能遇到的困难,特别设置了很多适合于加深对概念、定理以及基本方法理解
    的例题。习题我们也是精心挑选,选择难度适中、有助于理解的经典题目,一些技巧性过强
    而脱离线性代数本质的题目我们也会删去或给读者一定的提示。因此我们的讲义特别重视教学和
    习题和考试的一致性,这对于初次学习而言也是非常重要的,也是很多教材没有精心编排而忽略的,
    事实上这会特别影响读者阅读体验;
    \item 在本讲义的编排过程中,我们摒弃了传统的讲授思路。首先我们选择《大学数学:代数与几何》
    以及《线性代数应该这样学》作为参考教材,它们都是从抽象空间出发研究的,相比于一般的线性代数
    或高等代数教材更能深入本质。但我们也考虑到过于抽象的引入对初学者十分不友好,所以我们不断地
    强调我们的讲授逻辑,重视自然地引入概念,自然地推进对概念的研究,最后引申至这些概念对于我们之后
    的研究的重要性。因此编排中我们不断优化内容编排顺序,也添加了足量的补充内容,目的就是使得
    读者能够更自然地接受而非填鸭式地囫囵吞枣,能够真正体会到数学的自然之美而非在抽象的描述或是
    繁杂的技巧中迷失了方向,我想这对于每一个数学学习者而言都是非常关键的。
\end{enumerate}

\section*{阅读建议}
我们为以下四类读者提供如下阅读建议:
\begin{enumerate}
    \item 初学线性代数过程中的读者:那么请坐稳扶好,备好配套的《大学数学:代数与几何》以及
    《线性代数应该这样学》。这本讲义是很好的学习笔记,其中我们有大量的remark帮助读者理解教材中
    可能觉得很平凡的内容,也有大量编排合理的例题和习题帮助读者巩固知识。初学过程中很推荐读者
    阅读我们反复强调的一些逻辑和一些补充的内容从而尽快形成学习体系,这对于数学学习是非常关键的
    ——把握了主线,剩下的就只有一些细节留待补充。当然一些较难的习题不一定在第一次就要掌握,因此
    可以根据自己的接受程度合理选择;
    \item 希望重新学习加深理解的读者:我想这本讲义是非常适合第二次更为深入学习理解的,当然第二次
    学习可以适当略过一些基础和细节内容,但本讲义中很多深入的讨论、独特的思路和有意义的联系一定
    对你第二次学习有所裨益;
    \item 在学习其他方面知识时回顾线性代数基础的读者:无论是基础已经遗忘很多或是还有一定印象的
    读者,本讲义都可以为您提供帮助,因为本书逻辑完整,并且从基础讲起,非常有助于简要回顾一些概念
    帮助后续研究学习其他内容,相比于一些教材填鸭式的讲解更适合于在简短的内容中迅速把握住重点;
    \item 复习考试的读者:如前述本书特色所说,本讲义有大量的通过分点列举总结的内容,每节最后
    也有比较完整的内容总结和逻辑梳理,我们也准备了足量的例题和习题供读者参考。但因为本书是从
    最基础的讲起,并且非常重视一些细节,因此复习时读者可以略过一些过于基础和细节的内容,也可以
    选择性参考讲义中给出的证明等,习题也可以优先选择难度适中的,因为考试中不会出现很难的题目。
\end{enumerate}

\section*{例题与习题}
笔者坚信,没有适量习题练习是很难在初学时较为清晰地掌握线性代数这些抽象的思路以及
运算技巧的,因此本讲义提供了足量的例题与习题便于读者及时巩固学习的概念,并掌握一些
常用的技巧,拓展一些实用的结论,同时一些例题和习题也是讲义完整逻辑中不可或缺的一环。

讲义中的例题有部分是直接概念性的,因为考虑到有一些概念初学时太过抽象,或者一些公式
较为复杂,需要及时联系以熟悉使用.还有一些例题是非常经典的问题,其中的思想在很多其它
习题中都会使用到。基本上在每个重要概念/定理/方法介绍后笔者都会准备合适的例子,并且
都会直接在题目后给出答案。

讲义的习题均设置了A、B、C三组,从低至高区分了难度,读者可以根据自己的实际需求选择
合适难度的习题进行巩固提升。所有的习题都在习题答案分册中提供了解答或思路(教材习题)
可能直接引用),因此读者在思考中遇到困难时可以参考其中的思路,但我们并不推荐直接
参考答案将所有习题粗略过一遍,这样的学习效果十分有限。

\section*{授课建议}
非常欢迎在本讲义的基础上节选或改编出适合辅学授课的讲义,但请注意以下几点:
\begin{enumerate}
    \item 请遵循\href{https://creativecommons.org/licenses/by-nc-sa/4.0/deed.zh}
    {CC BY-NC-SA 4.0}协议,即署名、非商业性使用、相同方式共享;
    \item 本讲义由于笔者本人风格以及目的所在,因此内容较为细致,在授课时您应当有选择性
    地节选内容在课堂上讲授,一些细节性的内容可以在课后让学生自行阅读,否则在有限时间内
    很难讲授较为完整的体系;
    \item 同理,在授课过程中您可以选取对您的授课思路有帮助的经典例题或习题进行讲授。
    授课中题在精不在多,您应当根据自己的授课风格和时间安排合理地选择题目,以有助于
    学生理解以及掌握基本方法为宗旨。
\end{enumerate}

\section*{最后的话}
我们十分清楚,现在阅读这段话的你可能从小到大都对数学缺乏兴趣,也可能在未来与数学之间
不会再有很多的交集。我想很多同学都是经过填鸭式的应试教育而来,如果并非生来热爱,那种
传统的教学方式只能是不断地毁灭式打击学生的数学学习兴趣。

在序言中笔者也提到,这本讲义希望还原数学本原的自然之美,因此从引入到推进到引申,特别是
``未竟之美''部分,我们都尽可能地从自然的角度出发,然后不断深入,让读者能够看到人类目前
研究的模糊边界,能够看到数学的无限魅力。或许从小我们便接受过教育,说数学或许不能帮你买菜,
但其中的``思维方式''才是最重要的。我想,通过本讲义由浅及深的自然推进,读者大概可以体会
当年无数数学家在探索数学本原时的或许``灵光一现''又或许``站在巨人肩膀''背后的思维方式。
更重要的是,这就是追求真理的过程,是一代代数学家用自己有限的生命逼近宇宙无穷,通向崇高
理念世界的过程。我想读者无论是在学习哪个专业,这都是非常重要的精神品质。

尽管我们在编写过程中尽可能地考虑到了读者的阅读体验,但我们也不可能做到面面俱到,
如果内容编排上有什么不合理的地方,或者有什么地方不够清晰,欢迎您将您的阅读体验通过邮件
或直接在本讲义所在的\href{https://github.com/yhwu-is/Linear-Algebra-Left-Undone}
{GitHub仓库}提交issue。如果您希望加入我们的编写团队,将这一讲义传承下去,也欢迎您通过
GitHub仓库提交pull request。

愿诸君热爱数学,热爱对真理的追求。

\begin{flushright}
    \kaishu
    吴一航 \\
    浙江大学计算机科学与技术学院 \\
    yhwu\_is@zju.edu.cn \\
    2023 年 8 月
\end{flushright}