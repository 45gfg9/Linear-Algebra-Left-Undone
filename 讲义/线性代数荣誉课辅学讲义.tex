\documentclass{ctexbook}

\usepackage{geometry}
\geometry{a4paper}
%\usepackage[UTF8, heading = false, scheme = plain]{ctex} % 格式
\usepackage{ctex}
\usepackage[utf8]{inputenc}
\usepackage{bm}
\usepackage{graphicx} % 添加图片
% \usepackage{amsthm}
\usepackage{amsmath}
\renewcommand{\vec}[1]{\boldsymbol{#1}} % 生产粗体向量,而不是带箭头的向量
\usepackage{amssymb}
\usepackage{booktabs} % excel 导出的大表格
\usepackage{rotating}
\usepackage{extarrows}
\usepackage{enumitem}
\usepackage{xcolor}

\usepackage{tikz}
\usepackage{pgfplots}
\usetikzlibrary{arrows, arrows.meta, calc, intersections, decorations.pathreplacing, patterns, decorations.markings}

\usepackage{indentfirst}
\setlength{\parindent}{2em}

\usepackage{ntheorem}
\theoremheaderfont{\bf\heiti}
\theorembodyfont{\fangsong}

\usepackage{makeidx} % 名词索引
\makeindex

\usepackage{xparse}
\NewDocumentCommand{\keyterm}{smo}{%
    {\sffamily\heiti\bfseries{#2}%
    \IfNoValueF{#3}{({#3})}}%
    \IfBooleanF{#1}{\index{#2}}%
}

\usepackage{zhnumber}

% chapter 标题修改为第 * 讲
\ctexset{
    chapter={format={\centering\Huge\bfseries},name={第,讲},number=\arabic{chapter}},
    section={format={\raggedright\Large\bfseries},name={,},number={\arabic{chapter}.\arabic{section}}},
    subsection={format={\raggedright\large\bfseries},name={,},number={\arabic{chapter}.\arabic{section}.\arabic{subsection}}},
}

\usepackage{mismath} % 包含 rank, span 等命令

% hyperref 与 cleveref 需要最后引入
\usepackage{hyperref}
\usepackage{cleveref}
\hypersetup{
    colorlinks,
    pdfborder={0 0 0},
    bookmarksnumbered,
}

\newtheorem{definition}{定义}[chapter] % 中文
\newtheorem{example}{例}[chapter]
\newtheorem{lemma}{引理}[chapter]
\newtheorem{theorem}{定理}[chapter]
\newtheorem{corollary}{推论}[chapter]
\newenvironment{proof}{{\noindent\bf\heiti 证明}\quad\fangsong}{\hfill$\square$\par}
\newenvironment{solution}{{\noindent\bf\heiti 解}\quad\fangsong}{\par}

\renewcommand{\figureautorefname}{图}
\renewcommand{\tableautorefname}{表}
\renewcommand{\equationautorefname}{式}
\renewcommand{\theoremautorefname}{定理}
\renewcommand{\sectionautorefname}{节}
\newcommand{\exampleautorefname}{例}
\newcommand{\definitionautorefname}{定义}
\crefrangeformat{equation}{式~#3#1#4--#5#2#6}
\crefrangeformat{example}{例~#3#1#4--#5#2#6}

\DeclareMathOperator{\diag}{diag}

\title{\heiti 浙江大学 2023-2024 学年 \\ 线性代数荣誉课辅学讲义}
\author{2023-2024 学年线性代数 I/II(H)辅学授课 \\ 吴一航\ \ yhwu\_is@zju.edu.cn}

\begin{document}
\frontmatter
\maketitle

\songti

% 插入空页
{\null
\thispagestyle{empty}
\newpage}
\setcounter{page}{1}

\pdfbookmark[0]{目录}{contents}
\tableofcontents

\addtolength{\parskip}{.5em}

\mainmatter
\setcounter{page}{1} % 将页码计数设置为 1
\chapter{预备知识}

线性代数作为大学的第一门数学课,预修要求并不高. 我们默认读者具有基本的高中数学知识,因此关于集合、映射以及向量的基本知识我们不在此赘述. 这一讲我们将从基本代数结构开始,以便后续线性空间的引入,然后我们将介绍本书中常见的概念——等价类和最常用的算法之一——高斯消元法.

\section{基本代数结构}

我们选择从基本代数结构谈起,因为在以往的实践中我们深切地体会到直接引入线性空间的跳跃. 因此我们希望从更具象的例子开始,首先引入``代数结构''这一基本概念,然后在下一节中自然地引出线性空间的定义.

我们首先考察一个简单的例子:实数集$\mathbf{R}$,它是一个集合. 在初中我们便知道,在$\mathbf{R}$上我们可以定义加法和乘法两种运算. 本质而言,运算是一种映射(或者更通俗而言,函数):

\begin{center}
    \begin{tabular}{rrcl}
        $+\enspace\colon$      & $\mathbf{R}\times\mathbf{R}$ & $\to$     & $\mathbf{R}$ \\
                               & $(a,b)$                      & $\mapsto$ & $a+b$        \\
        $\times\enspace\colon$ & $\mathbf{R}\times\mathbf{R}$ & $\to$     & $\mathbf{R}$ \\
                               & $(a,b)$                      & $\mapsto$ & $a\times b$
    \end{tabular}
\end{center}

上面的定义中出现了一个新的记号,即两个集合之间出现了乘号,这实际上是集合的笛卡尔积运算,定义如下:

\begin{definition}
    设$A$和$B$是两个非空集合,我们把集合
    \[A\times B=\{(a,b) \mid a\in A, b\in B\}\]
    称为集合$A$和$B$的\keyterm{笛卡尔积}[Cartesian product].
\end{definition}

因此我们很容易理解$\mathbf{R}\times\mathbf{R}$是一个集合,它的元素是形如$(a,b)$的有序对,其中$a,b\in\mathbf{R}$. 事实上,我们可以将$\mathbf{R}\times\mathbf{R}$看作平面上的点集,其中的点$(a,b)$对应于平面上的一个点,这一点的横坐标为$a$,纵坐标为$b$.

我们回到运算的映射表示,我们发现$+$和$\times$两个映射以两个实数作为函数的自变量,函数值也是一个实数. 或许读者看到这里还是对运算的定义有些许迷茫,但如果我们回忆映射的基本定义$f:A\to B$,$a\mapsto f(a)$,并将加法乘法写成$+(2,3)=5$,$\times(2,3)=6$,想必就会恍然大悟:$+$和$\times$实际上就是函数名,函数做的事情就是输入两个自变量然后进行加法/乘法运算得到函数值.

在上述讨论中,我们所做的事情很简单,就是给定一个集合,然后在这一集合的元素之间定义运算. 实际上这就是代数系统的定义:
\begin{definition}[{\keyterm{代数系统}[algebraic system]}]
    一般地,我们把一个非空集合$X$和在$X$上定义的若干代数运算$f_1,\ldots,f_k$组成的系统称为\keyterm*{代数系统}(简称代数系),记作$\langle X : f_1,\ldots,f_k\rangle$.
\end{definition}

特别注意的是,代数系统上定义的运算必须保证封闭性,也就是运算后的结果必须仍然在集合$X$中.

不难理解,代数系统其中蕴含的性质与其中定义的运算具有的性质是关联很大的. 我们仍然以实数域为例,介绍在代数学中关心的几个运算性质. 我们首先讨论实数域上的加法运算,以下性质对于任意$a,b,c\in\mathbf{R}$都成立:

\begin{enumerate}
    \item 结合律:$(a+b)+c=a+(b+c)$;

    \item 单位元:存在一个元素0,使得$a+0=0+a=a$;

    \item 逆元:对于任意$a$,存在一个元素$-a$,使得$a+(-a)=(-a)+a=0$(0为单位元);

    \item 交换律:$a+b=b+a$.
\end{enumerate}

对于乘法运算(可记为$\cdot$或$\times$),单位元一般记为1(更一般的可以记为$e$),逆元记为$a^{-1}$. 事实上,我们可以给出更多的例子:
\begin{example}\label{ex:1:Abel 群}
    \begin{enumerate}
        \item 代数系统$\langle \mathbf{R}\backslash\{0\}:\circ\rangle$定义的一般乘法运算

        \item 代数系统$\langle \mathbf{R}^2:+\rangle$定义的平面向量的加法
    \end{enumerate}
    均满足上述四条运算性质.
\end{example}

事实上,我们可以对上面的定义做进一步的抽象. 我们可以忽略集合中元素的差异(元素可以是实数,也可以是上述例子中的平面向量等),同时也可以忽略运算定义的差异,只关心运算作用于集合元素的性质. 对于一般的代数系统$\langle G:\circ\rangle$,我们有如下定义:
\begin{definition}[群] \label{def:1:群}
    若运算$\circ$满足结合律,则称代数系统$\langle G:\circ\rangle$为\keyterm{半群}[semigroup];若在半群基础上存在单位元,则称之为\keyterm{含幺半群}[monoid];若在含幺半群基础上每个元素存在逆元,则称之为\keyterm{群}[group];若在群的基础上运算还满足交换律,则称之为\keyterm{Abel群}[Abelian group],也称\keyterm{交换群}[commutative group].
\end{definition}

\autoref{def:1:群} 给出了我们本节第一个要讨论的代数结构——群的定义. 简而言之,代数结构就是在集合上定义具有某些特定性质的运算后得到的一类代数系统. 事实上,教材中42--44页给出了大量抽象的例子有助于同学们理解上述一系列群的定义,并且我们在后续学习矩阵的时候也会遇到一些群结构,相信这些实例能使读者体会到``在集合上定义运算''的方式的多样与抽象.

为方便书写,对于\autoref{def:1:群} 定义的群$\langle G:\circ\rangle$我们可以简写为群$G$. 除此之外,我们还需要指出以下两点:
\begin{theorem}\label{thm:1:群的单位元逆元唯一}
    \begin{enumerate}
        \item 群的单位元唯一;

        \item 群的每个元的逆元唯一.
    \end{enumerate}
\end{theorem}

\begin{proof}
    \begin{enumerate}
        \item 设$e_1$和$e_2$都是群$G$的单位元,则
              \[e_1=e_1\circ e_2=e_2.\]

        \item 设$b$和$c$都是$a$的逆元,则
              \[b=b\circ e=b\circ(a\circ c)=(b\circ a)\circ c=e\circ c=c.\]
    \end{enumerate}
\end{proof}

其中第一点的证明直接使用了单位元的性质,第二点的证明则使用了结合律和逆元的性质. 这里关于唯一性的证明是非常重要的:我们只需假设要证明唯一的东西有两个,然后说明这两个必然相等即可. 这一思想在之后证明矩阵的逆唯一等问题时也会用到,因此此处特别给出证明强调.

事实上,在很多集合上我们不仅可以定义一种运算,也可以定义两种甚至更多运算,在代数结构中我们仅讨论最多两种运算的情况. 事实上,我们最开始的实数集合定义加法和乘法的例子便可以引入一个新的代数结构——域:
\begin{definition}[{\keyterm{域}[field]}]
    我们称代数系统$\langle F:+,\circ\rangle$为一个\keyterm*{域},如果
    \begin{enumerate}
        \item $\langle F:+\rangle$是交换群,其单位元记作0;

        \item $\langle F\backslash\{0\}:\circ\rangle$是交换群;

        \item 运算$\circ$对$+$满足左、右分配律,即
              \begin{gather*}
                  a\circ(b+c)=a\circ b+a\circ c \\
                  (b+c)\circ a=b\circ a+c\circ a
              \end{gather*}
    \end{enumerate}
\end{definition}

显然,实数域$\mathbf{R}$上定义一般的实数加法和乘法后构成一个域. 实际上我们熟悉的例如有理数、实数等集合关于一般的加法和乘法运算都构成域,因此我们会经常使用``有理数域''、``实数域''等说法. 我们称数集对数的加法和乘法构成的域为数域,注意此处运算的定义必须是数学分析中定义的数的加法和乘法,不能是自定义的运算.
\begin{theorem}
    关于数域,我们有如下两个结论:
    \begin{enumerate}
        \item 数集$F$对数的加法和乘法构成数域的充要条件为:$F$包含0,1且对数的加、减、乘、除(除数不为0)运算封闭;

        \item 任何数域都包含有理数域$\mathbf{Q}$,即$\mathbf{Q}$是最小的数域.
    \end{enumerate}
\end{theorem}

上述定理的证明可见教材46页. 事实上,如果加法和乘法的定义不是数的加法和乘法,我们可以定义除了数域之外的域,我们将在本讲介绍完等价类的概念后给出这样的例子.

当然,还有一种代数结构对于$\circ$运算的要求有所降低,但也有广泛的应用,这就是环:
\begin{definition}[环]
    我们称代数系统$\langle R:+,\circ\rangle$为一个\keyterm{环}[ring],如果
    \begin{enumerate}
        \item $\langle R:+\rangle$是交换群,其单位元记作0;

        \item $\langle R:\circ\rangle$是半群;

        \item 运算$\circ$对$+$满足左、右分配律,即
              \begin{gather*}
                  a\circ(b+c)=a\circ b+a\circ c \\
                  (b+c)\circ a=b\circ a+c\circ a
              \end{gather*}
    \end{enumerate}

    若关于$\circ$存在单位元,则称之为\keyterm{含幺环}[ring with identity],若进一步每个非0($+$运算单位元)元素关于$\circ$都有逆元,则称之为\keyterm{除环}[division ring]. 另外,若上述定义中$\circ$运算满足交换律,则称为\keyterm{交换环}[commutative ring],结合上述除环和交换环两个定义,我们可以发现,交换除环即为域.
\end{definition}

\begin{example}
    利用定义验证下述关于代数系统的结论:
    \begin{enumerate}
        \item 整数集$\mathbf{Z}$对整数的加法和乘法构成一个交换环,但不是域;

        \item 设$C[a,b]$是闭区间$[a,b]$上的连续函数的集合;它对函数的加法和乘法构成一个环;

        \item 设$Q(\sqrt{2})=\{a+b\sqrt{2} \mid a,b\in\mathbf{Q}\}$,则$Q(\sqrt{2})$是一个数域.
    \end{enumerate}
\end{example}

我想大部分读者都会对抽象出代数结构的原因表示不解,如果这个问题无法解答,我想在下一章直接引入抽象的线性空间更会引发同学们对于``学了这个有什么用''的怀疑. 我们可以举一些不那么贴切但具象的例子来说明这其中的意义. 读者高中阶段想必大都经受过解析几何的摧残,大家在拿到题目时总会首先观察到题目属于``定点''、``定值''或是``极值''等问题,大家将自动与自己做题的经验或技巧匹配用于解答这几类问题. 同理,在研究一个特定的代数系统(例如定义了加法和乘法的实数域)的性质时,我们可以首先将其归类为群、环或是域等,然后我们只需要利用群环域各自的性质来研究这个代数系统的性质,而不需要再去研究这个代数系统的具体定义. 在这一过程中我们实现了问题的``归约'',即将一个复杂的问题转化为一个简单的更为抽象的问题,正如将解决上千道解析几何问题转化为研究几种题型的技巧. 这一``归约''的思想在将来的学习生活中我们将经常遇见,在实际中例如投资股票时我们可以将投资转化为提高投资组合的期望收益而尽力降低方差(风险)的求取极值的问题,在理论中,例如在计算理论的学习中我们会学习更为形式化的对问题的归约,这在算法复杂性研究中是基础的思想. 对于这类抽象问题感兴趣的同学不妨可以选择数学科学学院的抽象代数等课程,或是阅读本讲义的``后继''教程\href{https://frightenedfoxcn.github.io/notes/series/alg-for-cs/}{《写给计算机系学生的代数》}作进一步的了解. 事实上,对于理论感兴趣的同学,抽象代数将是必不可少的基础课程,它将是密码学、量子计算、计算理论以及编程语言理论等诸多领域的必要基础.

当然,这段描述因为涉及的知识容量较大,大概无法说服每一个读者. 但我们会在学习线性空间、线性映射的过程中不断重复这些思想,直到读者具备的知识容量足够时,一定能领会其中的奥妙.

\section{复数域的引入}

本书前半段讨论的框架是实数域、复数域都适用的,当然为了简化,我们的例子大都来源于实数. 从多项式一讲开始,我们便会开始强调实数域和复数域结论的不同,因此我们有必要在此引入复数域.

直观来看,实数位于数轴上,复数则分布在二维平面上,因此我们可以先考虑平面点集$\mathbf{R}^2$,并在其上定义加法和乘法运算使其成为一个域. 我们回顾高中学习的平面向量知识,我们记$\vec{e}_1=(1,0)$,$\vec{e}_2=(0,1)$,则$\mathbf{R}^2$上的任一向量$\vec{u}=(x,y)$可写为$x\vec{e}_1+y\vec{e}_2$. 此外,我们仍沿袭高中对向量长度的定义,即$\lvert\vec{u}\rvert=\sqrt{x^2+y^2}$.

在\autoref{ex:1:Abel 群} 中我们已经验证了$\mathbf{R}^2$上的向量加法满足Abel群的条件,因此我们只需要定义$\mathbf{R}^2$上的乘法使得代数系统$\langle\mathbf{R}^2\backslash\{(0,0)\}:\circ\rangle$也为Abel群. 这一乘法的构造需要满足一些自然的条件,同时也能实现构成Abel群的要求. 事实上,我们有如下定理:
\begin{theorem}\label{thm:1:复数乘法构造}
    平面点集$\mathbf{R}^2$上存在唯一的乘法$\circ$,满足
    \begin{enumerate}
        \item (单位元) $\vec{u}\circ\vec{e}_1=\vec{e}_1\circ\vec{u}=\vec{u},\enspace\forall\vec{u}\in\mathbf{R}^2$;

        \item (长度可乘性) $\lvert\vec{u}\circ\vec{v}\rvert=\lvert\vec{u}\rvert\lvert\vec{v}\rvert$.
    \end{enumerate}
    此乘法满足交换律,且使得$\langle\mathbf{R}^2:+,\circ\rangle$成为域.
\end{theorem}

上述定理中第一个条件是非常自然的,因为在二维平面上,$\{(x,0) \mid x\in\mathbf{R}\}$实际上就是实数轴,因此$\vec{e}_1=(1,0)$相当于实数1,因此作为乘法单位元是非常自然的. 第二条长度可乘则看起来没那么自然,但在接下来的证明中我们将会了解到其意义.

\begin{proof}
    对任意向量$\vec{u}=(a,b)=a\vec{e}_1+b\vec{e}_2,\enspace \vec{v}=(c,d)=c\vec{e}_1+d\vec{e}_2$,我们利用乘法的第一条性质有
    \[\vec{u}\circ\vec{v}=ac\vec{e}_1+(ad+bc)\vec{e}_2+bd\vec{e}_2\circ\vec{e}_2.\]
    由此可见$\vec{u}\circ\vec{v}=\vec{v}\circ\vec{u}$,因此乘法满足交换律. 同时可知,要定义乘法,关键是定义$\vec{e}_2\circ\vec{e}_2$的值.

    记$\vec{e}_2\circ\vec{e}_2=(x,y)$,由长度可乘性知$x^2+y^2=1$,另一方面
    \[(\vec{e}_1+\vec{e}_2)\circ(\vec{e}_1-\vec{e}_2)=\vec{e}_1-\vec{e}_2\circ\vec{e}_2=(1-x,y).\]
    由$|\vec{e}_1+\vec{e}_2|=|\vec{e}_1-\vec{e}_2|=\sqrt{2}$以及长度可乘性可得
    \[4=|(\vec{e}_1+\vec{e}_2)\circ(\vec{e}_1-\vec{e}_2)|^2=(1-x)^2+y^2.\]
    由此求出$x=-1,\enspace y=0$. 这说明
    \[\vec{e}_2\circ\vec{e}_2=-\vec{e}_1.\]
    由此得乘法的定义$\vec{u}\circ\vec{v}=(ac-bd)\vec{e}_1+(ad+bc)\vec{e}_2$,即
    \[(a,b)\circ(c,d)=(ac-bd,ad+bc).\]
    可验证,此乘法以$\vec{e}_1$为单位元,等式$(ac-bd)^2+(ad+bc)^2=(a^2+b^2)(c^2+d^2)$表明乘法满足长度可乘性. 上述证明亦表明乘法唯一(只能这么构造$\vec{e}_2\circ\vec{e}_2$).

    接下来我们很容易验证$\langle\mathbf{R}^2:+,\circ\rangle$满足域的定义,我们留作习题供读者自行验证.
\end{proof}

在\autoref{thm:1:复数乘法构造} 赋予的乘法下,$\langle\mathbf{R}^2:+,\circ\rangle$称为复数域$\mathbf{C}$. 我们自然地将$\vec{e}_1$合理简记为1,同时$\vec{e}_2$简记为$\i$,因为此时$(a,b)$即为$a+b\i$,并且利用$\vec{e}_2^2=-\vec{e}_1$可知$\i^2=-1$,这与我们熟知的虚数单位的定义是统一的. 这一代数表示引入的相关概念,如实部、虚部、纯虚数,以及复数四则运算法则在高中阶段大家都已熟知,在此不再赘述.

非零复数$z=x+y\i$也可写为极坐标的形式,即$z=|z|(\cos\theta+\i\sin\theta)$,其中$|z|=\sqrt{x^2+y^2}$为复数的平面表示的模长,$\theta\in\mathbf{R}$为连接原点与$z$的有向线段与$x$轴正方向的夹角(在相差$2\pi$整数倍的意义下唯一). 我们称$\theta$为复数$z$的辐角. 关于复数的模长我们有经典的三角不等式:
\begin{theorem}
    设$z,w\in\mathbf{C}$,则有$|z+w|\leqslant|z|+|w|$.
\end{theorem}

这一定理的几何意义是非常显然的,我们将$z$和$w$放在平面直角坐标系中观察就可以明白这就是经典三角不等式的复数版本. 等号成立的条件也显而易见,即$z$和$w$要么至少一个为0,要么都非零且$z$和$w$位于从原点出发的同一条射线上. 严格的证明如下:

\begin{proof}
    \begin{align*}
        |z+w|^2 & =(z+w)(\overline{z}+\overline{w})       \\
                & =|z|^2+|w|^2+2\Re(z\overline{w})        \\
                & \leqslant|z|^2+|w|^2+2|z||\overline{w}| \\
                & =(|z|+|w|)^2.
    \end{align*}
    等号成立当且仅当$z\overline{w}$为非负实数,与前述直观可得的的条件是等价的.
\end{proof}

证明中用到了一些应当熟知的结论,如$|z|^2=z\overline{z}$等,我们默认读者具有这些基础知识,因此不在此赘述.

\section{等价关系}

我们时常需要讨论集合中元素之间的关系. 例如直线间的平行、垂直、相交,或是数之间的大于、等于、小于关系.``关系''在我们的讲义中将会多次出现,因此我们很有必要在此形式化定义这一概念,并强调其中一类特定的关系——等价关系.

我们首先从(二元)关系这一概念入手. 实际上,这里的二元关系和日常生活中的关系是紧密相连的,例如将全人类作为谈论的背景集合,那么$(\text{小头爸爸}, \text{大头儿子})$这一有序二元组是符合这一关系的,但$(\text{章鱼哥}, \text{海绵宝宝})$显然不符合. 因此我们可以将父子关系看作笛卡尔积集合$\text{人类}\times\text{人类}$的子集. 更一般化的,集合$A$中的关系可以由$A\times A$的子集
\[\{(a,b) \mid a,b\in A, \enspace a\,R\,b\}\]
来刻画,其中$R$是这个关系本身(实质上是两个元素之间的某种性质),例如之前讨论的父子关系,或是数学中的大于、小于或同余等. 事实上,反过来,由$A\times A$的子集可以确定一个关系,例如我把全世界所有的父子组合放在这个集合中,那么这个集合就定义了人类中的父子关系.
\begin{example}
    以下是一些关系的例子:
    \begin{enumerate}
        \item 设$A=\mathbf{R}$,则$A\times A$的子集
              \[\{(a,b)\in A\times A \mid a^2+b^2=1\}\]
              定义了一个关系$R$,即
              \[a\,R\,b \iff a^2+b^2=1.\]

        \item 设$A=\{1,2,3\}$,则$A\times A$的子集
              \[\{(1,1),(1,2),(1,3),(2,2),(2,3),(3,3)\}\]
              定义了一个关系$R$,即
              \[a\,R\,b \iff a\leqslant b.\]

        \item 设$A$为任意数集,定义在$A$上的函数$f$也是一种关系,集合$A\times A$的子集
              \[B=\{(a,b)\in A\times A \mid b=f(a)\}\]
              刻画了这一关系. 换言之,函数是一种特殊的关系,它要求$\forall a\in A$有且仅有一个元素$b\in A$使得$(a,b)\in B$,其中$B$为上述定义的$A\times A$的子集.

        \item 设$A=\mathbf{Z}$,关系$R$满足$a\,R\,b\iff a\equiv b \pmod n$,即模$n$同余,则$A\times A$的子集
              \[\{(a,b)\in A\times A \mid a\equiv b \pmod n\}\]
              可以刻画这一关系.
    \end{enumerate}
\end{example}

接下来我们要讨论一种特别的关系,即等价关系. 它对关系$R$有一定的规定:
\begin{definition}\label{def:1:等价关系}
    集合$A$中关系若满足以下条件:
    \begin{itemize}
        \item (自反性) $\forall a\in A, \enspace a\,R\,a$;

        \item (对称性) 若$a\,R\,b$,则$b\,R\,a$;

        \item (传递性) 若$a\,R\,b$,$b\,R\,c$,则$a\,R\,c$,
    \end{itemize}
    则称$R$为$A$的一个等价关系. 进一步地,若$R$是集合$A$的一个等价关系且$a,b\in A$,若$a\,R\,b$,则称$a$,$b$关于$R$是等价的,并把$A$中所有与$a$等价的元素集合
    \[\overline{a}=\{b\in A \mid b\,R\,a\}\]
    称为$a$所在的等价类,$a$称为这个等价类的代表元素,并记$\{\overline{a}\}$为所有等价类为元素构成的集族.
\end{definition}

我们可能需要一个例子来理解这些概念. 我们不难证明,初等数论中的同余关系是一种等价关系,以模3同余为例,我们取整体集合为正整数集合,对于3,它的等价类就是所有和3模3同余的元素集合,即所有3的倍数. 同理,对于1,它所在的等价类就是模3余1的全体正整数,2所在的等价类是全体模3余2的正整数. 除此之外,我们还发现一个特点,即这三个等价类将原集合分成了三个无交集的子集
\begin{gather*}
    \overline{0}=\{3k\mid k\in\mathbf{Z}\} \\
    \overline{1}=\{3k+1\mid k\in\mathbf{Z}\} \\
    \overline{2}=\{3k+2\mid k\in\mathbf{Z}\}
\end{gather*}
且它们的并集就是原集合,即这三个等价类构成了原集合的一个\keyterm{分划}[partition](即分为并为原集合且互不相交的子集). 这一结论对所有等价类都成立,是很直观的结论:
\begin{theorem}\label{thm:1:等价类的性质}
    设$R$是集合$A$的等价关系,则由所有不同的等价类构成的子集族$\{\overline{a}\}$是$A$的分划. 反之,我们也可以基于分划在$A$中定义等价关系.
\end{theorem}

证明这一定理需要一个引理:
\begin{lemma}
    设$R$是集合$A$的等价关系,$a,b\in A$,则$\overline{a}=\overline{b}\iff a\,R\,b$.
\end{lemma}
这一引理说明$a$和$b$等价当且仅当它们等价类相同,或者说在同一个等价类中,相信根据等价类的定义这是很显然的结论.

这一引理还有一个重要的推论:
\begin{corollary}\label{cor:1:等价类的性质}
    设$R$是集合$A$的等价关系,$a,b\in A$,则下面二者必成立其一:
    \begin{enumerate}
        \item $\overline{a}\cap\overline{b}=\varnothing$;
        \item $\overline{a}=\overline{b}$.
    \end{enumerate}
\end{corollary}
即等价类要么相等要么不相交,这一结论也是非常自然的,且由这一结论我们很容易证明\autoref{thm:1:等价类的性质}. 如果对这些定理的证明细节感兴趣的读者可以参看教材第5页的定理1.1和1.2.

进一步此我们可以定义商集的概念:
\begin{definition}[{\keyterm{商集}[quotient set]}]
    设$R$是集合$A$的等价关系,以关于$R$的等价类为元素的集合(实际上是集合构成的集合,又称集族)$\{\overline{a}\}$称为$A$对$R$的\keyterm*{商集},记为$A/R$. 由
    \[\pi(a) = \overline{a}, \enspace \forall a\in A\]
    定义的$A$到$A/R$上的映射$\pi$称为$A$到$A/R$上的自然映射.
\end{definition}
我们可以看到,自然映射$\pi$将$A$中的元素$a$映到自己所在的等价类$\overline{a}$. 基于上述定义,我们可以完成在基本代数结构一节中遗留的一个问题:我们能否定义非数域的域?答案是肯定的,如果同学们对密码学感兴趣的应当听闻过有限域这一概念,接下来我们将通过简单的例子来说明这一概念.

\begin{example}\label{ex:1:有限域}
    设$Z_n$是$\mathbf{Z}$关于模$n$同余关系$R$的商集,即
    \[Z_n=\mathbf{Z}/R=\{\overline{0},\overline{1},\ldots,\overline{n-1}\}.\]
    即$Z_n$中的元素是$n$个集合,其中第$i$个集合是全体模$n$余$i-1\enspace(i=1,2,\ldots,n)$的整数构成的集合.

    在$Z_n$上定义加法$\oplus$为$\overline{a}\oplus\overline{b}=\overline{a+b}$. 这里$a$和$b$并不一定要在$0$到$n-1$之间,因为事实上$\overline{a}=\overline{kn+a}\enspace(k\in\mathbf{Z})$. 我们只需对$a$,$b$以及$a+b$对$n$取模就可以将它们控制在$0$到$n-1$之间且表示的是同一个运算表达式(因为本质上只是我们选取了同一个等价类的不同代表元素进行计算,例如$n=3$时,$\overline{1}+\overline{2}=\overline{4}+\overline{8}=\overline{0}$).

    接下来我们需要定义乘法$\circ$,同样是一个自然的定义,即$\overline{a}\circ\overline{b}=\overline{ab}$. 我们很容易验证$\forall n\in\mathbf{Z}$且$n\geqslant 2$,$\langle Z_n:\oplus,\circ\rangle$构成一个含幺交换环. 教材43页例8和45页例3中有详细的证明,因为较为显然此处从略. 我们要讨论的是何时$\langle Z_n:\oplus,\circ\rangle$构成域,由此我们便构造了一个非数域的域,并且元素个数是有限的.

    我们这里可以给出结论:$\langle Z_n:\oplus,\circ\rangle$是域当且仅当$n$是素数. 这一结论的证明需要一些数论的知识,我们放在习题中供感兴趣的同学证明.
\end{example}

\section{高斯消元法}

高斯消元法是线性代数中最常用的算法之一,是之后解决大量问题所需要掌握的基本方法,同时也是考试中一定会考察的内容,无论是单独一个大题考察,还是嵌入在其它问题中. 教材中相关概念和算法的介绍已经非常详细,这里只作总结.

注意考试中单独考察解方程时,时间充足时建议将过程写完整,标明初等行变换的具体步骤,并且至少写出阶梯矩阵和行简化阶梯矩阵. 除此之外,需要保证计算中尽量减少错误,时间充足可以解完方程后将答案代入进行检查.

需要强调的是,不要认为本节内容很简单就放过了,实际上如果长期不计算高斯消元法很容易陷入眼高手低的窘境,因此希望各位同学熟悉高斯消元法的基本步骤并熟练应用.

一般的,对于一个由$m$个方程组成的$n$元(即变量数为$n$)线性方程组
\[ \begin{cases} \begin{aligned}
            a_{11}x_1+a_{12}x_2+\cdots+a_{1n}x_n & = b_1           \\
            a_{21}x_1+a_{22}x_2+\cdots+a_{2n}x_n & = b_2           \\
                                                 & \vdotswithin{=} \\
            a_{m1}x_1+a_{m2}x_2+\cdots+a_{mn}x_n & = b_m
        \end{aligned} \end{cases} \]
将其系数排列成矩阵
\[\begin{pmatrix}
        a_{11} & a_{12} & \cdots & a_{1n} \\
        a_{21} & a_{22} & \cdots & a_{2n} \\
        \vdots & \vdots & \ddots & \vdots \\
        a_{m1} & a_{m2} & \cdots & a_{mn}
    \end{pmatrix}\]
且记$\vec{b}=(b_1,b_2,\ldots,b_m)^\mathrm{T}$,若$\vec{b}=\vec{0}$则称此方程为齐次线性方程组,否则为非齐次线性方程组. 再将$n$个未知量记为$n$元列向量$X=(x_1,x_2,\ldots,x_n)^\mathrm{T}$,我们便可以把方程组简记为$AX=\vec{b}$.

令$\vec{\beta}_i=(a_{1i},a_{2i},\ldots,a_{mi})^\mathrm{T}$,即方程组系数矩阵的某一列,则方程组还可以记为$x_1\vec{\beta}_1+x_2\vec{\beta}_2+\cdots+x_n\vec{\beta}_n=\vec{b}$,这一形式将在之后多次见到.

在以上的记号下,我们可以将解线性方程组的过程转化为矩阵的初等行变换. 高斯消元法的一般步骤如下:
\begin{center}
    线性方程组$\overset{1}{\longrightarrow}$增广矩阵$\overset{2}{\longrightarrow}$阶梯矩阵$\overset{3}{\longrightarrow}$(行)简化阶梯矩阵$\overset{4}{\longrightarrow}$解
\end{center}

\begin{enumerate}[label=步骤\arabic*~]
    \item 只需要将线性方程组转化为$(A, \vec{b})$的形式,得到左$n$列为系数矩阵,最右列为列向量$\vec{b}$的$n+1$列的增广矩阵;

    \item 通过初等行变换后,得到教材P34(1--13)的形式的矩阵——阶梯矩阵. 阶梯矩阵系数全零行在最下方,并且非零行中,在下方的行的第一个非零元素一定在上方行的右侧(每行第一个非零元素称主元素);

    \item 将主元素化1后将主元素所在列的其他元素均通过初等行变换化为0即可;

    \item \label{item:1:解方程组}
          我们分三种情况讨论:
          \begin{enumerate}
              \item 有唯一解:没有全零行,最后一个主元素的行号与系数矩阵的列数相等,且行简化阶梯矩阵对角线上全为1,其余元素均为0,此时可以直接写出解;

              \item 无解:出现矛盾方程,即系数为0的行的行末元素不为0,此时直接写无解即可;

              \item 有无穷解:非上述情况. 此时设出自由未知量将其令为$k_1,k_2,\ldots$,然后代入增广矩阵对应的方程组即可. 注意选取自由未知量时,选取没有主元素出现的列对应的未知量会与标准答案更贴近(如教材P33选取$x_2,x_5$),当然选择其他作为自由未知量也可以.
          \end{enumerate}
\end{enumerate}

从高斯消元法开始,我们正式进入线性代数的学习. 实际上,上述 \ref*{item:1:解方程组} 中关于方程组解的情况的讨论我们是浮于表面,是基于算法最后得到的矩阵的形式进行的讨论,但事实上,这背后蕴含着更深刻的意义. 我们将会在接下来的十余个章节中讲述线性代数中的核心概念,并在\hyperref[chap:朝花夕拾]{朝花夕拾}中回过头来重新审视线性方程组解的问题. 相信在那时,经历十余章各式抽象概念和运算技巧的洗礼后再来回味这一问题的你,定有``守得云开见月明''之感,对线性代数的理解也会更深一层.

\vspace{2ex}
\centerline{\heiti \Large 内容总结}

本讲为了后续章节讲述方便引入了一些基本概念和算法. 尽管这是一门面向理工科应用的数学课,但我们仍然希望以最自然的方式引入概念,而非填鸭式地轰炸,因此我们首先从大家最熟悉的实数集合开始,讨论在集合上定义运算的方法:我们逐步加强条件,引入了三种基本的代数结构——群、环和域,并且给出了一些例子,并简单讨论了定义代数系统的意义. 事实上,下一讲开始要介绍的线性空间也是一种特殊的代数结构,因此首先引入代数结构对于我们自然展开接下来的讨论有很大的帮助,不至于让读者觉得非常突兀.

接下来我们也从域的定义入手,构造了$\mathbf{R}^2$上的乘法运算使其构成了一个域,并且我们发现这里的定义与高中学习的复数乘法是完全一致的. 之后我们引入了等价关系的概念,这一概念在后续的讲义中将会多次出现,其重要意义就是将一个集合划分成了几个等价的区域. 最后我们讨论了高斯消元法的一般步骤,这是我们接下来解决线性空间中各类问题绕不开的算法.

\vspace{2ex}
\centerline{\heiti \Large 习题}

\vspace{2ex}
{\kaishu 我这门课很简单,只有简单的加减乘除四则运算,甚至除法都不太需要.}
\begin{flushright}
    \kaishu
    ——浙江大学数学科学学院教授吴志祥
\end{flushright}

\centerline{\heiti A组}
\begin{enumerate}
    \item 完善\autoref{thm:1:复数乘法构造} 中的证明,即证明$\mathbf{R}^2$在平面向量加法和如\autoref*{thm:1:复数乘法构造} 定义的乘法下构成一个域.

    \item 完成教材48页第13题.

    \item 求齐次线性方程组$\begin{cases}
                  x_1+x_2+x_3+4x_4-3x_5=0   \\
                  2x_1+x_2+3x_3+5x_4-5x_5=0 \\
                  x_1-x_2+3x_3-2x_4-x_5=0   \\
                  3x_1+x_2+5x_3+6x_4-7x_5=0
              \end{cases}$的通解.

    \item 求非齐次线性方程组$\begin{cases}
                  x_1-x_2+2x_3-2x_4+3x_5=1     \\
                  2x_1-x_2+5x_3-9x_4+8x_5=-1   \\
                  3x_1-2x_2+7x_3-11x_4+11x_5=0 \\
                  x_1-x_2+-x_3-x_4+3x_5=3
              \end{cases}$的通解.

    \item 求解线性方程组$\begin{cases}
                  x_1+x_2+x_3=1   \\
                  x_1+2x_2-5x_3=2 \\
                  2x_1+3x_2-4x_3=5
              \end{cases}$.
\end{enumerate}

\centerline{\heiti B组}
\begin{enumerate}
    \item 设$A$是一个Abel群,$A$的运算是加法. 在$A$中定义乘法运算为$ab=0,\enspace\forall a,b\in A$. 证明:$A$为一个环(我们称这种环为\keyterm*{零环}[zero ring]).

    \item 证明:若集合$A$上的二元关系$R$满足
          \begin{enumerate}
              \item $a\,R\,a,\enspace\forall a\in A$;

              \item $\forall a,b,c\in A$,若$a\,R\,b$且$a\,R\,c$,则$b\,R\,c$.
          \end{enumerate}
          则$R$为$A$上的等价关系.
\end{enumerate}

\centerline{\heiti C组}
\begin{enumerate}
    \item 证明:\autoref{ex:1:有限域} 中定义的$\langle Z_n:\oplus,\circ\rangle$是域当且仅当$n$是素数.
          (提示:无论$n$是否为素数,$n\in\mathbf{Z}$且$n\geqslant 2$时$\langle Z_n:\oplus,\circ\rangle$为含幺交换环,因此是否为素数将决定这一结构中每个元素是否有逆元. 在初等数论中,我们熟知的裴蜀定理可以解决这一问题.)

    \item 本讲我们构造了$\mathbf{R}^2$上的乘法,从而定义了复数域的乘法运算. 本题希望探讨的是:$\mathbf{R}^3$无法构造出乘法使其成为一个域. 在高中的学习中我们知道,$\mathbf{R}^3$空间向量的一组基底为$\{\vec{e}_1=(1,0,0),\vec{e}_2=(0,1,0),\vec{e}_3=(0,0,1)\}$. 证明:$\mathbf{R}^3$没有乘法同时满足以下性质:
          \begin{enumerate}
              \item (单位元) $\forall \vec{u}\in\mathbf{R}^3,\enspace\vec{e}_1\cdot \vec{u}=\vec{u}\cdot \vec{e}_1$;

              \item (交换性) $\forall \vec{u},\vec{v}\in\mathbf{R}^3,\enspace\vec{u}\cdot \vec{v}=\vec{v}\cdot \vec{u}$;

              \item (长度可乘性) $\forall \vec{u},\vec{v}\in\mathbf{R}^3,\enspace|\vec{u}\cdot\vec{v}|=|\vec{u}||\vec{v}|$.
          \end{enumerate}
          按照如下思路给出详细证明过程:采用反证法. 假设乘法存在,则
          \begin{enumerate}
              \item 通过计算$(\vec{e}_1+\vec{e}_2)\cdot(\vec{e}_1-\vec{e}_2)$,$(\vec{e}_1+\vec{e}_3)\cdot(\vec{e}_1-\vec{e}_3)$,证明\[\vec{e}_2\cdot\vec{e}_2=\vec{e}_3\cdot\vec{e}_3=-\vec{e}_1.\]

              \item 证明$(\vec{e}_2+\vec{e}_3)\cdot(\vec{e}_2-\vec{e}_3)=0$得出矛盾.
          \end{enumerate}
\end{enumerate}

\chapter{线性空间}

本讲我们将开始回答第 1 讲最后留下的问题,即线性方程组有唯一解、无穷解或无解的本质原因. 这段旅程或许有些漫长,中间会有很多的铺垫,我们将从其中最为基础的概念——线性空间出发进行探讨.

回忆高斯消元法,方程组中每一行或一列都可以视为向量. 我们可以先看下面这个例子:
\begin{example}\label{ex:2:线性空间引入}
    考虑如下两个方程组
    \begin{multicols}{2}
        \item $\begin{cases}
                x_1+x_2+x_3=0   \\
                x_1+2x_2+3x_3=0 \\
                2x_1+3x_2+4x_3=0
            \end{cases}$

        \item $\begin{cases}
                x_1+x_2+x_3=0   \\
                x_1+2x_2+3x_3=0 \\
                x_1+3x_2+4x_3=0
            \end{cases}$
    \end{multicols}
    不难解得,第一个方程组有无穷解,第二个方程组有唯一解. 从高斯消元法的过程来看,第一个方程组的简化阶梯矩阵出现了全零行,其原因是显而易见的:因为方程组第一行和第二行相加正好是第三行,因此可以直接消去第三行,即三行的系数矩阵的三个行向量
    \[\alpha_1=(1,1,1),\enspace\alpha_2=(1,2,3),\enspace\alpha_3=(2,3,4)\]
    满足$\alpha_1+\alpha_2=\alpha_3$. 而第二个方程组系数矩阵行向量间没有类似的可互相消去的关系.
\end{example}

从上面这一例子中我们可以看出,方程组的解与系数矩阵的行向量之间的关系密切相关. 因此我们会有一个很自然的想法,即我们需要研究向量之间的关联. 受第1讲基本代数结构的启发,我们应当自然地想到我们需要引入一个代数结构,从而使得我们可以统一地研究向量间的关联,这一代数结构便是线性空间.

\section{线性空间的定义}

\term{线性空间}\index{xianxingkongjian@线性空间 (linear space)}是我们接触的第一个核心概念,作为一种代数结构,它需要在非空集合$V$上定义运算. 其中第一个运算是我们熟知的加法$+$. 在线性空间的定义中,我们要求$\langle V:+\rangle$构成Abel群,即其中元素满足如下运算律:
\begin{enumerate}
    \item (结合律) $\alpha+(\beta+\gamma)=(\alpha+\beta)+\gamma,\enspace\forall \alpha,\beta,\gamma \in V$;

    \item (加法单位元) $\exists 0 \in V$使得$\forall\alpha\in V$ 有 $\alpha+0=0+\alpha$;

    \item (逆元) $\forall\alpha\in V,\enspace \exists \beta \in V$,有$\alpha+\beta=\beta+\alpha=0$,记$\beta=-\alpha$;

    \item (交换律) $\forall\alpha, \beta\in V,\enspace \alpha+\beta=\beta+\alpha$.
\end{enumerate}

第二种运算和之前学习的其他代数结构不同,我们需要首先引入一个数域$\mathbf{F}$,接下来在$\mathbf{F}\times V$上定义取值于$V$的数乘运算,即$\mathbf{F}\times V$中的每个元素$(\lambda,\alpha)\mapsto \lambda\alpha\in V$,并且数乘运算满足以下性质:$\forall \alpha,\beta \in V,\enspace\forall \lambda,\mu\in\mathbf{F}$以及$\mathbf{F}$上的乘法单位元1,有
\begin{enumerate}
    \item $1\cdot \alpha=\alpha$;

    \item $\lambda(\mu\alpha)=(\lambda\mu)\alpha$;

    \item $(\lambda+\mu)\alpha=\lambda\alpha+\mu\alpha$;

    \item $\lambda(\alpha+\beta)=\lambda\alpha+\lambda\beta$.
\end{enumerate}

基于此,我们完整定义了一个线性空间,我们一般称集合$V$关于上述两种运算在域$\mathbf{F}$上构成一个线性空间,简称为$V$在域$\mathbf{F}$上的线性空间,记作$V(\mathbf{F})$. 如果$\mathbf{F}$是实(复)数域,则称$V$为实(复)数域上的线性空间. 关于线性空间的定义,我们还有如下说明:
\begin{enumerate}
    \item 线性空间还有一个重要的概念是运算封闭,即线性空间中的元素进行加法或数乘运算后,得到的元素仍然是属于线性空间的. 这一点是定义要求的,加法封闭是 Abel 群的要求,因为 Abel 群要求加法运算定义为映射 $V\times V\to V$,因此$V$中两个元素相加后必须仍在$V$中(事实上这是代数系统的共性),数乘注意前述定义中数乘运算``取值于$V$''的要求,即它是 $F\times V\to V$ 的映射;

    \item 特别注意线性空间定义在非空集合上,事实上根据加法构成Abel群的要求,最小的线性空间也必须至少包含加法单位元(可以记为$V=\{0\}$).

    \item 结合我们上一讲对公理化的研究,事实上我们到目前为止也只定义了上面的加法、数乘运算和几条规则,我们需要忘记其他任何规则,由此出发进行推导出一些看似显然但公理没有直接给出的重要运算性质:
          \begin{enumerate}
              \item 由于加法运算构成Abel群,因此加法零元和逆元是唯一的,并且我们可以定义减法运算为加上一个元素的逆,即$\alpha-\beta=\alpha+(-\beta)$;

              \item 事实上,根据公理中的性质,我们可以逐步得到$\lambda(\alpha-\beta)+\lambda\beta=\lambda((\alpha-\beta)+\beta)=\lambda((\alpha+(-\beta))+\beta)=\lambda(\alpha+((-\beta)+\beta))=\lambda(\alpha+\vec{0})=\lambda\alpha$,两边分别加$-(\lambda\beta)$即可以得到
                    \begin{equation}\label{eq:2:线性空间运算性质1}
                        \lambda(\alpha-\beta)=\lambda\alpha-\lambda\beta.
                    \end{equation}
                    上面推导过程中第一个等号来源于数乘分配律,第二个等号来源于减法的定义(加上逆元),第三个等号来源于加法结合律,第四个等号来源于逆元的定义(加起来等于向量加法零元$\vec{0}$),最后一个等号来源于加法单位元的定义. 事实上这一过程是非常清晰的. 需要注意的一点是,接下来为了区分$V$中的零元和数域中的数0,我们将$V$中零元加粗,请读者务必仔细区分.

                    除此之外,$(\lambda-\mu)\alpha+\mu\alpha=(\lambda-\mu+\mu)\alpha=\lambda\alpha$,两边分别加$-(\mu\alpha)$即可以得到
                    \begin{equation}\label{eq:2:线性空间运算性质2}
                        (\lambda-\mu)\alpha=\lambda\alpha-\mu\alpha.
                    \end{equation}
                    事实上,\autoref{eq:2:线性空间运算性质1} 和\autoref{eq:2:线性空间运算性质2} 可以视为数乘运算对减法也满足分配律(但我们必须时刻牢记在心,数的减法是常规的,向量的减法是加上向量的逆元).

              \item 在\autoref{eq:2:线性空间运算性质1} 中分别令$\alpha=\beta$和$\alpha=\vec{0}$,在\autoref{eq:2:线性空间运算性质2} 分别令$\lambda=\mu$和$\lambda=0$有如下四条性质:
                    \begin{enumerate}
                        \item $\lambda\cdot \vec{0}=\vec{0}$;

                        \item $\lambda(-\beta)=-(\lambda\beta)$;

                        \item $0\cdot \alpha=\vec{0}$;

                        \item $(-\mu)\alpha=-(\mu\alpha)$.
                    \end{enumerate}
                    我们详细证明前两条如何根据公理一步步推导得到,后两条请读者依照此自行证明.
                    \begin{proof}
                        \begin{enumerate}
                            \item 在\autoref{eq:2:线性空间运算性质1} 中令$\alpha=\beta$,则$\lambda(\alpha-\alpha)=\lambda\alpha-\lambda\alpha$,根据减法定义有$\alpha-\alpha=\alpha+(-\alpha)=\vec{0}$,且$\lambda\alpha-\lambda\alpha=\lambda\alpha+(-(\lambda\alpha))=\vec{0}$,因此$\lambda\cdot \vec{0}=\vec{0}$.

                            \item 在\autoref{eq:2:线性空间运算性质1} 中令$\alpha=\vec{0}$有$\lambda(\vec{0}-\beta)=\lambda\vec{0}-\lambda\beta$,根据减法定义有$\vec{0}-\beta=\vec{0}+(-\beta)=-\beta$(第二个等号来源于加法单位元性质),且$\lambda\vec{0}-\lambda\beta=\vec{0}-\lambda\beta=\vec{0}+(-(\lambda\beta))=-(\lambda\beta)$(第一个等号来源于刚刚证明的$\lambda\cdot \vec{0}=\vec{0}$,第二个等号来源于减法的定义,第三个等号来源于加法单位元性质),因此$\lambda(-\beta)=-(\lambda\beta)$.
                        \end{enumerate}
                    \end{proof}
                    特别地,当$\mu=1$时有$(-1)\alpha=-\alpha$. 即$-1$数乘一个元素可以得到该元素的逆元(虽然代入一般平面向量这一点非常显然,但是我们只能基于公理一步步推导得到这一显然的性质).

              \item 若$\lambda\alpha=\vec{0}$,则$\lambda=0$或$\alpha=\vec{0}$,这一点也是显然的,因为如果$\lambda\neq 0$,则$\lambda^{-1}$存在,从而$\alpha=1\alpha=(\lambda^{-1}\lambda)\alpha=\lambda^{-1}(\lambda\alpha)=\lambda^{-1}\vec{0}=\vec{0}$(这里的每一个等号都是能找到对应的,请读者自行判断).

                    最后,综合上述性质我们有方程$\lambda\beta+\lambda_1\alpha_1+\lambda_2\alpha_2+\cdots+\lambda_r\alpha_r=\vec{0}$在$\lambda\neq 0$时的解为$\beta=-\lambda^{-1}\lambda_1\alpha_1-\lambda^{-1}\lambda_2\alpha_2-\cdots-\lambda^{-1}\lambda_r\alpha_r$. 我们放在习题中供读者练习.
          \end{enumerate}
\end{enumerate}

或许同学们会疑惑为什么线性空间会要求上述这8条性质(加法、数乘各4条). 事实上,这里的加法交换律是可以被其他7条推出的,感兴趣的同学可以自行尝试证明. 其余的7条公理彼此独立,每一条均不可取消. 感兴趣的同学可以试试举出反例来说明其余7条中每一条均不能由其余各条推出.

我们发现线性空间中定义的运算规则与我们高中学习的平面向量的加法和数乘是非常类似的,我们回顾未竟专题一关于公理化的讨论,实际上这就可以视为从简单的向量加法和数乘抽象出来的一些规则. 而公理的诞生应当是要尽可能简洁,而且有足够的表达力——这一点我们将来基于这一定义不断推出线性空间的性质时就会发现非常足够(事实上你现在就能通过我们上面证明的运算性质初步感知到这一点,因为7条公理中任何一条的缺失都会使得上面某条显然而合理的性质不再满足,而我们未来需要的性质都可以由此导出),因此皮亚诺在1888年正式给出这一定义并沿用至今. 但我们需要知道他的工作也是基于前人(如格拉斯曼)的工作不断修正而来的,只是我们被动接受这一概念使得这一自然的过程变得很突兀. 当然这门课只要求你记忆这8条性质,并请务必牢记于心,考试可能要求你验证线性空间. 记忆难度也并不大,Abel 群4条性质都有名称标注,数乘运算也是易于记忆的结合律和分配律加单位元性质.

除此之外,公理化定义还有一个很重要的作用就是使得我们可以不仅仅在向量集合的背景下定义线性空间,这使得我们可以将对于很多结构的研究都转化为对于线性空间的研究. 接下来我们给出一些与向量无关的线性空间的例子:

\begin{example}
    几种非常常见的线性空间,希望读者能熟知其性质:
    \begin{enumerate}
        \item (多项式)$\mathbf{F}[x]_{n+1}=\{a_0+a_1x+\cdots+a_nx^n \mid a_i\in\mathbf{F}\}$关于多项式的加法和数乘构成线性空间,但
              \[\mathbf{F}[x]'_{n+1}=\{a_0+a_1x+\cdots+a_nx^n \mid a_i\in\mathbf{F}, a_n\neq 0\}\]
              不构成线性空间.

              注:书上常将多项式记为$\mathbf{F}[x]_{n+1}$,表示次数不超过$n$的多项式的集合,而《线性代数应该这样学》中使用 $\mathcal{P}_n(\mathbf{F})$ 表示相同的集合.

              注意常见记号:$(k_1p_1+k_2p_2)(x)=k_1p_1(x)+k_2p_2(x)$.

        \item (复数与实数)可以验证:复数集$\mathbf{C}$是数域$\mathbf{C}$或数域$\mathbf{R}$上的线性空间. 此处一定注意复数集$\mathbf{C}$在此处同时出现在集合和数域中.

              注意:这一例子表明,同一集合可以在不同数域上构成不同的线性空间,在下一讲接触维数的定义后,我们也将知道二者的维数是不一样的(见\autoref{ex:3:不同数域的维数}).

              当然,不同的集合也可以在同一个数域上构成不同的线性空间,例如$\mathbf{C(R)}$和$\mathbf{R(R)}$.

        \item 对$n$维实非零系数向量空间$V$定义如下加法运算
              \begin{gather*}
                  \alpha = (a_1, a_2, \ldots, a_n), \beta = (b_1, b_2, \ldots, b_n) \in V = \mathbf{R}_{+}^n, \\
                  \alpha \oplus \beta = (a_1b_1, a_2b_2, \ldots, a_nb_n).
              \end{gather*}
              定义如下数乘运算
              \[\forall \lambda \in \mathbf{R}, \lambda \circ \alpha = (a_1^\lambda, a_2^\lambda, \ldots, a_n^\lambda).\]
              则$V$构成线性空间.

        \item $V=\{f \mid x \in \mathbf{R}, f(x) \in \mathbf{C}$(即$f$是实变量复值函数),且$f(-x)=\overline{f(x)}$(后者为$f(x)$的共轭复数)$\}$,定义如下的加法和数乘运算:
              \begin{gather*}
                  (f \oplus g)(x) = f(x) + g(x) \\
                  (\lambda \circ f)(x) = \lambda f(x).
              \end{gather*}
              则$V$构成线性空间.
    \end{enumerate}
\end{example}

\begin{solution}
    \begin{enumerate}
        \item 我们对八条性质进行逐条验证即可.
              \begin{enumerate}
                  \item $\forall p_1(x), p_2(x), p_3(x) \in \mathbf{F}[x]_{n+1}=\{a_0+a_1x+\cdots+a_nx^n \mid a_i\in\mathbf{F}\}$,有
                        \begin{align*}
                                & (p_1(x) + p_2(x)) + p_3(x)                                                                                \\
                            ={} & ((a_{10} + a_{11}x + \cdots  + a_{1n}x^n) + (a_{20} + a_{21}x + \cdots  + a_{2n}x^n))                     \\
                            +{} & (a_{30} + a_{31}x + \cdots  + a_{3n}x^n)                                                                  \\
                            ={} & ((a_{10} + a_{20}) + (a_{11} + a_{21}) x + \cdots  + (a_{1n} + a_{2n}) x^n)                               \\
                            +{} & (a_{30} + a_{31}x + \cdots  + a_{3n}x^n)                                                                  \\
                            ={} & (((a_{10} + a_{20}) + a_{30}) + ((a_{11} + a_{21}) + a_{31})x + \cdots + ((a_{1n} + a_{2n}) + a_{3n})x^n) \\
                            ={} & ((a_{10} + (a_{20} + a_{30})) + (a_{11} + (a_{21} + a_{31}))x + \cdots + (a_{1n} + (a_{2n} + a_{3n}))x^n) \\
                            ={} & (a_{10} + a_{11}x + \cdots  + a_{1n}x^n)                                                                  \\
                            +{} & ((a_{20} + a_{21}x + \cdots  + a_{2n}x^n) + (a_{30} + a_{31}x + \cdots  + a_{3n}x^n))                     \\
                            ={} & p_1(x) + (p_2(x) + p_3(x))
                        \end{align*}
                        注意,在证明过程中,我们用了形式的加法定义(逐次数将系数相加),并诉诸域 $\mathbf{F}$ 上的结合律,这种诉诸基域性质的方式在以后的证明中会经常碰上.

                  \item 取定 $p_0(x) = 0 \in V$ 则有 $\forall p(x) \in \mathbf{F}[x]_{n+1}, p(x) + p_0(x) = p_0(x) + p(x)$.

                  \item $\forall p(x) = a_0 + a_1x + \cdots + a_nx^n \in \mathbf{F}[x]_{n+1}, \exists p^*(x) = -a_0 - a_1x - \cdots - a_nx^n \in \mathbf{F}[x]_{n+1}, p(x) + p^*(x) = p^*(x) + p(x) = p_0(x) = 0$.

                  \item $\forall p_1(x), p_2(x) \in \mathbf{F}[x]_{n+1}$有
                        \begin{align*}
                            p_1(x) + p_2(x)
                             & = (a_{10} + a_{11}x + \cdots + a_{1n}x^n) + (a_{20} + a_{21}x + \cdots + a_{2n}x^n) \\
                             & = (a_{10} + a_{20}) + (a_{11} + a_{21})x + \cdots + (a_{1n} + a_{2n})x^n            \\
                             & = (a_{20} + a_{10}) + (a_{21} + a_{11})x + \cdots + (a_{2n} + a_{1n})x^n            \\
                             & = (a_{20} + a_{21}x + \cdots + a_{2n}x^n) + (a_{10} + a_{11}x + \cdots + a_{1n}x^n) \\
                             & = p_2(x) + p_1(x).
                        \end{align*}

                  \item 取定 $\lambda = 1 \in \mathbf{F},\forall p(x) \in \mathbf{F}[x]_{n+1}, \lambda \cdot p(x) = p(x)$.

                  \item $\forall \lambda, \mu \in \mathbf{F}, p(x) \in \mathbf{F}[x]_{n+1}$有
                        \begin{align*}
                            \lambda(\mu p(x)) & = \lambda(\mu(a_0 + a_1x + \cdots + a_nx^n)) = \lambda(\mu a_0 + \mu a_1x + \cdots + \mu a_nx^n)                  \\
                                              & = \lambda \mu a_0 + \lambda \mu a_1x + \cdots + \lambda \mu a_nx^n = (\lambda \mu) (a_0 + a_1x + \cdots + a_nx^n) \\
                                              & = (\lambda \mu)p(x).
                        \end{align*}

                  \item $\forall \lambda, \mu \in \mathbf{F}, p(x) \in \mathbf{F}[x]_{n+1}$有
                        \begin{align*}
                            (\lambda + \mu) p(x)
                             & = (\lambda + \mu)(a_0 + a_1x + \cdots + a_nx^n)                                         \\
                             & = (\lambda + \mu)a_0 + (\lambda + \mu)a_1x + \cdots + (\lambda + \mu)a_nx^n             \\
                             & = \lambda a_0 + \mu a_0 + \lambda a_1x + \mu a_1x+ \cdots + \lambda a_nx^n + \mu a_nx^n \\
                             & = \lambda(a_0 + a_1x + \cdots + a_nx^n) + \mu(a_0 + a_1x + \cdots + a_nx^n)             \\
                             & = \lambda p(x) + \mu p(x).
                        \end{align*}
                        这里的第二行到第三行并没有诉诸对单项式的分配律,而是利用了性质 6 和域 $\mathbf{F}$ 上的分配律.

                  \item $\forall p_1(x), p_2(x) \in \mathbf{F}[x]_{n+1}, \lambda \in \mathbf{F}$有
                        \begin{align*}
                                & \lambda(p_1(x) + p_2(x))                                                                        \\
                            ={} & \lambda((a_{10} + a_{11}x + \cdots + a_{1n}x^n) + (a_{20} + a_{21}x + \cdots + a_{2n}x^n))      \\
                            ={} & \lambda((a_{10} + a_{20}) + (a_{11} + a_{21})x + \cdots + (a_{1n} + a_{2n})x^n)                 \\
                            ={} & \lambda(a_{10} + a_{20}) + \lambda(a_{11} + a_{21})x + \cdots + \lambda(a_{1n} + a_{2n})x^n     \\
                            ={} & \lambda(a_{10} + a_{11}x + \cdots + a_{1n}x^n) + \lambda(a_{20} + a_{21}x + \cdots + a_{2n}x^n) \\
                            ={} & \lambda p_1(x) + \lambda p_2(x).
                        \end{align*}
              \end{enumerate}
              但是对\[\mathbf{F}[x]'_{n+1}=\{a_0+a_1x+\cdots+a_nx^n \mid a_i\in\mathbf{F}, a_n\neq 0\}\]不构成线性空间,其原因在于我们无法找到一个零元$p_0(x)$满足$p(x) + p_0(x) = p_0(x) + p(x) = p(x)$.

        \item 同理我们应当对八条性质逐条验证,但我们在第一讲以及说明了全体复数构成一个域,因此$\mathbf{C}(\mathbf{C})$自动满足线性空间的所有条件,此处不再赘述. 除此之外,$\mathbf{C}(\mathbf{R})$的加法运算与实数无关(回顾线性空间定义,实数只用来参与数乘运算),因此加法Abel群事实上与$\mathbf{C}(\mathbf{C})$一致,都是群$\langle \mathbf{C}:+\rangle$,此处也不再验证. 因此这里只验证$\mathbf{C}(\mathbf{R})$数乘运算是否满足线性空间定义的要求:
              \begin{enumerate}
                  \item 取定 $1 \in \mathbf{R}, \forall \alpha = a+b\i \in \mathbf{C},\enspace a, b \in \mathbf{R},\enspace 1 \cdot \alpha = 1 \cdot (a+b\i) = a+b\i = \alpha$.

                  \item $\forall \lambda, \mu \in \mathbf{R},\enspace \alpha = a+b\i \in \mathbf{C},\enspace a, b \in \mathbf{R}$,
                        \begin{align*}
                            \lambda(\mu \alpha) = \lambda(\mu (a+b\i)) = \lambda(\mu a+\mu b\i) = \lambda \mu a + \lambda \mu b\i = (\lambda \mu)(a+b\i) = (\lambda \mu)\alpha.
                        \end{align*}

                  \item $\forall \lambda, \mu \in \mathbf{R},\enspace \alpha = a+b\i \in \mathbf{C},\enspace a, b \in \mathbf{R}$,
                        \begin{align*}
                            (\lambda + \mu) \alpha
                             & = (\lambda a + \lambda b\i) + (\mu a + \mu b\i)            \\
                             & = \lambda(a+b\i)+\mu(a+b\i) = \lambda \alpha + \mu \alpha.
                        \end{align*}

                  \item $\forall \lambda \in \mathbf{R},\enspace \alpha_1 = a_1+b_1\i, \alpha_2 = a_2+b_2\i \in \mathbf{C},\enspace a_i, b_i \in \mathbf{R},\enspace i = 1, 2$,
                        \begin{align*}
                            \lambda(\alpha_1+\alpha_2)
                             & = \lambda((a_1+b_1\i)+(a_2+b_2\i)) = \lambda((a_1+a_2)+(b_1+b_2)\i)                             \\
                             & = \lambda(a_1+a_2)+\lambda(b_1+b_2)\i = (\lambda a_1+\lambda b_1\i)+(\lambda a_2+\lambda b_2\i) \\
                             & = \lambda(a_1+b_1\i)+\lambda(a_2+b_2\i) = \lambda \alpha_1 + \lambda \alpha_2.
                        \end{align*}
              \end{enumerate}
              所以$\mathbf{C}(\mathbf{C})$和$\mathbf{C}(\mathbf{R})$均构成线性空间.

        \item 这里定义的``加法''和``数乘''与一般的不同,不过也只需要验证八条性质就行.
              \begin{enumerate}
                  \item $\forall \alpha = (a_1, a_2, \ldots, a_n), \beta = (b_1, b_2, \ldots, b_n), \gamma = (c_1, c_2, \ldots, c_n) \in V, $
                        \begin{align*}
                            (\alpha \oplus \beta) \oplus \gamma
                             & = ((a_1, a_2, \ldots, a_n)\oplus (b_1, b_2, \ldots, b_n)) \oplus (c_1, c_2, \ldots, c_n)  \\
                             & = (a_1b_1, a_2b_2, \ldots, a_nb_n) \oplus (c_1, c_2, \ldots, c_n)                         \\
                             & = (a_1b_1c_1, a_2b_2c_2, \ldots, a_nb_nc_n)                                               \\
                             & = (a_1, a_2, \ldots, a_n)\oplus (b_1c_1, b_2c_2, \ldots, b_nc_n)                          \\
                             & = (a_1, a_2, \ldots, a_n) \oplus ((b_1, b_2, \ldots, b_n) \oplus (c_1, c_2, \ldots, c_n)) \\
                             & = \alpha \oplus (\beta \oplus \gamma)
                        \end{align*}

                  \item 取定 $e = (1, 1, \ldots , 1) \in V,\enspace \forall \alpha = (a_1, a_2, \ldots, a_n) \in V$,
                        \begin{align*}
                            e \oplus \alpha & =(1, 1, \ldots , 1) \oplus (a_1, a_2, \ldots, a_n) =(a_1, a_2, \ldots, a_n) = \alpha \\
                                            & =(a_1, a_2, \ldots, a_n) \oplus (1, 1, \ldots , 1) =\alpha \oplus e.
                        \end{align*}

                  \item $\forall \alpha = (a_1, a_2, \ldots, a_n) \in V,\enspace \exists \beta = \left(\dfrac{1}{a_1}, \dfrac{1}{a_2}, \ldots, \dfrac{1}{a_n}\right),\enspace \alpha \oplus \beta = \beta \oplus \alpha = e$.

                  \item $\forall \alpha = (a_1, a_2, \ldots, a_n), \beta = (b_1, b_2, \ldots, b_n) \in V$,
                        \begin{align*}
                            \alpha \oplus \beta
                             & = (a_1, a_2, \ldots, a_n) \oplus (b_1, b_2, \ldots, b_n) = (a_1b_1, a_2b_2, \ldots, a_nb_n)                        \\
                             & = (b_1a_1, b_2a_2, \ldots, b_na_n) = (b_1, b_2, \ldots, b_n) \oplus (a_1, a_2, \ldots, a_n) = \beta \oplus \alpha.
                        \end{align*}

                  \item 取定 $\lambda = 1 \in \mathbf{R},\enspace \forall \alpha = (a_1, a_2, \ldots, a_n) \in V$,
                        \[\lambda \circ \alpha = (a_1^\lambda, a_2^\lambda, \ldots, a_n^\lambda) = (a_1, a_2, \ldots, a_n) = \alpha.\]

                  \item $\forall \lambda, \mu \in \mathbf{R}, \forall \alpha \in = (a_1, a_2, \ldots, a_n) \in V$,
                        \begin{align*}
                            \lambda \circ(\mu \circ \alpha)
                             & = \lambda \circ(\mu \circ (a_1, a_2, \ldots, a_n)) = \lambda \circ (a_1^\mu, a_2^\mu, \ldots, a_n^\mu) \\
                             & = (a_1^{\lambda\mu}, a_2^{\lambda\mu}, \ldots, a_n^{\lambda\mu}) = (\lambda \mu)\circ \alpha.
                        \end{align*}

                  \item $\forall \lambda, \mu \in \mathbf{R},\enspace \forall \alpha \in = (a_1, a_2, \ldots, a_n) \in V$,
                        \begin{align*}
                            (\lambda + \mu) \circ \alpha
                             & = (\lambda + \mu) \circ (a_1, a_2, \ldots, a_n) = (a_1^{\lambda + \mu}, a_2^{\lambda + \mu}, \ldots, a_n^{\lambda + \mu})                                              \\
                             & = (a_1^\lambda a_1^\mu, a_2^\lambda a_2^\mu, \ldots, a_n^\lambda a_n^\mu) = (a_1^\lambda, a_2^\lambda, \ldots, a_n^\lambda) \oplus (a_1^\mu, a_2^\mu, \ldots, a_n^\mu) \\
                             & = (\lambda \circ (a_1, a_2, \ldots, a_n)) \oplus (\mu \circ (a_1, a_2, \ldots, a_n))                                                                                   \\
                             & = (\lambda \circ \alpha) \oplus (\mu \circ \alpha).
                        \end{align*}

                  \item $\forall \lambda \in \mathbf{R},\enspace \alpha = (a_1, a_2, \ldots, a_n), \beta = (b_1, b_2, \ldots, b_n) \in V$,
                        \begin{align*}
                            \lambda \circ (\alpha \oplus \beta)
                             & = \lambda \circ ((a_1, a_2, \ldots, a_n) \oplus (b_1, b_2, \ldots, b_n))                                            \\
                             & = \lambda \circ (a_1b_1, a_2b_2, \ldots , a_nb_n) = ((a_1b_1)^\lambda, (a_2b_2)^\lambda, \ldots , (a_nb_n)^\lambda) \\
                             & = (a_1^\lambda b_1^\lambda, a_2^\lambda b_2^\lambda, \ldots , a_n^\lambda b_n^\lambda)                              \\
                             & = (a_1^\lambda, a_2^\lambda, \ldots, a_n^\lambda) \oplus (b_1^\lambda, b_2^\lambda, \ldots, b_n^\lambda)            \\
                             & = (\lambda \circ (a_1, a_2, \ldots, a_n)) \oplus (\lambda \circ (b_1, b_2, \ldots, b_n))                            \\
                             & = (\lambda \circ \alpha) \oplus (\lambda \circ \beta).
                        \end{align*}
              \end{enumerate}
              所以$V$构成在此``加法''和``数乘''下的线性空间.

        \item 这题主要注意需要验证封闭的性质是什么就可以了.
              \begin{enumerate}
                  \item $\forall f, g, h \in V$,
                        \begin{align*}
                            ((f \oplus g) \oplus h)(x) & = (f \oplus g)(x)+h(x)                                \\
                                                       & = (f(x)+g(x))+h(x) = f(x)+(g(x)+h(x))                 \\
                                                       & = f(x)+ (g \oplus h)(x) = (f \oplus (g \oplus h))(x).
                        \end{align*}

                  \item 取定 $e(x)=0,\enspace \forall x \in \mathbf{R},\enspace e(-x)=0=\overline{e(x)},\enspace \forall f \in V$,
                        \begin{align*}
                            (f \oplus e)(x) = f(x) + e(x) = f(x) = e(x) + f(x) = (e \oplus f)(x).
                        \end{align*}

                  \item $\forall f \in V,\enspace \exists g \in V,\enspace g(x) := -f(x),\enspace \forall x \in \mathbf{R}$,
                        \begin{gather*}
                            g(-x) = -f(-x) = -\overline{f(x)} = \overline{g(x)} \\
                            (f \oplus g)(x) = f(x)+g(x) = 0 = e(x) = g(x) + f(x) = (g \oplus f)(x).
                        \end{gather*}

                  \item $\forall f, g \in V,\enspace (f \oplus g)(x) = f(x)+g(x) = g(x)+f(x) = (g \oplus f)(x)$.

                  \item 取定 $\lambda = 1 \in \mathbf{R},\enspace \forall f \in V,\enspace (\lambda \circ f)(x) = \lambda f(x) = f(x)$.

                  \item $\forall \lambda, \mu \in \mathbf{R},\enspace f \in V$,
                        \[(\lambda \circ (\mu \circ f))(x) = \lambda((\mu \circ f)(x)) = \lambda (\mu f(x)) = (\lambda \mu) f(x) = ((\lambda \mu) \circ f)(x).\]

                  \item $\forall \lambda, \mu \in \mathbf{R},\enspace f \in V$,
                        \begin{align*}
                                & ((\lambda + \mu) \circ f)(x) = (\lambda + \mu)f(x) = \lambda f(x) + \mu f(x)           \\
                            ={} & (\lambda \circ f)(x) + (\mu \circ f)(x) = ((\lambda \circ f) \oplus (\mu \circ f))(x).
                        \end{align*}

                  \item $\forall \lambda \in \mathbf{R},\enspace f, g \in V$,
                        \begin{align*}
                                & (\lambda \circ (f \oplus g))(x) = \lambda((f \oplus g)(x)) = \lambda (f(x)+g(x)) = \lambda f(x) + \lambda g(x) \\
                            ={} & (\lambda \circ f)(x) + (\lambda \circ g)(x) = ((\lambda \circ f) \oplus (\lambda \circ g))(x).
                        \end{align*}
              \end{enumerate}
              所以$V$构成在此``加法''和``数乘''下的线性空间.
    \end{enumerate}
\end{solution}

在上例以及习题中我们可以看到很多特殊的线性空间,它们集合中的元素不一定是数或向量,运算也不一定是熟知的数的运算和向量的数乘,对这些空间我们需要学会熟练判断,从而加深对``在集合上定义运算''的理解.

\section{线性子空间}

我们首先介绍线性子空间的定义:
\begin{definition}[线性子空间] \index{xianxingkongjian!zi@线性子空间 (linear subspace), 子空间 (subspace)}
    设$W$是线性空间$V(\mathbf{F})$的非空子集,如果$W$对$V$中的运算也构成域$\mathbf{F}$上的线性空间,则称$W$是$V$的\term{线性子空间}(简称\term{子空间}).
\end{definition}

请一定注意定义中的非空子集,建议验证子空间时先验证非空. 接下来自然的问题便是,什么时候$V$的子集$W$对$V$中的运算也构成域$\mathbf{F}$上的线性空间?事实上这一条件是惊人地简单与美观的:
\begin{theorem}\label{thm:2:子空间判别}
    线性空间$V(\mathbf{F})$的非空子集$W$为$V$的子空间的充分必要条件是$W$对于$V(\mathbf{F})$的线性运算封闭.
\end{theorem}

这表明只要子空间非空且其中的元素满足对原空间的加法和数乘运算封闭即可构成原空间的子空间. 这一定理的证明也非常简单,必要性显然(构成线性空间必须满足运算封闭),充分性我们只需要作如下思考:
\begin{enumerate}
    \item 结合律、分配律运算律是一定不变的,例如我们回顾加法结合律的定义$a+(b+c)=(a+b)+c,\enspace\forall a,b,c\in V$,由于这一性质对于任意$V$中元素成立,则若$a,b,c\in W\subseteq V$也必有这一性质成立(更通俗而言就是子集$W$中的元素也是$V$中的,因此必然受$V$中运算性质的限制);

    \item 我们根据上面的原则对8条性质一一验证,发现加法单位元和逆元仍不能保证存在,因为这不仅与运算法则相关,更与集合中元素的存在相关——取子集可能使得加法单位元和逆元被拿掉. 但在定理要求的数乘封闭性下这是不可能的:由于$\mathbf{F}$是数域,因此所有有理数都是其子集,因此$0,-1\in\mathbf{F}$. $\forall \alpha\in V$,我们由于数乘封闭性可知,$0\cdot\alpha=0\in W$,$(-1)\cdot\alpha=-\alpha\in W$,因此$W$中也有加法单位元和逆元.
\end{enumerate}

证明具体书写见教材62--63页. 下面我们来看两个常见的例子体会子空间的判别方法:
\begin{example}\label{ex:2:常见子空间}
    回答下列关于子空间的判定问题:
    \begin{enumerate}
        \item \label{item:2:常见子空间:1}
              说明$\mathbf{R}[x]_2$是$\mathbf{R}[x]_3$的子空间;

        \item \label{item:2:常见子空间:2}
              判断$W_1=\left\{(x,y,z) \,\middle|\, \dfrac{x}{3}=\dfrac{y}{2}=z\right\},\enspace W_2=\{(x,y,z) \mid x+y+z=1,\enspace x-y+z=1\}$是否为$\mathbf{R}^3$的子空间;

        \item \label{item:2:常见子空间:3}
              (线性方程组的解)试说明齐次线性方程组$AX=0$的解集是线性空间$\mathbf{F}^n$的一个子空间,但非齐次线性方程组的解不再构成线性空间(因为加法运算不封闭,具体见教材P62的2.2节开头的例子以及P86习题3(3).
    \end{enumerate}
\end{example}

\begin{solution}
    \begin{enumerate}
        \item 只需证明$\mathbf{R}[x]_2 \subseteq \mathbf{R}[x]_3$,以及$\mathbf{R}[x]_2$对$\mathbf{R}[x]_3$中的加法和数乘封闭即可.

              $\forall v \in \mathbf{R}[x]_2$,可被写作$v=a+bx,a,b \in \mathbf{R}$. 又有$\mathbf{R}[x]_3=\{a+bx+cx^2,a,b,c \in \mathbf{R}\}$,取$c=0$,有$v=a+bx \in \mathbf{R}[x]_3$,因此$\mathbf{R}[x]_2 \subseteq \mathbf{R}[x]_3$.

              对于$\mathbf{R}[x]_3$中的加法和数乘:
              \[mv_1+nv_2=m(a_1+b_1x)+n(a_2+b_2x)=(ma_1+na_2)+(mb_1+nb_2)x \in \mathbf{R}[x]_3\]
              所以$\mathbf{R}[x]_2$是$\mathbf{R}[x]_3$的子空间.

        \item 对 $W_1$: 引入参数$t$,
              \[W_1=\left\{(3t,2t,t) \,\middle|\, \frac{x}{3} = \frac{y}{2} = z = t\right\}\]
              对于$\forall v_1, v_2 \in W_1, v_1 = (3t_1, 2t_1, t_1), v_2 = (3t_2, 2t_2, t_2)$,有
              \begin{align*}
                  av_1 + bv_2 & = (3at_1 + 3bt_2, 2at_1 + 2bt_2, at_1 + bt_2)           \\
                              & = (3(at_1 + bt_2), 2(at_1 + bt_2), at_1 + bt_2) \in W_1
              \end{align*}
              故$W_1$封闭,是 $\mathbf{R}^3$ 的子空间.

              对 $W_2$: 有反例. 取 $u_1 = (1, 0, 0), u_2 = (0, 0, 1) \in W_2$,但 $W_1 + W_2 = (1, 0, 1)$ 不满足 $x + y + z = 1$,故 $W_2$ 不封闭,不是 $\mathbf{R}^3$ 的子空间.

        \item 设齐次线性方程组 $AX=0$ 的解构成的集合是 $W_1$,$\forall X_1, X_2 \in W_1$,有 $AX_1 = AX_2 = 0$,所以 $\forall a, b \in \mathbf{F}$,
              \[A(a X_1 + b X_2) = A(a X_1) + A(b X_2) = a AX_1 + b AX_2 = 0\]
              故 $W_1$ 封闭,是 $\mathbf{F}^n$ 的子空间.

              设非齐次线性方程组 $AX = \beta,\enspace \beta \in \mathbf{F}^m,\enspace \beta \neq 0$ 的解构成的集合是 $W_2$,$\forall X_1, X_2 \in W_2$,有 $AX_1 = AX_2 = \beta$,所以 $A(X_1 + X_2) = AX_1 + AX_2 = 2\beta \neq \beta$. 故 $W_2$ 不封闭,不是 $\mathbf{F}^n$ 的子空间.
    \end{enumerate}
\end{solution}

上例中 \ref*{item:2:常见子空间:2} 表明过原点的直线/平面构成三维空间的子空间,不过原点的无法保持线性性. 事实上 \ref*{item:2:常见子空间:2} 和 \ref*{item:2:常见子空间:3} 在表述同一个问题,\ref*{item:2:常见子空间:2} 从几何角度描述了 \ref*{item:2:常见子空间:3} 中齐次/非齐次线性方程组的解集. 事实上,在定义了子空间后, 如果一个线性空间的子集也构成线性空间,我们就可以对其进行同样的研究. 这一想法在我们后续的内容中十分重要, 现在需要大家先熟知子空间的定义和判别.

最后我们需要注意一个名词的定义. 线性空间有两个子空间称为平凡子空间,即仅含零元的子集$\{0\}$和其自身$V$. 而其它子空间称为非平凡子空间.

\section{线性表示 \quad 线性扩张}

在高中平面向量的学习中我们知道,两个单位向量$(1,0)$和$(0,1)$可以表示出整个平面的所有向量,高中我们也称这样的向量为平面向量的基底. 接下来我们将二维平面扩展至任意线性空间,同样讨论有关于``表示''、``基底''的问题.

我们首先来看线性组合和线性表示的概念:
\begin{definition}
    设$V(\mathbf{F})$是一个线性空间,$\alpha_i\in V,\enspace\lambda_i\in \mathbf{F}\enspace(i=1,2,\ldots,m)$,则向量$\alpha=\lambda_1\alpha_1+\lambda_2\alpha_2+\cdots+\lambda_m\alpha_m$称为向量组$\alpha_1,\alpha_2,\ldots,\alpha_m$在域$\mathbf{F}$的线性组合,或说$\alpha$在域$\mathbf{F}$上可用向量组$\alpha_1,\alpha_2,\ldots,\alpha_m$线性表示.
\end{definition}
这和我们高中所学的用向量的基底表示其他向量是完全一致的. 基于此,我们给出线性扩张的定义:
\begin{definition}[线性扩张] \index{xianxingkuozhang@线性扩张 (linear span)}
    设$S$是线性空间$V(\mathbf{F})$的非空子集,我们称
    \[ \spa(S)=\{\lambda_1\alpha_1+\cdots+\lambda_k\alpha_k \mid \lambda_1,\ldots,\lambda_k\in\mathbf{F},\enspace\alpha_1,\ldots,\alpha_k\in S,\enspace k\in\mathbf{N}_+\} \]
    为$S$的\term{线性扩张},即$S$中所有有限子集在域$\mathbf{F}$上的一切线性组合组成的$V(\mathbf{F})$的子集.
\end{definition}
注意,$\spa$参考的是《线性代数应该这样学》的记号,《大学数学——代数与几何》中使用$L$表示线性扩张. 考虑到本讲义记号统一性,我们采用更加常用并且不会与之后其它定义的记号冲突的$\spa$.

下面的定理告诉我们可以通过线性扩张构造子空间:
\begin{theorem}\label{thm:2:线性扩张构造子空间}
    线性空间$V(\mathbf{F})$的非空子集$S$的线性扩张$\spa(S)$是$V$中包含$S$的最小子空间.
\end{theorem}
仍然利用平面向量进行直观的理解,平面(也显然在平面向量加法和数乘下构成线性空间)$\mathbf{R}^2$可以由向量$(1,0)$和$(0,1)$扩张而成. 由这一定理的结果我们可以将一个向量组的线性扩张称为向量组的张成空间. 这一定理的证明思想非常重要,因此在此给出:

\begin{proof}
    \begin{enumerate}
        \item 首先我们证明$\spa(S)$是$V$的子空间.
              \begin{enumerate}
                  \item $\spa(S)$非空:由于$S$非空,且$S\subseteq\spa(S)$显然成立:取$\lambda=1,\enspace\forall s\in S,\enspace \lambda s=s\in\spa(S)$. 因此$\spa(S)$非空;

                  \item 设$\alpha,\beta\in\spa(S)$,则存在$\lambda_1,\ldots,\lambda_k\in\mathbf{F},\enspace \alpha_1,\ldots,\alpha_k\in S,\enspace\mu_1,\ldots,\mu_l\in\mathbf{F},\enspace\beta_1,\ldots,\beta_l\in S$,使得
                        \begin{gather*}
                            \alpha=\lambda_1\alpha_1+\cdots+\lambda_k\alpha_k \\
                            \beta=\mu_1\beta_1+\cdots+\mu_l\beta_l
                        \end{gather*}
                        因此我们可以得到$\spa(S)$
                        \begin{enumerate}
                            \item 关于加法封闭:$\alpha+\beta=\lambda_1\alpha_1+\cdots+\lambda_k\alpha_k+\mu_1\beta_1+\cdots+\mu_l\beta_l\in\spa(S)$;

                            \item 关于数乘封闭:$\lambda\alpha=\lambda\lambda_1\alpha_1+\cdots+\lambda\lambda_k\alpha_k\in\spa(S)$(数域关于乘法运算封闭,故$\lambda\lambda_i\in\mathbf{F},\enspace i=1,\ldots,k$).
                        \end{enumerate}
              \end{enumerate}
              综上,$\spa(S)$是$V$的子空间;

        \item 接下来我们证明$\spa(S)$是包含$S$的最小子空间. 设$W$是$V$的任一子空间,我们只需证明$\spa(S)\subseteq W$.

              事实上,类似于前面$S\subseteq\spa(S)$的证明我们有$S\subseteq W$,故$S$中元素都在$W$中. 且由\autoref{thm:2:子空间判别} 可知子空间中元素一定关于加法、数乘封闭,因此$\forall {\alpha}=\lambda_1\alpha_1+\cdots+\lambda_k\alpha_k\in\spa(S)$. 由于$\alpha_1,\ldots,\alpha_k\in S\subseteq W,\enspace\lambda_1,\ldots,\lambda_k\in\mathbf{F}$,因此$\alpha\in W$,从而$\spa(S)\subseteq W$,由此得证.
    \end{enumerate}
\end{proof}

上述证明的重要性在于,我们在这一个证明中练习了子集的证明方法、子空间的充要条件以及对于``最小''问题证明的一般方法. 希望读者能掌握其中的每一个思想与技巧. 此外,这一定理有很强的直观性,因为线性扩张实际上就是将子集中的元素进行无穷次重复的线性组合,将所有可能经过线性运算获得的向量都生成了,因此线性扩张的结果一定保障了线性运算的封闭.

最后我们再说明有限维线性空间和无限维线性空间的定义,本课程研究的内容都在有限维线性空间,如果少数时间拓展至无限维空间我们会给出说明:
\begin{definition}
    $V(\mathbf{F})$称为有限维线性空间,如果$V$中存在一个有限子集$S$使得$\spa(S)=V$,反之称为无限维线性空间.
\end{definition}

\begin{example}
    证明:$\mathbf{R}[x]_3$是有限维线性空间,$\mathbf{R}[x]$是无限维线性空间.
\end{example}

\begin{proof}
    \begin{enumerate}
        \item 显然$\mathbf{R}[x]_3$的有限子集$S=\{1,x,x^2\}$可以张成$\mathbf{R}[x]_3$,因此$\mathbf{R}[x]_3$是有限维线性空间;

        \item 对于$\mathbf{R}[x]$,我们只需证明其任意有限子集都无法张成其本身. 我们取其任意有限子集,则其中多项式元素的次数一定有最大值,我们记为$m$,那么$z^{m+1}$以及更高次数的无法被表示,因此$\mathbf{R}[x]$是无限维线性空间.
    \end{enumerate}
\end{proof}

\vspace{2ex}
\centerline{\heiti \Large 内容总结}

本讲我们追随着第一讲最末尾关于线性方程组为什么无解、有唯一解或无穷解的问题,展开我们对线性方程组一般理论的讨论. 我们首先通过一个例子引入我们为什么要研究线性空间——因为我们需要了解向量之间的关联,从直觉上这与线性方程组解的情况是有关系的. 我们给出了线性空间的定义——其核心仍然是在集合上定义满足一定条件的运算,事实上就是对我们高中就熟知的向量加法数乘规则的抽象,然后我们讨论了基于这一公理化的定义我们可以得到的性质. 我们介绍了线性空间的子空间的定义与判别方法,引入了线性表示、线性扩张的概念并说明了我们如何通过线性扩张得到子空间——这一定理蕴含着所谓``闭包''的思想,我们将在未来讨论仿射子集时再次见到,实际上是非常符合几何直观的.

事实上,这一讲的内容是比较抽象的,因为线性空间的定义实际上就是将我们熟知的向量加法数乘运算抽象出来,从而适用于所有有类似结构的集合,因此读者在学习时可能会自动带入一些高中平面向量的直观,然后发现显然的问题不用证,复杂的问题摸不着头脑,但读者应当在未竟专题一中训练了基于定义和公理的数学证明思想,我们也尽力给出大量经典的例子,将推导过程写得非常详细,所以整体而言思路应当是清晰易懂的.

\vspace{2ex}
\centerline{\heiti \Large 习题}

\vspace{2ex}
{\kaishu 1520年以来,全世界只有85个机构存活至今,其中50家是大学. 大学依靠梦想、希望生存下去——这就是大学的历史.}
\begin{flushright}
    \kaishu
    ——美国哥伦比亚大学校长L·C·柏林格
\end{flushright}

\centerline{\heiti A组}
\begin{enumerate}
    \item 检验下列集合对指定的加法和数乘运算是否构成实数域上的线性空间.
          \begin{enumerate}
              \item 有理数集$\mathbf{Q}$对普通的数的加法和乘法;

              \item 集合$\mathbf{R}^2$对通常的向量加法和如下定义的数量乘法:$\lambda\cdot(x,y)=(\lambda x,y)$;

              \item $\mathbf{R}_+^n$(即$n$元正实数向量)对如下定义的加法和数乘运算:
                    \begin{gather*}
                        (a_1,\ldots,a_n)+(b_1,\ldots,b_n)=(a_1b_1,\ldots,a_nb_n) \\
                        \lambda\cdot(a_1,\ldots,a_n)=(a_1^\lambda,\ldots,a_n^\lambda)
                    \end{gather*}

              \item 请继续完成教材P86第二章习题第1题第(9)--(11)问关于函数的加法数乘定义线性空间的问题.
          \end{enumerate}

    \item 请完成教材P86--87第二章习题第3题. 第(5)问平常问题较多,实际上就是要判断满足一定条件的多项式是否构成子空间.
\end{enumerate}

\centerline{\heiti B组}
\begin{enumerate}
    \item 证明:已知线性空间$V(\mathbf{F})$,$\lambda,\lambda_1,\ldots,\lambda_r\in\mathbf{F}$,$\beta,\alpha_1,\ldots,\alpha_r\in V$,有$\lambda\beta+\lambda_1\alpha_1+\lambda_2\alpha_2+\cdots+\lambda_r\alpha_r=\vec{0}$在$\lambda\neq 0$时的解为$\beta=-\lambda^{-1}\lambda_1\alpha_1-\lambda^{-1}\lambda_2\alpha_2-\cdots-\lambda^{-1}\lambda_r\alpha_r$.

    \item 设$V$是一个线性空间,$W$是$V$的子集,证明:$W$是$V$的子空间$\iff \spa(W)=W$.
\end{enumerate}

\centerline{\heiti C组}
\begin{enumerate}
    \item 设$\mathbf{E}$是域$\mathbf{F}$的一个子域.
          \begin{enumerate}
              \item 证明:$\mathbf{F}$关于自身的加法和乘法构成一个$\mathbf{E}$上的向量空间,并举一例;

              \item 举例说明:$\mathbf{E}\enspace(\mathbf{E}\neq \mathbf{F})$不是$\mathbf{F}$上的线性空间;

              \item 证明:若$V$是$\mathbf{F}$上的一个线性空间,则$V$也是$\mathbf{E}$上的一个线性空间.
          \end{enumerate}

    \item 考虑在第一章定义的有限域 $\mathbf{F}_4$ 和 $\mathbf{Z}_2$. 证明:$\mathbf{Z}_2$ 可以看作 $\mathbf{F}_4$ 的一个子域. 并给出 $\mathbf{F}_4$ 在 $\mathbf{Z}_2$ 上的线性空间结构. 验证 $\mathbf{Z}_p$ 一定是 $\mathbf{F}_{p^n}$ 的一个子域.
\end{enumerate}

\chapter{有限维线性空间}

在第二讲开头的\autoref{ex:2:线性空间引入} 中,我们讨论了齐次线性方程组解的个数与方程组系数矩阵
行向量间没有可互相消去的关系之间的联系.本节我们将这种``可互相消去的关系''进行形式化定义.另一方面,
在第二讲最后探讨线性扩张的概念时,一个很自然的问题便是:一个有限维线性空间最少可以由多少个向量
线性扩张而来?循此路径,我们将在本讲探寻线性空间的最基本的结构属性.

\section{线性相关性}
\subsection{线性相关性的定义}
本节我们将形式化定义在引言中我们提到的``可相互消去的关系''——线性相关性,同时这一定义也可以解决
引言中提到的关于有限维线性空间至少需要多少个向量张成的问题.
\begin{definition}
    设$V(\mathbf{F})$是一个线性空间,$\alpha_1,\alpha_2,\ldots,\alpha_m\in V$,若存在
    不全为0的$\lambda_1,\lambda_2,\ldots,\lambda_m\in\mathbf{F}$,使得
    \[\lambda_1\alpha_1+\lambda_2\alpha_2+\cdots+\lambda_m\alpha_m=0\]
    成立,则称$\alpha_1,\alpha_2,\ldots,\alpha_m$\keyterm{线性相关}[linearly dependent],
    否则称\keyterm{线性无关}[linearly independent](即系数只能为0).
\end{definition}

很显然,\autoref{ex:2:线性空间引入} 中的方程组1系数矩阵的三个行向量$\alpha_1,\alpha_2,\alpha_3$
满足$\alpha_1+\alpha_2-\alpha_3=0$,因此满足线性相关的定义,方程组2的系数矩阵三个行向量
$\beta_1,\beta_2,\beta_3$的线性组合则只有$0\cdot\beta_1+0\cdot\beta_2+0\cdot\beta_3$等于0,
因此符合线性无关的定义.

事实上,直接由定义我们还可以导出以下关于零向量的结论:
\begin{enumerate}
    \item 线性空间中单个向量$\alpha$线性相关的充要条件是$\alpha$为零向量;

    \item 任何含零向量的向量组都线性相关.
\end{enumerate}

需要注意的是,很多时候线性相关和线性无关的证明就是基于定义,请务必牢牢掌握.我们先来看几个基本的例子:
\begin{example}
    \begin{enumerate}[label=(\arabic*)]
        \item 判断$\mathbf{R}^3$中向量$(1,1,0),(0,1,1),(1,0,-1)$的线性相关性;

        \item 判断$\mathbf{R}^3$中向量$(1,-3,1),(-1,2,-2),(1,1,3)$的线性相关性;

        \item 判断$\mathbf{R}[x]_3$中$p_1(x)=1+x,\enspace p_2(x)=1-x,\enspace p_3(x)=x+x^2$的线性相关性;

        \item 判断连续函数全体构成的线性空间中$1,\enspace \sin^2x,\enspace \cos^2x$的线性相关性;

        \item 判断连续函数全体构成的线性空间中$1,\enspace 2^x,\enspace 2^{-x}$的线性相关性.
    \end{enumerate}
\end{example}
注意上述 (3) 到 (5) 题为不能代入特殊的$x$值来说明,例如 (3) 令$x=0$得到线性相关的做法是错误的,因为
(3) 中线性空间就是多项式构成的线性空间,其中的元素就是多项式,不能代入值.注意 (5) 是特殊题型,
需要构造更多的方程来求解这一问题.

\subsection{线性相关性的定理}
实际上,除了定义之外,线性相关性还有大量的等价描述.我们将在本节介绍常见的等价描述,它们是
理解线性空间结构等后续内容的基础,因此希望读者对以下结论及其证明十分熟练并且要有深刻的理解.
我们的主线思路是从不同方面理解线性相关性:
\begin{enumerate}
    \item 从线性组合看(定义)

          向量组线性相关$\iff$它们有系数不全为0的线性组合等于零向量;

          向量组线性无关$\iff$它们只有系数全为0的线性组合才会等于零向量.
    \item 从线性表示看(教材定理2.3)
          \begin{theorem}
              线性空间$V(\mathbf{F})$中的向量组$\alpha_1,\alpha_2,\ldots,\alpha_m\enspace(m \geqslant 2)$线性相关的充分必要条件是
              $\alpha_1,\alpha_2,\ldots,\alpha_m$中有一个向量可由其余向量在域$\mathbf{F}$上线性表示.
          \end{theorem}
          这一定理等价描述为,向量组线性无关的充分必要条件是其中的向量无法互相表示.这是显然的,因为向量组能互相表示
          利用定义可以轻松写出非零系数的线性表示.总结一下即为:

          向量组线性相关$\iff$其中至少有一个向量可以由其余向量线性表示;

          向量组线性无关$\iff$其中每一个向量都不能由其余向量线性表出.
    \item 从齐次线性方程组看(教材P66例3,实际上这一点与定义十分类似)

          列向量组$\alpha_1,\alpha_2,\ldots,\alpha_m$线性相关$\iff$齐次线性方程组$x_1\alpha_1+x_2\alpha_2+\cdots+x_m\alpha_m=0$有非零解;

          列向量组$\alpha_1,\alpha_2,\ldots,\alpha_m$线性无关$\iff$齐次线性方程组$x_1\alpha_1+x_2\alpha_2+\cdots+x_m\alpha_m=0$只有零解.
    \item 从向量组与它的部分组的关系看(教材P67例6)

          如果向量组的一个部分组线性相关,那么整个向量组也线性相关;

          如果向量组线性无关,那么它的任何一个部分组也线性无关.
    \item 从向量组线性表示一个向量的方式看(教材定理2.4)
          \begin{theorem}\label{thm:3:线性无关等价表示唯一}
              若向量组$\alpha_1,\alpha_2,\ldots,\alpha_m$线性无关,而向量组$\beta,\alpha_1,\alpha_2,\ldots,\alpha_m$线性相关,
              则$\beta$可由$\alpha_1,\alpha_2,\ldots,\alpha_m$线性表示,且表示法唯一.
          \end{theorem}
          这一定理证明十分经典,特别是唯一性的证明需要掌握,因此此处我们给出证明:

          \begin{proof}
            由于向量组$\beta,\alpha_1,\alpha_2,\ldots,\alpha_m$线性相关,故存在不全为0的$\lambda_0,\lambda_1,\ldots,\lambda_m$使得
            \begin{equation}\label{eq:3:线性无关等价定理}
                \lambda_0\beta+\lambda_1\alpha_1+\lambda_2\alpha_2+\cdots+\lambda_m\alpha_m=0,
            \end{equation}
            其中$\lambda_0$必不为0,因为如果将$\lambda_0=0$代入\autoref{eq:3:线性无关等价定理},则由于向量组
            $\alpha_1,\alpha_2,\ldots,\alpha_m$线性无关,必有$\lambda_1=\lambda_2=\cdots=\lambda_m=0$,
            与$\lambda_0,\lambda_1,\ldots,\lambda_m$不全为0的假设矛盾.

            因此我们有
            \[\beta=-\frac{\lambda_1}{\lambda_0}\alpha_1-\frac{\lambda_2}{\lambda_0}\alpha_2-\cdots-\frac{\lambda_m}{\lambda_0}\alpha_m.\]
            由此我们知道$\beta$可由$\alpha_1,\alpha_2,\ldots,\alpha_m$线性表示.接下来我们证明表示方式的唯一性.假设有两种表示方法:
            \begin{gather*}
                \beta=\mu_1\alpha_1+\mu_2\alpha_2+\cdots+\mu_m\alpha_m, \\
                \beta=\nu_1\alpha_1+\nu_2\alpha_2+\cdots+\nu_m\alpha_m.
            \end{gather*}
            两式相减可得
            \[0=(\mu_1-\nu_1)\alpha_1+(\mu_2-\nu_2)\alpha_2+\cdots+(\mu_m-\nu_m)\alpha_m.\]
            由于$\alpha_1,\alpha_2,\ldots,\alpha_m$线性无关,因此$\mu_i-\nu_i=0(i=1,2,\cdots,m)$,即$\mu_i=\nu_i(i=1,2,\cdots,m)$,
            因此表示方式唯一.
          \end{proof}

          事实上关于这一定理我们有一个直接的推论
          \begin{corollary}
            若向量组外另一向量可由这一组向量线性表示,则
            \begin{enumerate}
                \item 向量组线性无关$\iff$表示方式唯一;

                \item 向量组线性相关$\iff$表示方式有无穷多种.
            \end{enumerate}
          \end{corollary}
          推论的证明非常简单,此处考虑到读者可能处于初学阶段,给出证明范例:
          
          \begin{proof}
            我们设向量组为$\alpha_1,\alpha_2,\ldots,\alpha_m$,向量组外的向量为$\beta$.
            对于(a),向量组线性无关$\Rightarrow$表示方式唯一就是\autoref{thm:3:线性无关等价表示唯一}
            的直接结论,因此我们只需考虑表示方式唯一$\Rightarrow$向量组线性无关.利用反证法,假设向量组线性相关,
            则存在不全为0的$\lambda_1,\lambda_2,\ldots,\lambda_m$使得
            \begin{equation}\label{eq:3:线性无关等价推论1}
                0=\lambda_1\alpha_1+\lambda_2\alpha_2+\cdots+\lambda_m\alpha_m.
            \end{equation}
            由于$\beta$可由$\alpha_1,\alpha_2,\ldots,\alpha_m$线性表示,因此存在$\mu_1,\mu_2,\ldots,\mu_m$使得
            \begin{equation}\label{eq:3:线性无关等价推论2}
                \beta=\mu_1\alpha_1+\mu_2\alpha_2+\cdots+\mu_m\alpha_m.
            \end{equation}
            事实上,我们只需将\autoref{eq:3:线性无关等价推论1}两边乘以任意的$k\in\mathbf{F}$($\mathbf{F}$为
            向量组所在线性空间定义的数域),然后加到\autoref{eq:3:线性无关等价推论2}的两边即可得到
            \[\beta=(\mu_1+k\lambda_1)\alpha_1+(\mu_2+k\lambda_2)\alpha_2+\cdots+(\mu_m+k\lambda_m)\alpha_m.\]
            因此表示方式不唯一(且有无穷多种),与假设矛盾,因此向量组线性无关.事实上这一证明也将(b)中向量组线性无关
            $\Rightarrow$表示方式有无穷多种证明给出,(b)的另一边同样用反证法可以回到(a)的证明,由此推论得证.
          \end{proof}
\end{enumerate}

\section{基与维数}
\subsection{引入:向量组的秩与极大线性无关组}
在上一节中我们介绍了很基本的线性无关的等价表述,现在我们回到我们的主线,即我们希望解决有限维线性空间至少需要多少个
向量张成的问题,接下来的讨论将逐步逼近问题的答案.
\begin{lemma}\label{lemma:3:线性相关性引理}
    设$\alpha_1,\alpha_2,\ldots,\alpha_m$线性相关,则有$j\in\{1,2,\cdots,m\}$使得:
    \begin{enumerate}
        \item $\alpha_j=\spa(\alpha_1,\alpha_2,\ldots,\alpha_{j-1})$;
        \item 从$\alpha_1,\alpha_2,\ldots,\alpha_m$中删去向量$\alpha_j$,剩余向量的张成空间仍等于
                $\spa(\alpha_1,\alpha_2,\ldots,\alpha_m)$.
    \end{enumerate}
\end{lemma}
可能大家看见1的记号可能又有些许陌生了,但只需简单回顾线性扩张的定义,我们知道证明1就是证明$\alpha_j$可以被
$\alpha_1,\alpha_2,\ldots,\alpha_{j-1}$线性表示.这一结论初看和\autoref{thm:3:线性无关等价表示唯一}很类似,
但细看发现不太一样:我们要求必须有一个向量可以由排列在它前面的向量线性表示,而非被其余所有向量线性表示.因此这一结论
并不平凡,证明的过程中也有一个技巧,我们给出证明供读者参考学习:

\begin{proof}
    由于$\alpha_1,\alpha_2,\ldots,\alpha_m$线性相关,因此存在不全为0的$\lambda_1,\lambda_2,\ldots,\lambda_m$使得
    \[\lambda_1\alpha_1+\lambda_2\alpha_2+\cdots+\lambda_m\alpha_m=0.\]
    设$j$是$\{1,2,\cdots,m\}$中使得$\lambda_j\neq 0$的最大者,则有
    \begin{equation}\label{eq:3:线性相关性引理}
        \alpha_j=-\frac{\lambda_1}{\lambda_j}\alpha_1-\frac{\lambda_2}{\lambda_j}\alpha_2-\cdots-\frac{\lambda_{j-1}}{\lambda_j}\alpha_{j-1}.
    \end{equation}
    因此$\alpha_j$可由$\alpha_1,\alpha_2,\ldots,\alpha_{j-1}$线性表示,即$\alpha_j\in\spa(\alpha_1,\alpha_2,\ldots,\alpha_{j-1})$,
    故1得证.

    接下来我们证明2.首先$\spa(\alpha_1,\ldots,\alpha_{j-1},\alpha_{j+1},\ldots,\alpha_m)\subseteq\spa(\alpha_1,\alpha_2,\ldots,\alpha_m)$是显然的,
    因为任意被$\alpha_1,\ldots,\alpha_{j-1},\alpha_{j+1},\ldots,\alpha_m$线性表示的向量实际上也是被
    $\alpha_1,\alpha_2,\ldots,\alpha_m$线性表示了,只是$\alpha_j$前的系数恒为0.

    然后证明另一边包含关系,即$\spa(\alpha_1,\alpha_2,\ldots,\alpha_m)\subseteq\spa(\alpha_1,\ldots,\alpha_{j-1},\alpha_{j+1},\ldots,\alpha_m)$.
    任取$\beta\in\spa(\alpha_1,\cdots,\alpha_m)$,则存在$\mu_1,\mu_2,\ldots,\mu_m$使得
    \[\beta=\mu_1\alpha_1+\mu_2\alpha_2+\cdots+\mu_m\alpha_m.\]
    将$\alpha_j$用\autoref{eq:3:线性相关性引理}表示,代入上式可得任意$\spa(\alpha_1,\cdots,\alpha_m)$中的向量都可以由
    $\alpha_1,\ldots,\alpha_{j-1}$,\\$\alpha_{j+1},\ldots,\alpha_m$线性表示,因此
    $\beta\in\spa(\alpha_1,\alpha_2,\ldots,\alpha_{j-1},\alpha_{j+1},\ldots,\alpha_m)$,故引理得证.
\end{proof}

事实上1中证明最核心的步骤就是取$j$是$\{1,2,\cdots,m\}$中使得$\lambda_j\neq 0$的最大者,这一最大者是一定存在的,
因为首先存在$\lambda_i\neq 0$,其次$\lambda_i\neq 0$的个数是有限的,因此一定存在最大者,这一证明的技巧十分重要,通俗的记忆方法
为``从右往左检查,找到第一个不为0的系数(即最大的不为0的系数)'',我们给出一个推论,推论的证明思想就是如此,
我们放在习题中供读者练习:
\begin{corollary}
    $\alpha_1,\alpha_2,\ldots,\alpha_m$线性相关(其中$\alpha_1\neq 0$)的充要条件是存在一个向量$\alpha_i(1<i\neq m)$可由
    $\alpha_1,\alpha_2,\ldots,\alpha_{i-1}$线性表示,且表示法唯一.
\end{corollary}
事实上这一推论也可以作为线性无关的等价表述之一.

接下来我们继续我们的主线思路,事实上\autoref{lemma:3:线性相关性引理}的2给我们了一个很重要的启示,即对于线性相关的向量组,
我们丢弃其中某些(可以被其他向量线性表示)的向量后,张成的空间是不变的.因此我们可以重复丢弃这样的向量,并仍然保持张成空间不变.
一个自然的问题是,这样丢弃的操作直到什么时候停止呢?

事实上答案也是非常自然的,即我们最后一次从向量组中丢弃向量(并保证张成的空间不变)后,剩余的向量组恰好线性无关时即可停止丢弃.
原因非常简单,因为如果这最后一次不丢弃,则根据\autoref{lemma:3:线性相关性引理}我们一定还能选出一个向量,使得丢弃这一向量
后仍能保持张成空间不变.但一旦丢弃向量后向量组线性无关,这时一定不能继续丢弃,例如这时剩余的线性无关向量组为
$\beta_1,\cdots,\beta_m$,这时丢弃其中任意一个$\beta_i(i\in\{1,2,\cdots,m\})$,则原向量组张成的空间中,至少
$\beta_i$无法被剩余向量组线性表示(否则$\beta_i$可以被$\beta_1,\cdots,\beta_{i-1},\beta_{i+1},\cdots,\beta_m$线性表示,
则$\beta_1,\cdots,\beta_m$必线性相关),因此我们一定不能继续丢弃.

我们可以将上述过程形式化地表达为以下算法:
\begin{enumerate}
    \item 
\end{enumerate}
事实上,上述算法的每一步都是可操作的并且有绝对确定的结果,因此我们总能为一个向量组找到最短的可以张成同一线性空间的子向量组.

在上述过程中我们可以引入两个重要的概念,即向量组的秩和极大线性无关组:
\begin{definition}
    设向量组$S=\{\alpha_1,\alpha_2,\ldots,\alpha_m\}$张成的线性空间为$V$,若存在$S$的一个线性无关向量组
    $B=\{\alpha_{k1},\alpha_{k2},\ldots,\alpha_{kr}\}$,使得$V=\spa(B)$,则称$B$为$S$的一个
    \keyterm{极大线性无关组}[maximal linearly independent set],并称极大线性无关组的长度$r=r(S)$为$S$的
    \keyterm{秩}[rank].
\end{definition}
定义中``极大''一词我们只需简单思考前述过程即可明白其含义,因为我们要求丢弃后的向量组一旦线性无关就要停止继续丢弃向量,
因此这一剩余向量组的长度一定是所有线性无关向量组中最大的.

要注意的是,极大线性无关组在本讲义、教材甚至其它教材(如丘维声老师的高等代数)中的定义都有所不同,实际上不同的版本只是为了
顺应不同讲解思路而提出的,本质上并无区别,相信读者在完全理解本节内容后能认识到这一点.

由此我们关于有限维线性空间至少需要多少个向量张成的问题有了初步的解答,即如果我们已知这一线性空间是可以由某一向量组张成的,
那么这一向量组的秩(即极大线性无关组的长度)就是张成空间需要的最少向量个数.可能初看这一段话,其中出现的``极大''和``最小''
容易导致思维的混乱,但我们可以用一句话清晰地总结:极大线性无关组的长度就是张成空间需要的最少向量个数(如果仍然混乱,我们可以
回忆丢弃向量的过程:我们不断丢弃向量得到``最小''的仍然满足张成空间不变的向量组,而这一向量组必须是所有线性无关向量组中最长的,
因为向量组丢到线性无关后不能再丢了).

\subsection{向量组的性质}
事实上,我们会有一个自然的疑问,即极大线性无关组的长度是否唯一?我们在丢弃向量的时候,如果向量的排序不同,我们丢弃的次序也可能
不同,因此我们最终得到的极大线性无关组是有可能不同的.但长度不同表明向量组的秩不唯一,这样向量组的秩就失去了很多研究价值——数学喜欢
唯一确定的,例如数学分析中表达式的极限不唯一我们会称其极限不存在;又例如定积分的值如果可以是不唯一的,那么我们一定会
重新思考积分的定义,否则面积、体积甚至物理中的很多问题都会产生意义不明的多解.

因此我们需要尝试证明极大线性无关组的长度是唯一的,我们从下面这一非常重要的定理开始:
\begin{theorem}\label{thm:3:线性表示}
    设$V(\mathbf{F})$中向量组$ \beta_1,\beta_2,\ldots,\beta_s $的每个向量可由另一向量组$\alpha_1,\alpha_2,\ldots,\alpha_r$
    线性表示.若$s>r$,则$ \beta_1,\beta_2,\ldots,\beta_s $线性相关.
\end{theorem}
这一定理的等价(逆否)命题为,$ \beta_1,\beta_2,\ldots,\beta_s $线性无关则必有$s\leqslant r$.

这一定理可通俗概括为:多的向量组可以被少的向量组线性表示,多的一定线性相关.反过来说,线性无关的向量只能被等长或更长的向量组线性表示.
定理的证明思想上非常简单,但写起来可能有些许复杂,我们给出证明:

\begin{proof}
    
\end{proof}

事实上,\autoref{thm:3:线性表示}因其重要性又被称为源泉定理,因为我们可以基于此得到大量的推论,下面我们将给出几个简单的作为代表,
习题中会出现更为复杂的应用:
\begin{example}\label{ex:3:线性表示推论}
    证明以下\autoref{thm:3:线性表示}的推论:
    \begin{enumerate}[label=(\arabic*)]
        \item 任意$n$维线性空间中的$n+1$个向量必线性相关,反之,$n-1$个向量无法张成$n$维线性空间;
        \item 若向量组$B_1$可以被向量组$B_2$线性表示,则有$r(B_1)\leqslant r(B_2)$;
        \item 设$B_1$和$B_2$是两个线性无关向量组,若$B_1$可以被$B_2$线性表示,$B_2$也可以被$B_1$线性表示,则$B_1$和$B_2$长度相等.
    \end{enumerate}
\end{example}
\begin{proof}
    
\end{proof}

事实上,\autoref{ex:3:线性表示推论}(3)中两个向量组$B_1$和$B_2$可以互相表示也可以称$B_1$和$B_2$等价.这里的等价和\autoref{def:1:等价关系}
中描述的等价关系一致,即向量组等价同样满足自反性、对称性和传递性,即
\begin{enumerate}
    \item 自反性:
    \item 对称性:
    \item 传递性:
\end{enumerate}
三个条件的成立是显然的,我们不再赘述,接下来我们基于等价向量组的定义给出\autoref{thm:3:线性表示}的进一步结论,直至证明向量组的秩唯一:
\begin{corollary}
    关于等价的向量组,我们有如下结论:
    \begin{enumerate}
        \item 向量组与其极大线性无关组等价;
        \item 向量组的任意两个极大线性无关组等价;
        \item 向量组的任意两个极大线性无关组长度相等,即向量组的秩唯一.
    \end{enumerate}
\end{corollary}
\begin{proof}
    
\end{proof}

由此我们证明了向量组的秩是唯一的,因此这一定义对我们将来的研究非常友好.
\subsection{基与维数}
在前几小节中,我们讨论了这一问题:给定向量组$B$,我们能否选出一个长度最小的向量组$B_1$使其张成的空间与$B$能张成的空间相同.
接下来我们讨论更一般化的情形,即我们不给定向量组$B$,直接讨论能张成一个线性空间的线性无关向量组.
\begin{definition}
    若线性空间$V(\mathbf{F})$的有限子集$B=\{\alpha_1,\alpha_2,\ldots,\alpha_n\}$线性无关,且$\spa(B) = V$,则称$B$为$V$的一组基,
    并称$n$为$V$的维数,记作$\dim V = n$.
\end{definition}

关于基与维数的定义,我们有以下几点需要强调:
\begin{enumerate}
    \item 我们有一个自然的问题:有限维线性空间是否一定有基,若是,则上述定义的基和维数对所有有限维线性空间都是存在的.
    事实上结论是显然的.根据定义,有限维线性空间$V$一定能被其某一有限子集$S$张成,我们根据求取极大线性无关组的算法取出$S$的极大线性无关组$B$,
    则$B$一定是$V$的基.

    \item 由第一点我们发现,基的存在依赖于极大线性无关组的存在,二者只是在定义上有差别:极大线性无关组是一个向量组的最短等价向量组,
    而基是张成线性空间的最短向量组.但二者本质统一,实际上极大线性无关组就是它能张成的线性空间的一组基,其长度(向量组的秩)也就是线性空间的维数.

    \item 有限维线性空间的基不一定唯一,但它们的长度必定唯一(即维数唯一).这一推导和向量组的秩唯一完全一致.我们可以假设有限维线性空间$V$有两组基$B_1$和$B_2$,
    根据基的定义(即它们可以张成$V$,也就是可以表示出$V$中的所有向量).因此$B_1$中的每一个向量都可以由$B_2$线性表示,反之亦然,因此$B_1$和$B_2$等价,
    由此我们可以得到$B_1$和$B_2$的长度相等,即因此有限维线性空间维数唯一.
    
    \item 我们还需要提及一个概念:自然基.例如三维空间的自然基为$e_1=(1,0,0),e_2=(0,1,0),e_3(0,0,1)$. $n$维空间也有类似的推广
    (即$n$个只有一位为 1 其余全为 0 的向量,此后没有特殊说明$e_i$就表示$\mathbf{R}^n$第$i$位为1,其余位置为0的自然基).
    对于多项式我们则将$1,x,x^2,\ldots$称为自然基,矩阵、函数等构成的线性空间也有相关的常用的基.
\end{enumerate}

事实上,定义出基和维数之后我们对线性空间的研究方式就更明朗了:我们从开始的令人眼花缭乱的八条运算性质,利用这些线性运算的特点导出线性扩张与子空间的关联,
然后经过线性相关性的讨论最终得到线性空间的本质结构实际上就是可以由基经过一系列线性运算扩张而来,因此我们对线性空间的研究很多时候只需要研究其基和维数即可,
由此我们的抽象上升一层,即我们不需要观察线性空间中无限个向量,事实上只需要研究有限个向量的性质即可对整个线性空间有较为全面的了解.实际上这一思想与之后我们
得到矩阵等讨论是密切相关的,因此在我们整个向着对线性方程组解的结构的讨论的路径中也称得上是一块关键的里程碑.

我们经常会遇到验证线性空间的基的问题(求解基的题目最后往往也需要验证你写出的向量组确实是基),我们主要有如下两个角度:
\begin{enumerate}
    \item 根据定义,我们只需验证基的两个条件:线性无关和张成空间.线性无关利用定义即可,张成空间则需要验证任意向量都可以由基线性表示.
    \item 若我们能确认线性空间$V$的维数$\dim V$,那么我们只需找到$\dim V$个线性无关的向量即可,因为它们必然是$V$的基.这一结论的证明
    是容易的,在下面的例题中我们给出一个更一般的结论的证明供读者参考.
\end{enumerate}
\begin{example}
    在$n$维线性空间$V$中,$n$个向量$\alpha_1,\ldots,\alpha_n$线性无关的充要条件是它们可以线性表示出$V$中的任意向量.
\end{example}
\begin{proof}
    
\end{proof}

除此之外,我们也在此给出一些求解或验证线性空间的基和维数的基本例题,在习题以及后续章节中会有更多的例子.
\begin{example}\label{ex:3:不同数域的维数}
    证明:线性空间$\mathbf{C(C)}$维数为1,不同于线性空间$\mathbf{C(R)}$维数为2.
\end{example}
\begin{proof}
    
\end{proof}
\begin{example}
    证明:1,$(x-5)^2$,$(x-5)^3$是$\mathbf{R}[x]_4$的子空间$U$的一组基,其中$U$定义为
    \[U=\{p\in\mathbf{R}[x]_4:p'(5)=0\}.\]
\end{example}
\begin{proof}
    
\end{proof}

我们在后续讨论中经常会涉及子空间和原空间之间的关联,特别是它们的基之间的关联,下面这一定理能很好地满足我们的需求:
\begin{theorem}
	如果$W$是$n$维线性空间$V$的一个子空间,则$W$的基可以扩充为$V$的基.
\end{theorem}
这一定理的应用非常广泛,事实上笔者认为这一定理结论重要性高于证明,因此不在此给出证明,对证明感兴趣的读者可以参看教材70页.

实际上还有关于向量组的秩、基与维数有关的很多结论,事实上都可以由前述的定理推导而来,很多结论事实上都非常自然,我们将习题中展示.
考虑到本讲概念、定理内容多而杂.我们在本讲最后也会给出一个思维导图,读者可以参考.

\subsection{极大线性无关组的求法}
我们在前述讲解中实际上已经给出一个求解极大线性无关组的方法,但那一方法适用于证明极大线性无关组一定存在,
如果考试中要求取极大线性无关组我们应当考虑教材71页给出的``通用而简便''的方法.事实上教材中给出的方法以及解释
已经非常细致,我们只总结其关键步骤,读者可以参考教材中进行细致的学习.
\begin{lemma}
    极大线性无关组的求法

    我们将题目给定的向量按列排成矩阵,然后将矩阵作初等变换化成阶梯矩阵,找到主元所在的列,提取出原矩阵对应列的向量即可.
\end{lemma}

注意极大线性无关组是不唯一的,但上面给出了一个程式化的方法.实际上如果能一眼看出结果的也不必如此麻烦
(当然题目直接要求极大线性无关组还是应当写具体过程的).
\begin{example}\label{ex:3:求解极大线性无关组}
	求向量组
    \[\{\alpha_1=(1,-1,2,4),\alpha_2=(0,3,1,2),\alpha_3=(3,0,7,14),\alpha_4=(1,-1,2,0),\alpha_5=(2,1,5,6)\}\]
	的极大线性无关组和秩.
\end{example}
\begin{solution}

\end{solution}

学会求解极大线性无关组后,我们还能解决一个重要的问题,就是如何扩张一个线性无关向量组成为线性空间的一组基.之前我们只说明了这样的扩张
是存在的,但具体如何取到并没有给出.虽然在未来实际应用中我们大部分时候可能只需要扩充一两个向量就行,很多时候我们随手取或者
依靠之后的行列式等工具就很好解决.但实践中我们发现很多同学在教材没给出固定算法的情况下完全无法接受``随手取''这样的描述,因此
在此笔者还是给出一种虽然暴力但一定有效的算法.

设线性空间$V$维数为$n$,我们已有的线性无关向量组为$B=\{\alpha_1,\alpha_2,\ldots,\alpha_s\}(s<n)$,我们的目标是将这个向量组扩充为
$V$的一组基$B'=\{\alpha_1,\alpha_2,\ldots,\alpha_s,\alpha_{s+1},\ldots,\alpha_n\}$,我们的算法如下:
\begin{enumerate}
    \item 首先,如果$V$不是$\mathbf{F}^n$空间,我们取$B$在$V$的任意一组基下的坐标(如果有自然基最好取自然基方便计算);
    \item 任取$V$的一组基$B_0=\{\beta_1,\beta_2,\ldots,\beta_n\}$,这组基和前面取的是否一致无所谓(看了后面的例子就明白了).
    我们得到了一个新的向量组$B_1=\{\alpha_1,\alpha_2,\ldots,\alpha_s,\beta_1,\beta_2,\ldots,\beta_n\}$;
    \item 求$B_1$的极大线性无关组即可,特别注意最后选向量的时候不能把$\alpha_1,\alpha_2,\ldots,\alpha_s$扔掉了,只能扔后面的向量,
    因为我们求的是从$\alpha_1,\alpha_2,\ldots,\alpha_s$扩充来的一组基;
    \item 最后将我们上面得到的坐标结合第一步取的$V$的基得到由$B$扩充而来的一组基.
\end{enumerate}
\begin{example}
    设$V=\mathbf{R}[x]_4$,我们已有向量组$B=\{1+x,x^3+x^2+3x\}$,请将其扩充为$V$的一组基.
\end{example}
\begin{solution}

\end{solution}

\section{向量的坐标}
坐标的概念实际上我们已经熟悉,例如高中所学的平面向量的坐标表示就是向量在二维平面的基$(0,1),(1,0)$
下的坐标表示.我们现在将这个概念拓展到更一般的线性空间:
\begin{definition}
	设$B=\{\beta_1,\beta_2,\cdots,\beta_n\}$是$n$维线性空间$V(\mathbf{F})$的一组基,如果$V$中元素$\alpha$
	表示为$\alpha=a_1\beta_1+a_2\beta_2+\cdots+a_n\beta_n$,则其系数组$a_1,a_2,\cdots,a_n$称为$\alpha$在
	基$B$下的坐标,记为$\alpha_B=(a_1,a_2,\cdots,a_n)$.
\end{definition}
\begin{example}
	分别求$p(x)=a_0+a_1x+a_2x^2$在基$B_1=\{1,x,x^2\}$和$B_2=\{1,x-1,(x-1)^2\}$下的坐标.
\end{example}

关于向量的定义我们有以下几点需要强调:
\begin{enumerate}
    \item 若向量$\alpha$在基$\beta_1,\beta_2,\cdots,\beta_n$下的坐标为$\alpha_B=(a_1,a_2,\cdots,a_n)$,则我们也可以写为
    \[\alpha=(\beta_1,\beta_2,\cdots,\beta_n)\begin{pmatrix}
        a_1\\a_2\\\vdots\\a_n
    \end{pmatrix}.\]
    实际上我们在学习矩阵乘法后就会意识到这一记号是很自然的,因为这样的表示就等价于
    \[\alpha=a_1\beta_1+a_2\beta_2+\cdots+a_n\beta_n,\]
    但当前没有学习矩阵乘法,因此我们只能将其视为一种记号.
    \item 坐标与向量是一一对应的:一个坐标可以确定唯一的向量,一个向量在基下表示的系数也必然唯一(因为基是线性无关的);
    \item 坐标保持元素间的线性运算关系不变:$(\alpha+\beta)_B=\alpha_B+\beta_B$和$(\lambda\alpha)_B=\lambda\alpha_B$成立,
    例如$\mathbf{R}[x]_3$中的向量$\alpha=x^2+2x+1$和$\beta=2x^2+3x+1$,则$\alpha+\beta=3x^2+5x+2$,对应于向量运算,
    我们有$(\alpha+\beta)_B=(3,5,2)=(1,2,0)+(2,3,2)=\alpha_B+\beta_B$.
    证明如下:
    
    \begin{proof}
        
    \end{proof}
    \item 由以上两点我们可以知道:我们对各种各样的$n$维线性空间的研究都可以首先通过坐标转化为$\mathbf{F}^n$中的元素进行研究,例如
    \begin{example}\label{ex:3:转化为坐标}
        求$\mathbf{R}[x]_4$中向量组$\{p_1=x^3-x^2+2x+4,p_2=3x^2+x+2,p_3=3x^3+7x+14,p_4=x^3-x^2+2x,p_5=2x^3+x^2+5x+6\}$
        的极大线性无关组.
    \end{example}
    我们首先将所有多项式先转化为坐标,然后就会发现和\autoref{ex:3:求解极大线性无关组}完全一致,最后将坐标转回多项式即可.

    事实上,将任意的线性空间转化为$\mathbf{F}^n$研究的思想是非常重要的,因为这可以带来进一步的抽象,即我们甚至可以遮蔽线性空间基的特点,
    只关注其维数进行研究,这与此后线性空间的同构以及矩阵表示都有密不可分的联系.事实上,我们一直都在使用这一基本思想,我们每次设线性空间有一组基
    $\alpha_1,\cdots,\alpha_n$时,事实上我们只关注其维数$n$而遮蔽了基的特点:它可以是向量,可以是多项式,可以是矩阵、函数等等,
    但这些都不重要,我们都可以这些元素视为几何空间$\mathbf{F}^n$中的向量,获得更直观的理解,从而可以忽视一些使我们理解困难的细节.
    \item 容易验证$\mathbf{R}^n$中的向量在自然基下的坐标实际上就是向量本身,例如$(x,y,z)=xe_1+ye_2+ze_3$,故在$\mathbf{R}^3$自然基
    下的坐标仍然为$(x,y,z)$,需要牢记,有时可以加速解题.
\end{enumerate}

\vspace{2ex}
\centerline{\heiti \Large 内容总结}
本节内容相对而言概念和定理非常多,涉及的题型也很多,因此我们在这里给出一个思维导图,供读者捋顺思路:

事实上,与其他内容风格不一样的是,本讲中很大一部分的定理我们都给出了证明,一方面是为了提升阅读体验,防止在
初学时就被多个``显然''等词汇困惑,另一方面也是希望读者能够从这些比较规范的证明中得到一些证明的技巧.

也许读到这里很多读者都会有些迷惑与焦急——为什么我们仿佛在学习很多看起来十分抽象而且似乎没什么实际应用的
知识呢?(抽象金字塔)
\vspace{2ex}

\centerline{\heiti \Large 习题}
\vspace{2ex}
{\kaishu }
\begin{flushright}
    \kaishu

\end{flushright}
\centerline{\heiti A组}
\begin{enumerate}
    \item 请先完成教材P87-88第二章习题第10题的判断题;
	\item 证明:如果向量组线性相关,把每个向量去掉$m$个位置一致的分量,得到的缩短组仍线性相关;
	如果向量组线性无关,把每个向量添加$m$个位置一致的分量,得到的缩短组仍线性无关;
	\item $a$取何值时,$\beta_1=(1,3,6,2)^\mathrm{T},\beta_2=(2,1,2,-1)^\mathrm{T},\beta_3=(1,-1,a,-2)^\mathrm{T}$
	线性无关?
	\item 设$\alpha_1,\alpha_2,\cdots,\alpha_n\in\mathbf{F}^n$,证明:$\alpha_1,\alpha_2,\cdots,\alpha_n$线性无关
	的充要条件是$\mathbf{F}^n$中任一向量都可以由它们线性表示.
	\item 设$S_1=\{\alpha_1,\cdots,\alpha_s\},S_2=\{\beta_1,\cdots,\beta_t\}$是向量空间$V$的两个线性无关的子集,证明:
	$\alpha_1,\cdots,\alpha_s,\beta_1,\cdots,\beta_t$线性无关$\iff L(S_1)\cap L(S_2)=\{0\}$.
    \item 已知$\alpha_1=(1,2,4,3),\alpha_2=(1,-1,-6,6),\alpha_3=(-2,-1,2,-9),\alpha_4=(1,1,-2,7),\beta=(4,2,4,a)$.
	\begin{enumerate}[label=(\arabic*)]
        \item 求子空间$L(\alpha_1,\alpha_2,\alpha_3,\alpha_4)$的维数和一组基;
        \item 求$a$的值使得$\beta\in W$,并求$\beta$在(1)所选基下的坐标.
    \end{enumerate}
	\item 证明:$B=\{1,x-a,(x-a)^2\}(a\neq 0)$是$\mathbf{R}[x]_3$的一组基,并求$\mathbf{R}[x]_3$的
	自然基$\{1,x,x^2\}$中每个向量关于基$B$的坐标.
	\item 已知向量组$A=\{\alpha_1,\alpha_2,\alpha_3\},B=\{\alpha_1,\alpha_2,\alpha_3,\alpha_4\},C=\{\alpha_1,\alpha_2,\alpha_3,\alpha_5\}$
	的秩分别为$r(A)=r(B)=3,r(C)=4$,证明:$\{\alpha_1,\alpha_2,\alpha_3,\alpha_5-\alpha_4\}$的秩为4.
	\item 设向量组$\alpha_1,\alpha_2,\cdots,\alpha_s$的秩为$r$,在其中任取$m$个向量$\alpha_{i1},\alpha_{i2},\cdots,\alpha_{im}$,
	证明:向量组$\alpha_{i1},\alpha_{i2},\cdots,\alpha_{im}$的秩$\ge r+m-s$.
	\item 已知$\alpha_1,\alpha_2,\cdots,\alpha_n$线性无关,且$\alpha_1,\alpha_2,\cdots,\alpha_n,\beta,\gamma$线性相关,
	证明:要么$\beta,\gamma$可以由$\alpha_1,\alpha_2,\cdots,\alpha_n$线性表示,要么$\alpha_1,\alpha_2,\cdots,\alpha_n,\beta$
	与$\alpha_1,\alpha_2,\cdots,\alpha_n,\gamma$等价.
\end{enumerate}
\centerline{\heiti B组}
\begin{enumerate}
    \item $\alpha_1,\alpha_2,\ldots,\alpha_m$线性相关(其中$\alpha_1\neq 0$)的充要条件是存在一个向量$\alpha_i(1<i\neq m)$可由
    $\alpha_1,\alpha_2,\ldots,\alpha_{i-1}$线性表示,且表示法唯一.
    \item 证明以下两个结论:
    \begin{enumerate}[label=(\arabic*)]
        \item 设$U$和$W$都是$V$的非零子空间,如果$U\subseteq W$,那么$\dim U \leqslant \dim W$;
        \item 设$U$和$W$都是$V$的非零子空间,$U\subseteq W$,且$\dim U = \dim W$,则$U = W$.
    \end{enumerate}
    \item 已知$\alpha_1\neq 0$,则$\alpha_1,\alpha_2,\cdots,\alpha_n$线性相关的充要条件是存在$i(2\le i\le n)$
	使得$\alpha_i$可由$\alpha_1,\alpha_2,\cdots,\alpha_{i-1}$线性表示,且表示法唯一.
	\item 设向量组$\alpha_1,\alpha_2,\cdots,\alpha_n$线性无关,在向量组$\beta,\alpha_1,\alpha_2,\cdots,\alpha_n$中至多有一个向量$\alpha_i(1\le i\le r)$
	可被其前面的$i$个向量$\beta,\alpha_1,\alpha_2,\cdots,\alpha_{i-1}$线性表示.
	\item 证明:$1,e^{\lambda_1\cdot x},e^{\lambda_2\cdot  x}(\lambda_1\neq\lambda_2$且均不为0$)$线性无关.
	\item 设线性空间$V(\mathbf{F})$中,向量$\beta$是$\alpha_1,\cdots,\alpha_r$的线性组合,但不是$\alpha_1,\cdots,\alpha_{r-1}$的线性组合,
	证明:$L(\alpha_1,\cdots,\alpha_{r-1},\alpha_r)=L(\alpha_1,\cdots,\alpha_{r-1},\beta)$.
\end{enumerate}
\centerline{\heiti C组}
\begin{enumerate}
	\item 已知$m$个向量$\alpha_1,\alpha_2,\cdots,\alpha_m$线性相关,但其中任意$m-1$个都线性无关,证明:
	\begin{enumerate}
        \item 若$k_1\alpha_1+\cdots+k_m\alpha_m=0$,则$k_1,\cdots,k_m$全为0或全不为0;
        \item 若存在两个等式
        \[k_1\alpha_1+\cdots+k_m\alpha_m=0,\]
        \[l_1\alpha_1+\cdots+l_m\alpha_m=0,\]
        其中$l_1\neq 0$,证明:$\cfrac{k_1}{l_1}=\cdots=\cfrac{k_m}{l_m}$.
    \end{enumerate}
    \item (替换定理)设$\alpha_1,\alpha_2,\cdots,\alpha_r$线性无关,且可以被
	$\{\beta_1,\beta_2,\cdots,\beta_n\}$线性表示,则可以从$\{\beta_1,\beta_2,\cdots,\beta_n\}$
	选出$r$个向量替换成$\alpha_1,\alpha_2,\cdots,\alpha_r$后得到与$\{\beta_1,\beta_2,\cdots,\beta_n\}$
	等价的新向量组(注:可以使用数学归纳法证明).
	\item 设线性空间$V=\mathbf{F^n}$,证明:
	\begin{enumerate}[label=(\arabic*)]
        \item 存在$V$的子空间$W$,使得$W$的任一非零向量的分量均不为0;
        \item 若$V$的子空间$W$的任一非零向量的分量均不为0,则$\dim W=1$;
        \item 若$V$的子空间$W$的任一非零向量的零分量个数均不超过$r$,则$\dim W \le r+1$.
    \end{enumerate}
\end{enumerate}

\chapter{线性空间的运算}

在前述章节中我们对(有限维)线性空间中的基本概念以及研究的基本问题进行了了解. 事实上,很多时候我们还需要研究不同线性空间进行运算后得到的新集合的性质,本节我们将详细展开讨论这一问题.

\section{线性空间的交、并、和}

\begin{definition}
    设$W_1,W_2$是线性空间$V(\mathbf{F})$的两个子空间,则
    \begin{align*}
        W_1 \cap W_2 & =\{\alpha \mid \alpha\in W_1 \text{~且~} \alpha\in W_2\}            \\
        W_1 \cup W_2 & =\{\alpha \mid \alpha\in W_1 \text{~或~} \alpha\in W_2\}            \\
        W_1 + W_2    & =\{\alpha_1+\alpha_2 \mid \alpha_1\in W_1,\enspace\alpha_2\in W_2\}
    \end{align*}
    分别称为$W_1$和$W_2$的交、并、和.
\end{definition}

交与并的定义实际上与集合交与并的定义类似,而和的定义可能有些许反直觉. 我们可以通过一个例子来体会为什么要定义子空间的和.
\begin{example}\label{ex:4:子空间运算}
    在$\mathbf{R}^3$中,我们设
    \[\alpha_1=(0,0,1),\ \alpha_2=(0,1,0),\ \alpha_3=(1,0,0).\]
    令$\mathbf{R}^3$子空间$W_1=\spa(\alpha_1,\alpha_2)$,$W_2=\spa(\alpha_1,\alpha_3)$,则$W_1$实际上是$yOz$平面,$W_2$是$xOz$平面,因此我们根据交与并的概念(实际上就是集合取交集和并集)得到$W_1 \cap W_2=\spa(\alpha_1)$(即$z$坐标轴).

    进一步考察并集,事实上显然$W_1 \cup W_2$得到的集合表示$xOz$和$yOz$平面上所有的点. 事实上我们发现,$W_1 \cup W_2$得到的集合关于向量加法、数乘运算并不封闭,例如只需取$\alpha_2+\alpha_3=(0,1,1)$就不在$W_1 \cup W_2$中,因此不再是$\mathbf{R}^3$的子空间.

    接下来我们考察二者之和. 事实上$W_1+W_2=\mathbf{R}^3$. 原因在于
    \begin{enumerate}
        \item $\forall \beta\in W_1 + W_2$,由子空间和的定义可知有$\beta=\beta_1+\beta_2$,其中$\beta_1\in W_1\subseteq \mathbf{R}^3$,$\beta_2\in W_2\subseteq \mathbf{R}^3$,由于$\mathbf{R}^3$是线性空间,其中元素关于加法运算封闭,因此$\beta=\beta_1+\beta_2\in \mathbf{R}^3$,即$W_1+W_2\subseteq \mathbf{R}^3$;

        \item $\mathbf{R}^3$中任一向量$(x,y,z)$总能写成$(x,y,z)=(0,y,z)+(x,0,0)$的形式,其中$(0,y,z)$在$W_1$中,$(x,0,0)$在$W_2$中,因此根据子空间和的定义可知$\mathbf{R}^3\subseteq W_1 + W_2$成立.
    \end{enumerate}
    综上,我们得到$W_1+W_2=\mathbf{R}^3$.
\end{example}

从上面证明$W_1+W_2=\mathbf{R}^3$的过程中我们可以提炼出证明子空间的和等于某一空间的一般方法:本质而言仍然是证明集合相等,因此证明两边包含即可. 证明子空间的和属于某一空间是平凡的,如上述证明的第一部分;第二部分证明某一空间属于子空间和只需要将该空间中任意向量都可分解为各个子空间中向量的和即可.

事实上,根据\autoref{ex:4:子空间运算} 我们发现,子空间$W_1$和$W_2$的交与和仍然是线性空间,但是它们的并不是线性空间. 事实上,我们可以证明如下定理:
\begin{theorem}
    设$W_1,W_2$是线性空间$V(\mathbf{F})$的两个子空间,则
    \begin{enumerate}
        \item $W_1 \cap W_2$是$V$的子空间;

        \item $W_1 + W_2$是$V$的子空间;

        \item $W_1 \cup W_2$为$V$的子空间$\iff W_1 \subseteq W_2$或$W_2 \subseteq W_1 \iff W_1 \cup W_2=W_1+W_2$.
    \end{enumerate}
\end{theorem}

定理前两条的证明见教材74页,第三条我们留作习题供读者练习,因为在考试中有出现过. 前两条还可以进行推广,即$V$的有限个子空间的交与和仍然是$V$的子空间.

除此之外,这一定理也告诉我们为什么需要研究子空间的和而更少研究子空间的并:因为子空间的和仍然是线性空间. 直观理解实际上就是和的定义中出现了两个子空间的向量的加法,而构成子空间的核心就是运算封闭,因此这一定义为子空间的和仍构成子空间提供了保证,因此这一定义也是十分自然的.

\section{覆盖定理}

\begin{theorem}[覆盖定理] \label{thm:4:覆盖定理} \index{fugaidingli@覆盖定理}
    设$V_1,V_2,\ldots,V_s$是线性空间$V$的$s$个非平凡子空间,证明:$V$中至少存在一个向量不属于$V_1,V_2,\ldots,V_s$中的任何一个,即$V_1 \cup V_2 \cup \cdots \cup V_s\subsetneq V$.
\end{theorem}

覆盖定理表明任何一个线性空间都不能被自身有限个非平凡子空间通过并得到. 初看可能有些不够自然,但我们可以从简单的几何意义获得直观的理解:有限条直线的并不可能是一个平面. 下面我们利用数学归纳法进行证明.

\begin{proof}
    \begin{enumerate}
        \item 当$s=2$时,由于$V_1,V_2$是非平凡子空间,因此$V$中存在$\alpha\notin V_1$. 若$\alpha\notin V_2$,则结论已经成立. 若$\alpha\in V_2$,由$V_2$非平凡知存在$\beta\notin V_2$. 我们考虑$\alpha+\beta$和$2\alpha+\beta$,则必有这两个向量都不属于$V_2$(否则有$\beta\in V_2$),并且这两个向量也不能同时属于$V_1$(否则两个向量相减等于$\alpha$也属于$V_1$,矛盾). 这就说明这两个向量中至少有一个既不在$V_1$中也不在$V_2$中,因此结论成立.

        \item 对于$s>2$,假设命题对$s-1$个子空间成立,即$V$中存在向量$\alpha\notin V_1\cup V_2\cup\cdots\cup V_{s-1}$. 若$\alpha\notin V_s$,则结论成立. 若$\alpha\in V_s$,由$V_s$非平凡知存在$\beta\notin V_s$. 我们考虑$\alpha+\beta,2\alpha+\beta,\ldots,s\alpha+\beta$,则与归纳基础中同样的原因,必有这$s$个向量都不属于$V_s$,且这$s$个向量中不可能存在两个向量同属于一个$V_i\enspace(i=1,2,\ldots,s-1)$,因此这$s$个向量中至少有一个不在$V_1\cup V_2\cup\cdots\cup V_s$中,因此结论成立.
    \end{enumerate}
\end{proof}

本质而言$s>2$的情况就是将$s-1$个子空间的并视为一个整体,然后套用$s=2$的情况证明. 若将这一定理的条件限制在$V$为有限维线性空间,我们也可以利用Vandermonde行列式的方法证明,详见\autoref{ex:13:行列式证明覆盖定理}. 事实上覆盖定理在习题中也有出现,例如教材91--92页第8、9题,都是覆盖定理的直接证明. 我们下面再给出一个例子供读者应用覆盖定理:
\begin{example}
    $V_1,V_2,\ldots,V_s$是线性空间$V$的$s$个非平凡子空间,证明:存在$V$的一组基$\alpha_1,\alpha_2,\ldots,\alpha_n$都不在$V_1,V_2,\ldots,V_s$中.
\end{example}

\begin{proof}
    由\autoref{thm:4:覆盖定理},$V$中存在向量$\alpha_1\notin V_1\cup V_2\cup\cdots\cup V_s$. 继续取$\alpha_2\notin V_1\cup V_2\cup\cdots\cup V_s\cup\spa(\alpha_1)$,则一定有$\alpha_1,\alpha_2$线性无关. 继续取$\alpha_3\notin V_1\cup V_2\cup\cdots\cup V_s\cup\spa(\alpha_1,\alpha_2)$,则一定有$\alpha_1,\alpha_2,\alpha_3$线性无关. 以此类推,最终得到一组基$\alpha_1,\alpha_2,\ldots,\alpha_n$都不在$V_1,V_2,\ldots,V_s$中.
\end{proof}

\section{维数公式}

\begin{theorem}[维数公式]\label{thm:4:维数公式}
    设$W_1,W_2$是线性空间$V(\mathbf{F})$的两个子空间,则
    \[\dim W_1+\dim W_2=\dim(W_1+W_2)+\dim(W_1\cap W_2).\]
\end{theorem}
上式称为子空间的维数公式,区别于下一专题中的线性映射基本定理的维数公式. 这一定理的证明思想非常重要,因此此处我们给出证明.

\begin{proof}
    设$\dim W_1=s,\enspace \dim W_2=t,\enspace \dim(W_1\cap W_2)=r$. 设$W_1\cap W_2$的一组基为$\alpha_1,\alpha_2,\ldots,\alpha_r$,则可以扩充为$W_1$的一组基,记为$\alpha_1,\alpha_2,\ldots,\alpha_r,\beta_1,\ldots,\beta_{s-r}$;也可以扩充为$W_2$的一组基,记为$\alpha_1,\alpha_2,\ldots,\alpha_r,\gamma_1,\ldots,\gamma_{t-r}$. 则我们有
    \[W_1+W_2=\spa(\alpha_1,\ldots,\alpha_r,\beta_1,\ldots,\beta_{s-r},\gamma_1,\ldots,\gamma_{t-r})\]
    (如果对这一步有疑问可以回顾\autoref{ex:4:子空间运算} 中的证明). 由此,我们要证$\dim (W_1+W_2)=s+t-r$,只需证$\alpha_1,\ldots,\alpha_r,\beta_1,\ldots,\beta_{s-r},\gamma_1,\ldots,\gamma_{t-r}$线性无关. 为此,我们设
    \begin{equation}\label{eq:4:维数公式证明1}
        a_1\alpha_1+\cdots+a_r\alpha_r+b_1\beta_1+\cdots+b_{s-r}\beta_{s-r}+c_1\gamma_1+\cdots+c_{t-r}\gamma_{t-r}=0,
    \end{equation}
    即
    \begin{equation}\label{eq:4:维数公式证明2}
        a_1\alpha_1+\cdots+a_r\alpha_r+b_1\beta_1+\cdots+b_{s-r}\beta_{s-r}=-c_1\gamma_1-\cdots-c_{t-r}\gamma_{t-r}.
    \end{equation}
    显然,\autoref{eq:4:维数公式证明2} 等号两端的向量分别属于$W_1$和$W_2$,因此它们都属于$W_1\cap W_2$,因此都可以被$W_1\cap W_2$的基线性表示,即
    \[-c_1\gamma_1-\cdots-c_{t-r}\gamma_{t-r}=d_1\alpha_1+\cdots+d_r\alpha_r,\]
    即
    \begin{equation}\label{eq:4:维数公式证明3}
        c_1\gamma_1+\cdots+c_{t-r}\gamma_{t-r}+d_1\alpha_1+\cdots+d_r\alpha_r=0.
    \end{equation}
    由于$\alpha_1,\ldots,\alpha_r,\gamma_1,\ldots,\gamma_{t-r}$是$W_2$的基,因此\autoref{eq:4:维数公式证明3} 所有系数都为0,即$c_1=\cdots=c_{t-r}=d_1=\cdots=d_r=0$. 代入\autoref{eq:4:维数公式证明2} 后,由于$\alpha_1,\dots,\alpha_r,\beta_1,\dots,\beta_{s-r}$是$W_1$的基,因此可得$a_1=\cdots=a_r=b_1=\cdots=b_{s-r}=0$,因此,代入\autoref{eq:4:维数公式证明1} 后可知$\alpha_1,\ldots,\alpha_r,\beta_1,\ldots,\beta_{s-r},\gamma_1,\ldots,\gamma_{t-r}$必定线性无关(因为根据前述证明所有系数只能为0),故得证.
\end{proof}

总结而言,这一定理证明用到的思想就是``设小扩大''. 我们设出最小空间$V_1\cap V_2$的基,然后分别扩充为$V_1$和$V_2$的基,然后观察要证明的等式和已知的联系,然后利用\autoref{eq:4:维数公式证明2} 构造等式两边属于不同空间的向量这一技巧即可. 下面是一个证明思想类似的例子,需要用到矩阵的相关知识,暂未学到的同学可以先略过本题:
\begin{example}
    已知$A,B$分别是数域$\mathbf{F}$上的$l \times k$和$k \times n$矩阵,$X$是$n \times 1$的列向量. 证明:所有满足$ABX=0$的$BX$构成一个线性空间$V$,且$\dim V = r(B) - r(AB)$.
\end{example}

\begin{proof}
    $V$是线性空间只需要说明其中元素关于加法数乘封闭即可,因为这样$V$就是$\mathbf{F}^k$的子空间. 这一证明非常基本,我们在此略过.

    记$V_1=\{X\mid BX=0\},\enspace V_2=\{X\mid ABX=0\}$,则$V_1\subseteq V_2$,因为$\forall X\in V_1$,有$BX=0$,因此$ABX=A0=0$,即$X\in V_2$,因此$V_1\subseteq V_2$. 利用``设小扩大''的思想,取$V_1$的一组基$\alpha_1,\ldots,\alpha_r$,则可以扩充为$V_2$的一组基,记为$\alpha_1,\ldots,\alpha_r,\alpha_{r+1},\ldots,\alpha_m$,则$r=n-r(B)$,$s=n-r(AB)$,于是
    \begin{align*}
        V & =\{BX\mid ABX=0\}                                                \\
          & =\spa(B\alpha_1,\ldots,B\alpha_r,B\alpha_{r+1},\ldots,B\alpha_m) \\
          & =\spa(B\alpha_{r+1},\ldots,B\alpha_m).
    \end{align*}
    下面证明$B\alpha_{r+1},\ldots,B\alpha_m$线性无关. 为此,设
    \[c_{r+1}B\alpha_{r+1}+\cdots+c_mB\alpha_m=0,\]
    则
    \[B(c_{r+1}\alpha_{r+1}+\cdots+c_m\alpha_m)=0,\]
    因此$c_{r+1}\alpha_{r+1}+\cdots+c_m\alpha_m\in V_1$,因此存在$c_1,\ldots,c_r$使得
    \[c_{r+1}\alpha_{r+1}+\cdots+c_m\alpha_n=c_1\alpha_1+\cdots+c_r\alpha_r,\]
    即
    \[c_{r+1}\alpha_{r+1}+\cdots+c_m\alpha_m-c_1\alpha_1-\cdots-c_r\alpha_r=0.\]
    由于$\alpha_1,\ldots,\alpha_m$线性无关,因此
    \[c_{r+1}=\cdots=c_m=c_1=c_2=\cdots=c_r=0,\]
    因此$B\alpha_{r+1},\ldots,B\alpha_m$线性无关,因此$V$的维数为$s-r=(n-r(AB))-(n-r(B))=r(B)-r(AB)$,得证.
\end{proof}

\section{线性空间的直和}

我们将来证明或者利用和空间时,很多时候都是利用和空间定义进行向量分解. 我们特别重视分解唯一时的情形,因为这对我们的研究很有帮助,这时的和即为直和. 严谨而言,我们有如下定义:
\begin{definition}
    设$W_1,W_2$是线性空间$V(\mathbf{F})$的两个子空间. 若$W_1 \cap W_2=\{0\}$,则$W_1+W_2$叫做$W_1$与$W_2$的\term{直和}\index{zhihe@直和 (direct sum)},记作$W_1\oplus W_2$.

    进一步地,若$V=W_1\oplus W_2$,则称$W_1,W_2$为\term{互补子空间}\index{xianxingkongjian!hubu@互补子空间 (complementary subspaces)},或$W_1$是$W_2$的补空间,或$W_2$是$W_1$的补空间.
\end{definition}

直和有以下等价的命题,我们证明或者利用直和都可以任意选择:
\begin{theorem}\label{thm:4:直和等价命题}
    对于子空间$W_1,W_2$,下列命题等价:
    \begin{enumerate}
        \item $W_1+W_2$是直和,即$W_1 \cap W_2=\{0\}$;

        \item $W_1+W_2$中的每个向量$\alpha$的分解式$\alpha=\alpha_1+\alpha_2\enspace(\alpha_1\in W_1,\enspace\alpha_2\in W_2)$唯一;

        \item 零向量的分解式$\vec{0}=\alpha_1+\alpha_2 \enspace(\alpha_1\in W_1,\enspace\alpha_2\in W_2)$仅当$\alpha_1=\alpha_2=\vec{0}$时成立;

        \item $\dim (W_1+W_2)=\dim W_1+\dim W_2$.
    \end{enumerate}
\end{theorem}

定理的证明是基本的,可以参考教材76页. 在实际运用中我们要非常熟悉这些等价条件,因为都可能使用到.

我们也可以定义有限个子空间的直和,即$V=W_1\oplus W_2\oplus\cdots\oplus W_n \iff W_i \cap \sum\limits_{j \neq i}W_j=\{0\}$,即每个子空间与其余子空间的和的交都是$\{0\}$. 等价命题也是上述定理的推广,例如唯一分解、$\vec{0}$的分解以及维数公式推广,此处不再赘述,详见教材76页. 除此之外,我们还有一个与多空间直和相关的定理:
\begin{theorem}\label{thm:4:多空间直和}
    若$V=V_1\oplus V_2,\enspace V_1=V_{11}\oplus\cdots\oplus V_{1s},\enspace V_2=V_{21}\oplus\cdots\oplus V_{2t}$,则
    \[V=V_{11}\oplus\cdots\oplus V_{1s}\oplus V_{21}\oplus\cdots\oplus V_{2t}\]
\end{theorem}
这一定理的证明是很简单的,实际上利用零向量分解唯一即可.

在习题中我们证明直和一般有两种思路,一种是先证和,再证直和,我们来看一个例子(没有学到矩阵的可以先略过):
\begin{example}
    数域$\mathbf{F}$上所有$n$阶方阵组成的线性空间$V=\mathbf{M}_n(\mathbf{F})$,$V_1$表示所有对称矩阵组成的集合,$V_2$表示所有反对称矩阵组成的集合. 证明:$V_1,V_2$都是$V$的子空间,且$V=V_1\oplus V_2$.
\end{example}

\begin{proof}
    首先证明和. 事实上,对于任意矩阵$A\in V$,有
    \[A=B+C,\enspace B=\frac{1}{2}(A+A^T),\enspace C=\frac{1}{2}(A-A^T),\]
    其中$B$是对称矩阵,$C$是反对称矩阵,即$B\in V_1$,$C\in V_2$,因此$V_1+V_2=V$(因为$V$中任意元素都可以写成$V_1$和$V_2$元素和的形式,根据和的定义可知成立).

    下面证明直和. 我们有如下三种方法:
    \begin{enumerate}
        \item 利用零向量分解唯一:设$O$是$n$阶零矩阵,设$O=B+C$,其中$B$是对称矩阵,$C$是反对称矩阵. 由于$B$是对称矩阵,因此$B^T=B$,由于$C$是反对称矩阵,因此$C^T=-C$,因此
              \[O=O^T=(B+C)^T=B^T+C^T=B-C\]
              解得$B=C=O$,因此零向量分解唯一,故直和得证;

        \item 利用$V_1\cap V_2=\{0\}$:设$A\in V_1\cap V_2$,则$A=A^T=-A$,因此$A=-A$,即$A=O$,因此$V_1\cap V_2=\{0\}$,故直和得证;

        \item 利用$\dim V_1+\dim V_2=\dim V$:这一方法较为复杂,我们简单阐述思想. 设$E_{ij}$是第$i$行第$j$列元素为1,其余元素为0的矩阵,则$V$的一组基为$E_{ij},\enspace i,j=1,2,\ldots,n$,$V_1$的一组基为$E_{ij}+E_{ji},\enspace i<j$和$E_{ii},\enspace i=1,2,\ldots,n$,$V_2$的一组基为$E_{ij}-E_{ji},\enspace i<j$,则$\dim V_1=\dfrac{n(n-1)}{2},\enspace \dim V_2=\dfrac{n(n-1)}{2}$,因此$\dim V_1+\dim V_2=n^2$,因此$\dim V_1+\dim V_2=\dim V$,故直和得证.
    \end{enumerate}
\end{proof}

还有一种证明$V=V_1\oplus V_2$的方式是先令$W=V_1+V_2$,先证明和为直和(即交为$\{0\}$)再证$W=V$即可,下面是一个例子:
\begin{example}
    设$A$是数域$\mathbf{F}$上的一个$n$阶可逆方阵,$A$的前$r$个行向量组成的矩阵为$B$,后$n-r$个行向量组成的矩阵为$C$,$n$元线性方程组$BX=0$与$CX=0$的解空间分别为$V_1,V_2$. 证明:$\mathbf{F}^n=V_1\oplus V_2$.
\end{example}

\begin{proof}
    先记$W=V_1+V_2$,若$\alpha\in V_1\cap V_2$,则$B\alpha=C\alpha=0$,所以
    \[A\alpha=\begin{pmatrix}
            B \\
            C
        \end{pmatrix}\alpha=0,\]
    由于$A$可逆,因此$\alpha=0$,即$V_1\cap V_2=\{0\}$,因此$V_1+V_2$是直和,因此只需证$W=\mathbf{F}^n$即可. 事实上,我们知道$r(B)=r,r(C)=n-r$,因此$\dim V_1=n-r,\enspace \dim V_2=n-(n-r)=r$,所以
    \[\dim W=\dim V_1+\dim V_2=n-r+r=n=\dim \mathbf{F}^n,\]
    又$W=V_1\oplus V_2\subseteq \mathbf{F}^n$,因此$W=\mathbf{F}^n$,故得证.
\end{proof}

最后我们要提醒读者注意的是,有限维线性空间的一个子空间的补空间并不唯一,如下面的例子:
\begin{example}
    在$\mathbf{R}^3$中,$W_1=\spa(\alpha_1)$,则其补空间根据直和的维数公式可知为2,记为$W_2=\spa(\alpha_2,\alpha_3)$. 实际上只需要$\alpha_1,\alpha_2,\alpha_3$线性无关即可,事实上这样的选择是有无穷种的,因为$W_1$本质表示一条直线,故$W_2$是不包含$W_1$且不与$W_1$平行的平面即可,这样$\alpha_2,\alpha_3$是$W_2$任意一组基都可以.
\end{example}

\vspace{2ex}
\centerline{\heiti \Large 内容总结}

本讲我们介绍了线性空间之间的三种运算——交、并、和. 和的概念初次见到可能有些许抽象,但经过一些例子之后我们应当能理解为什么线性空间不同于普通集合,更常用``和''这一运算. 关于并我们给出了一些构成线性空间的条件以及一个重要的覆盖定理,读者了解即可. 关于交与和我们给出了一个维数公式,它不仅结论非常重要,``设小扩大''的证明思想也是在未来非常常用的. 进一步地,我们讨论了直和的概念以及它的等价条件,以及证明直和的两种思路. 我们必须要重视直和这一概念,因为它在未来关于线性变换矩阵约化表示的讨论中起到重要的桥梁作用.

除此之外,线性空间之间还有一种基于笛卡尔积的运算,我们将在后续讨论同构的时候作为一个应用讨论,将直和与笛卡尔积联系起来.

\vspace{2ex}
\centerline{\heiti \Large 习题}

\vspace{2ex}
When language has been well chosen, one is astonished to find that all demonstrations made for a known object apply immediately to many new objects: nothing requires to be changed, not even the terms, since then names have become the same.
\begin{flushright}
    ——H. Poincaré
\end{flushright}

\centerline{\heiti A组}
\begin{enumerate}
    \item 设$V=\{(a_1,a_2,a_3,a_4) \mid a_1+a_2+a_3+a_4=0\}$,$W=\{(a_1,a_2,a_3,a_4) \mid a_1-a_2-a_3+a_4=0,a_1+a_2+a_3-a_4=0\}$.
          \begin{enumerate}
              \item 证明:$V$和$W$为$\mathbf{R}^4$的子空间;

              \item 分别求$V \cap W$,$V+W$以及$W$的补空间的维数与一组基.
          \end{enumerate}

    \item 设 $f_1=-1+x,\ f_2=1-x^2,\ f_3=1-x^3,\ g_1=x-x^2,\ g_2=x+x^3,\ V_1=\spa(f_1,f_2,f_3),\ V_2=\spa(g_1,g_2)$,求:
          \begin{enumerate}
              \item $V_1+V_2$ 的基和维数;

              \item $V_1 \cap V_2$ 的基和维数;

              \item $V_2$ 在 $\mathbf{R}[x]_4$ 空间的补.
          \end{enumerate}

    \item 在数域$\mathbf{F}$上,已知$V_1,V_2$分别为方程组$x_1+x_2+\cdots+x_n=0$与$x_1=x_2=\cdots=x_n$的解空间. 证明:$\mathbf{F}^n=V_1\oplus V_2$.
\end{enumerate}

\centerline{\heiti B组}
\begin{enumerate}
    \item 已知$V_1$是线性方程组\[\begin{cases}
                  3x_1+4x_2-5x_3+7x_4=0 \\
                  4x_1+11x_2-13x_3+16x_4=0
              \end{cases}\]
          的解空间,$V_2$是线性方程组\[\begin{cases}
                  2x_1-3x_2+3x_3-2x_4=0 \\
                  7x_1-2x_2+x_3+3x_4=0
              \end{cases}\]
          的解空间,分别求$V_1 \cap V_2$与$V_1+V_2$的基和维数.

    \item 设$W_1,W_2$是线性空间$V(\mathbf{F})$的两个子空间. 证明以下命题等价:
          \begin{enumerate}
              \item $W_1 \cup W_2$为$V$ 的子空间;

              \item $W_1 \subseteq W_2$或$W_2 \subseteq W_1$;

              \item $W_1 \cup W_2=W_1+W_2$.
          \end{enumerate}

    \item 设\[W_1=\left\{\begin{pmatrix}
                  x & -x \\ y & z
              \end{pmatrix} \,\middle|\, x,y,z\in \mathbf{F} \right\},W_2=\left\{\begin{pmatrix}
                  a & b \\ -a & c
              \end{pmatrix} \,\middle|\, a,b,c\in \mathbf{F} \right\}.\]

          \begin{enumerate}
              \item 证明:$W_1,W_2$是$\mathbf{M}_2(\mathbf{F})$的子空间,并求$\dim W_1,\dim W_2,\dim(W_1+W_2),\dim(W_1\cap W_2)$;

              \item 求$W_1\cap W_2$的一组基,并求$A=\begin{pmatrix}
                            3 & -3 \\ -3 & 1
                        \end{pmatrix}$关于这组基的坐标.
          \end{enumerate}

    \item 设$V$是域$\mathbf{F}$上的$n$维线性空间,$\alpha_1,\alpha_2,\ldots,\alpha_n$是$V$的一组基,且
          \begin{gather*}
              V_1=\spa(\alpha_1+2\alpha_2+\cdots+n\alpha_n) \\
              V_2=\left\{k_1\alpha_1+k_2\alpha_2+\cdots+k_n\alpha_n \,\middle|\, k_1+\dfrac{k_2}{2}+\cdots+\dfrac{k_n}{n}=0\right\}
          \end{gather*}
          证明:
          \begin{enumerate}
              \item $V_2$是$V$的子空间;

              \item $V=V_1\oplus V_2$.
          \end{enumerate}

    \item 设$\mathbf{F}$为数域,$V_1=\{A\in\mathbf{F}^{n\times n} \mid A^\mathrm{T}=A\},\enspace
              V_2=\{A\in\mathbf{F}^{n\times n} \mid A^\mathrm{T}=-A\},\enspace V_3=\{A\in\mathbf{F}^{n\times n} \mid A\text{~为上三角矩阵}\}$.
          \begin{enumerate}
              \item 证明:$V_1,V_2,V_3$都是$\mathbf{F}^{n\times n}$的子空间;

              \item 证明:$\mathbf{F}^{n\times n}=V_1+V_3$但不为直和,$\mathbf{F}^{n\times n}=V_2\oplus V_3$.
          \end{enumerate}

    \item 已知$V_1,V_2$是有限维线性空间$V$的子空间,且$\dim(V_1+V_2)=\dim(V_1 \cap V_2)+1$. 证明:要么$V_1 \subseteq V_2$,要么$V_2 \subseteq V_1$.

    \item 证明:和$\sum\limits_{i=1}^{s}V_i$为直和的充要条件是$V_i \cap \sum\limits_{j=1}^{i-1}V_j=\{0\},\enspace i=1,2,\ldots,s$.

    \item 判断下列说法是否正确:
          \begin{enumerate}
              \item 若$V \subseteq V_1 \cup V_2 \cup \cdots \cup V_s$,则$V=(V_1 \cap V)\cup(V_2 \cap V)\cup\cdots\cup(V_s \cap V)$;

              \item 若$V \subseteq V_1+V_2+\cdots +V_s$,则$V=(V_1 \cap V)+(V_2 \cap V)+\cdots+(V_s \cap V)$.
          \end{enumerate}

    \item 设$V$为有限维线性空间,$V_1$为其非零子空间. 证明:存在唯一的子空间$V_2$,使得$V=V_1\oplus V_2$的充要条件为$V_1=V$.
\end{enumerate}

\centerline{\heiti C组}
\begin{enumerate}
    \item 设$V$是域$\mathbf{F}$上的$n$阶对称矩阵关于矩阵加法和数乘运算构成的线性空间,令
          \[U=\{A\in V \mid \tr(A)=0\},\enspace W=\{\lambda E \mid \lambda\in\mathbf{F}\}.\]
          \begin{enumerate}
              \item 证明:$U,W$为$V$的子空间;

              \item 分别求$U,W$的一组基和维数;

              \item 证明:$V=U\oplus W$.
          \end{enumerate}

    \item 设$W_0,W_1,W_2,\ldots,W_s$是线性空间$V$的$s+1$个非平凡子空间,且$W_0 \subseteq W_1 \cup W_2 \cup \cdots \cup W_s$. 证明:必存在$i$使得$W_0\subseteq W_i$.
\end{enumerate}

\chapter{线性映射}

在前几讲的学习中,我们从开始的 8 条运算性质出发,利用这些线性运算的特点导出线性扩张与子空间的关联,然后经过线性相关性的讨论最终得到线性空间的本质结构实际上就是可以由基经过一系列线性运算扩张而来,因此我们对线性空间的研究很多时候只需要研究其基和维数即可,由此我们对线性空间的研究和描述就可以转为研究基和维数——这是线性空间的基本结构属性. 当然我们最后也讨论了线性空间之间的运算. 从本讲开始我们将研究不同线性空间之间的关联,我们的手段是定义两个线性空间之间的线性映射,由此发掘出比较不同线性空间之间最本质的差别是什么,使我们的抽象更深一层,然后在抽象的制高点将抽象转化为具象,讨论矩阵这一对线性映射的``有形''描述和线性映射本身的联系,为后文详细讨论矩阵作铺垫.

\section{线性映射的定义}

\subsection{线性映射的定义}

\begin{definition}{线性映射}{线性映射的定义} \index{xianxingyingshe@线性映射 (linear map)}
    从线性空间$V_1(\mathbf{F})$到$V_2(\mathbf{F})$的一个映射$\sigma$是线性的,如果$\forall \alpha,\beta \in V_1$和$\forall \lambda,\mu \in \mathbf{F}$都有
    \begin{equation}\label{eq:5:线性映射}
        \sigma(\lambda\alpha+\mu\beta)=\lambda\sigma(\alpha)+\mu\sigma(\beta).
    \end{equation}

    从线性空间$V$到自身的线性映射$\sigma$也叫作$V$上的\term{线性变换},在有的教材中也称为\term{算子}\index{xianxingyingshe!xianxingbianhuan@线性变换 (linear transformation), 算子 (operator)}. 从线性空间$V(\mathbf{F})$到域$\mathbf{F}$的线性映射$f$叫作$V$上的线性泛函(或称线性函数,线性形式)\index{xianxingyingshe!xianxingfanhan@线性泛函 (linear functional)}.

    为方便称呼,我们称对于$V_1(\mathbf{F})$到$V_2(\mathbf{F})$的线性映射$\sigma$,$V_1(\mathbf{F})$是其出发空间,$V_2(\mathbf{F})$是其到达空间,也可简记为$\sigma: V_1\to V_2$.
\end{definition}
实际上,上述定义式 \ref*{eq:5:线性映射} 可以分拆为以下二式:
\begin{gather}
    \tag{加性} \sigma(\alpha+\beta)=\sigma(\alpha)+\sigma(\beta) \\
    \tag{齐次性} \sigma(\lambda\alpha)=\lambda\sigma(\alpha)
\end{gather}

根据定义,我们容易知道熟悉的过原点的一次函数是线性映射,而不过原点的一次函数不代表线性映射. 这似乎与平常的称呼不同,因为一次函数我们经常都称它们为``线性的'',这里我们必须强调的是,至少在线性代数的框架下,我们研究的``线性''性质是包含加性和齐次性两条要求的,事实上不过原点的一次函数我们可以视作非齐次线性方程,这里的``非齐次''的含义便很清晰了.

另一方面,如果我们不将一次函数视为映射,而将视为平面点集(不过原点的一条直线),我们可以回顾线性子空间中的描述,我们说过原点的直线构成平面的线性子空间,但不过原点的直线不是,我们判断的依据是不过原点的直线内的两点关于加法和数乘不封闭——仔细一想,是不是与线性映射定义中不满足加性、齐次性是同样的道理呢?

最后,我们需要强调一个非常容易被忽视但却很重要的一点. 事实上我们在\autoref{ex:运算与同构} 中关于$\mathbf{R}$与$\mathbf{R}^+$同构的讨论,以及坐标同构的讨论中已经引出了这一点,即线性映射$\sigma:V_1\to V_2$满足的表达式
\[\sigma(\alpha+\beta)=\sigma(\alpha)+\sigma(\beta)\]
中左边的加法是出发空间$V_1$中定义的加法运算,右边是到达空间$V_2$中定义的加法运算,二者并不是同一个加法,数乘也类似. %TODO: 同构

本小节最后我们讨论线性映射的两个重要的性质:
\begin{theorem}{}{线性映射零元性质}
    设$\sigma$是线性空间$V_1(\mathbf{F})$到$V_2(\mathbf{F})$的线性映射,则$\sigma(0_1)=0_2$.
\end{theorem}
注意定理中$0_1$为出发空间$V_1$中的零元,$0_2$代表到达空间$V_2$中的零元. 这只是为了区分两个空间零元而引入的记号,实际上下标也可以省略,直接写为0也可以.

\begin{proof}
    根据线性性,$\sigma(0+0)=\sigma(0)+\sigma(0)$,两边同时减去$\sigma(0)$可知$\sigma(0)=0$.
\end{proof}

\begin{theorem}{}{线性映射保相关性}
    设$\sigma$是线性空间$V_1(\mathbf{F})$到$V_2(\mathbf{F})$的线性映射,如果$V_1$中向量$\alpha_1,\alpha_2,\ldots,\alpha_n$线性相关,则$\sigma(\alpha_1),\sigma(\alpha_2),\ldots,\sigma(\alpha_n)$也线性相关.

    反之,若$\sigma(\alpha_1),\sigma(\alpha_2),\ldots,\sigma(\alpha_n)$线性无关,则$\alpha_1,\alpha_2,\ldots,\alpha_n$必线性无关.
\end{theorem}
这一性质表明线性映射保持线性相关性. 定理中两个描述互为逆否命题,因此我们可以只证明前者.

\begin{proof}
    设$\alpha_1,\alpha_2,\ldots,\alpha_n$线性相关,则存在不全为0的数$\lambda_1,\lambda_2,\ldots,\lambda_n$使得$\lambda_1\alpha_1+\lambda_2\alpha_2+\cdots+\lambda_n\alpha_n=0$,于是
    \[\sigma(\lambda_1\alpha_1+\lambda_2\alpha_2+\cdots+\lambda_n\alpha_n)=\lambda_1\sigma(\alpha_1)+\lambda_2\sigma(\alpha_2)+\cdots+\lambda_n\sigma(\alpha_n)=0\]
    因此存在不全为零的数$\lambda_1,\lambda_2,\ldots,\lambda_n$使得$\lambda_1\sigma(\alpha_1)+\lambda_2\sigma(\alpha_2)+\cdots+\lambda_n\sigma(\alpha_n)=0$,因此$\sigma(\alpha_1),\sigma(\alpha_2),\ldots,\sigma(\alpha_n)$线性相关.
\end{proof}

需要注意的是,线性映射可能将线性无关的向量组映射为线性相关的向量组,例如
\begin{example}{}{}
    设$\sigma$是线性空间$\mathbf{R}^2$到$\mathbf{R}^2$的线性映射,定义$\sigma(x,y)=(x+y,x+y)$,则$\sigma$将线性无关的向量组$(1,0),(0,1)$映射为线性相关的向量组$(1,1),(1,1)$.
\end{example}

\subsection{线性映射举例}

事实上,``线性性''在数学中是一个非常基本的性质,我们首先来看以下几个在数学中非常基本的概念,它们都是线性映射的例子:
\begin{example}{}{}
    数学分析与概率论中的线性映射:
    \begin{enumerate}
        \item (极限) $\displaystyle\lim_{n\to +\infty}(\lambda a_n+\mu b_n)=\lambda\lim_{n\to +\infty}a_n+\mu\displaystyle\lim_{n\to +\infty}b_n$;

        \item (求导) $(\lambda f(x)+\mu g(x))'=\lambda f'(x)+\mu g'(x)$;

        \item (积分) $\displaystyle\int_a^b(\lambda f(x)+\mu g(x))\,\mathrm{d}x=\lambda\int_a^bf(x)\,\mathrm{d}x+\mu\displaystyle\int_a^bg(x)\,\mathrm{d}x$;

        \item (数学期望) $\mathrm{E}(\lambda X+\mu Y)=\lambda \mathrm{E}(X)+\mu \mathrm{E}(Y)$.
    \end{enumerate}
\end{example}

有的读者可能会有另外的疑惑:上面的例子为什么能称其为线性映射?它们是从线性空间到线性空间的映射吗?事实上,上面的例子中到达空间都是数域——这是符合定义的,出发空间对于极限而言就是任意有极限的数列构成的线性空间,对于求导、求积分而言就是任意可导、可积的函数构成的线性空间,对于数学期望而言就是期望存在的随机变量构成的线性空间. 读者可以自行验证这些的确构成线性空间,此处不再赘述.

相信看到这里,我们便能逐渐理解线性性在数学中的基础地位. 很多时候一些初看有些抽象的概念,当我们将其与学过的知识联系时,便会真切地体会到一种相通的美感. 事实上很多时候学习过程就是如此,当我们知识储量不断上升的时候,我们会不断发现很多宝贵的思想跨越学科,凝聚着人类智慧的结晶,这种感觉是非常美妙的. 更重要的是,一旦我们知道它们是线性映射之后,我们便可以用之后我们将要讨论的所有线性映射相关的性质来研究它们,这便是一个抽象的概念给我们带来的力量.

接下来希望读者阅读教材3.1节例1--9了解其它常见的线性映射,特别是其中的几何意义(虽然不会直接考察,但是对理解有帮助). 其中例1、7、8、9希望读者当做练习,熟悉线性映射定义的验证,这在考试中也是常见的. 例4、5中的放缩与错切是常见的几何变换,例2中旋转变换在之后有很多的应用场景,其几何意义能帮助我们理解很多内容. 例3镜面变换在内积空间中会详细介绍,例6投影变换将在幂等矩阵中我们会再次提及.
\begin{example}{}{}
    写出下列映射的出发空间和到达空间,并判断其是否为线性映射:
    \begin{enumerate}
        \item $\sigma(x_1,x_2)=(x_1-x_2,x_1,x_1+x_2)$;

        \item $\sigma(x_1,x_2)=(x_1x_2,x_1+x_2)$;

        \item $\sigma(p(x))=p(x+1)-p(x),\enspace\forall p(x) \in \mathbf{R}[x]_n$;

        \item $\sigma(p(x))=p(a),\enspace\forall p(x)$,其中$a$为常数;

        \item $\sigma(\xi)=2\xi+\xi_0$,其中$\xi_0$是线性空间$V$中的一个固定向量.
    \end{enumerate}
\end{example}

\begin{solution}
    \begin{enumerate}
        \item 出发空间为 $ \mathbf{R}^2 $,到达空间为 $ \mathbf{R}^3 $. $ \sigma $ 是线性映射.

              $ \forall (x_1, x_2), (y_1, y_2) \in \mathbf{R}^2,\enspace k_1, k_2 \in \mathbf{R} $, 有
              \begin{align*}
                      & \sigma(k_1(x_1, x_2) + k_2(y_1, y_2))                                                                     \\
                  ={} & ((k_1 x_1 + k_2 y_1) - (k_1 x_2 + k_2 y_2), k_1 x_1 + k_2 y_1, (k_1 x_1 + k_2 y_1) + (k_1 x_2 + k_2 y_2)) \\
                  ={} & k_1(x_1 - x_2, x_1, x_1 + x_2) + k_2(y_1 - y_2, y_1, y_1 + y_2)                                           \\
                  ={} & k_1 \sigma(x_1, x_2) + k_2 \sigma(y_1, y_2)
              \end{align*}

        \item 出发空间为 $ \mathbf{R}^2 $,到达空间为 $ \mathbf{R}^2 $. $ \sigma $ 不是线性映射.

              $ \forall (x_1, x_2), (y_1, y_2) \in \mathbf{R}^2,\enspace k_1, k_2 \in \mathbf{R} $, 有
              \begin{align*}
                  \sigma((x_1, x_2) + (y_1, y_2))
                   & = \sigma(x_1 + y_1, x_2 + y_2)                                        \\
                   & = ((x_1 + y_1)(x_2 + y_2), ((x_1 + y_1) + (x_2 + y_2))                \\
                   & = (x_1 x_2 + x_1 y_2 + y_1 x_2 + y_1 y_2, x_1 + y_1 + x_2 + y_2)      \\
                   & = (x_1 x_2 + y_1 y_2, x_1 + y_1 + x_2 + y_2) + (x_1 y_2 + y_1 x_2, 0) \\
                   & = \sigma(x_1, x_2) + \sigma(y_1, y_2) + (x_1 y_2 + y_1 x_2, 0)        \\
                   & \neq \sigma(x_1, x_2) + \sigma(y_1, y_2)
              \end{align*}

        \item 出发空间为 $ \mathbf{R}[x]_n $,到达空间为 $ \mathbf{R}[x]_{n - 1} $. $ \sigma $ 是线性映射.

              $ \forall p_1(x), p_2(x) \in \mathbf{R}[x]_n,\enspace k_1, k_2 \in \mathbf{R} $, 有
              \begin{align*}
                      & \sigma(k_1 p_1(x) + k_2 p_2(x))                     \\
                  ={} & (k_1 p_1 + k_2 p_2)(x + 1) - (k_1 p_1 + k_2 p_2)(x) \\
                  ={} & k_1(p_1(x + 1) - p_1(x)) + k_2(p_2(x + 1) - p_2(x)) \\
                  ={} & k_1 \sigma(p_1(x)) + k_2 \sigma(p_2(x))
              \end{align*}

        \item 出发空间为 $ \mathbf{R}[x] $,到达空间为 $ \mathbf{R}[x] $. $ \sigma $ 是线性映射.

              $ \forall p_1(x), p_2(x) \in \mathbf{R}[x],\enspace k_1, k_2 \in \mathbf{R} $, 有
              \begin{align*}
                      & \sigma(k_1 p_1(x) + k_2 p_2(x))         \\
                  ={} & (k_1 p_1 + k_2 p_2)(a)                  \\
                  ={} & k_1 p_1(a) + k_2 p_2(a)                 \\
                  ={} & k_1 \sigma(p_1(x)) + k_2 \sigma(p_2(x))
              \end{align*}

        \item 出发空间为 $ V $,到达空间为 $ V $. 当 $ \xi_0 = \vec{0} $ 时, $ \sigma(\xi) = 2 \xi $ 是线性映射.

              $ \forall \xi_1, \xi_2 \in V $, 有
              \begin{align*}
                  \sigma(\xi_1 + \xi_2) & = 2(\xi_1 + \xi_2)              \\
                                        & = 2 \xi_1 + 2 \xi_2             \\
                                        & = \sigma(\xi_1) + \sigma(\xi_2)
              \end{align*}

              当 $ \xi_0 \neq \vec{0} $ 时, $ \sigma $ 不是线性映射.

              $ \forall \xi_1, \xi_2 \in V $, 有
              \begin{align*}
                  \sigma(\xi_1 + \xi_2) & = 2(\xi_1 + \xi_2) + \xi_0              \\
                                        & = 2 \xi_1 + 2 \xi_2 + \xi_0             \\
                                        & = \sigma(\xi_1) + \sigma(\xi_2) - \xi_0 \\
                                        & \neq \sigma(\xi_1) + \sigma(\xi_2)
              \end{align*}
    \end{enumerate}
\end{solution}

\subsection{线性映射的基本运算}

我们在之前的学习中已经了解,连续函数关于函数的加法数乘运算可以构成线性空间,事实上线性映射可以视为特殊的函数,因此我们希望在本节讨论怎样的运算定义能使其构成线性空间,除此之外也介绍线性映射乘法(即复合)和逆运算.

我们需要首先说明一个记号,我们把线性空间$V_1(\mathbf{F})$到$V_2(\mathbf{F})$的所有线性映射组成的集合记作$\mathcal{L}(V_1,V_2)$(类似于将定义在$[a,b]$上取值于实数集的连续函数全体记为$C[a,b]$). 如果是出发空间与到达空间均为$V$的线性变换全体,我们可以简记为$\mathcal{L}(V)$. 我们希望在该集合上定义线性空间,于是需要定义其中元素(线性映射)的加法和数乘运算:
\begin{definition}{}{}
    设$\sigma,\tau\in \mathcal{L}(V_1,V_2)$,规定$\sigma$与$\tau$之和及$\lambda$与$\sigma$的数乘$\lambda\sigma$分别为
    \begin{gather*}
        (\sigma+\tau)(\alpha)=\sigma(\alpha)+\tau(\alpha),\enspace\forall\alpha\in V_1 \\
        (\lambda\sigma)(\alpha)=\lambda(\sigma(\alpha)),\enspace\forall\alpha\in V_1
    \end{gather*}
\end{definition}

\begin{theorem}{}{线性映射全体构成线性空间}
    $\mathcal{L}(V_1,V_2)$与上述定义的线性映射加法和数乘构成域$\mathbf{F}$上的线性空间.
\end{theorem}

下面讨论线性映射的复合. 设$\sigma \in \mathcal{L}(V_1,V_2),\enspace\tau \in \mathcal{L}(V_2,V_3)$,则$\tau\sigma$是$\mathcal{L}(V_1,V_3)$中的元素,且$\tau\sigma(\alpha)=\tau(\sigma(\alpha)),\enspace\forall \alpha \in V_1$.
\begin{theorem}{}{}
    上述定义的映射$\tau\sigma$是线性映射.
\end{theorem}
注意:在上述定义中一定注意$\sigma$和$\tau$的顺序,我们需要先使用$\sigma$将$V_1$中的元素映射到$V_2$,然后再用外层的$\tau$将这个结果映射到$V_3$.

下面定义逆映射. 如果可逆映射$\sigma:V_1 \to V_2$的逆映射为$\sigma^{-1}$,则有$\sigma^{-1}\sigma=I_{V_1}$且$\sigma\sigma^{-1}=I_{V_2}$. 其中$I_{V}$的含义为$V$上的恒等映射,即$I_V(\alpha)=\alpha,\enspace \forall \alpha \in V$.
\begin{theorem}{}{}
    上述定义的逆映射$\sigma^{-1}$为线性映射.
\end{theorem}

上述三个定理的证明是非常基本的,详细的证明在教材103--104页,读者可以先自行尝试,如果不会证明则说明对于线性空间和线性映射的定义熟悉程度仍需提高,因为这里的证明都只需要机械地套用定义.

\section{线性映射的像与核}

我们在之前的讨论中已经了解了线性映射的定义与运算,接下来我们需要关心的问题是:定义出的线性映射能将出发空间完整映射到到达空间吗,还是到达空间中有些向量无法被映到?线性映射是否可以是单射?单射的充要条件又是什么?这与我们研究一般的映射的思路是类似的. 因此我们希望在本节讨论线性映射的像和核.
\begin{definition}{}{}
    设$\sigma$是线性空间$V_1(\mathbf{F})$到$V_2(\mathbf{F})$的线性映射. $V_1$的所有元素在$\sigma$下的像组成的集合
    \[\sigma(V_1)=\{\beta \mid \beta=\sigma(\alpha),\enspace \alpha \in V_1\}\]
    称为$\sigma$的\term{像}(或\term{值域})\index{xiang@像 (image), 值域 (range)},记作$\im \sigma$,或记作 $\operatorname{range}\sigma$.

    $V_2$的零元$0_2$在$\sigma$下的完全原像
    \[\sigma^{-1}(0_2)=\{\alpha \mid \sigma(\alpha)=0_2,\enspace \alpha \in V_1\}\]
    称为$\sigma$的\term{核}(或\term{零空间})\index{he@核 (kernel), 零空间 (null space)},记作$\ker \sigma$,或记作 $\operatorname{null}\sigma$.
\end{definition}

关于像与核的定义,我们需要强调以下几点:
\begin{enumerate}
    \item 实际上,像空间的定义就类似于函数的值域,核空间可以视为到达空间中0的原像集合,因此理解起来是很简单的;

    \item 注意线性映射的像和核分别是$V_2$和$V_1$的子空间. 同样地,若$W_1$和$W_2$分别是$V_1$和$V_2$的子空间,则$\sigma(W_1)$和$\sigma^{-1}(W_2)$也分别是$V_2$和$V_1$的子空间. 前者证明见教材102页,后者我们作为习题留给读者,实际上都非常简单,只是为读者熟悉定义而在此处提及.
\end{enumerate}

接下来我们要讨论如何计算线性映射的像与核,这在考试中非常常见,请务必牢记,无论线性映射有多么复杂多么抽象,基本的方法都是:
\begin{enumerate}
    \item 设出发空间的一组基为$B=\{\alpha_1,\alpha_2,\ldots,\alpha_n\}$,则像空间
          \[\im \sigma=\sigma(V_1)=\spa(\sigma(\alpha_1),\sigma(\alpha_2),\ldots,\sigma(\alpha_n)).\]
          即线性映射在出发空间一组基下的像的线性扩张,解答时写出极大线性无关组然后扩张即可;

          当然读者可能质疑其合理性,因为与定义不完全一致. 我们可以证明这一方法是合理的,即线性映射在出发空间一组基下像的线性扩张就是其像空间.

          \begin{proof}
              首先我们知道$\sigma(V_1)$包含$\sigma(\alpha_1),\sigma(\alpha_2),\ldots,\sigma(\alpha_n)$,并且是$V_2$的子空间. 又根据\autoref{thm:线性扩张构造子空间},$\sigma(V_1)$是包含$\sigma(\alpha_1),\sigma(\alpha_2),\ldots,\sigma(\alpha_n)$的最小子空间,因此我们可以得到$\spa(\sigma(\alpha_1),\sigma(\alpha_2),\ldots,\sigma(\alpha_n))\subseteq\sigma(V_1)$.

              接下来证明另一半包含. 根据线性扩张定义可知只需证$V_1$中任意元素的像都可以被$\sigma(\alpha_1),\sigma(\alpha_2),\ldots,\sigma(\alpha_n)$线性表示. 任取$\alpha\in V_1$,则$\alpha$可由$V_1$一组基$\{\alpha_1,\alpha_2,\ldots,\alpha_n\}$线性表示为$\alpha=\lambda_1\alpha_1+\lambda_2\alpha_2+\cdots+\lambda_n\alpha_n$,于是,
              \[\sigma(\alpha)=\sigma(\lambda_1\alpha_1+\lambda_2\alpha_2+\cdots+\lambda_n\alpha_n)=\lambda_1\sigma(\alpha_1)+\lambda_2\sigma(\alpha_2)+\cdots+\lambda_n\sigma(\alpha_n)\]
              即$\sigma(\alpha)$可由$\sigma(\alpha_1),\sigma(\alpha_2),\ldots,\sigma(\alpha_n)$线性表示,即出发空间任意向量在$\sigma$下的像都可以由$\sigma(\alpha_1),\sigma(\alpha_2),\ldots,\sigma(\alpha_n)$线性表示,因此$\sigma(V_1)\subseteq\spa(\sigma(\alpha_1),\sigma(\alpha_2),\ldots,\sigma(\alpha_n))$.
          \end{proof}

    \item 核空间可以直接利用定义令$\sigma(\alpha)=0$,利用解线性方程组得到解集即为结果,注意也许表示为线性扩张的形式.
\end{enumerate}

\begin{example}{}{}
    已知$\mathbf{R}^3$到$\mathbf{R}^2$的映射$\sigma$为$\sigma(x_1,x_2,x_3)=(x_1+x_2,x_2-x_3)$,求$\sigma$的像和核.
\end{example}

\begin{solution}
    \begin{itemize}
        \item 首先求像空间. 取出发空间$\mathbf{R}^3$的一组基$B=\{(1,0,0),(0,1,0),(0,0,1)\}$,则$\im \sigma=\sigma(\mathbf{R}^3)=\spa(\sigma(1,0,0),\sigma(0,1,0),\sigma(0,0,1))=\spa((1,0),(1,1),(0,-1))$. 根据求解极大线性无关组的方法(或者这么简单的情况瞪眼法也可以)得到像空间$\im \sigma=\spa((1,0),(0,-1))=\mathbf{R}^2$.

        \item 接下来求解核空间. 设$\sigma(\alpha)=0$,其中$\alpha=(x_1,x_2,x_3)$,即$\sigma(x_1,x_2,x_3)=(x_1+x_2,x_2-x_3)=(0,0)$,解得解向量为$k(-1,1,1),\enspace k\in\mathbf{R}$,写成线性扩张的形式为$\spa((-1,1,1))$.
    \end{itemize}
\end{solution}

事实上教材102页例3给出了另一种求像空间的方法,但是为了防止读者混淆这一方法和之后线性映射矩阵表示的方法,希望读者能按照笔者介绍的方法求解.

事实上,研究一般映射我们会很在意映射是否为单射或满射. 是否为满射通过我们介绍的求解像空间的方法是很好判断的,但单射似乎并不能直接利用像空间或者核空间直接判断,但我们只需稍作转化就可以发现单射和核空间有着密不可分的联系:
\begin{theorem}{}{单射与核空间}
    线性映射$\sigma$是单射当且仅当$\ker \sigma=\{0\}$.
\end{theorem}
这个定理告诉我们,线性映射是单射和0的逆像只有0是等价的. 这一结论也是非常强的,因为我们知道一般的函数是不满足这一结论的,例如$f(x)=x^2$,虽然只有$f(0)=0$,但在$\mathbf{R}$上显然不是单射. 这一定理证明非常简单,希望读者掌握:

\begin{proof}
    首先我们证明$\sigma$是单射时$\ker\sigma=\{0\}$. 事实上$\sigma$是单射意味着任意到达空间中的元素的逆象唯一,又线性映射必须满足$\sigma(0)=0$,则0的逆象唯一为0是显然的.

    接下来我们证明$\ker \sigma=\{0\}$时$\sigma$是单射. 事实上,$\sigma(\alpha)=\sigma(\beta)$等价于$\sigma(\alpha)-\sigma(\beta)=0$,即$\sigma(\alpha-\beta)=0$,由于$\ker \sigma=\{0\}$,因此$\alpha-\beta=0$,即$\alpha=\beta$,因此$\sigma$是单射.
\end{proof}

\section{线性映射的确定}

我们知道,两个函数相等当且仅当它们的定义域相等且对于任意定义域内的元素,它们的函数值相等. 线性映射则有更好的性质,即有限维空间上的线性映射可以被基上的像唯一确定,即
\begin{theorem}{}{线性映射唯一确定}
    已知线性映射$\sigma,\tau\in \mathcal{L}(V_1,V_2)$,且有$V_1$的基$B=\{\alpha_1,\alpha_2,\ldots,\alpha_n\}$. 若$\sigma(\alpha_i)=\tau(\alpha_i),\enspace\forall \alpha_i \in B$,则有$\sigma=\tau$.
\end{theorem}
即映射在一组基上的像确定了,则映射是唯一的. 证明非常简单:

\begin{proof}
    实际上我们只需证明$\sigma(\alpha)=\tau(\alpha),\enspace\forall \alpha \in V_1$即可,因为这就是一般映射相等的条件. 事实上,任取$\alpha \in V_1$,则$\alpha$可由$B$线性表示为$\alpha=\lambda_1\alpha_1+\lambda_2\alpha_2+\cdots+\lambda_n\alpha_n$,于是
    \begin{gather*}
        \sigma(\alpha)=\sigma(\lambda_1\alpha_1+\lambda_2\alpha_2+\cdots+\lambda_n\alpha_n)=\lambda_1\sigma(\alpha_1)+\lambda_2\sigma(\alpha_2)+\cdots+\lambda_n\sigma(\alpha_n) \\
        \tau(\alpha)=\tau(\lambda_1\alpha_1+\lambda_2\alpha_2+\cdots+\lambda_n\alpha_n)=\lambda_1\tau(\alpha_1)+\lambda_2\tau(\alpha_2)+\cdots+\lambda_n\tau(\alpha_n)
    \end{gather*}
    由于$\sigma(\alpha_i)=\tau(\alpha_i),\enspace\forall \alpha_i \in B$,因此$\sigma(\alpha)=\tau(\alpha)$,即$\sigma=\tau$.
\end{proof}

事实上这与之前证明求解像空间的方法合理性(即证线性映射在一组基上的像的线性扩张就是线性映射的像空间)是完全相通的. 这里也蕴含了一个数学的基本想法. 我们发现线性映射比一般的函数要求更高,因为它要求了两个运算性质,我们说这里构成线性映射的条件比构成一般函数的条件``更强''. 更强的要求必然带来更美妙的结果,例如线性映射可被基上的像唯一确定,而一般函数则不存在这样的性质. 抑或是未来如果有同学学习复变函数时,那时我们研究的``全纯函数''比数学分析中常见的连续函数要求更强,因此会有大量在数学分析中无法想象的非常漂亮的结果. 值得一提的是,复变函数这样美妙的结果直接带来了\nameref{thm:代数学基本定理}的非常简便的证明,这在数学史上是非常重要的里程碑.

\begin{theorem}{}{线性映射构造}
    设$B=\{\alpha_1,\alpha_2,\ldots,\alpha_n\}$是$V_1$的基,$S=\{\beta_1,\beta_2,\ldots,\beta_n\}$是$V_2$中任意$n$个向量,则存在唯一的$\sigma\in \mathcal{L}(V_1,V_2)$使得$\sigma(\alpha_i)=\beta_i,\enspace i=1,2,\ldots,n$.
\end{theorem}
这一定理即教材107--108页定理3.6,证明也是简单的,只需先定义出这个映射. $\forall \alpha \in V_1$,则$\alpha$可由$B$线性表示为$\alpha=\lambda_1\alpha_1+\lambda_2\alpha_2+\cdots+\lambda_n\alpha_n$,于是定义
\[\sigma(\alpha)=\sigma(\lambda_1\alpha_1+\lambda_2\alpha_2+\cdots+\lambda_n\alpha_n)=\lambda_1\beta_1+\lambda_2\beta_2+\cdots+\lambda_n\beta_n\]
即可满足条件,因为我们可以验证这是线性映射(见教材108页),并且唯一性在\autoref{thm:线性映射唯一确定} 中已经说明. 实际上对于初学而言难点在于定义,实际上证明后会发现这一定义太自然了,就是向着线性性定义的,因此很多构造不需要太复杂的想法,自然的、满足要求的是最好的.

最后我们讨论一个初学时容易困惑的问题,如下例所示:
\begin{example}{}{线性映射判断1}
    是否存在$\mathbf{R}^2$到$\mathbf{R}^3$的线性映射$\sigma$使得$\sigma(1,0)=(1,0,0),\enspace\sigma(0,1)=(0,1,0),\enspace\sigma(1,1)=(0,0,1)$?
\end{example}

初学时感到困难是因为不能熟练应用线性映射的各类性质,找不到映射定义也不敢下结论不存在,或者发现必要条件都满足了却不敢构造. 我们这里给出几个解决策略:
\begin{enumerate}
    \item \label{item:5:线性映射判断1:1}
          根据\autoref{thm:线性映射零元性质} 可知,如果我们发现题目给定的条件无法满足将出发空间零元映射至到达空间零元则一定不是线性映射;

    \item \label{item:5:线性映射判断1:2}
          根据\autoref{thm:线性映射保相关性} 可知,如果我们发现映射将线性相关的向量组映射到了线性无关向量组,则一定不是线性映射;

    \item \label{item:5:线性映射判断1:3}
          一定不存在从低维线性空间到高维线性空间的满射,原因是简单的:我们取低维出发空间的一组基$B=\{\alpha_1,\alpha_2,\ldots,\alpha_m\}$,则它们的像的线性扩张$\spa(\sigma(\alpha_1),\sigma(\alpha_2),\ldots,\sigma(\alpha_m))$就是像空间. 我们取高维到达空间的一组基$B_2=\{\beta_1,\beta_2,\ldots,\beta_n\}$,则由于维数更高有$n>m$. 由于$\sigma$是满射,因此$\sigma(\alpha_1),\sigma(\alpha_2),\ldots,\sigma(\alpha_m)$可以线性表示出$B_2$中任意向量,根据\autoref{thm:线性表示} 可知(这是长的向量可以被短的向量线性表示),向量组$B_2$线性相关,但我们知道这是一组基,因此矛盾!

          这一性质在下一讲介绍了线性映射基本定理后会有更简单的证明,但此处的证明也是很重要的,体现了\autoref{thm:线性表示} 作为源泉定理的重要性.

    \item 如果题目给定的映射不违反上述线性映射的必要条件,那我们可以按照\autoref{thm:线性映射构造} 构造出相应的映射.
\end{enumerate}

根据上面的描述,我们发现 \ref*{item:5:线性映射判断1:1} 中的映射定义违反了明显违反了不能是从低维到高维满射的条件. 实际上,$\sigma$也将线性相关的向量组映射到了线性无关的向量组,并且根据定义,$\sigma(0)=\sigma((1,0)+(0,1)-(1,1))=(1,1,-1)$,因此所有的必要条件都被违反了,因此这一映射一定不是线性映射. 事实上这一例子也表明很多时候三个必要条件可能是同时违反的,因为它们都是由基本的线性映射和线性相关性质推导而来,并非完全独立的判据.

我们还需强调的是,如果\autoref{thm:线性映射构造} 前提成立,即题目给我们的是一组基下的像,则一定不会违反上述三个必要条件. 对于条件 \ref*{ex:线性映射判断1},给定一组基$\alpha_1,\ldots,\alpha_n$,我们要凑出$\sigma(0)$只能通过$\sigma(0)=\sigma(0\alpha_1+\cdots+0\alpha_n)=0\sigma(\alpha_1)+\cdots+0\sigma(\alpha_n)=0$,因此不可能违反  \ref*{item:5:线性映射判断1:1}. 对于 \ref*{item:5:线性映射判断1:2},我们给定的是基,因此不存在将线性相关向量组映射到线性无关向量组的情况. 对于 \ref*{item:5:线性映射判断1:3},如果题目给出的是低维到高维的映射,由于我们只给出了低维出发空间的基下的像,这些像不可能张成整个高维到达空间(原理和$n-1$个向量无法张成$n$维空间一致),因此也不可能违反 \ref*{item:5:线性映射判断1:3},因此\autoref{thm:线性映射构造} 并不与我们的必要条件相矛盾,相反,如果题目给出的是一组基下的像,我们就可以毫无顾虑地说映射一定存在.

最后我们再看一个例子来练习我们上面的策略:
\begin{example}{}{线性映射判断2}
    是否存在$\mathbf{R}^3$到$\mathbf{R}^2$的线性映射$\sigma$使得$\sigma(1,-1,1)=(1,0),\enspace\sigma(1,1,1)=(0,1)$?
\end{example}

\begin{solution}
    事实上这里的定义完全不违反上述的必要条件,因此我们考虑证明这一线性映射存在. 根据\autoref{thm:线性映射构造},我们考虑构造出$\sigma$在一组基(我们取自然基)下的像,然后我们就可以根据\autoref*{thm:线性映射构造} 知道这一映射一定存在.

    实际上,根据提给条件我们有
    \begin{gather*}
        \sigma(1,-1,1)=\sigma(e_1-e_2+e_3)=\sigma(e_1)-\sigma(e_2)+\sigma(e_3)=(1,0) \\
        \sigma(1,1,1)=\sigma(e_1+e_2+e_3)=\sigma(e_1)+\sigma(e_2)+\sigma(e_3)=(0,1)
    \end{gather*}
    我们希望解出$\sigma(e_1),\sigma(e_2),\sigma(e_3)$,这样就可以直接根据\autoref*{thm:线性映射构造} 构造出这一线性映射. 但这一方程组只有2个方程却有三个未知量. 事实上我们可以任意定义$\sigma(e_3)=(0,0)$,然后解方程组得到
    \[\sigma(e_1)=\dfrac{1}{2} (1,1),\enspace \sigma(e_2)=\dfrac{1}{2} (-1,1)\]
    又由$\sigma(e_3)=(0,0)$,根据\autoref*{thm:线性映射构造},满足题目条件的线性映射存在.
\end{solution}
需要注意的是题目中$\sigma(e_3)$不一定要定义为$(0,0)$,这样只是为了计算方便,事实上定义成任何值都可以得到$\sigma$在一组基下的像,从而根据\autoref*{thm:线性映射构造} 得到这一线性映射存在. 如果题目要求我们写出映射也并不复杂,根据我们在\autoref*{thm:线性映射构造} 中的构造方法,我们可以写出$\forall\alpha=(x,y,z)=xe_1+ye_2+ze_3$,
\[\sigma(\alpha)=\sigma(xe_1+ye_2+ze_3)=x\sigma(e_1)+y\sigma(e_2)+z\sigma(e_3)=\dfrac{1}{2} (x-y,x+y)\]
符合题目条件. 事实上根据$\sigma(e_3)$定义的不唯一,我们可以得到不同的线性映射,这里只是给出一种可能的解.

\section{线性映射基本定理}

本节将要介绍的定理是线性代数最重要的定理之一,因其重要性也被冠以(有限维线性空间)线性映射基本定理的名号. 为了介绍这一定理,我们需要首先引入一个概念:线性映射的秩. 在线性映射像求解的讨论中我们有$\im \sigma=\sigma(V_1)=\spa(\sigma(\alpha_1),\sigma(\alpha_2),\ldots,\sigma(\alpha_n))$. 我们基于此定义线性映射的秩:
\begin{definition}{}{}
    设$\sigma\in \mathcal{L}(V_1,V_2)$,如果$\sigma(V_1)$是$V_2$的有限维子空间,则$\sigma(V_1)$的维数称为$\sigma$的秩,记作$r(\sigma)$,即$r(\sigma)=\dim \sigma(V_1)$.
\end{definition}

这一定义是平凡的,简单理解线性映射的秩即为线性映射像空间的维数. 基于这一定理我们便可以介绍本节的核心定理:

\begin{theorem}{线性映射基本定理}{线性映射基本定理}
    设$\sigma \in \mathcal{L}(V_1,V_2)$,若$\dim V_1=n$,则
    \[r(\sigma)+\dim\ker\sigma=n.\]
\end{theorem}
简而言之,这一定理表明:线性映射的秩(或者说线性映射像空间维数)与核空间维数之和等于出发空间的维数. 这一定理的证明非常重要,在之后的很多讨论中还会用到这一思想,因此我们给出详细的证明并阐述其中的思想:

\begin{proof}
    证明的思路和\nameref{thm:线性空间维数公式}的证明思路类似,即``设小扩大''.

    我们设$\dim\ker\sigma=k$,并设$\ker\sigma$的一组基为$\alpha_1,\alpha_2,\ldots,\alpha_k$. 我们将其扩充为$V_1$的一组基,记为$\alpha_1,\alpha_2,\ldots,\alpha_k,\alpha_{k+1},\ldots,\alpha_n$.

    根据定理要证明的等式和前述假设,我们只需证$r(\sigma)=n-k$,即证明像空间维数为$n-k$. 我们知道像空间为$\spa(\sigma(\alpha_1),\sigma(\alpha_2),\ldots,\sigma(\alpha_n))$,其中根据我们的假设,$\sigma(\alpha_1)=\sigma(\alpha_2)=\cdots=\sigma(\alpha_k)=0$(因为它们是核空间的基),因此像空间为$\spa(\sigma(\alpha_{k+1}),\ldots,\sigma(\alpha_n))$. 我们只需证明这一向量组是线性无关的即可,因为这样这$n-k$个向量就可以构成像空间的一组基,从而证明了$r(\sigma)=n-k$.

    我们设$c_{k+1}\sigma(\alpha_{k+1})+\cdots+c_n\sigma(\alpha_n)=0$,即
    \[\sigma(c_{k+1}\alpha_{k+1}+\cdots+c_n\alpha_n)=0\]
    故$c_{k+1}\alpha_{k+1}+\cdots+c_n\alpha_n \in \ker\sigma$,因此可以被$\alpha_1,\alpha_2,\ldots,\alpha_k$线性表示. 于是有
    \[c_{k+1}\alpha_{k+1}+\cdots+c_n\alpha_n=c_1\alpha_1+\cdots+c_k\alpha_k\]
    即
    \[c_1\alpha_1+\cdots+c_k\alpha_k-c_{k+1}\alpha_{k+1}-\cdots-c_n\alpha_n=0\]
    由于$\alpha_1,\alpha_2,\ldots,\alpha_n$是$V_1$的一组基,因此$c_1=\cdots=c_k=c_{k+1}=\cdots=c_n=0$,故$\sigma(\alpha_{k+1}),\ldots,\sigma(\alpha_n)$线性无关,命题得证.
\end{proof}

事实上这一定理也被称为线性映射``维数公式'',但为了与\nameref{thm:线性空间维数公式}区分,本讲义中我们称这一定理为线性映射基本定理. 读者可以比较一下两个``维数公式''的证明,二者都使用了``设小扩大''的思想,都将要证明的结论转化为证明一组向量是线性无关的,但其中证明线性无关的方法略有不同,读者可以仔细体会.

下面我们给出一个证明思想上类似的例子供读者练习:
\begin{example}{}{}
    设$\sigma$为有限维线性空间$V$上的线性变换,$W$是$V$的子空间,证明:
    \[\dim\sigma(W)+\dim(\ker\sigma \cap W)=\dim W.\]
\end{example}

\begin{proof}
    与\nameref{thm:线性映射基本定理}证明类似,我们``设小扩大''. 设$\dim W=n,\enspace\dim\ker\sigma\cap W=k$,设$\ker\sigma\cap W$的一组基为$\alpha_1,\alpha_2,\ldots,\alpha_k$,我们将其扩充为$W$的一组基,记为
    \[\alpha_1,\alpha_2,\ldots,\alpha_k,\alpha_{k+1},\ldots,\alpha_n.\]

    根据定理要证明的等式和前述假设,我们只需证$\dim\sigma(W)=n-k$. 我们知道像空间为$\sigma(W)=\spa(\sigma(\alpha_1),\sigma(\alpha_2),\ldots,\sigma(\alpha_n))$,其中根据我们的假设,$\sigma(\alpha_1)=\sigma(\alpha_2)=\cdots=\sigma(\alpha_k)=0$,因此像空间为$\spa(\sigma(\alpha_{k+1}),\ldots,\sigma(\alpha_n))$. 我们只需证明这一向量组是线性无关的即可,因为这样这$n-k$个向量就可以构成像空间的一组基,从而证明了$\dim\sigma(W)=n-k$.

    我们设
    \[c_{k+1}\sigma(\alpha_{k+1})+\cdots+c_n\sigma(\alpha_n)=0,\]
    即$\sigma(c_{k+1}\alpha_{k+1}+\cdots+c_n\alpha_n)=0$,故$c_{k+1}\alpha_{k+1}+\cdots+c_n\alpha_n\in\ker\sigma$,因此可以被$\alpha_1,\alpha_2,\ldots,\alpha_k$线性表示. 于是有
    \[c_{k+1}\alpha_{k+1}+\cdots+c_n\alpha_n=c_1\alpha_1+\cdots+c_k\alpha_k,\]
    即
    \[c_1\alpha_1+\cdots+c_k\alpha_k-c_{k+1}\alpha_{k+1}-\cdots-c_n\alpha_n=0,\]
    由于$\alpha_1,\alpha_2,\ldots,\alpha_n$是$W$的一组基,因此$c_1=\cdots=c_k=c_{k+1}=\cdots=c_n=0$,故$\sigma(\alpha_{k+1}),\ldots,\sigma(\alpha_n)$线性无关,命题得证.
\end{proof}

基于线性映射基本定理,我们可以得到如下定理:
\begin{theorem}{}{双射等价条件}
    对$\sigma \in \mathcal{L}(V_1,V_2)$且$\dim V_1=\dim V_2=n$,下列条件等价:
    \begin{enumerate}
        \item \label{item:6:双射等价条件:1}
              $\ker \sigma=\{0\}$;

        \item \label{item:6:双射等价条件:2}
              $\sigma$为单射;

        \item $\sigma$为满射;

        \item $\sigma$为双射(可逆);

        \item $r(\sigma)=n$.
    \end{enumerate}
\end{theorem}

我们需要注意的是,上述 \ref*{item:6:双射等价条件:1} 与 \ref*{item:6:双射等价条件:2} 等价不是基于线性映射基本定理得到的,而是在前述\autoref{thm:单射与核空间} 中已经证明的. 其余等价性的证明也是非常简单,只需要简单套用维数公式即可.

实际上,线性映射基本定理还隐藏着一个我们之前以及介绍过的结论,即不可能存在从低维空间到高维空间的满射. 利用反证法,假设存在这样的映射$\sigma:V_1\to V_2$,则核空间维数$\dim\ker\sigma=n-r(\sigma)=\dim V_1-\dim V_2<0$,这显然是不合理的. 当然这一结论有一对称形式也成立,即不存在高维空间到低维空间的单射,证明类似,不再赘述.

还需注意的是,这一定理前提要求是有限维空间上的线性变换,因为我们可以给出如下例子:

\begin{example}{}{}
    设$V$是全体定义在实数域,取值于实数域的可微函数关于一般的函数加法和数乘构成的线性空间,$\sigma:V\to V$定义为$\sigma(f)=f'$,即求导变换,则$\sigma$是线性变换,且显然$\sigma$是满射(因为任意连续函数$g$一定黎曼可积,所以一定能在$V$中找到原函数使得原函数的导数为$g$),但$\sigma$不是单射,例如对于$g(x)=2x$有$f(x)=x^2+C$($C$为任意常数)都可以有$f'(x)=g(x)$. 因此我们发现这里定义在无限维线性空间$V$中的线性变换使得上面的定理中单射满射不等价.
\end{example}

\section{像与核的进一步讨论}

关于线性变换的像和核有很多的包含关系或等式等结论,实际上很多问题都来源于线性映射基本定理及其推论,本节我们主要探讨这一话题.

我们首先说明几个重要的原则:
\begin{enumerate}
    \item 解决此类问题大多需要综合利用维数公式及其推论,需要将题给条件转化为合适的等价表述然后解决;

    \item 注意集合相等的证明方式,实际上就是两个集合互相包含. 实际上很多时候一边的包含是显然的,只需证明另一边;

    \item 时刻注意线性映射的像和核的定义,线性空间的交、和与直和的概念,例如看到像需要想到其存在原像,看到和与直和要想到将向量分拆等.
\end{enumerate}

接下来我们看一些经典的结论(已知$V$为有限维线性空间,$\sigma\in \mathcal{L}(V,V)$),其中结论 \ref*{item:6:像与核的进一步讨论:1} 最为常见:

\begin{enumerate}
    \item \label{item:6:像与核的进一步讨论:1}
          若$\sigma$为幂等变换(即$\sigma^2=\sigma$)有$V=\ker\sigma\oplus\im \sigma$;

          \begin{proof}
              回忆直和的证明方法,我们这里利用先证明和为直和(即交为$\{0\}$)再证等号成立的方法. 设$\alpha\in\ker\sigma\cap\im \sigma$,则$\sigma(\alpha)=0$,且存在$\beta\in V$使得$\sigma(\beta)=\alpha$,因此利用$\sigma^2=\sigma$有
              \[0=\sigma(\alpha)=\sigma(\sigma(\beta))=\sigma^2(\beta)=\sigma(\beta)=\alpha,\]
              即$\ker\sigma\cap\im \sigma=\{0\}$,因此和为直和. 又由\nameref{thm:线性映射基本定理}可知,$\dim V=\dim\ker\sigma+\dim\im \sigma$,因此$V=\ker\sigma\oplus\im \sigma$.
          \end{proof}

    \item 关于核空间,我们有如下定理,这一定理在之后讨论矩阵标准形的时候非常有用:
          \begin{theorem}{}{核空间性质}
              我们有如下关于核空间增长与停止增长的性质:
              \begin{enumerate}
                  \item $\{0\}=\ker \sigma^0\subseteq\ker \sigma^1\subseteq\cdots\subseteq \ker \sigma^k\subseteq\ker \sigma^{k+1}\subseteq\cdots$;

                  \item 设$m$是非负整数使得$\ker \sigma^m=\ker \sigma^{m+1}$,则
                        \[\ker \sigma^m=\ker \sigma^{m+1}=\ker \sigma^{m+2}=\ker \sigma^{m+3}=\cdots\]

                  \item 令$n=\dim V$,则$\ker \sigma^n=\ker \sigma^{n+1}=\ker \sigma^{n+2}=\cdots$.
              \end{enumerate}
          \end{theorem}

          \begin{proof}
              \begin{enumerate}
                  \item 设$i>j\geqslant 0$,则$\forall\alpha\in\ker\sigma^j$,即$\sigma^j(\alpha)=0$,则$\sigma^i(\alpha)=\sigma^{i-j}(\sigma^j(\alpha))=0$,即$\alpha\in\ker\sigma^i$,因此$\ker\sigma^j\subseteq\ker\sigma^i$,故$\ker \sigma^0\subseteq\ker \sigma^1\subseteq\cdots\subseteq \ker \sigma^k\subseteq\ker \sigma^{k+1}\subseteq\cdots$.

                        这一点表明核空间随着线性变换的幂次增长而增长(至少不减),下面一点将说明这一不减序列一旦某个包含符号可以取等号,那么此后的项都相等.

                  \item 任取$k>0$,由(1)可知$\ker \sigma^{m+k}\subseteq\ker \sigma^{m+k+1}$,故只需证$\ker \sigma^{m+k+1}\subseteq\ker\sigma^{m+k}$. 事实上,设$\alpha\in\ker \sigma^{m+k+1}$,则$0=\sigma^{m+k+1}(\alpha)=\sigma^{m+1}(\sigma^k(\alpha))$,即$\sigma^k(\alpha)\in\ker\sigma^{m+1}$. 又$\ker \sigma^m=\ker \sigma^{m+1}$,则$\sigma^k(\alpha)\in\ker\sigma^m\implies\sigma^m(\sigma^k(\alpha))=0\implies\sigma^{m+k}(\alpha)=0\implies\alpha\in\ker\sigma^{m+k}$,故$\ker \sigma^{m+k+1}\subseteq\ker\sigma^{m+k}$,因此$\ker \sigma^{m+k+1}=\ker\sigma^{m+k}$,故$\ker \sigma^m=\ker \sigma^{m+1}=\ker \sigma^{m+2}=\ker \sigma^{m+3}=\cdots$.

                  \item 由上一点知我们只需证$\ker \sigma^n=\ker \sigma^{n+1}$. 反证法,若$\{0\}=\ker\sigma^0\subsetneqq\ker\sigma^1\subsetneqq\cdots\subsetneqq\ker\sigma^{n+1}$,则这一递增链条每处严格包含于的维数必然增加1,因此$\dim\ker\sigma^{n+1}\geqslant n+1>n$,但我们知道$\ker\sigma^{n+1}$是$V$的子空间,因此矛盾!故命题成立.
              \end{enumerate}
          \end{proof}
          对于像空间而言也有类似于\autoref{thm:核空间性质} 的定理,证明方法也是类似的,我们放在习题中供读者思考.

    \item 存在正整数$m$使得$V=\im \sigma^m+\ker\sigma^m$(和前述性质思想类似,我们放在习题中供读者思考);

    \item 下列条件等价:
          \begin{enumerate}
              \item $V=\ker\sigma\oplus\im \sigma$;

              \item $\ker\sigma \cap \im \sigma=\{0\}$;

              \item $\ker\sigma=\ker\sigma^2$;

              \item $\im \sigma=\im \sigma^2$;

              \item $r(\sigma^2)=r(\sigma)$.
          \end{enumerate}

          \begin{proof}
              \begin{enumerate}
                  \item $(1)\implies(2)$:由直和的定义显然;

                  \item $(2)\implies(3)$:由\autoref{thm:核空间性质} 可知$\ker\sigma\subseteq\ker\sigma^2$,又任取$\alpha\in\ker\sigma^2$,则
                        \[0=\sigma^2(\alpha)=\sigma(\sigma(\alpha)),\]故$\sigma(\alpha)\in\ker\sigma\cap\im\sigma$,即$\sigma(\alpha)=0$,因此$\alpha\in\ker\sigma$,故$\ker\sigma^2\subseteq\ker\sigma$,因此$\ker\sigma=\ker\sigma^2$;

                  \item $(3)\implies(4)$:令$\alpha\in\im\sigma^2$,故存在$\beta\in V$使得$\alpha=\sigma^2(\beta)=\sigma(\sigma(\beta))$,即$\alpha\in\im\sigma$,因此$\im\sigma^2\subseteq\im\sigma$,又由(3)知
                        \[\dim\im\sigma^2=n-\dim\ker\sigma^2=n-\dim\ker\sigma=\dim\im\sigma,\]
                        因此$\im\sigma^2=\im\sigma$;

                  \item $(4)\implies(5)$:根据线性映射的秩的定义(等于像空间维数)显然;

                  \item $(5)\implies(1)$:利用先证明和为直和(即交为$\{0\}$)再证等号成立的方法. 事实上$r(\sigma^2)=r(\sigma)$即$\dim\im\sigma^2=\dim\im\sigma$,又$(3)\implies(4)$证明了$\im\sigma^2\subseteq\im\sigma$,故$\im \sigma=\im \sigma^2$. 事实上,设$\beta_1,\ldots,\beta_s$为$\im\sigma$的一组基,则
                        \[\im\sigma^2=\spa(\sigma(\beta_1),\ldots,\sigma(\beta_s)),\]
                        由$\dim\im\sigma^2=\dim\im\sigma$可知$\sigma(\beta_1),\ldots,\sigma(\beta_s)$是$\im\sigma^2$的一组基,由$\im \sigma=\im \sigma^2$可知这也是$\im\sigma$的一组基$. \forall\alpha\in\ker\sigma\cap\im\sigma$,设
                        \[\alpha=k_1\beta_1+\cdots+k_s\beta_s,\]
                        由于
                        \[0=\sigma(\alpha)=k_1\sigma(\beta_1)+\cdots+k_s\sigma(\beta_s),\]
                        由$\sigma(\beta_1),\ldots,\sigma(\beta_s)$是一组基可知$k_1=\cdots=k_s=0$,因此$\alpha=0$,故$\ker\sigma\cap\im\sigma=\{0\}$,因此和为直和. 又由\nameref{thm:线性映射基本定理}可知,$\dim V=\dim\ker\sigma+\dim\im \sigma$,因此$V=\ker\sigma\oplus\im \sigma$.
              \end{enumerate}
          \end{proof}

    \item $\dim(\ker\sigma+\im \sigma) \geqslant \dfrac{n}{2}$,等号成立充要条件为$\ker\sigma=\im \sigma$.

          \begin{proof}
              这一结论的证明需要结合两个维数公式. 事实上,由线性空间维数公式有
              \[\dim(\ker\sigma+\im \sigma)=\dim\ker\sigma+\dim\im \sigma-\dim(\ker\sigma \cap \im \sigma)=n-\dim(\ker\sigma \cap \im \sigma)\]
              因此只需证明$\dim(\ker\sigma+\im \sigma) \leqslant \dfrac{n}{2}$.

              我们用反证法,我们知道$\ker\sigma\cap \im \sigma$是$\ker\sigma$和$\im\sigma$的子空间,因此
              \begin{gather*}
                  \dim(\ker\sigma\cap \im \sigma) \leqslant \dim\ker\sigma \\
                  \dim(\ker\sigma\cap \im \sigma) \leqslant \dim\im\sigma
              \end{gather*}
              故若$\dim(\ker\sigma\cap \im \sigma)>\dfrac{n}{2}$,则有
              \[\dim\ker\sigma+\dim\im \sigma>n\]
              与线性映射基本定理矛盾,因此$\dim(\ker\sigma+\im \sigma) \geqslant \dfrac{n}{2}$成立.

              接下来我们讨论取等条件. 充分性显然,因为此时
              \begin{gather*}
                  \dim\ker\sigma=\dim\im\sigma=\dfrac{n}{2} \\
                  \ker\sigma\cap \im \sigma=\ker\sigma=\im\sigma
              \end{gather*}
              故
              \[\dim(\ker\sigma+\im \sigma)=n-\dim(\ker\sigma \cap \im \sigma)=\dfrac{n}{2}\]
              成立.

              接下来我们讨论必要性. 由
              \[\dim(\ker\sigma+\im \sigma)=n-\dim(\ker\sigma \cap \im \sigma)\]
              可知$\dim(\ker\sigma\cap \im \sigma)=\dfrac{n}{2}$,由$\ker\sigma\cap \im \sigma$是$\ker\sigma$和$\im\sigma$的子空间可知
              \begin{align*}
                  \dim\ker\sigma & \geqslant\dfrac{n}{2} \\
                  \dim\im \sigma & \geqslant\dfrac{n}{2}
              \end{align*}
              又由线性映射基本定理,$\dim\ker\sigma+\dim\im \sigma=n$,因此
              \[\dim\ker\sigma=\dim\im \sigma=\dfrac{n}{2}\]
              即子空间维数与原空间相等,故必有$\ker\sigma=\im \sigma=\ker\sigma\cap \im \sigma$成立(回顾线性空间$U\subseteq V$且$\dim U=\dim V$则$U=V$).
          \end{proof}
\end{enumerate}

\section{可逆与同构}

同构是直至目前线性代数中最重要的概念,本节中我们只讨论基本的概念和性质,在下一讲中我们将结合线性映射矩阵表示深入探讨同构的深层意义.

\subsection{线性空间同构的概念}

\begin{definition}{同构}{} \index{tonggou@同构 (isomorphism)}
    如果由线性空间$V_1(\mathbf{F})$到$V_2(\mathbf{F})$存在一个线性双射$\sigma$,则称$V_1(\mathbf{F})$和$V_2(\mathbf{F})$是\term{同构的},记作$V_1(\mathbf{F}) \cong V_2(\mathbf{F})$. $\sigma$称为$V_1(\mathbf{F})$到$V_2(\mathbf{F})$的一个\term{同构映射}\index{tonggou!yingshe@映射 (isomorphism map)}.
\end{definition}

根据定义我们发现,同构映射实际上就是线性双射. 关于同构的概念,我们有以下几点需要强调:
\begin{enumerate}
    \item 特别注意:同构是线性空间之间的关系,同构映射才是描述线性映射的;

    \item 事实上,同构也是一种等价关系,这一点很容易验证,读者可以自行尝试(可能传递性略有困难,实际上只需说明线性双射复合后仍是线性双射即可);

    \item 同构映射的逆映射也是同构映射,即线性双射的逆映射仍然是线性双射. 除此之外,两个同构映射的复合也是同构的. 这两个性质证明是容易的,我们放在习题中供读者验证.

    \item 对同构映射$\sigma$,$V_1$中向量组$ \alpha_1,\alpha_2,\ldots,\alpha_m $与$V_2$中对应的$ \sigma(\alpha_1),\sigma(\alpha_2),\ldots,\sigma(\alpha_m) $有相同的线性相关性.

          \begin{proof}
              我们已知一般的线性映射将线性相关的向量组映射为线性相关的向量组,因此对于同构映射,我们只需证明它能将线性无关的向量组映射为线性无关的向量组即可.

              设$V_1$中$\alpha_1,\alpha_2,\ldots,\alpha_m$线性无关,我们考察$\sigma(\alpha_1),\sigma(\alpha_2),\ldots,\sigma(\alpha_m)$的线性相关性. 设
              \[c_1\sigma(\alpha_1)+c_2\sigma(\alpha_2)+\cdots+c_m\sigma(\alpha_m)=0,\]
              即
              \[\sigma(c_1\alpha_1+c_2\alpha_2+\cdots+c_m\alpha_m)=0.\]
              因为$\sigma$是线性双射,因此$\sigma$首先必须是单射,因此$\ker\sigma=\{0\}$,所以
              \[c_1\alpha_1+c_2\alpha_2+\cdots+c_m\alpha_m=0\]
              由$\alpha_1,\alpha_2,\ldots,\alpha_m$线性无关,故$c_1=c_2=\cdots=c_m=0$,即$\sigma(\alpha_1),\sigma(\alpha_2),\ldots,\sigma(\alpha_m)$线性无关,证毕.
          \end{proof}

          这一结论比一般的线性映射更强,对于一般的线性映射只有将线性相关的向量组映射为线性相关的向量组,无法保证将线性无关的向量组映射为线性无关的向量组,但同构映射可以保证,因为它是线性双射. 这一性质也是本质的,因为双射具有``一一对应''的属性,因此直觉也告诉我们,线性空间的基在线性双射(同构映射)下的像应当对应于像空间的一组基.

          我们可以更进一步得到下面的结论:
          \begin{theorem}{}{同构保秩}
              设$\sigma$是$V_1$到$V_2$的同构映射,$S_1=\{\alpha_1,\alpha_2,\ldots,\alpha_m\}$是$V_1$的任意一组向量,$S_2=\{\sigma(\alpha_1),\sigma(\alpha_2),\ldots,\sigma(\alpha_m)\}$,则$r(S_1)=r(S_2)$,即同构映射保持映射前后向量组秩不变.
          \end{theorem}
          \begin{proof}
              反证法. 假设$r(S_1)\neq r(S_2)$,我们从以下两方面导出矛盾:
              \begin{enumerate}
                  \item 若$r(S_1)>r(S_2)$,取$S_1$的极大线性无关组,记为$S_1'$,则$r(S_1')=r(S_1)>r(S_2)$. 又$S_1'$在$\sigma$下的像$S_2'$为$S_2$的子向量组,因此$r(S_2')\leqslant r(S_2)$. 但我们有同构映射保持线性无关性,因此$r(S_2')=r(S_1')=r(S_1)>r(S_2)$,矛盾!因此这种情况不可能;

                  \item 若$r(S_1)<r(S_2)$,取$S_2$的极大线性无关组,记为$S_2'$,则$r(S_2')=r(S_2)>r(S_1)$. 如前所述,同构映射的逆仍为同构映射,考虑$\sigma$的逆$\sigma^{-1}$,$S_2'$在$\sigma^{-1}$下的像$S_1'$为$S_1$的子向量组,因此$r(S_1')\leqslant r(S_1)$. 但我们有同构映射保持线性无关性,因此$r(S_1')=r(S_2')=r(S_2)>r(S_1)$,矛盾!因此这种情况不可能.
              \end{enumerate}
          \end{proof}
\end{enumerate}

我们讨论几个经典的一一对应的例子.
\begin{enumerate}
    \item 第一个例子是坐标映射:在有限维向量空间的向量的坐标一节中,我们说明了一个向量在一组基下坐标唯一,而一个坐标对应唯一一个向量,并且也证明了坐标运算的线性性,因此坐标映射是同构映射,并且是经典的同构映射. 它可以建立起任何一个$n$维线性空间$V(\mathbf{F})$与几何向量空间$\mathbf{F}^n$之间的一一对应(同构映射),即任意$n$维线性空间$V(\mathbf{F})\cong\mathbf{F}^n$. 这一点之后会强调多次,需牢记;

    \item 第二个例子将在本节后续线性映射矩阵表示中描述并证明,目前我们只给出结论,读者不必惊恐于不理解其中的记号,因为接下来的核心任务之一就是证明存在这一同构映射:若$\dim V_1(\mathbf{F})=m$,$\dim V_2(\mathbf{F})=n$,则$\mathcal{L}(V_1,V_2) \cong \mathbf{F}^{m \times n}$.
\end{enumerate}

\subsection{同构的等价条件}

下面我们给出同构的等价条件:
\begin{theorem}{}{同构的等价条件}
    两个线性空间$V_1(\mathbf{F})$和$V_2(\mathbf{F})$同构的充要条件是它们的维数相等.
\end{theorem}

\begin{proof}
    必要性:设$V_1(\mathbf{F})$和$V_2(\mathbf{F})$同构,即存在线性双射(故至少是单射)$\sigma:V_1\to V_2$. 由线性映射基本定理,
    \[\dim V_1=\dim\ker\sigma+\dim\im\sigma=\dim\im\sigma=\dim V_2.\]
    故必要性成立.

    下证明充分性,即证两维数相等的线性空间之间存在线性双射. 设$\dim V_1=\dim V_2=n$,设$V_1$的一组基为$\alpha_1,\alpha_2,\ldots,\alpha_n$,$V_2$的一组基为$\beta_1,\beta_2,\ldots,\beta_n$,根据\autoref{thm:线性映射构造} 可知,我们可以定义线性映射$\sigma:V_1\to V_2$,使得
    \begin{equation}\label{eq:6:构造同构}
        \sigma(\alpha_1)=\beta_1,\sigma(\alpha_2)=\beta_2,\ldots,\sigma(\alpha_n)=\beta_n.
    \end{equation}
    接下来只需证明$\sigma$是线性双射即可. 事实上$\sigma$是单射是显然的,因为若$\sigma(\alpha)=0$,其中$\alpha\in V_1$,则$\alpha$可以写为$\alpha=k_1\alpha_1+k_2\alpha_2+\cdots+k_n\alpha_n$,则有
    \[\sigma(\alpha)=\sigma(k_1\alpha_1+k_2\alpha_2+\cdots+k_n\alpha_n)=k_1\sigma(\alpha_1)+k_2\sigma(\alpha_2)+\cdots+k_n\sigma(\alpha_n)=0,\]
    又由\autoref{eq:6:构造同构} 以及$\beta_1,\beta_2,\ldots,\beta_n$线性无关可知$k_1=k_2=\cdots=k_n=0$,因此$\alpha=0$,即$\sigma$是单射. 由\autoref{thm:双射等价条件}(或直接根据线性映射基本定理)可知,$\sigma$是线性双射,证毕.
\end{proof}

我们需要指出,同构是目前为止最重要的概念. 它统一了前面所学的所有主干内容,将线性空间可以按维数划分为不同的等价类,并将抽象再升一层,表明线性空间最本质的特点在于维数,因为我们可以通过同构建立起对所有维数相同的线性空间之间的一一对应. 更重要的是我们可以通过坐标映射建立起任何一个$n$维线性空间$V(\mathbf{F})$与几何向量空间$\mathbf{F}^n$之间的同构映射,从而遮蔽所有线性空间自身基的特色(例如有的线性空间中的元素是矩阵、函数、数列等),进而可以将所有对有限维线性空间的研究转为对简单向量空间的研究. 从此以后的大部分研究中,我们再提到线性空间,我们只需要说出线性空间的维数,就相当于给出了几乎所有的信息. 或许目前对上面的说法的认识还不够深刻,但在下一讲中我们将通过线性映射矩阵表示的讨论进一步加深理解.

下面我们通过几个例题来应用同构的等价条件,也同时进一步了解几个常见的同构的例子:
\begin{example}{}{}
    指出下面各组内的两个线性空间是否同构,若同构可以进一步思考同构映射的构造:
    \begin{enumerate}
        \item 最高次不超过$n-1$的多项式构成的线性空间$\mathbf{R}[x]_n$与$\mathbf{R}^n$;

        \item 全体复数在实数域上的线性空间$\mathbf{C}(\mathbf{R})$与$\mathbf{R}^2$;

        \item 全体二元复向量$\mathbf{C}^2$在实数域上构成的线性空间$\mathbf{C}^2(\mathbf{R})$与$\mathbf{R}[x]_4$;

        \item 全体二元复向量$\mathbf{C}^2$在复数域上构成的线性空间$\mathbf{C}^2(\mathbf{C})$与$\mathcal{L}(\mathbf{R}^4,\mathbf{R})$.
    \end{enumerate}
\end{example}

\begin{solution}
    \begin{enumerate}
        \item 同构,因为二者维数均为$n$. 同构映射非常简单,因为$\mathbf{R}[x]_n$在基$\{1,x,\ldots,x^{n-1}\}$下的坐标就在$\mathbf{R}^n$中,因此同构映射就是这一坐标映射:
              \[\sigma:\mathbf{R}[x]_n\to\mathbf{R}^n,\quad a_0+a_1x+\cdots+a_{n-1}x^{n-1}\mapsto(a_0,a_1,\ldots,a_{n-1}).\]
              我们很容易验证这一映射是线性双射,因此是同构映射.

        \item 同构,因为二者维数均为2(回忆\autoref{ex:不同数域的维数}). 这一同构映射同上一小问理,$\mathbf{C}(\mathbf{R})$在基$\{1,i\}$下的坐标就在$\mathbf{R}^2$中,因此同构映射就是这一坐标映射:
              \[\sigma:\mathbf{C}(\mathbf{R})\to\mathbf{R}^2,\quad a+bi\mapsto(a,b).\]
              事实上这就是将复数在二维平面中的表示,我们很容易验证这一映射是线性双射,因此是同构映射..

        \item 同构,因为二者维数均为4. 这一同构映射也非常简单,因为$\mathbf{C}^2(\mathbf{R})$在基
              \[\{(1,0),(i,0),(0,1),(0,i)\}\]
              下的坐标和$\mathbf{R}[x]_4$在基$\{1,x,x^2,x^3\}$下的坐标都在$\mathrm{R}^n$中,可以将它们一一对应,因此同构映射就是这一坐标映射:
              \[\sigma:\mathbf{C}^2(\mathbf{R})\to\mathbf{R}[x]_4,\quad (a+bi,c+di)\mapsto a+bx+cx^2+dx^3.\]
              我们很容易验证这一映射是线性双射,因此是同构映射.

        \item 不同构,因为$\mathbf{C}^2(\mathbf{C})$的维数为2,而$\mathcal{L}(\mathbf{R}^4,\mathbf{R})$的维数为4.
    \end{enumerate}
\end{solution}

\section{线性空间的积}

实际上,在上一讲中,我们已经通过线性空间的交与和,学习了如何通过一些线性空间构造一个新的线性空间,本节我们将讨论从多个线性空间构造新的线性空间的另一种方法. 但我们的目标不仅限于此,通过定义积空间,我们将重新审视同构的概念,或许同构并不仅仅是维数相等这么简单的事情.

\subsection{线性空间的积的定义与性质}

我们熟知集合有笛卡尔积运算,而线性空间是定义在集合上的代数结构,因此我们有一个自然的问题,即我们能否在多个线性空间的对应的集合的笛卡尔积上定义加法和数乘运算,使其成为一个线性空间?

答案是肯定的,但我们需要首先声明的一点是,构成笛卡尔积的这些线性空间必须定义在同一个数域上,否则新集合上的数乘我们将很难定义,因为数域不同我们将很难选择数乘的常数应该选择来自于哪个线性空间的数域.
\begin{definition}{}{积空间}
    设$V_1,V_2,\ldots,V_n$是数域$\mathbf{F}$上的线性空间,我们有如下三个定义:
    \begin{enumerate}
        \item 线性空间的积:
              \[V_1 \times V_2 \times \cdots \times V_n=\{(v_1,v_2,\ldots,v_n)\mid v_i \in V_i,\enspace i=1,2,\ldots,n\};\]

        \item 规定$V_1 \times V_2 \times \cdots \times V_n$上加法和数乘运算:
              \begin{enumerate}
                  \item 加法:$(v_1,v_2,\ldots,v_n)+(u_1,u_2,\ldots,u_n)=(v_1+u_1,v_2+u_2,\ldots,v_n+u_n)$;

                  \item 数乘:$\lambda(v_1,v_2,\ldots,v_n)=(\lambda v_1,\lambda v_2,\ldots,\lambda v_n)$.
              \end{enumerate}
    \end{enumerate}
\end{definition}

事实上我们很容易验证上述定义的线性空间的积在定义的加法和数乘运算下构成线性空间,我们将放在习题中供读者练习. 接下来我们要研究这一线性空间的性质. 事实上,我们早在有限维线性空间一节中就说明了,一个线性空间的核心结构就是其基和维数,因此我们首先研究它们. 事实上,对于积空间,它的基和维数的确定是非常符合我们的直觉的,我们来看一个例子:
\begin{example}{}{}
    求积空间$\mathbf{R}[x]_3\times\mathbf{R}^2$的一组基.
\end{example}

\begin{solution}
    我们知道$\mathbf{R}[x]_3$的一组基为$1,x,x^2$,而$\mathbf{R}^2$的一组基为$(1,0),(0,1)$. 很自然的想法是:我们可以先取$\mathbf{R}[x]_3$的一组基,$\mathbf{R}^2$的位置置零,然后反之取$\mathbf{R}^2$的一组基,$\mathbf{R}[x]_3$的位置置零,即$(1,(0,0)),\ (x,(0,0)),\ (x^2,(0,0)),\ (0,(1,0)),\ (0,(0,1))$. 我们很容易可以证明上述向量组满足基的两个条件:线性无关和张成空间.
\end{solution}

上述例子中的基的构造方法是很自然的,而且我们会发现,在这样取基的情况下积空间的维数很显然就是各个线性空间的维数之和. 我们可以很容易地推广到一般情况:
\begin{theorem}{}{积空间维数}
    设$V_1,V_2,\ldots,V_n$是数域$\mathbf{F}$上的有限维线性空间,则$V_1 \times V_2 \times \cdots \times V_n$是有限维线性空间,且
    \[\dim(V_1 \times V_2 \times \cdots \times V_n)=\dim V_1+\dim V_2+\cdots+\dim V_n.\]
\end{theorem}

\begin{proof}
    证明非常简单直接:我们取$V_i$的一组基,对这组基中每个向量,我们取$V_1 \times V_2 \times \cdots \times V_n$中的这样的向量:其中第$j$个位置为此向量,其余位置为零向量,这样我们遍历所有$V_i$和每个$V_i$的基向量我们就得到了$V_1 \times V_2 \times \cdots \times V_n$的一组基(线性无关和张成性是很容易验证的),这组基的长度(即维数)为$\dim V_1+\dim V_2+\cdots+\dim V_n$.
\end{proof}

\subsection{线性空间的积与直和}

本节我们将通过线性空间的积的角度来讨论线性空间的和的性质. 事实上,我们的手段就是构造线性映射,然后利用线性映射基本定理来得到结论. 我们的主要定理如下:
\begin{theorem}{}{积与直和}
    设$U_1,U_2,\ldots,U_n$是$V$的子空间,我们定义线性映射$\sigma:U_1 \times U_2 \times \cdots \times U_n \to U_1+U_2+\cdots+U_n$,使得$\sigma(u_1,u_2,\ldots,u_n)=u_1+u_2+\cdots+u_n$,则$U_1+U_2+\cdots+U_n$是$V$的直和$\iff \sigma$是同构映射.
\end{theorem}

\begin{proof}
    回顾同构映射实际上就是线性双射,我们分成两部分给出证明:
    \begin{enumerate}
        \item 充分性:$\sigma$是双射,则$\sigma$首先是单射. 根据单射的等价条件,我们有$\ker \sigma=\{0\}$,即$u_1+u_2+\cdots+u_n=0$必须有$u_1=u_2=\cdots=u_n=0$,而这正是\autoref{thm:直和等价命题} 中直和的等价条件;

        \item 必要性:设$U_1+U_2+\cdots+U_n$是直和,我们证明$\sigma$是单的、满的:
              \begin{enumerate}
                  \item 单射:设$\sigma(u_1,u_2,\ldots,u_n)=0$,即$u_1+u_2+\cdots+u_n=0$,由直和的等价条件可知$u_1=u_2=\cdots=u_n=0$,即$\sigma$的核空间只有出发空间零元,故是单射;

                  \item 满射:实际上是由这个线性映射的定义直接保证的. $\forall u \in U_1+U_2+\cdots+U_n$,根据和的定义一定有分解$u=u_1+u_2+\cdots+u_n$,其中$u_i \in U_i$,因此根据$\sigma$的定义$\sigma(u_1,u_2,\ldots,u_n)=u$,即任意$u$我们都可找到原像,故是满射.
              \end{enumerate}
              至于线性性,我相信读者已经能够很容易验证,因此我们不再赘述. 因此$\sigma$是线性双射,即是同构映射.
    \end{enumerate}
\end{proof}

通过这一定理我们可以直接得出以下结论:
\begin{theorem}{}{}
    设$U_1,U_2,\ldots,U_n$是有限维线性空间$V$的子空间,则$U_1+U_2+\cdots+U_n$是$V$的直和$\iff \dim(U_1+U_2+\cdots+U_n)=\dim U_1+\dim U_2+\cdots+\dim U_n$.
\end{theorem}

\begin{proof}
    根据\autoref{thm:积与直和},$U_1+U_2+\cdots+U_n$是$V$的直和$\iff \sigma$是同构映射. 同构映射的出发空间和到达空间维数相等,因此$\sigma$是双射$\iff \dim(U_1 \times U_2 \times \cdots \times U_n)=\dim(U_1+U_2+\cdots+U_n)$,最后根据\autoref{thm:积空间维数} 积空间的维数可知定理成立.
\end{proof}

由此,我们通过在积空间上定义映射,结合\nameref{thm:线性映射基本定理}(或同构)得到了\autoref{thm:直和等价命题} 中关于维数的命题. 总结而言,在积空间的讨论中我们展现了一个比较完整地学习路径:从定义积空间的想法(来源于集合的笛卡尔积),到如何自然地定义出这一空间的加法和数乘运算,然后研究构造出的空间的基本结构有什么特点,然后进一步构造其上线性映射,得到一些其他的结论. 这一路径的每一步都是非常自然的,而且是学习一个数学概念的常见思路,希望读者不仅是在线性代数中体会到这种学习路径,在其他数学课甚至其他学科中都能总结出这样一条引入—定义—性质—应用的自然路径.

\subsection{自然同构}

本节我们将重新审视同构这一概念,我想本节的内容不是主线必须的,事实上对于同构这一概念,在有限维线性空间的视角下理解为维数相等在大部分场合下已经足够,但如果你希望在之后更深地理解对偶等章节,本节内容会提供一些基本的概念.\footnote{严谨而言,本节内容需要用到范畴论的概念,但这里我们为了避免引入大量的概念,将范畴论的术语转化为线性代数中的术语,并且大大简化我们的讨论,也就是说,这里仅仅是相关内容的冰山一角. 本书的最后一个未竟专题将会展开范畴论的讨论,感兴趣的读者可以参考.}

为了重新审视同构,我们首先来看两个例子. 第一个例子需要我们回顾\autoref{thm:积与直和},实际上我们可以直接得到如下同构:
\begin{equation} \label{eq:5:积与直和自然同构}
    V_1\times V_2\times\cdots\times V_n\cong V_1\oplus V_2\oplus\cdots\oplus V_n.
\end{equation}
这一同构可以由映射
\begin{equation} \label{eq:5:积与直和自然同构映射}
    \begin{aligned}
        \sigma:V_1\times V_2\times\cdots\times V_n & \to V_1\oplus V_2\oplus\cdots\oplus V_n, \\
        (v_1,v_2,\ldots,v_n)                       & \mapsto v_1+v_2+\cdots+v_n
    \end{aligned}
\end{equation}
决定. 第二个例子则更加简单,我们知道对任何一个$n$维线性空间$V$有
\begin{equation} \label{eq:5:V与Fn同构}
    V\cong\mathbf{F}^n.
\end{equation}
这一同构可以由映射
\begin{equation} \label{eq:5:V与Fn同构映射}
    \begin{aligned}
        \sigma:V & \to\mathbf{F}^n,            \\
        v        & \mapsto(v_1,v_2,\ldots,v_n)
    \end{aligned}
\end{equation}
决定,其中$v_1,v_2,\ldots,v_n$是$v$在$V$的某一组基下的坐标. 相信读者读到这里已经发现了问题:\autoref{eq:5:积与直和自然同构映射} 中的映射我们并没有强调基的选取,然而\autoref{eq:5:V与Fn同构映射} 中的映射却依赖于基的选取,当$V$的基选取不一致时,$v\in V$的坐标会变化,因此映射$\sigma$的定义也会变化.

这一差异引入了在线性空间中的自然同构的概念:
\begin{definition}{自然同构(线性空间版本)}{}
    设$V_1,V_2$为两个线性空间,称两者自然同构,如果在某组$V_1$的基$\alpha_{11},\ldots,\alpha_{1n}$和某组$V_2$的基$\alpha_{21},\ldots,\alpha_{2n}$下可以定义一个线性映射$\sigma_\alpha$ 满足:
    \begin{enumerate}
        \item $\sigma_\alpha$是一个从$V_1$到$V_2$的同构;
        \item 对于任意$V_1$的基$\beta_{11},\ldots,\beta_{1n}$和$V_2$的基$\beta_{21},\ldots,\beta_{2n}$,如果能定义出同构映射$\sigma_\beta$,则必须有$\sigma_\beta$与$\sigma_\alpha$都是同一个线性映射.
    \end{enumerate}
\end{definition}

读者可能会觉得这个定义有点语焉不详. 但实际上,我们可以回顾\autoref{thm:线性映射唯一确定},为了定义一个线性映射,我们往往需要写出它把出发空间的基中的每个向量映射到了哪里,这也是上面通过基定义映射的依据. 而为了把我们前面例子中映射定义与基无关的要求,我们就引入了第二条性质.

于是,接下来我们便可以按照这一定义验证两个例子中的同构是否是自然同构:
\begin{proof}
    \begin{enumerate}
        \item
        \item
    \end{enumerate}
\end{proof}

最后我们再看一个例子,我们希望进一步看到构造同构映射带来的研究问题的方便性:
\begin{example}{}{}
    设$V_1,V_2,\ldots,V_n,W$是数域$\mathbf{F}$上的线性空间,证明:$\mathcal{L}(V_1 \times V_2 \times \cdots \times V_n,W)$与$\mathcal{L}(V_1,W) \times \mathcal{L}(V_2,W) \times \cdots \times \mathcal{L}(V_n,W)$同构.
\end{example}
有的读者可能看见这题就会觉得非常简单,因为有限维线性空间的前提下二者维数显然相同,然而我们这里并未限定有限维线性空间,因此需要读者自己构造同构映射. 事实上同构映射的构造是很自然的,因为我们知道任何线性空间都与向量空间有一个很自然的映射,我们只需要将两部分映射复合即可.

\begin{solution}
    $\forall f\in \mathcal{L}(V_1 \times V_2 \times \cdots \times V_n,W)$,我们定义$f_i:V_i\to W(i=1,2,\ldots,m)$满足
    \[f_i(v_i)=f(0,\ldots,0,v_i,0,\ldots,0),\]
    其中$v_i$位于第$i$个位置,其余位置为零向量.

    定义$\varphi:\mathcal{L}(V_1 \times V_2 \times \cdots \times V_n,W)\to \mathcal{L}(V_1,W) \times \mathcal{L}(V_2,W) \times \cdots \times \mathcal{L}(V_n,W)$,使得$\varphi(f)=(f_1,f_2,\ldots,f_m)$,则接下来我们要验证$\varphi$就是我们要求的同构映射.
\end{solution}

\vspace{2ex}
\centerline{\heiti \Large 内容总结}

本讲我们开始讨论两个线性空间之间的关联,引入了线性映射这一概念. 我们讨论了``线性性''这一基本的性质,它将经常出现在我们数学学习过程中,并且我们也讨论了基于线性性这一要求能得到映射具有怎样的性质——如将0元映射到0元,将线性相关的向量组映射到线性相关的向量组(反之不一定). 接下来我们进一步构造了线性映射的加法和数乘,从而使得$V_1$到$V_2$的全体线性映射构成一个线性空间,这一空间记作$\mathcal{L}(V_1,V_2)$. 我们还讨论了线性映射的像和核,它们分别是到达空间和出发空间的子空间,我们还详细讨论了如何计算它们. 最后我们讨论了线性映射的确定,即线性映射在一组基下的像唯一确定,这一定理的思想是非常重要的,它表明关于线性映射的研究完全可以限制在在一组基下的研究,也讨论了一个基本的问题:即是否存在满足特定要求的线性映射. 事实上,以上所有的讨论都基于``线性''这一性质,因此掌握本节中的各种证明有助于读者深入体会基于``线性''能通过怎样的一般证明手段得到怎样的结果.

\vspace{2ex}
\centerline{\heiti \Large 习题}

\vspace{2ex}
I argue that set theory should not be based on membership, as in Zermelo-Frankel set theory, but rather on isomorphism-invariant structure.
\begin{flushright}
    ——W. Lawvere
\end{flushright}

\centerline{\heiti A组}
\begin{enumerate}
    \item 设$\sigma: V_1\to V_2$是线性映射. 证明:$\sigma(W_1)$和$\sigma^{-1}(W_2)$分别是$V_2$和$V_1$的子空间.

    \item 设$\sigma,\tau \in \mathcal{L}(V,V)$且$\sigma^2=\sigma$,$\tau^2=\tau$. 证明:
          \begin{enumerate}
              \item $\sigma^k=\sigma$(幂等变换);

              \item 若$(\sigma+\tau)^2=\sigma+\tau$,则$\sigma\tau=\theta$(零变换);

              \item 设$\sigma\tau=\tau\sigma$,则$(\sigma+\tau-\sigma\tau)^2=\sigma+\tau-\sigma\tau$.
          \end{enumerate}

    \item 是否存在$\mathbf{R}^3$到$\mathbf{R}^2$的线性映射$\sigma$使得$\sigma(1,-1,1)=(1,0)$,$\sigma(1,1,1)=(0,1)$?

    \item 是否存在$\mathbf{R}^2$到$\mathbf{R}^3$的线性映射$\sigma$使得$\sigma(3,2)=(1,0,0)$,$\sigma(1,5)=(1,1,0)$,$\sigma(-1,4)=(1,1,1)$?

    \item 求$\sigma(x_1,x_2,\ldots,x_n)=(x_1,0,\ldots,0)$的像、核与秩.
\end{enumerate}

\centerline{\heiti B组}
\begin{enumerate}
    \item 已知
          \begin{gather*}
              \alpha_1=(1,-1),\enspace\alpha_2=(2,-1),\enspace\alpha_3=(-3,2) \\
              \beta_1=(1,0),\enspace\beta_2=(0,1),\enspace\beta_3=(1,1)
          \end{gather*}
          是否存在$\sigma\in \mathcal{L}(\mathbf{R}^2,\mathbf{R}^2)$,使得$\sigma(\alpha_i)=\beta_i,\enspace i=1,2,3$?

    \item 设$\alpha_1,\alpha_2$是线性空间$V(\mathbf{F})$的一组基,$x_1\alpha_1+x_2\alpha_2 \in V$. 定义$T(x_1\alpha_1+x_2\alpha_2)=r_1x_1\alpha_1+r_2x_2\alpha_2$,其中$r_1,r_2$是域$\mathbf{F}$中的两个常数. 证明:$T$是$V$上的一个线性变换. 当$V=\mathbf{R}^2$时,说明$T$的几何意义.

    \item 已知$\mathbf{R}$上的线性变换$\sigma(x_1,x_2)=(x_1-x_2,x_1+x_2)$,$\tau(x_1,x_2)=(x_1-x_2,x_2-x_1)$.
          \begin{enumerate}
              \item 求$\sigma^2(x_1,x_2)$;

              \item $\sigma$是否可逆?如可逆,求$\sigma^{-1}(x_1,x_2)$;

              \item 求$\xi\in \mathcal{L}(\mathbf{R}^2,\mathbf{R}^2)$,使得$\xi\tau=\theta$(零变换).
          \end{enumerate}

    \item 已知$\mathbf{R}^3$上的两个线性变换$\sigma,\tau$为:
          \begin{gather*}
              \sigma(x_1,x_2,x_3)=(x_3,0,0) \\
              \tau(x_1,x_2,x_3)=(x_1+x_2+x_3,x_1-x_2,0)
          \end{gather*}
          \begin{enumerate}
              \item 求$r(\sigma),\enspace r(\tau),\enspace \im\sigma,\enspace \ker\sigma$;

              \item 求$r(\tau\sigma),\enspace r(\sigma\tau),\enspace r(\sigma+\tau)$;

              \item 求$\im\tau+\ker\tau$.
          \end{enumerate}

    \item 设 $\sigma$ 是线性空间 $V$ 上的线性变换,如果 $\sigma^{k-1}(\alpha) \neq 0$,但 $\sigma^{k}(\alpha) = 0$,证明:\\
          $\alpha,\sigma(\alpha),\ldots,\sigma^{k-1}(\alpha)\enspace(k>0)$ 线性无关(本题还有对应的矩阵版本,解法基本一致).

    \item 设$\mathbf{R}[x]_3$是次数小于3的实系数多项式和全体零多项式一起组成的集合关于多项式加法和数乘多项式运算构成的实数域上的线性空间.
          \begin{enumerate}
              \item 证明:$W=\{f(x)\in \mathbf{R}[x]_3 \mid f(1)=0\}$是$\mathbf{R}[x]_3$的一个子空间,并求$W$的维数和一组基;

              \item 定义从$\mathbf{R}[x]_3$到$\mathbf{R}$的线性映射$\sigma(f(x))=f(1)$,证明:$\sigma$为线性映射,并求$\im\sigma$和$\dim\ker\sigma$;

              \item 设$f,g,h \in \mathbf{R}[x]_3$且$f(1)=g(1)=h(1)=0$,证明:$f,g,h$线性相关.
          \end{enumerate}
\end{enumerate}

\centerline{\heiti C组}
\begin{enumerate}
    \item 设 $V(\mathbf{F})$ 是一个 $n$ 维线性空间,$\sigma \in \mathcal{L}(V,V)$. 证明:
          \begin{enumerate}
              \item 在 $\mathbf{F}[x]$ 中有一个次数不高于 $n^2$ 的多项式 $p(x)$ 使 $p(\sigma) = \theta$;

              \item $\sigma$ 可逆$\iff$有一常数项不为 0 的多项式 $p(x)$ 使 $p(\sigma) = \theta$.
          \end{enumerate}

    \item 已知$\sigma_1,\sigma_2,\ldots,\sigma_s$是线性空间$V$上的$s$个两两不同的线性变换,证明:在$V$中必存在向量$\alpha$使得$\sigma_1(\alpha),\sigma_2(\alpha),\ldots,\sigma_s(\alpha)$也两两不同.
\end{enumerate}

\chapter{线性映射基本定理}

在上一讲的讨论中我们定义了线性映射的基本概念,讨论了由其定义直接引出的性质.本节我们将深入讨论
线性映射像空间与核空间之间的关联,从而引出我们目前为止最核心的概念——同构,因为同构使得我们研究
的抽象层次更上一层,为我们在下一讲中在这抽象的制高点获得最具象的表达形式——矩阵作铺垫.

\section{线性映射的秩}
通过对线性映射像的求解的讨论我们有
$\im \sigma=\sigma(V_1)=\spa(\sigma(\alpha_1),\sigma(\alpha_2),\ldots,\sigma(\alpha_n))$.
我们基于此定义线性映射的秩:
\begin{definition}
    设$\sigma\in \mathcal{L}(V_1,V_2)$,如果$\sigma(V_1)$是$V_2$的有限维子空间,则
    $\sigma(V_1)$的维数称为$\sigma$的秩,记作$r(\sigma)$,即$r(\sigma)=\dim \sigma(V_1)$.
\end{definition}

这一定义是平凡的,简单理解线性映射的秩即为线性映射像空间的维数.

\section{线性映射基本定理}
这一定理是线性代数最重要的定理之一,因其重要性也被冠以(有限维线性空间)线性映射基本定理的名号:
\begin{theorem}\label{thm:6:线性映射基本定理}
    设$\sigma \in \mathcal{L}(V_1,V_2)$,若$\dim V_1=n$,则
    \[r(\sigma)+\dim\ker\sigma=n.\]
\end{theorem}
简而言之,这一定理表明:线性映射的秩(或者说线性映射像空间维数)与核空间维数之和等于出发空间的维数.
这一定理的证明非常重要,在之后的很多讨论中还会用到这一思想,因此我们给出详细的证明并阐述其中的思想:

\begin{proof}
    证明的思路和线性空间维数公式\autoref{thm:4:维数公式}的证明思路类似,即``设小扩大''.我们设
    $\dim\ker\sigma=k$,并设$\ker\sigma$的一组基为$\alpha_1,\alpha_2,\ldots,\alpha_k$.
    我们将其扩充为$V_1$的一组基,记为$\alpha_1,\alpha_2,\ldots,\alpha_k,\alpha_{k+1},\ldots,\alpha_n$.

    根据定理要证明的等式和前述假设,我们只需证$r(\sigma)=n-k$,即证明像空间维数为$n-k$.
    我们知道像空间为$\spa(\sigma(\alpha_1),\sigma(\alpha_2),\ldots,\sigma(\alpha_n))$,其中根据我们的假设,
    $\sigma(\alpha_1)=\sigma(\alpha_2)=\cdots=\sigma(\alpha_k)=0$(因为它们是核空间的基),因此像空间为
    $\spa(\sigma(\alpha_{k+1}),\ldots,\sigma(\alpha_n))$,我们只需证明这一向量组是线性无关的即可,因为这样
    这$n-k$个向量就可以构成像空间的一组基,从而证明了$r(\sigma)=n-k$.

    我们设$c_{k+1}\sigma(\alpha_{k+1})+\cdots+c_n\sigma(\alpha_n)=0$,即
    \[\sigma(c_{k+1}\alpha_{k+1}+\cdots+c_n\alpha_n)=0.\]
    故$c_{k+1}\alpha_{k+1}+\cdots+c_n\alpha_n \in \ker\sigma$,因此可以被$\alpha_1,\alpha_2,\ldots,\alpha_k$线性表示,
    于是有
    \[c_{k+1}\alpha_{k+1}+\cdots+c_n\alpha_n=c_1\alpha_1+\cdots+c_k\alpha_k.\]
    即
    \[c_1\alpha_1+\cdots+c_k\alpha_k-c_{k+1}\alpha_{k+1}-\cdots-c_n\alpha_n=0.\]
    由于$\alpha_1,\alpha_2,\ldots,\alpha_n$是$V_1$的一组基,因此$c_1=\cdots=c_k=c_{k+1}=\cdots=c_n=0$,
    故$\sigma(\alpha_{k+1}),\ldots,\sigma(\alpha_n)$线性无关,命题得证.
\end{proof}

事实上这一定理也被称为线性映射``维数公式'',但为了与线性空间维数公式\autoref{thm:4:维数公式}区分,本讲义中
我们称这一定理为线性映射基本定理.读者可以比较一下两个``维数公式''的证明,二者都使用了``设小扩大''的思想,都将
要证明的结论转化为证明一组向量是线性无关的,但其中证明线性无关的方法略有不同,读者可以仔细体会.

下面我们给出一个证明思想上类似的例子供读者练习:
\begin{example}
    设$\sigma$为有限维线性空间$V$上的线性变换,$W$是$V$的子空间,证明:
    \[\dim\sigma(W)+\dim(\sigma^{-1}(0) \cap W)=\dim W.\]
\end{example}
\begin{proof}
    
\end{proof}
基于线性映射基本定理,我们可以得到如下定理:
\begin{theorem}\label{thm:6:双射等价条件}
    对$\sigma \in \mathcal{L}(V_1,V_2)$且$\dim V_1=\dim V_2=n$,下列条件等价:
    \begin{enumerate}
        \item $\ker \sigma=\{0\}$;
        \item $\sigma$为单射;
        \item $\sigma$为满射;
        \item $\sigma$为双射(可逆);
        \item $r(\sigma)=n$.
    \end{enumerate}
\end{theorem}

显然这一定理前提要求是有限维空间上的线性变换.我们需要注意的是,上述1与2等价不是基于线性映射基本定理得到的,
而是在前述\autoref{thm:5:单射与核空间}中已经证明的.其余等价性的证明也是非常简单,只需要简单套用维数公式即可.

实际上,线性映射基本定理还隐藏着一个我们之前以及介绍过的结论,即不可能存在从低维空间到高维空间的满射.利用反证法,
假设存在这样的映射$\sigma:V_1\to V_2$,则核空间维数$\dim\ker\sigma=n-r(\sigma)=\dim V_1-\dim V_2<0$,这显然是
不合理的.当然这一结论有一对称形式也成立,即不存在高维空间到低维空间的单射,证明类似,不再赘述.

\section{像与核的进一步讨论}
关于线性变换的像和核有很多的包含关系或等式等结论,实际上很多问题都来源于线性映射基本定理及其推论,本节我们主要探讨这一话题.

我们首先说明几个重要的原则:
\begin{enumerate}
    \item 解决此类问题大多需要综合利用维数公式及其推论,需要将题给条件转化为合适的等价表述然后解决;

    \item 注意集合相等的证明方式,实际上就是两个集合互相包含.实际上很多时候一边的包含是显然的,只需证明另一边;

    \item 时刻注意线性映射的像和核的定义,线性空间的交、和与直和的概念,例如看到像需要想到其存在原像,看到和与直和要想到将向量分拆等.
\end{enumerate}

接下来我们看一些经典的结论(已知$V$为有限维线性空间,$\sigma\in \mathcal{L}(V,V)$),、其中结论1最为常见:

\begin{enumerate}
    \item 若$\sigma$为幂等变换(即$\sigma^2=\sigma$)有$V=\ker\sigma\oplus\im \sigma$;

    \begin{proof}
        
    \end{proof}

    \item 关于核空间,我们有如下定理,这一定理在之后讨论矩阵标准形的时候非常有用:
    \begin{theorem}\label{thm:6:核空间性质}
        我们有如下关于核空间增长与停止增长的性质:
        \begin{enumerate}
            \item $\{0\}=\ker \sigma^0\subset\ker \sigma^1\subset\cdots\subset
            \ker \sigma^k\subset\ker \sigma^{k+1}\subset\cdots$;
    
            \item 设$m$是非负整数使得$\ker \sigma^m=\ker \sigma^{m+1}$,则
            \[\ker \sigma^m=\ker \sigma^{m+1}=\ker \sigma^{m+2}=\ker \sigma^{m+3}=\cdots\]
    
            \item 令$n=\dim V$,则$\ker \sigma^n=\ker \sigma^{n+1}=\ker \sigma^{n+1}=\cdots$.
        \end{enumerate}
    \end{theorem}
    \begin{proof}
        
    \end{proof}
    对于像空间而言也有类似于\autoref{thm:6:核空间性质}的定理,证明方法也是类似的,我们放在习题中供读者思考.
    \item 存在正整数$m$使得$V=\im \sigma^m+\ker\sigma^m$(和前述性质思想类似,我们放在习题中供读者思考);
    \item 下列条件等价:
    \begin{enumerate}[label=(\arabic*)]
        \item $V=\ker\sigma\oplus\im \sigma$;
        \item $r(\sigma^2)=r(\sigma)$;
        \item $\ker\sigma=\ker\sigma^2$;
        \item $\ker\sigma \cap \im \sigma=\{0\}$;
        \item $\im \sigma=\im \sigma^2$.
    \end{enumerate}
    \begin{proof}
        
    \end{proof}

    \item $\dim(\ker\sigma+\im \sigma) \geqslant \cfrac{n}{2}$,等号成立充要条件为$\ker\sigma=\im \sigma$.
    
    \begin{proof}
        这一结论的证明需要结合两个维数公式.事实上,由线性空间维数公式有
        \[\dim(\ker\sigma+\im \sigma)=\dim\ker\sigma+\dim\im \sigma-\dim(\ker\sigma \cap \im \sigma)=n-\dim(\ker\sigma \cap \im \sigma),\]
        因此只需证明$\dim(\ker\sigma+\im \sigma) \geqslant \cfrac{n}{2}$.

        我们用反证法,我们知道$\ker\sigma\cap \im \sigma$是$\ker\sigma$和$\im\sigma$的子空间,因此
        \begin{gather*}
            \dim(\ker\sigma\cap \im \sigma) \leqslant \dim\ker\sigma \\
            \dim(\ker\sigma\cap \im \sigma) \leqslant \dim\im\sigma
        \end{gather*}
        故若$\dim(\ker\sigma+\im \sigma)>\cfrac{n}{2}$,则有
        \[\dim\ker\sigma+\dim\im \sigma>n\]
        与线性映射基本定理矛盾,因此$\dim(\ker\sigma+\im \sigma) \geqslant \cfrac{n}{2}$成立.

        接下来我们讨论取等条件.充分性显然,因为此时$\dim\ker\sigma=\dim\im\sigma=\cfrac{n}{2}$,且
        $\ker\sigma\cap \im \sigma=\ker\sigma=\im\sigma$,故$\dim(\ker\sigma+\im \sigma)=n-\dim(\ker\sigma \cap \im \sigma)=\cfrac{n}{2}$成立.

        接下来我们讨论必要性.由$\dim(\ker\sigma+\im \sigma)=n-\dim(\ker\sigma \cap \im \sigma)$可知
        $\dim(\ker\sigma\cap \im \sigma)=\cfrac{n}{2}$,由$\ker\sigma\cap \im \sigma$是$\ker\sigma$和$\im\sigma$的子空间
        可知$\dim\ker\sigma\geqslant\cfrac{n}{2},\dim\im \sigma\geqslant\cfrac{n}{2}$,又由线性映射基本定理,
        $\dim\ker\sigma+\dim\im \sigma=n$,因此$\dim\ker\sigma=\dim\im \sigma=\cfrac{n}{2}$,即子空间维数与原空间
        相等,故必有$\ker\sigma=\im \sigma=\ker\sigma\cap \im \sigma$成立(回顾线性空间$U\subset V$且$\dim U=\dim V$则$U=V$).
    \end{proof}
\end{enumerate}

\section{可逆与同构}
同构是直至目前线性代数中最重要的概念,本节中我们只讨论基本的概念和性质,在下一讲中我们将结合线性映射矩阵表示深入探讨
同构的深层意义.
\subsection{线性空间同构的概念}
\begin{definition}
    如果由线性空间$V_1(\mathbf{F})$到$V_2(\mathbf{F})$存在一个线性双射$\sigma$,则称
    $V_1(\mathbf{F})$和$V_2(\mathbf{F})$是\keyterm{同构的},记作$V_1(\mathbf{F}) \cong V_2(\mathbf{F})$.
    $\sigma$称为$V_1(\mathbf{F})$到$V_2(\mathbf{F})$的一个\keyterm{同构映射}[isomorphism].
\end{definition}

根据定义我们发现,同构映射实际上就是线性双射.关于同构的概念,我们有以下几点需要强调:
\begin{enumerate}
    \item 特别注意:同构是线性空间之间的关系,同构映射才是描述线性映射的;
    \item 事实上,同构也是一种等价关系,这一点很容易验证,读者可以自行尝试(可能传递性略有困难,实际上只需说明线性双射
    复合后仍是线性双射即可);
    \item 同构映射的逆映射也是同构映射,即线性双射的逆映射仍然是线性双射.除此之外,两个同构映射的复合也是同构的.
    这两个性质证明是容易的,我们放在习题中供读者验证.
    \item 对同构映射$\sigma$,$V_1$中向量组$ \alpha_1,\alpha_2,\ldots,\alpha_m $与$V_2$中对应的
    $ \sigma(\alpha_1),\sigma(\alpha_2),\ldots,\sigma(\alpha_m) $有相同的线性相关性,证明如下:

    \begin{proof}
        我们已知一般的线性映射将线性相关的向量组映射为线性相关的向量组,因此对于同构映射,我们只需证明
        它能将线性无关的向量组映射为线性无关的向量组即可.
        
        设$V_1$中$\alpha_1,\alpha_2,\ldots,\alpha_m$线性无关,我们考察$\sigma(\alpha_1),\sigma(\alpha_2),\ldots,\sigma(\alpha_m)$
        的线性相关性,设
        \[c_1\sigma(\alpha_1)+c_2\sigma(\alpha_2)+\cdots+c_m\sigma(\alpha_m)=0,\]
        即
        \[\sigma(c_1\alpha_1+c_2\alpha_2+\cdots+c_m\alpha_m)=0.\]
        因为$\sigma$是线性双射,因此$\sigma$首先必须是单射,因此$\ker\sigma=\{0\}$,因此
        \[c_1\alpha_1+c_2\alpha_2+\cdots+c_m\alpha_m=0,\]
        由$\alpha_1,\alpha_2,\ldots,\alpha_m$线性无关,故$c_1=c_2=\cdots=c_m=0$,即$\sigma(\alpha_1),\sigma(\alpha_2),\ldots,\sigma(\alpha_m)$
        线性无关,证毕.
    \end{proof}
    
    这一结论比一般的线性映射更强,对于一般的线性映射只有将线性相关的向量组映射为线性相关的向量组,无法保证将线性无关的向量组
    映射为线性无关的向量组,但同构映射可以保证,因为它是线性双射.这一性质也是本质的,因为双射具有``一一对应''的属性,因此直觉
    也告诉我们,线性空间的基在线性双射(同构映射)下的像应当对应于像空间的一组基.

    我们可以更进一步得到下面的结论:
    \begin{theorem}\label{thm:6:同构保秩}
        设$\sigma$是$V_1$到$V_2$的同构映射,$S_1=\{\alpha_1,\alpha_2,\ldots,\alpha_m\}$是$V_1$的任意一组向量,
        $S_2=\{\sigma(\alpha_1),\sigma(\alpha_2),\ldots,\sigma(\alpha_m)\}$,则$r(S_1)=r(S_2)$,即同构映射保持映射前后向量组秩不变.
    \end{theorem}
    \begin{proof}
        反证法.假设$r(S_1)\neq r(S_2)$,我们从以下两方面导出矛盾:
        \begin{enumerate}
            \item 若$r(S_1)>r(S_2)$,取$S_1$的极大线性无关组,记为$S_1'$,则$r(S_1')=r(S_1)>r(S_2)$.又$S_1'$在$\sigma$下的像
            $S_2'$为$S_2$的子向量组,因此$r(S_2')\leqslant r(S_2)$.但我们有同构映射保持线性无关性,因此$r(S_2')=r(S_1')=r(S_1)>r(S_2)$,
            矛盾!因此这种情况不可能;

            \item 若$r(S_1)<r(S_2)$,取$S_2$的极大线性无关组,记为$S_2'$,则$r(S_2')=r(S_2)>r(S_1)$.如前所述,同构映射的逆仍为同构映射,考虑$\sigma$
            的逆$\sigma^{-1}$,$S_2'$在$\sigma^{-1}$下的像$S_1'$为$S_1$的子向量组,因此$r(S_1')\leqslant r(S_1)$.但我们有同构映射保持线性无关性,
            因此$r(S_1')=r(S_2')=r(S_2)>r(S_1)$,矛盾!因此这种情况不可能.
        \end{enumerate}
    \end{proof}
\end{enumerate}


我们讨论几个经典的一一对应的例子.
\begin{enumerate}
    \item 第一个例子是坐标映射:在有限维向量空间的向量的坐标一节中,我们说明了一个向量在一组基下坐标唯一,而一个坐标对应唯一一个向量,
    并且也证明了坐标运算的线性性,因此坐标映射是同构映射,并且是经典的同构映射.它可以建立起任何一个$n$维线性空间$V(\mathbf{F})$与几何向量空间
    $\mathbf{F}^n$之间的一一对应(同构映射),即任意$n$维线性空间$V(\mathbf{F})\cong\mathbf{F}^n$.这一点之后会强调多次,需牢记;

    \item 第二个例子将在下一将线性映射矩阵表示中描述并证明,目前我们只给出结论,读者不必惊恐于不理解其中的记号,
    因为下一讲的核心任务之一就是证明存在这一同构映射:若$\dim V_1(\mathbf{F})=m$,$\dim V_2(\mathbf{F})=n$,则
    $\mathcal{L}(V_1,V_2) \cong \mathbf{F}^{m \times n}$.
\end{enumerate}

\subsection{同构的等价条件}
下面我们给出同构的等价条件:
\begin{theorem}\label{thm:6:同构的等价条件}
    两个线性空间$V_1(\mathbf{F})$和$V_2(\mathbf{F})$同构的充要条件是它们的维数相等.
\end{theorem}

\begin{proof}
    必要性:设$V_1(\mathbf{F})$和$V_2(\mathbf{F})$同构,即存在线性双射(故至少是单射)$\sigma:V_1\to V_2$.
    由线性映射基本定理,
    \[\dim V_1=\dim\ker\sigma+\dim\im\sigma=\dim\im\sigma=\dim V_2.\]
    故必要性成立.

    下证明充分性,即证两维数相等的线性空间之间存在线性双射.设$\dim V_1=\dim V_2=n$,设$V_1$的一组基为
    $\alpha_1,\alpha_2,\ldots,\alpha_n$,$V_2$的一组基为$\beta_1,\beta_2,\ldots,\beta_n$,
    根据\autoref{thm:5:线性映射构造}可知,我们可以定义线性映射$\sigma:V_1\to V_2$,使得
    \begin{equation}\label{eq:6:构造同构}
        \sigma(\alpha_1)=\beta_1,\sigma(\alpha_2)=\beta_2,\ldots,\sigma(\alpha_n)=\beta_n.
    \end{equation}
    接下来只需证明$\sigma$是线性双射即可.事实上$\sigma$是单射是显然的,因为若$\sigma(\alpha)=0$,
    其中$\alpha\in V_1$,则$\alpha$可以写为$\alpha=k_1\alpha_1+k_2\alpha_2+\cdots+k_n\alpha_n$,
    则有
    \[\sigma(\alpha)=\sigma(k_1\alpha_1+k_2\alpha_2+\cdots+k_n\alpha_n)=k_1\sigma(\alpha_1)+k_2\sigma(\alpha_2)+\cdots+k_n\sigma(\alpha_n)=0,\]
    又由\autoref{eq:6:构造同构}以及$\beta_1,\beta_2,\ldots,\beta_n$线性无关可知$k_1=k_2=\cdots=k_n=0$,因此$\alpha=0$,即$\sigma$是单射.
    由\autoref{thm:6:双射等价条件}(或直接根据线性映射基本定理)可知,$\sigma$是线性双射,证毕.
\end{proof}

我们需要指出,同构是目前为止最重要的概念,它统一了前面所学的所有主干内容,将线性空间可以按维数划分为不同的等价类,
并将抽象再升一层,表明线性空间最本质的特点在于维数,因为我们可以通过同构建立起对所有维数相同的线性空间之间的
一一对应,更重要的是我们可以通过坐标映射建立起任何一个$n$维线性空间$V(\mathbf{F})$与几何向量空间
$\mathbf{F}^n$之间的同构映射,从而遮蔽所有线性空间自身基的特色(例如有的线性空间中的元素是矩阵、函数、数列等),
从而对所有有限维线性空间的研究都可以转为对简单向量空间的研究.从此以后的大部分研究中,我们再提到线性空间,
我们只需要说出线性空间的维数,就相当于给出了几乎所有的信息.
或许目前对上面的说法的认识还不够深刻,但在下一讲中我们将通过线性映射矩阵表示的讨论进一步加深理解.

下面我们通过几个例题来应用同构的等价条件,也同时进一步了解几个常见的同构的例子:
\begin{example}
    指出下面各组内的两个线性空间是否同构,若同构可以进一步思考同构映射的构造:
    \begin{enumerate}
        \item 最高次不超过$n-1$的多项式构成的线性空间$\mathbf{R}[x]_n$与$\mathbf{R}^n$;

        \item 全体复数在实数域上的线性空间$\mathbf{C}(\mathbf{R})$与$\mathbf{R}^2$;

        \item 全体二元复向量$\mathbf{C}^2$在实数域上构成的线性空间$\mathbf{C}^2(\mathbf{R})$与$\mathbf{R}[x]_4$;

        \item 全体二元复向量$\mathbf{C}^2$在复数域上构成的线性空间$\mathbf{C}^2(\mathbf{C})$与$\mathcal{L}(\mathbf{R}^4,\mathbf{R})$.
    \end{enumerate}
\end{example}
\begin{solution}

\end{solution}

\vspace{2ex}
\centerline{\heiti \Large 内容总结}

\vspace{2ex}

\centerline{\heiti \Large 习题}
\vspace{2ex}
{\kaishu }
\begin{flushright}
    \kaishu

\end{flushright}
\centerline{\heiti A组}
\begin{enumerate}
    \item 同构映射的逆、复合仍然是同构映射.
\end{enumerate}
\centerline{\heiti B组}
\begin{enumerate}
    \item
\end{enumerate}
\centerline{\heiti C组}
\begin{enumerate}
    \item
\end{enumerate}

\chapter{线性映射矩阵表示}

\section{线性映射矩阵表示}
\begin{definition}\label{def:7:线性映射矩阵表示}
    设$B_1=\{\varepsilon_1,\varepsilon_2,\ldots,\varepsilon_n\}$是$V_1(F)$的基,$B_2=\{\alpha_1,\alpha_2,\cdots,\alpha_m\}$是$V_2(F)$的基.
    则线性映射$\sigma \in \mathcal{L}(V_1,V_2)$被它作用于基$B_1$的像
    \[\sigma(B_1)=\{\sigma(\varepsilon_1),\sigma(\varepsilon_2),\ldots,\sigma(\varepsilon_n)\}\]
    所唯一确定,而$\sigma(B_1)$是$V_2$的子空间,于是其中元素都可以被基$B_2$线性表示,即
    \[ \left\{
     \begin{array}{rcl}
        \sigma(\varepsilon_1)&=&a_{11}\alpha_1+a_{21}\alpha_2+\ldots+a_{m1}\alpha_m \\
        \sigma(\varepsilon_2)&=&a_{12}\alpha_1+a_{22}\alpha_2+\ldots+a_{m2}\alpha_m \\
        &\vdots& \\
        \sigma(\varepsilon_n)&=&a_{1n}\alpha_1+a_{2n}\alpha_2+\ldots+a_{mn}\alpha_m
     \end{array}
    \right. \]

    我们将$\sigma(B_1)=\{\sigma(\varepsilon_1),\sigma(\varepsilon_2),\ldots,\sigma(\varepsilon_n)\}$
    关于基$B_2$的坐标排列成矩阵$\mathbf{M}(\sigma)$,即
    \[\mathbf{M}(\sigma)=\begin{pmatrix}
        a_{11} & a_{12} & \cdots & a_{1n} \\
        a_{21} & a_{22} & \cdots & a_{2n} \\
        \vdots & \vdots & \ddots & \vdots \\
        a_{m1} & a_{m2} & \cdots & a_{mn}
    \end{pmatrix}\]
\end{definition}
更通俗来说,线性映射矩阵表示就是将线性映射在一组基上的像在另一组基下的坐标表示按列排列得到的结果.
这一整体过程我们也可以用如下记号表示:
\begin{equation}\label{eq:7:线性映射矩阵表示}
    (\sigma(\epsilon_1),\sigma(\epsilon_2),\ldots,\sigma(\epsilon_n))=(\alpha_1,\alpha_2,\ldots,\alpha_m)\mathbf{M}(\sigma).
\end{equation}

\begin{example}\label{ex:7:矩阵表示1}
    已知$\sigma \in \mathcal{L}(\mathbf{R}^3,\mathbf{R}^3)$且$\sigma(x_1,x_2,x_3)=(x_1+x_2,x_1-x_3, x_2)$
    \begin{enumerate}[label=(\arabic*)]
        \item 求$\sigma$的像空间和核空间;

        \item 求$\sigma$关于$\mathbf{R}^3$自然基的矩阵.
    \end{enumerate}
\end{example}
\begin{solution}
    \begin{enumerate}[label=(\arabic*)]
        \item 求像空间和核空间的方法我们在之前已经介绍过,我们为了计算方便取$\mathbf{R}^3$的自然基$e_1,e_2,e_3$计算有:
        \[\im\sigma=\spa(\sigma(e_1),\sigma(e_2),\sigma(e_3))=\spa((1,1,0),(1,0,1),(0,-1,0))=\mathbf{R}^3\]
        对于核空间,解方程$\sigma(\alpha)=0$即可,我们也可以用更简洁的方式书写:
        \[\ker\sigma=\{(x_1,x_2,x_3)\mid \sigma(x_1,x_2,x_3)=(0,0,0)\}=\{(0,0,0)\}\]
        即方程只有零解,核空间可以记为$\ker\sigma=\{0\}$(只含零元的空间的一般记法).
        \item 我们根据\autoref{def:7:线性映射矩阵表示},我们应先写出$\sigma$在出发空间一组基(按题目要求是$\mathbf{R}^3$
        自然基)下的像,并将像表示为到达空间基(按题目要求是$\mathbf{R}^3$自然基)的线性组合,即
        \begin{gather*}
            \sigma(e_1)=(1,1,0)=e_1+e_2=(e_1,e_2,e_3)\begin{pmatrix}
                1 \\ 1 \\ 0
            \end{pmatrix} \\
            \sigma(e_2)=(1,0,1)=e_1+e_3=(e_1,e_2,e_3)\begin{pmatrix}
                1 \\ 0 \\ 1
            \end{pmatrix} \\
            \sigma(e_3)=(0,-1,0)=-e_2=(e_1,e_2,e_3)\begin{pmatrix}
                0 \\ -1 \\ 0
            \end{pmatrix}
        \end{gather*}
        接下来我们把坐标依次按列称矩阵就得到了本题需要求解的矩阵:
        \[\mathbf{M}(\sigma)=\begin{pmatrix}
            1 & 1 & 0 \\
            1 & 0 & -1 \\
            0 & 1 & 0
        \end{pmatrix}\]
    \end{enumerate}
\end{solution}

有趣的是,在结合我个人的学习经历以及过往辅学的经验后,我总结出了第二问的一种常见的错误解法,这里我需要加粗强调,下面这种
解法是\textbf{完全错误的!!!},这里展示这一解法是为了让读者将前面所学的知识完全厘清:

\begin{solution}
    (\textbf{错误解法!!!})$\sigma(x_1,x_2,x_3)=(x_1+x_2,x_1-x_3, x_2)=(x_1,x_2,x_3)\begin{pmatrix}
        1 & 1 & 0 \\
        1 & 0 & 1 \\
        0 & -1 & 0
    \end{pmatrix}$
\end{solution}

我们惊奇地发现,这一结果和我们前面得到的标准答案在向量的排列方式上发生了变化,即标准答案的1、2、3行变为了这里的1、2、3列,
我们需要强调两点:
\begin{enumerate}
    \item 为什么这种解法是错误的:我们可以直接比较\autoref{eq:7:线性映射矩阵表示}和这一解法中,\autoref{eq:7:线性映射矩阵表示}
    的等号左边是$n$个向量在$\sigma$下的像,而上述解法$\sigma(x_1,x_2,x_3)$只是$\sigma$在一个向量下的像,这显然是不一样的!!!
    同样,等号右边括号内\autoref{eq:7:线性映射矩阵表示}是到达空间的一组基,而上述解法中仍然只是一个向量.我们从未定义过这样解题
    的结果是什么,所以千万不能做这种无意义的事!!!

    容易导致混淆的原因可能在于$(x,y,z)$向量是排列成一行的,可能看起来和$(e_1,e_2,e_3)$有点相似,但如果我们将后者拆分成
    $((1,0,0),(0,1,0),(0,0,1))$,你还会混淆吗?

    \item 为什么会出现行列互换这样的错误:事实上
    \[\sigma(x,y,z)=\sigma(xe_1+ye_2+ze_3)=x\sigma(e_1)+y\sigma(e_2)+z\sigma(e_3)=(x,y,z)\begin{pmatrix}
        \sigma(e_1) \\ \sigma(e_2) \\ \sigma(e_3)
    \end{pmatrix},\]
    这里将$\sigma(e_1),\sigma(e_2),\sigma(e_3)$的结果按行排列成矩阵,而标准答案是将$\sigma(e_1),\sigma(e_2),\sigma(e_3)$
    在$\mathbf{R}^3$自然基下的坐标按列排列成矩阵,回忆$\mathbf{R}^n$向量在自然基下坐标是其本身这一性质,标准答案就是将
    $\sigma(e_1),\sigma(e_2),\sigma(e_3)$按列排列成矩阵,由此我们解释了行列互换发生的原因.
\end{enumerate}

这也就是为什么我强调读者不要参考教材102页例3求解像空间的方法来求解像空间——很容易导致这里矩阵表示犯这样的错误,
并且容易导致初学时无法区分求解像空间和线性映射矩阵表示的方法.在这里我必须再次强调:在没有完全熟练掌握这些概念和方法前,
不要乱用方法!!!

还需需要特别强调的一点是,
之后我们会经常看见两种记号,即$(\sigma(\varepsilon_1),\sigma(\varepsilon_2),\ldots,\sigma(\varepsilon_n))$
和$\sigma(\varepsilon_1,\varepsilon_2,\ldots,\varepsilon_n)$.实际上是等价的,等价原因是
$(\sigma(\varepsilon_1),\sigma(\varepsilon_2),\ldots,\sigma(\varepsilon_n))A=(\sigma(\varepsilon_1,\varepsilon_2,\ldots,\varepsilon_n))A=\sigma((\varepsilon_1,\varepsilon_2,\ldots,\varepsilon_n)A)$成立,
这一性质在之后会有运用,证明并不复杂,可以自行尝试或参考我的矩阵辅学授课.

\section{线性映射矩阵表示的进一步讨论}
\subsection{一组简单的例子}
\begin{example}\label{example:5:矩阵表示1}
    已知$\sigma \in \mathcal{L}(\mathbf{R}^3,\mathbf{R}^3)$且$\sigma(x_1,x_2,x_3)=(x_1+x_2,x_1-x_3, x_2)$
    \begin{enumerate}[label=(\arabic*)]
        \item 求$\sigma$的像空间和核空间;

        \item 求$\sigma$关于$\mathbf{R}^3$自然基的矩阵.
    \end{enumerate}
\end{example}

\begin{example}\label{ex:6:矩阵表示2}
    设$A=\begin{pmatrix}1 & 0 & 2 \\ -1 & 2 & 1 \\ 1 & 2 & 5\end{pmatrix}$为两个三维线性空间之间的线性映射$\sigma$对应的矩阵,
    求$\sigma$的像空间和核空间.
\end{example}

\begin{example}\label{ex:6:矩阵表示3}
    已知3阶矩阵$A=\begin{pmatrix}
        1 & 0 & 1 \\ 0 & -1 & 0 \\ -1 & 1 & -1
    \end{pmatrix}$. 定义$\mathbf{F}^{3 \times 3}$上的线性变换$\sigma(X)=AX,\enspace X \in \mathbf{F}^{3 \times 3}$.
    求$\sigma$的像和核.
\end{example}
实际上,例题2.4.1和2.4.3都是属于已知映射求像和核的题目,具体方法在像和核一节已经讲述,并且求矩阵表示也是根据上面的定义
即可,都是程式化的.然而例7则有不同,但此题与例2.4.1、2.4.2也有关联.实际上
此类问题像空间就是以矩阵列空间为坐标的向量的线性扩张,核空间是以矩阵零空间的基(即$AX=0$的基础解系)为坐标的向量的线性扩张,
推导见例7解析或我的矩阵辅学,希望各位同学能掌握推导并理解这三个例题之间的关系与区别. % TODO 编号系统 autoref

\subsection{一些相似的定理}
\begin{theorem} \label{thm:6:线性映射对向量坐标的影响}
    \textbf{\heiti 线性映射对向量坐标的影响}

    设$\sigma \in \mathcal{L}(V_1,V_2)$关于$V_1$和$V_2$的基$B_1$和基$B_2$的矩阵为$A=(a_{ij})_{m \times n}$,
    且$\alpha$与$\sigma(\alpha)$在基$B_1$和基$B_2$下的坐标分别为$X$和$Y$,则$Y=AX$.
\end{theorem}
上述即教材定理4.1,这一定理给出一个向量经过线性映射之后,其坐标的变化. 我们可以用下图表示:

\begin{figure}[h]
    \centering
    \begin{tikzpicture}[>=Stealth]
        \node (V) at (0,0) {$V$};
        \node (W) at (3,0) {$W$};
        \node (Fn) at (0,-3) {$\mathbf{F}^n$};
        \node (Fm) at (3,-3) {$\mathbf{F}^m$};
        \draw[->,thick] (V) -- node[below]{表示矩阵:$A$} (W);
        \draw[<->,thick] (V) -- node[right]{同构} (Fn);
        \draw[<->,thick] (W) -- node[left]{同构} (Fm);
        \draw[->,thick] (Fn) -- node[above]{$\sigma(\alpha)=A\alpha$} (Fm);
    \end{tikzpicture}
\end{figure}

图中我们可以看出通过坐标映射后得到的新映射即为\autoref{thm:6:线性映射对向量坐标的影响} 描述的映射.

在描述下一定理之前,我们首先介绍过渡矩阵(变换矩阵)的概念.
\begin{definition}
    设$B_1=\{\alpha_1,\alpha_2,\ldots,\alpha_n\}$与$B_2=\{\beta_1,\beta_2,\ldots,\beta_n\}$是线性空间
    $V(\mathbf{F})$的任意两组基,$B_2$中每个基向量被基$B_1$表示为
    \[ \left\{
    \begin{array}{rcl}
        \beta_1&=&a_{11}\alpha_1+a_{21}\alpha_2+\cdots+a_{n1}\alpha_n \\
        \beta_2&=&a_{12}\alpha_1+a_{22}\alpha_2+\cdots+a_{n2}\alpha_n \\
        &\vdots& \\
        \beta_n&=&a_{1n}\alpha_1+a_{2n}\alpha_2+\cdots+a_{nn}\alpha_n
    \end{array}
    \right. \]
    将上式用矩阵表示为
    \[(\beta_1,\beta_2,\cdots,\beta_n)=(\alpha_1,\alpha_2,\cdots,\alpha_n)\begin{pmatrix}
        a_{11} & a_{12} & \cdots & a_{1n} \\
        a_{21} & a_{22} & \cdots & a_{2n} \\
        \vdots & \vdots & \ddots & \vdots \\
        a_{n1} & a_{n2} & \cdots & a_{nn}
    \end{pmatrix}\]
    我们将这一矩阵称为即$B_1$变为基$B_2$的变换矩阵(或过渡矩阵).
\end{definition}
简单而言就是将$B_2$中的向量在$B_1$下的坐标按列排列.需要注意表述中是$B_1$变为基$B_2$还是反过来,
这两个矩阵互逆.注意过渡矩阵一定是基与基之间的表示矩阵,并且过渡矩阵一定可逆.
\begin{theorem}
    \textbf{\heiti 基的选择对向量坐标的影响}

    设线性空间$V$的两组基为$B_1$和$B_2$,且基$B_1$到$B_2$的变换矩阵(过渡矩阵)为$A$,如果
    $\xi \in V(\mathbf{F})$,且在$B_1$和$B_2$下的坐标分别为$X$和$Y$,则$Y=A^{-1}X$.
\end{theorem}
上述即教材定理4.10,描述同一个向量在不同基下坐标之间的关系.事实上,这与本节同构关系紧密,因为
同构意味着两个线性空间结构一致,故同构映射可以保持向量组的线性关系不变.在同构关系下,
线性组合对应线性组合,线性无关对应线性无关,线性相关对应线性相关.我们有如下定理:
\begin{theorem}
    设$(\alpha_1,\alpha_2,\ldots,\alpha_n)$是线性无关的向量组,且
    \[(\beta_1,\beta_2,\ldots,\beta_s)=(\alpha_1,\alpha_2,\ldots,\alpha_n)A\]
    则向量组$(\beta_1,\beta_2,\ldots,\beta_s)$的秩等于矩阵$A$的秩.
\end{theorem}
定理的证明需要用到坐标映射是同构映射这一事实,我们不难发现等式左侧向量组与$A$的列向量组是等价的.
事实上我们也可以由此发现,过渡矩阵一定是可逆矩阵.
\begin{theorem}
    已知$\beta_i=a_{1i}\alpha_1+a_{2i}\alpha_2+\cdots+a_{ni}\alpha_n\enspace(i=1,2,\ldots,n)$,
    且$A=(a_{ij})$可逆,则$\alpha_1,\alpha_2,\ldots,\alpha_n$与$\beta_1,\beta_2,\ldots,\beta_n$
    等价.
\end{theorem}
实际上这一定理与上一定理的思想都是类似的,我们可以看一个例题练习一下:
\begin{example}
    已知$\beta_1=\alpha_2+\alpha_3,\enspace\beta_2=\alpha_1+\alpha_3,\enspace\beta_3=\alpha_1+\alpha_2$,
    证明$\alpha_1,\alpha_2,\alpha_3$与$\beta_1,\beta_2,\beta_3$等价.
\end{example}
\begin{theorem}
    \textbf{\heiti 基的选择对映射矩阵的影响}

    设线性变换$\sigma \in \mathcal{L}(V,V)$,$B_1=\{\alpha_1,\ldots,\alpha_n\}$和$B_2=\{\beta_1,\ldots,\beta_n\}$
    是线性空间的$V(\mathbf{F})$的两组基,基$B_1$变为基$B_2$的变换矩阵为$C$,如果$\sigma$在基$B_1$下的矩阵为$A$,
    则$\sigma$关于基$B_2$所对应的矩阵为$C^{-1}AC$.
\end{theorem}
上述即教材定理7.4,研究同一个映射在不同基下表示矩阵之间的关系.实际上我们将在下一专题初等矩阵一节进一步讨论.
这一定理的证明需要用到我们之前描述的两种线性映射矩阵表示的统一性.

\vspace{2ex}
\centerline{\heiti \Large 内容总结}

\vspace{2ex}

\centerline{\heiti \Large 习题}
\vspace{2ex}
{\kaishu }
\begin{flushright}
    \kaishu

\end{flushright}
\centerline{\heiti A组}
\begin{enumerate}
    \item 设$\sigma: V_1\to V_2$是线性映射,证明:$\sigma(W_1)$和$\sigma^{-1}(W_2)$分别是$V_2$和$V_1$的子空间
\end{enumerate}
\centerline{\heiti B组}
\begin{enumerate}
    \item
\end{enumerate}
\centerline{\heiti C组}
\begin{enumerate}
    \item
\end{enumerate}

\chapter{线性映射的应用}

\section{线性空间的积}


\section{线性空间的商}


\section{对偶}


\vspace{2ex}
\centerline{\heiti \Large 内容总结}

\vspace{2ex}

\centerline{\heiti \Large 习题}
\vspace{2ex}
{\kaishu }
\begin{flushright}
    \kaishu

\end{flushright}
\centerline{\heiti A组}
\begin{enumerate}
    \item
\end{enumerate}
\centerline{\heiti B组}
\begin{enumerate}
    \item
\end{enumerate}
\centerline{\heiti C组}
\begin{enumerate}
    \item
\end{enumerate}

\chapter{矩阵基本运算}

\section{矩阵基本运算}
\subsection{基本概念}
\begin{enumerate}
    \item 矩阵的加法来源于线性映射的加法,矩阵相加要求两矩阵行列数一致,相加时只需对应位置元素相加即可;
    \item 矩阵的数乘来源于线性映射的数乘,计算只需矩阵的每个元素乘以常数即可;
    \item 矩阵的乘法来源于线性映射的复合,计算时要求前一个矩阵的列数等于后一个矩阵的行数,矩阵$A$与$B$
    相乘结果中第$i$行第$j$列元素为矩阵$A$的第$i$行与矩阵$B$的第$j$列对应位置元素相乘后求和的结果,
    即对于$A=(a_{ij})_{m \times n}$和$B=(b_{ij})_{n \times l}$,矩阵$C=AB=(c_{ij})_{m \times l}$,且
    $c_{ij}=a_{i1}b_{1j}+a_{i2}b_{2j}+\cdots+a_{in}b_{nj}\enspace(i=1,\ldots,m,\enspace j=1,\ldots,l)$.
\end{enumerate}

\subsection{基本性质}
\begin{enumerate}
    \item 回顾上一专题中$m \times n$矩阵构成的线性空间$\mathbf{M}_{m \times n}(\mathbf{F})$;
    \item 回顾矩阵乘法的基本性质:
    \begin{enumerate}[label=(\arabic*)]
        \item $(AB)C=A(BC)$(结合律)
        \item $\lambda(AB)=(\lambda A)B=A(\lambda B),\enspace \lambda \in \mathbf{F}$
        \item $A(B+C)=AB+AC$(左分配律)
        \item $(B+C)P=BP+CP$(右分配律)
        \item $A^kA^m=A^{k+m},\enspace (A^k)^m=A^{km}$,其中$A$为方阵,$k,m$为任意整数. 负整数对应于逆矩阵的情况.
    \end{enumerate}
    \item 回顾矩阵多项式的定义(利用线性映射多项式在基下的矩阵表示定义),
    并注意其交换性以及可因式分解性.
\end{enumerate}
\begin{example}
    展开矩阵多项式$(A+\lambda E)^n$.
\end{example}
\begin{example}
    设$f(x),g(x) \in \mathbf{F}[x],\enspace A,B \in \mathbf{M}_n(\mathbf{F})$. 证明:
    \[f(A)g(A)=g(A)f(A)\]
    \begin{enumerate}
        \item 如果$AB=BA$,则$f(A)g(B)=g(B)f(A)$;

        \item 设$f(x)=1+x+\cdots+x^{m-1}$,$g(x)=1-x$,$A=\begin{pmatrix}
            a & b \\ 0 & a
        \end{pmatrix}$,计算$f(A)g(A)$.
    \end{enumerate}
\end{example}
其他需要注意的性质:
\begin{enumerate}
    \item 矩阵乘法不一定满足交换律(即$AB$不一定等于$BA$).但是注意数量矩阵和任何矩阵相乘都是可交换的,因此求矩阵的幂次时,可以将其转化为$(A+\mu E)^n$(其中$E$为单位矩阵,$\mu$为常数)类型,然后利用二项式展开即可.很多情况下$A$都会是幂零矩阵,此时结果为有限项.
    \item $A\neq O$且$B\neq O$不能推出$AB\neq O$.例如线性方程组$AX = 0$有非零解,若$B$的各列均为方程非零解,则$AB = O$.
    \item 消去律也不一定满足:即$AB = AC$不一定$A = B$.原因在于$AB=AC \implies A(B-C)=O$,由(2)可知不一定$B = C$.
\end{enumerate}

\subsection{矩阵可交换问题}
一般来说在本课程中此类问题直接设可交换矩阵的每一个元素都是未知数即可,一些特殊的技巧
(使用关于一些特殊形状矩阵的结论)以及涉及到之后才能学到的知识的方法我们在这里也不展开了.我们只讨论一个基本的技巧,即
\[\forall t,\enspace AB=BA \iff (A-tE)B=B(A-tE)\]
此处的$t$根据矩阵的对角线上元素来决定,原则是使得其余矩阵与$A-tE$相乘的计算过程更为简单(一般是使得0元素更多),这样解方程也会更轻松.
我们来看一个简单的例子:
\begin{example}
    求与矩阵$A=\begin{pmatrix}
        3 & 0 & 0 \\ -1 & 3 & 0 \\ 0 & -1 & 3
    \end{pmatrix}$可交换的矩阵.
\end{example}

关于可交换我们有以下定理,证明并不是很复杂(教材习题中有出现):
\begin{theorem}
    \begin{enumerate}
        \item 与主对角元两两互异的对角矩阵可交换的方阵只能是对角矩阵;

        \item 准对角矩阵$A$每个对角分块内对角线元素相同,但不同对角块之间不同,则与$A$可交换的矩阵只能是准对角矩阵;

        \item 与所有$n$级可逆矩阵可交换的矩阵为数量矩阵;

        \item 与所有$n$级矩阵可交换的矩阵为数量矩阵.
    \end{enumerate}
\end{theorem}

\section{矩阵转置}
\subsection{基本概念}
实际上,矩阵的转置就是第$i$行变成了第$i$列,或者抽象表达为:
\[A=(a_{ij})_{m \times n},\enspace A^\mathrm{T}=(a'_{ji})_{n \times m},\enspace a_{ij}=a'_{ji}\]
写成矩阵形式就是:
\begin{definition}
    设$A=\begin{pmatrix}
        a_{11} & a_{12} & \cdots & a_{1n} \\
        a_{21} & a_{22} & \cdots & a_{2n} \\
        \vdots & \vdots & \ddots & \vdots \\
        a_{m1} & a_{m2} & \cdots & a_{mn}
    \end{pmatrix}$,称$\begin{pmatrix}
        a_{11} & a_{21} & \cdots & a_{m1} \\
        a_{12} & a_{22} & \cdots & a_{m2} \\
        \vdots & \vdots & \ddots & \vdots \\
        a_{1n} & a_{2n} & \cdots & a_{mn}
    \end{pmatrix}$为矩阵$A$的转置,记作$A^\mathrm{T}$.
\end{definition}

\subsection{基本性质}
\begin{enumerate}
    \item $(A^\mathrm{T})^\mathrm{T}=A$

    \item $(A+B)^\mathrm{T}=A^\mathrm{T}+B^\mathrm{T}$

    \item $(\lambda A)^\mathrm{T}=\lambda A^\mathrm{T},\enspace \lambda \in \mathbf{F}$

    \item $(AB)^\mathrm{T}=B^\mathrm{T}A^\mathrm{T}$,$(A_1A_2\cdots A_n)^\mathrm{T}=A_n^\mathrm{T}\cdots A_2^\mathrm{T}A_1^\mathrm{T}$

    \item $(A^\mathrm{T})^{-1}=(A^{-1})^\mathrm{T}$

    \item $(A^\mathrm{T})^m=(A^m)^\mathrm{T}$
\end{enumerate}

以上证明大都是平凡的,可以自己尝试完成.
\subsection{对阵矩阵与反对称矩阵}
\begin{definition}
    设$A=(a_{ij})_{n \times n}$,如果$\forall i,j=1,2,\ldots,n$均有$a_{ij}=a_{ji}$,
    则称$A$为对称矩阵. 若均有$a_{ij}=-a_{ji}$,则称$A$为反对称矩阵.
\end{definition}
易得$A$为对称矩阵的充要条件为$A=A^\mathrm{T}$,$A$为反对称矩阵的充要条件为$A=-A^\mathrm{T}$.
\begin{example}
    证明以下几点性质:
    \begin{enumerate}
        \item 反对称矩阵主对角元均为0;

        \item $AA^\mathrm{T}$和$A^\mathrm{T}A$均为对称矩阵;

        \item 设$A,B$为$n$阶对称和反对称矩阵,则$AB+BA$是反对称矩阵;

        \item 对称矩阵的乘积不一定对称;

        \item 可逆的对称(反对称)矩阵的逆矩阵也是对称(反对称)矩阵.
    \end{enumerate}
\end{example}

\section{初等矩阵}
\subsection{基本概念与性质}
\begin{definition}
    将单位矩阵$E$做一次初等变换得到的矩阵称为初等矩阵,与三种初等行、列变换对应的三类初等矩阵为:
    \begin{enumerate}
        \item 将单位矩阵第$i$行(或列)乘$c$,得到初等倍乘矩阵$E_i(c)$;

        \item 将单位矩阵第$i$行乘$c$加到第$j$行,或将第$j$列乘$c$加到第$i$列,得到初等倍加矩阵$E_{ij}(c)$;

        \item 将单位矩阵第$i,j$行(或列)对换,得到初等对换矩阵$E_{ij}$.
    \end{enumerate}
\end{definition}
请各位同学以矩阵形式写出以上三类矩阵.注意:
\begin{enumerate}
    \item 倍加变化请一定注意$i$和$j$在行列的情况下的不同;

    \item 三类矩阵不是三个矩阵,例如行列选择不唯一,常数选择不唯一;

    \item 注意三种初等矩阵都是可逆的,且$E_i^{-1}(c)=E_i\left(\dfrac{1}{c}\right)$,$E_{ij}^{-1}(c)=E_{ij}(-c)$,$E_{ij}^{-1}=E_{ij}$;

    \item 三种初等矩阵的转置:$E_i^\mathrm{T}(c)=E_i(c)$,$E_{ij}^\mathrm{T}(c)=E_{ji}(c)$,$E_{ij}^\mathrm{T}=E_{ij}$;
\end{enumerate}

初等矩阵大家非常关心为什么左乘代表行变换,右乘代表列变换.以右乘为例,我们来看矩阵$A$和$B$相乘的任一列结果.我们可以将矩阵$A$
按列做分块矩阵得到$\begin{pmatrix}\alpha_1,\ldots,\alpha_n\end{pmatrix}$,$\alpha_i$即表示$A$的第$i$列.然后矩阵$B$的第$j$列为列向量$(x_1,\ldots,x_n)^\mathrm{T}$,
由于矩阵$A$与$B$相乘结果第$j$列就是$A$与$B$的第$j$列相乘结果(回顾矩阵乘法的计算方式),则有$B$的第$i$列等于
$x_1\alpha_1+\cdots+x_n\alpha_n$即为$A$的全部列向量的线性组合,故右乘矩阵$A$得到矩阵的任一列都是$A$的全部列向量的线性组合,
所以右乘可以代表列变换.注意我这里并没有限制矩阵$B$为初等矩阵或可逆矩阵.

实际上左乘表示行变换可以用类似方法说明,只需按行对$B$分块即可.这一思想是特别重要的,在很多时候如果我们意识到左右乘是对被乘矩阵的行列
重新线性组合,思路会清晰很多.

关于初等矩阵还有一个相当重要的定理:

\begin{theorem}
    任意可逆矩阵都可以被表示为若干个初等矩阵的乘积.
\end{theorem}
定理证明只需要回忆高斯消元法可以将可逆矩阵化为单位矩阵即可.

利用矩阵初等变换我们可以获得本学期需要学习的三个矩阵标准形,因此这一内容虽然很基本但是非常重要:
\begin{enumerate}
    \item 相抵矩阵:本章已学习的内容,在之后会详细说明;
    \item 相似矩阵:若$P$为初等矩阵,对矩阵做$P^{-1}AP$变换即可得到与$A$相似的矩阵;
    \item 相合矩阵:两个矩阵,其中一个可以通过做相同的初等行列变换的到另一个矩阵(若$P$为初等矩阵,
    $P^{\mathrm{T}}AP$就是对$A$做了一次相同的初等行列变换).
\end{enumerate}
请同学们思考:如何从线性映射矩阵表示的角度理解初等变换与标准形的关系?在B组习题中将有练习进行体会
(实际上对矩阵表示的基做``初等变换''就是对表示矩阵做了初等变换,这两种变换行列方向不一致且矩阵互逆).

\section{矩阵的逆}
\subsection{基本概念}
\begin{definition}
    设$A \in \mathbf{M}_n(\mathbf{F})$. 若存在$B \in \mathbf{M}_n(\mathbf{F})$使得$AB=BA=E$,则称矩阵$A$可逆,
    并把$B$称为$A$的逆矩阵,记作 $ B = A^{-1} $.
\end{definition}
注意,逆矩阵定义基于方阵,非方阵没有上述逆矩阵.广义逆矩阵允许非方阵,但那是另一个定义,
我们不需要掌握.对于可逆矩阵,注意以下两个定理:
\begin{theorem}
    可逆矩阵$A$的逆矩阵唯一.
\end{theorem}
\begin{theorem}
    $AB=E \iff A$与$B$互为逆矩阵.
\end{theorem}
这两个定理的证明教材中有,特别注意唯一性的证明,反证法的思路一定要掌握,十分经典.
还需要强调的一点是,逆矩阵来源于逆映射.
\subsection{基本性质}
\begin{enumerate}
    \item 注意没有加法性质(请举出反例),对于数乘有$(\lambda A)^{-1}=\lambda^{-1}A^{-1}$;

    \item $(AB)^{-1}=B^{-1}A^{-1},\enspace (A_1A_2\cdots A_k)^{-1}=A_k^{-1}\cdots A_2^{-1}A_1^{-1}$;

    \item $(A^k)^{-1}=(A^{-1})^k,\enspace A^kA^m=A^{k+m},\enspace (A^k)^m=A^{km}$;

    \item 若$A$和$B$可逆,则$A\neq O$且$B\neq O$能推出$AB\neq O$,并且$A$可逆且$AB=O$可以推出$B=O$,除此之外还有消去律成立,即$A$则有$AB=AC \implies B=C$成立.
\end{enumerate}

还需要熟练掌握可逆矩阵的几个等价条件:
\begin{theorem}
    设$A \in \mathbf{M}_n{\mathbf{F}}$,则下列命题等价:
    \begin{enumerate}
        \item $A$可逆;

        \item $r(A)=n$;

        \item $A$的$n$个行(列)向量线性无关;

        \item 齐次线性方程组$AX=0$只有零解;

        \item $|A|\neq 0$.
    \end{enumerate}
\end{theorem}
\begin{example}
    已知矩阵 $A=\begin{pmatrix}a & b & c \\ d & e & f \\ h & x & y\end{pmatrix}$ 的逆是 $A^{-1}=\begin{pmatrix}-1 & -2 & -1 \\ 2 & 1 & 0 \\ 0 & -3 & -1\end{pmatrix}$,

$B=\begin{pmatrix}a-2b & b-3c & -c \\ d-2e & e-3f & -f \\ h-2x & x-3y & -y\end{pmatrix}$.求矩阵 $X$ 满足:

\[X+\left(B(A^TB^2)^{-1}A^T\right)^{-1}=X\left(A^2(B^TA)^{-1}B^T\right)^{-1}(A+B)\]
\end{example}

\subsection{逆矩阵的求解(基本方法)}
\begin{enumerate}
    \item 利用解线性方程组的方法:假设$AX=b$,使用高斯消元法求解;

    \item 利用初等矩阵的方法(初等行变换为常用方法).
\end{enumerate}

注意,基于初等变换的方法是非常重要的,我们很多时候不要被题目吓到去采用其他
偏门的方法,实际上很多时候拿到一个具体的矩阵求逆,使用的方法就是初等行变换.

\begin{example}
    用上述两种方法求矩阵$A=\begin{pmatrix}1 & -1 & 1 \\ 0 & 1 & 2 \\ 1 & 0 & 4\end{pmatrix}$的逆矩阵.
\end{example}

\subsection{矩阵方程}
\begin{enumerate}
    \item 考虑以下情形(其中出现的矩阵除$X$外均可逆,$X$不一定是列向量):
    \begin{enumerate}[label=(\arabic*)]
        \item $AX=B \implies X=A^{-1}B, \enspace XA=B \implies X=BA^{-1}$;
        \item $AXB=C \implies X=A^{-1}CB^{-1}$;
    \end{enumerate}
    \item 考虑以下情形:$AX=B$但$A$不可逆($X$不一定是列向量),直接高斯消元即可;
    \item 考虑以下求解方式的合理性:
    \begin{enumerate}[label=(\arabic*)]
        \item 若求$A^{-1}$,只需对$(A,E)$只做初等行变换,可以得到$(E,A^{-1})$;
        \item 若求$A^{-1}B$,只需对$(A,B)$只做初等行变换,可以得到$(E,A^{-1}B)$;
        \item 若求$BA^{-1}$,只需对$\begin{pmatrix}
            A \\ B
        \end{pmatrix}$只做初等列变换,可以得到$\begin{pmatrix}
            E \\ BA^{-1}
        \end{pmatrix}$;
        \item 对$\begin{pmatrix}
            A & E \\ E & O
        \end{pmatrix}$的前$n$行与$n$列做相同的行列变换,可以得到$\begin{pmatrix}
            P^\mathrm{T}AP & P^\mathrm{T} \\ P & O
        \end{pmatrix}$.
    \end{enumerate}
\end{enumerate}

\begin{example}
    设$A=\begin{pmatrix}1 & 0 & 0 \\ 1 & 1 & 0 \\ 1 & 1 & 1\end{pmatrix},\
    B=\begin{pmatrix}0 & 1 & 1 \\ 1 & 0 & 1 \\ 1 & 1 & 0\end{pmatrix}$,求矩阵$X$满足:
    \[AXA+BXB=AXB+BXA+A(A-B)\]
\end{example}

\vspace{2ex}
\centerline{\heiti \Large 内容总结}

\vspace{2ex}

\centerline{\heiti \Large 习题}
\vspace{2ex}
{\kaishu }
\begin{flushright}
    \kaishu

\end{flushright}
\centerline{\heiti A组}
\begin{enumerate}
    \item
\end{enumerate}
\centerline{\heiti B组}
\begin{enumerate}
    \item
\end{enumerate}
\centerline{\heiti C组}
\begin{enumerate}
    \item
\end{enumerate}

\chapter{矩阵运算进阶}

\section{分块矩阵}
\subsection{运算性质}
\begin{definition}
    一般的,对于$m \times n$矩阵$A$,如果在行的方向分成$s$块,在列的方向分成$t$
    块,就得到$A$的一个$s \times t$分块矩阵,记作$A=(A_{kl})_{s \times t}$,其中
    $A_{kl}\enspace(k=1,\ldots,s,\enspace l=1,\ldots,t)$称为$A$的子块.
\end{definition}
实际上上述表示方法就是将一般矩阵表示$A=(a_{ij})_{m \times n}$中的$a_{ij}$替换为了小块矩阵,
字母含义并无变化,内层代表索引,外层代表总行列数(只是分块矩阵是块索引和块数).
我们接下来考察分块矩阵的运算性质.
\begin{enumerate}
    \item 分块矩阵的加法:设分块矩阵$A=(A_{kl})_{s \times t}$,$B=(B_{kl})_{s \times t}$,如果$A$与$B$
    对应的子块$A_{kl}$和$B_{kl}$都是同型矩阵,则\[A+B=(A_{kl}+B_{kl})_{s \times t}\]
    由此我们看到分块矩阵加法要求小块形状和行列分块数都一致,实际上回顾一般矩阵加法要求矩阵完全同型即可理解这一要求.

    \item 分块矩阵的数乘:设分块矩阵$A=(A_{kl})_{s \times t}$,$\lambda$是一个数,则
    \[\lambda A=(\lambda A_{kl})_{s \times t}\]
    实际上数乘最好理解,因为如此计算的效果相当于一般矩阵数乘的效果,即给每个元素
    都乘以一个常数$\lambda$.

    \item 分块矩阵的乘法:设$A=(a_{ij})_{m \times n}$,$B=(b_{ij})_{n \times p}$,如果
    把$A$,$B$分别分块为$r \times s$和$s \times t$分块矩阵,且$A$的列分块法与$B$的行分块法相同
    (注意这些条件始终保证可乘性成立),则
    \[AB=\begin{pmatrix}
        A_{11} & A_{12} & \cdots & A_{1s} \\
        A_{21} & A_{22} & \cdots & A_{2s} \\
        \vdots & \vdots & \ddots & \vdots \\
        A_{r1} & A_{r2} & \cdots & A_{rs}
    \end{pmatrix}\begin{pmatrix}
        B_{11} & B_{12} & \cdots & B_{1t} \\
        B_{21} & B_{22} & \cdots & B_{2t} \\
        \vdots & \vdots & \ddots & \vdots \\
        B_{s1} & B_{s2} & \cdots & B_{st}
    \end{pmatrix}=C=(C_{kl})_{r \times t}\]
    其中$C$是$r \times t$分块矩阵,且$C_{kl}$与一般矩阵计算类似,即为$A$第$k$行块$B$的$l$列块对应元素相乘后相加,即
    \[C_{kl}=A_{k1}B_{1l}+A_{k2}B_{2l}+\cdots+A_{ks}B_{sl},\enspace k=1,\ldots,r,\enspace l=1,\ldots,t\]

    \item 分块矩阵的转置:大、小矩阵都要转置,这是分块矩阵与普通矩阵的一大性质差异;即$s \times t$分块矩阵$A=(A_{kl})_{s \times t}$
    转置后$A^\mathrm{T}=(B_{lk})_{t \times s}$为$t \times s$分块矩阵,且$B_{lk}=A_{kl}^\mathrm{T}$.
    例如$\begin{pmatrix}
        A_{11} & A_{12} \\ A_{21} & A_{22}
    \end{pmatrix}^\mathrm{T}=\begin{pmatrix}
        A_{11}^\mathrm{T} & A_{21}^\mathrm{T} \\ A_{12}^\mathrm{T} & A_{22}^\mathrm{T}
    \end{pmatrix}$.
\end{enumerate}

补充以下注意事项:
\begin{enumerate}
    \item 常见的行列分块方法:将矩阵按行/列分块,注意$A(\beta_1,\ldots,\beta_n)=(A\beta_1,\ldots,A\beta_n)$成立,
    但当$A$在右侧时并不可乘,按行分块也有对称的结论;

    \item 注意分块矩阵求逆,可以直接使用设未知数的方式完成,也可以利用下面即将介绍的分块矩阵初等变换进行解决;

    \item 分析分块矩阵与普通矩阵的运算性质的异同:分块矩阵转置需要注意大小都要转置,注意分块矩阵每一块仍为矩阵,所以当普通矩阵元素的求倒数
    对应于小块的求逆,加法乘法一定要块对应等,但实际上其他很多性质都是将单个元素推广为一块.
\end{enumerate}

\begin{example}
    设\[A=\begin{pmatrix}
        1 & 2 & 0 & 0 & 0 \\
        2 & 5 & 0 & 0 & 0 \\
        0 & 0 & -2 & 1 & 0 \\
        0 & 0 & 0 & -2 & 1 \\
        0 & 0 & 0 & 0 & -2
    \end{pmatrix},\enspace B=\begin{pmatrix}
        1 & 0 & 1 & 0 \\
        -1 & 2 & 3 & 0 \\
        1 & 2 & 0 & 4 \\
        0 & 1 & 2 & 4 \\
        0 & 0 & 1 & 4
    \end{pmatrix}\]
    利用分块矩阵的方法,求$A^2,\enspace AB,\enspace A^\mathrm{T},\enspace A^{-1}$.
\end{example}

\subsection{分块矩阵初等变换(打洞法)*}
分块矩阵的初等变换实际上可以视为一般矩阵初等变换的推广,实际上也有三种相应的推广形式,
即交换两行、对某一行乘以一个可逆矩阵以及对某一行左乘矩阵后加到另一行.它们的计算性质
以及可逆性质的证明比较繁琐,我们这里略去,直接应用即可.实际使用的时候,很多时候都是
使用一种将分块矩阵中的小块视为常数来处理.

分块矩阵初等行变换的一个重要的应用就是``打洞法'',常用于分块矩阵求逆的运算,在之后行列式的一些技巧性处理中也很常见.
例如:
\begin{enumerate}
    \item 当$A$可逆时,我们可以通过初等行变换消去$C$:
    \[ \begin{pmatrix}
        E & O \\ -CA^{-1} & E
    \end{pmatrix}\begin{pmatrix}
        A & B \\ C & D
    \end{pmatrix}=\begin{pmatrix}
        A & B \\ O & D-CA^{-1}B
    \end{pmatrix} \]
    可以继续做列变换消去$B$:
    \[ \begin{pmatrix}
        A & B \\ O & D-CA^{-1}B
    \end{pmatrix}\begin{pmatrix}
        E & -A^{-1}B \\ O & E
    \end{pmatrix}=\begin{pmatrix}
        A & O \\ O & D-CA^{-1}B
    \end{pmatrix} \]
    \item 特别地,对于对称矩阵$\begin{pmatrix}A & B \\ B^\mathrm{T} & D\end{pmatrix}$,其中$A$和$D$也是对称方阵,
    则$A$可逆时,可以通过合同变换消除$B$和$B^\mathrm{T}$,即
    \[ \begin{pmatrix}
        E & -A^{-1}B \\ O & E
    \end{pmatrix}^\mathrm{T}\begin{pmatrix}
        A & B \\ B^\mathrm{T} & D
    \end{pmatrix}\begin{pmatrix}
        E & -A^{-1}B \\ O & E
    \end{pmatrix}=\begin{pmatrix}
        A & O \\ O & D-B^\mathrm{T}A^{-1}B
    \end{pmatrix} \]
\end{enumerate}
接下来求取逆矩阵就很容易了,因为分块对角矩阵求逆矩阵就是对每个小对角块求逆,十分简单,所以解决此类问题
首先要利用分块矩阵初等变换进行对角化(一定注意区分行列变换的左右乘),然后如果$PAQ=\Lambda$,
其中$P$和$Q$为分块初等矩阵,$\Lambda$为分块对角矩阵,利用分块对角矩阵的逆容易计算的
特点计算$Q^{-1}A^{-1}P^{-1}=\Lambda^{-1}$,即可得到$A^{-1}=Q\Lambda^{-1}P$.
\begin{example}
    当$D$可逆时,仿照上面的步骤对角化分块矩阵$\begin{pmatrix}A & B \\ C & D\end{pmatrix}$并求逆矩阵.
\end{example}

\subsection{分块矩阵与数学归纳法}
分块矩阵经常运用在数学归纳法中,我们在之后的课程中也会经常用到这样的思想,
这一思想基于以下内容:

对于$\begin{pmatrix}
    A_1 & \alpha \\ \beta & a_{nn}
\end{pmatrix}$,假设$A_1$可逆,我们有
\[\begin{pmatrix}
    E_{n-1} & 0 \\ -\beta A_1^{-1} & 1
\end{pmatrix}\begin{pmatrix}
    A_1 & \alpha \\ \beta & a_{nn}
\end{pmatrix}=\begin{pmatrix}
    A_1 & \alpha \\ 0 & a_{nn}-\beta A_1^{-1}\alpha
\end{pmatrix}\]

\begin{example}
    若$n$阶矩阵$A$的各阶左上角子块矩阵都可逆,则存在主对角元全为1的下三角矩阵$L$和上三角矩阵$U$,使得$A=LU$($L$-$U$分解).
\end{example}

\section{特殊矩阵}
本节将会介绍一些常见的特殊矩阵以及它们常用的基本性质,还有一些将在特征值专题中讲解.
\subsection{对角矩阵}
我们一般记主对角矩阵为$\diag(d_1,d_2,\dots,d_n)$,准对角矩阵为$\diag(A_1,A_2,\dots,A_n)$.
下面是对角矩阵的一个基本定理,它很简单,但是很重要:
\begin{theorem}
    设$A$是一个$s \times n$矩阵,把$A$写成列向量与行向量的形式,分别为

    \[ A = \begin{pmatrix}\alpha_1 & \alpha_2 & \cdots & \alpha_n\end{pmatrix} = \begin{pmatrix} \beta_1 \\ \beta_2 \\ \vdots \\ \beta_n \end{pmatrix} \]
    则
    \begin{gather*}
        \begin{pmatrix}\alpha_1 & \alpha_2 & \cdots & \alpha_n\end{pmatrix}
        \begin{pmatrix}
            d_1 & & & \\
            & d_2 & & \\
            & & \ddots & \\
            & & & d_n
        \end{pmatrix} = \begin{pmatrix}d_1\alpha_1 & d_2\alpha_2 & \cdots & d_n\alpha_n\end{pmatrix} \\
        \begin{pmatrix}
            d_1 & & & \\
            & d_2 & & \\
            & & \ddots & \\
            & & & d_n
        \end{pmatrix} \begin{pmatrix} \beta_1 \\ \beta_2 \\ \vdots \\ \beta_n \end{pmatrix} = \begin{pmatrix} d_1\beta_1 \\ d_2\beta_2 \\ \vdots \\ d_n\beta_n \end{pmatrix}
    \end{gather*}

    即$A$右乘对角矩阵$\diag(d_1,d_2,\ldots,d_n)$相当于给$A$的第$i$列元素都乘以$d_i$,
    $A$左乘对角矩阵$\diag(d_1,d_2,\ldots,d_n)$相当于给$A$的第$i$行元素都乘以$d_i$.
\end{theorem}
\begin{theorem}
    (请自行完成以下内容的补充)

    对角矩阵以及准对角矩阵的三则运算、可逆性以及逆运算、乘方运算等规则.
\end{theorem}

\subsection{上(下)三角矩阵}
\begin{theorem}
    已知$A,B$都是上三角矩阵,且设$A$的主对角元素分别为$a_{11},\ldots,a_{nn}$,B的主对角元素分别为
    $b_{11},\ldots,b_{nn}$,则
    \begin{enumerate}
        \item $A^{\mathrm{T}}, B^\mathrm{T}$都是下三角矩阵;

        \item $AB$仍然是上三角矩阵,且$AB$的主对角元素为$a_{11}b_{11},\ldots,a_{nn}b_{nn}$;

        \item $A$可逆的充要条件是其主对角元均不为0,且$A$可逆时,$A^{-1}$也是上三角矩阵,并且$A^{-1}$的主对角元素分别为$a_{11}^{-1},\ldots,a_{nn}^{-1}$.
    \end{enumerate}
\end{theorem}

\begin{example}
    已知$A_1,\ldots,A_n$是$n$个对角元都为0的上三角矩阵,证明:$A_1A_2\cdots A_n=O$.
\end{example}

\subsection{基本矩阵}
只有一个元素为1,其余元素全为0的矩阵称为基本矩阵,第$i$行第$j$列元素为1的基本矩阵记为$E_{ij}$,
他们具有如下性质(可以回忆左右乘对应行列变换):
\begin{theorem}
    \begin{enumerate}
        \item $AE_{ij}$的结果就是把$A$的第$i$列移到第$j$列的位置,其余元素都为0的矩阵;

        \item $E_{ij}B$的结果就是把$B$的第$j$行移到第$i$行的位置,其余元素都为0的矩阵;

        \item $E_{ik}E_{kj}=E_{ij}$,当$k \neq l$时,有$E_{ik}E_{lj}=O$.
    \end{enumerate}
\end{theorem}

\subsection{其他矩阵}
其他矩阵如正交矩阵、置换矩阵、幂等矩阵、幂零矩阵等,有的会在稍后介绍部分性质,有的
则会在课程进行中或者后续课程中再见到它们.

\section{矩阵可交换问题}
一般来说在本课程中此类问题直接设可交换矩阵的每一个元素都是未知数即可,一些特殊的技巧
(使用关于一些特殊形状矩阵的结论)以及涉及到之后才能学到的知识的方法我们在这里也不展开了.我们只讨论一个基本的技巧,即
\[\forall t,\enspace AB=BA \iff (A-tE)B=B(A-tE)\]
此处的$t$根据矩阵的对角线上元素来决定,原则是使得其余矩阵与$A-tE$相乘的计算过程更为简单(一般是使得0元素更多),这样解方程也会更轻松.
我们来看一个简单的例子:
\begin{example}
    求与矩阵$A=\begin{pmatrix}
        3 & 0 & 0 \\ -1 & 3 & 0 \\ 0 & -1 & 3
    \end{pmatrix}$可交换的矩阵.
\end{example}

关于可交换我们有以下定理,证明并不是很复杂(教材习题中有出现):
\begin{theorem}
    \begin{enumerate}
        \item 与主对角元两两互异的对角矩阵可交换的方阵只能是对角矩阵;

        \item 准对角矩阵$A$每个对角分块内对角线元素相同,但不同对角块之间不同,则与$A$可交换的矩阵只能是准对角矩阵;

        \item 与所有$n$级可逆矩阵可交换的矩阵为数量矩阵;

        \item 与所有$n$级矩阵可交换的矩阵为数量矩阵.
    \end{enumerate}
\end{theorem}

\section{矩阵方程}
\begin{enumerate}
    \item 考虑以下情形(其中出现的矩阵除$X$外均可逆,$X$不一定是列向量):
    \begin{enumerate}[label=(\arabic*)]
        \item $AX=B \implies X=A^{-1}B, \enspace XA=B \implies X=BA^{-1}$;
        \item $AXB=C \implies X=A^{-1}CB^{-1}$;
    \end{enumerate}
    \item 考虑以下情形:$AX=B$但$A$不可逆($X$不一定是列向量),直接高斯消元即可;
    \item 考虑以下求解方式的合理性:
    \begin{enumerate}[label=(\arabic*)]
        \item 若求$A^{-1}$,只需对$(A,E)$只做初等行变换,可以得到$(E,A^{-1})$;
        \item 若求$A^{-1}B$,只需对$(A,B)$只做初等行变换,可以得到$(E,A^{-1}B)$;
        \item 若求$BA^{-1}$,只需对$\begin{pmatrix}
            A \\ B
        \end{pmatrix}$只做初等列变换,可以得到$\begin{pmatrix}
            E \\ BA^{-1}
        \end{pmatrix}$;
        \item 对$\begin{pmatrix}
            A & E \\ E & O
        \end{pmatrix}$的前$n$行与$n$列做相同的行列变换,可以得到$\begin{pmatrix}
            P^\mathrm{T}AP & P^\mathrm{T} \\ P & O
        \end{pmatrix}$.
    \end{enumerate}
\end{enumerate}

\begin{example}
    设$A=\begin{pmatrix}1 & 0 & 0 \\ 1 & 1 & 0 \\ 1 & 1 & 1\end{pmatrix},\
    B=\begin{pmatrix}0 & 1 & 1 \\ 1 & 0 & 1 \\ 1 & 1 & 0\end{pmatrix}$,求矩阵$X$满足:
    \[AXA+BXB=AXB+BXA+A(A-B)\]
\end{example}

\section{矩阵的逆进阶求法}
\subsection{给定多项式求逆矩阵}
此类题目相信大家已经有所见识,实际上就是通过一些初中所学的因式分解等基本变换得到需要求逆的矩阵与另一个矩阵相乘
可以得到单位矩阵(的一个倍数).
\begin{example}
    设$A$为非零矩阵,且$A^3=O$,证明:$E+A$和$E-A$都可逆.
\end{example}

\begin{example}
    若$X$,$Y$是两个列向量,且$X^\mathrm{T}Y=2$,证明:
    \begin{enumerate}
        \item $(XY^\mathrm{T})^k=2^{k-1}(XY^{\mathrm{T}})$;

        \item 如果$A=E+XY^\mathrm{T}$,则$A$可逆,并求其逆矩阵.
    \end{enumerate}
\end{example}

\subsection{利用分块矩阵初等变换*}
我们在前面已经讲解过了打洞法的基础题型,这里再给出一些例子:
\begin{example}
    设$A$、$B$为$n$阶矩阵,证明:若$E\pm AB$可逆,则$E\pm BA$可逆.
\end{example}
\begin{example}
    设$A$为$n$阶矩阵,$B$、$C$分别为$n \times m$、$m \times n$阶矩阵,
    证明:$E_m+CA^{-1}B$可逆$\iff A+BC$可逆.
\end{example}

\subsection{求逆的分式思想*}
虽然矩阵没有除法运算,但是我们如果将$(E-A)^{-1}$写成$\dfrac{E}{E-A}$,再类比泰勒展开
\[\frac{1}{1-x}=1+\sum_{n=1}^\infty x^n \quad x\in (-1,1)\]我们可以得到(不严谨!只能用来解题的时候当作初步的思路!)
\[(E-A)^{-1}=\frac{E}{E-A}=E+A+A^2+\cdots\]

\begin{example}
    已知方阵$A$满足$A^k=O$,其中$k$是一个正整数,求$E-A$的逆.
\end{example}

\begin{example}
    设$A$,$B$分别是$n \times m$和$m \times n$的矩阵,且$E_n \pm AB$可逆,则$E_m \pm BA$可逆.
\end{example}
不难发现这一例是前述4.3.1节中最后一个例题的推广. % TODO 编号

\subsection{提逆思想*}
这一思想的来源是矩阵逆没有加减相关的运算法则,因此我们需要提逆产生一些乘积项来解决问题.
\begin{example}
    设$A$是$n$阶方阵,且$E-A$,$E+A$和$A$都可逆,证明:$(E-A^{-1})^{-1}+(E-A)^{-1}=E$.
\end{example}

\section{矩阵的迹*}
\subsection{基本概念与性质}
\begin{definition}
    一个方阵$A$的所有主对角元素之和称为$A$的迹,记为$\tr(A)$.
\end{definition}
迹的常见性质如下:
\begin{theorem}
    已知$A, B$是两个$n$阶矩阵,$k$是一个常数,则
    \begin{enumerate}
        \item $\tr(kA) = k\tr(A)$;

        \item $\tr(A+B)=\tr(A)+\tr(B)$;

        \item $\tr(AB)=\tr(BA)$;

        \item 如果$A$是实矩阵,则$A=O \iff \tr(A^{\mathrm{T}}A)=0$.
    \end{enumerate}
\end{theorem}

我们先来看一个基本的例子练习一下:
\begin{example}
    证明:不存在方阵$A$、$B$使得$AB-BA=E$.
\end{example}

接下来我们介绍一个性质,这一性质的证明需要利用矩阵的秩中讲到的一些技巧:
\begin{theorem}
    已知 $n$ 阶矩阵 $A$ 的秩为 1,证明:$A^k=\tr(A)^{k-1}A$.
\end{theorem}
这一定理的证明需要用到矩阵的分解,在2020年吴志祥老师班期中考试有出现,可以参考
辅学网站上的解答.

当然我们还会在特征值一章再次见到矩阵的迹,相关内容在最后一个专题会展开讲述.

\subsection{幂零矩阵}
幂零矩阵是一种特殊的矩阵,幂零矩阵$A$存在一个正整数$k$使得$A^k=O$,
它具有如下性质(部分需要用到特征值,所以最后一个专题还会提及):
\begin{theorem}
    若$n$阶矩阵$A$为幂零矩阵,则
    \begin{enumerate}
        \item $A^n=O$;

        \item $A\pm E$均为可逆矩阵;

        \item 幂零矩阵对应的线性变换一定存在一个矩阵表示使得矩阵为上三角矩阵且对角线元素全为0;

        \item $A$ 为幂零矩阵 $\iff \forall k \in \mathbf{N}_+,\enspace\tr(A^k)=0$.
    \end{enumerate}
\end{theorem}

\begin{example}
    若$A$、$B$为两个$n$阶矩阵且满足$AB-BA=A$,证明:
    \begin{enumerate}
        \item $A$不可逆;

        \item $A$是幂零矩阵.
    \end{enumerate}
\end{example}

\section{矩阵的幂}
\begin{enumerate}
    \item 找规律

    在矩阵的转置中我们已经见识了一种找规律的方式,下面是一种类似的题型:
    \begin{example}
        计算$(PAQ)^k$,其中
        \[P=\begin{pmatrix}2 & 3 \\ 1 & 2\end{pmatrix},\enspace A=\begin{pmatrix}2 & 0 \\ 0 & -1\end{pmatrix},\enspace Q=\begin{pmatrix}2 & -3 \\ -1 & 2\end{pmatrix}\]
    \end{example}

    \begin{example}
        设$A=\begin{pmatrix}0 & -1 & 0 \\ 1 & 0 & 0 \\ 0 & 0 & -1 \end{pmatrix},\enspace P^{-1}AP=B$,求$B^{2004}-2A^2$.
    \end{example}

    还有一种找规律基于幂等矩阵,显然幂等矩阵的任意次方都与其平方相等是很好的性质,另一种找规律基于对合矩阵,即平方等于单位矩阵的矩阵,我们这里
    主要与大家分享关于幂零矩阵的方法,例子如下:
    \begin{example}
        求$A=\begin{pmatrix}a & 1 & 0 & 0 \\ 0 & a & 1 & 0 \\ 0 & 0 & a & 0 \\ 0 & 0 & 0 & a \end{pmatrix}^n$.
    \end{example}
    在上例中,我们采用将矩阵分为$A=tE+B$的方法,会发现矩阵$B$为上三角矩阵且对角线上全为0,是典型的幂零矩阵,利用这一性质可以快速解题.
    \item 数学归纳法
    \begin{example}
        求$A=\begin{pmatrix}\cos\alpha & \sin\alpha \\ -\sin\alpha & \cos\alpha\end{pmatrix}^n$.
    \end{example}
    这一问题对应我们常见的旋转变换.
    \begin{example}
        证明$\begin{pmatrix}
            a & c \\ 0 & b
        \end{pmatrix}^n=\begin{pmatrix}
            a^n & (a^{n-1}+a^{n-2}b+\dots+b^{n-1})c \\ 0 & b^n
        \end{pmatrix}$.
    \end{example}
    \item 利用秩为1的矩阵

    在我们有关秩的讨论中已经提到了如果$A$是秩为1的矩阵,那么$A^n=\tr(A)^{n-1}A$,我们可以利用这一性质解决问题.
    \begin{example}
        已知$M$是秩为 1 的矩阵,记$\tr(M)=b$,讨论$(aE+M)^n$的计算结果.
    \end{example}

    \begin{example}
        已知$A$是数域$P$上的一个2阶方阵,且存在正整数$l$使得$A^l=O$,证明:$A^2=O$.
    \end{example}
    事实上,我们在幂零矩阵的讨论中已经提及了上例的一般情况.

    \begin{example}
        已知数列$\{a_n\},\enspace\{b_n\}$满足$a_0=-1,\enspace b_0=3$,且
        \[\begin{cases}
            a_n=3a_{n-1}+b_{n-1}+2^{n-1} \\ b_n=2a_{n-1}+4b_{n-1}+2^n
        \end{cases}\]
        求$\{a_n\}\enspace\{b_n\}$的通项公式.
    \end{example}
    \item 利用初等矩阵的性质
    \begin{example}
        设$A$为三阶矩阵,$P$为三阶可逆矩阵,$P^{-1}AP=B$,其中$P=\begin{pmatrix}
            0 & 2 & -1 \\ 1 & 1 & 2 \\ -1 & -1 & -1
        \end{pmatrix}$,$B=\begin{pmatrix}
            0 & 0 & -1 \\ 0 & -1 & 0 \\ -1 & 0 & 0
        \end{pmatrix}$,求$A^{2022}$.
    \end{example}

    \item 利用对角化

    若一个矩阵$A$可对角化,即存在可逆矩阵$P$使得$A=P^{-1}\Lambda P$(其中$\Lambda$为对角矩阵),
    在这种形式下$A$的幂是很好求的.
    \begin{example}
        已知$A=\begin{pmatrix}
            0 & \cfrac{1}{2} & \cfrac{1}{2} \\ 1 & -\cfrac{1}{2} & \cfrac{1}{2} \\ 1 & -\cfrac{1}{2} & \cfrac{1}{2}
        \end{pmatrix}$,求$A^n$.
    \end{example}

\end{enumerate}

\vspace{2ex}
\centerline{\heiti \Large 内容总结}

\vspace{2ex}

\centerline{\heiti \Large 习题}
\vspace{2ex}
{\kaishu }
\begin{flushright}
    \kaishu

\end{flushright}
\centerline{\heiti A组}
\begin{enumerate}
    \item
\end{enumerate}
\centerline{\heiti B组}
\begin{enumerate}
    \item
\end{enumerate}
\centerline{\heiti C组}
\begin{enumerate}
    \item
\end{enumerate}

\section*{11 矩阵的秩}
\addcontentsline{toc}{section}{11 矩阵的秩}

\vspace{2ex}

\centerline{\heiti A组}
\begin{enumerate}
    \item 取 $\mathbf{R^4}$ 标准基 $\varepsilon_1,\varepsilon_2,\varepsilon_3,\varepsilon_4$.
    那么 $(\alpha_1,\alpha_2,\alpha_3,\alpha_4)=(\varepsilon_1,\varepsilon_2,\varepsilon_3,\varepsilon_4)A,(\beta_1,\beta_2,\beta_3,\beta_4)=(\varepsilon_1,\varepsilon_2,\varepsilon_3,\varepsilon_4)B.$
    其中 \[A=\begin{pmatrix}1 & 1 & 1 & 1 \\ 1 & 1 & -1 & -1 \\ 1 & -1 & 1 & -1 \\ 1 & -1 & -1 & 1\end{pmatrix},B=\begin{pmatrix}1 & 2 & 1 & 0 \\ 1 & 1 & 1 & 1 \\ 0 & 3 & 0 & -1 \\ 1 & 1 & 0 & -1\end{pmatrix}.\] 
    由此可知 \[(\beta_1,\beta_2,\beta_3,\beta_4)=(\varepsilon_1,\varepsilon_2,\varepsilon_3,\varepsilon_4)B=(\alpha_1,\alpha_2,\alpha_3,\alpha_4)A^{-1}B.\]
    过渡矩阵 \[A^{-1}B=\dfrac{1}{4}\begin{pmatrix}3 & 7 & 2 & -1 \\ 1 & -1 & 2 & 3 \\ -1 & 3 & 0 & -1 \\ 1 & -1 & 0 & -1\end{pmatrix}\]
    另外,容易求得 $\xi$ 在 $\alpha_1,\alpha_2,\alpha_3,\alpha_4$ 下的坐标为 $\begin{pmatrix}0 \\ \frac{1}{2} \\ \frac{1}{2} \\ 0\end{pmatrix}$
    \item 证明:考虑矩阵 $A$ 的行向量组的极大线性无关组,若添加的一行可由其极大线性无关组线性表示,则秩不变. 否则秩增加 $1$. 
    \item 证明:设 $A$ 的行向量组为 $\{\alpha_1,\alpha_2,\cdots,\alpha_s\}$,$r(A)=r$; $B$ 的行向量组为 $\{\alpha_1,\alpha_2,\cdots,\alpha_m\},r(B)=k$.
    不妨设:$B$ 的行向量组的极大线性无关组为 $\{\alpha_1,\alpha_2,\cdots,\alpha_k,\alpha_{i_1},\cdots,\alpha_{i_{r-k}}\}$,其中 $\{\alpha_{i_1},\cdots,\alpha_{i_{r-k}}\}$(共 $r-k$ 个向量)是包含在 $\{\alpha_{m+1},\cdots,\alpha_s\}$(共 $s-m$ 个向量)之中的. 显然有
    \[r-k \leq s-m,\]
    即
    \[r(B)=k\ge r+m-s=r(A)+m-s.\]
\end{enumerate}

\centerline{\heiti B组}
\begin{enumerate}
    \item 已知 $r(A+B) \leq r(A)+r(B)$,把 $B$ 写成 $-B$ 则有 $r(A-B) \leq r(A)+r(-B)=r(A)+r(B)$. 不等式右半部分得证.
    
    另外,$r(A)=r(A-B+B) \leq r(A-B)+r(B)$,从而 $r(A-B) \ge r(A)-r(B)$,当然,加个绝对值也是没有问题的:$r(A-B) \ge \lvert r(A)-r(B) \rvert$. 同理,有 $r(A+B) \ge \lvert r(A)+r(B) \rvert$. 证毕.
    \item $V$ 的基 $B_1$ 到 $B_2$ 的过渡矩阵 $P$ 具有下述形式:
    \[P=\begin{pmatrix}\mathrm{Im} & B_1 \\ 0 & B_2\end{pmatrix}\]
    其中 $B_1,B_2$ 分别是域 $\mathbf{F}$ 上 $m\times (n-m),(n-m)\times (n-m)$ 矩阵,
    \[\beta_j=b_{j1}\delta_1+\cdots+b_{jm}\delta_m+b_{j,m+1}\delta_{m+1}+\cdots+b_{jn}a_n\]
    其中 $j=m+1,\cdots,n$. 于是
    \[\beta_j+W=b_{j,m+1}(\alpha_{m+1}+W)+\cdots+b_{jn}(\alpha_n+W)\]
    因此商空间 $V/W$ 的基 $\alpha_{m+1}+W,\cdots,\alpha_n+W$ 到 $\beta_{m+1}+W,\cdots,\beta_n+W$ 的过渡矩阵是 $B_2$.
    \item 设 $\beta_1=\alpha_1+\alpha_2,\cdots,\beta_{n-1}=\alpha_{n-1}+\alpha_n,\beta_n=\alpha_n+\alpha_1$. 由于
    \[(\beta_1,\beta_2,\cdots,\beta_n) = (\alpha_1,\alpha_2,\cdots,\alpha_n)A,\]
    其中
    \[A=\begin{pmatrix}1 & & & & 1 \\ 1 & 1 & & & \\ & 1 & \ddots & & \\ & & \ddots & 1 & \\ & & & 1 & 1\end{pmatrix}.\]
    则有 $\lvert A \rvert = 1 + (-1)^{n+1}=2\neq 0$,所以 $A$ 可逆,可得 $\{\beta_1,\beta_2,\cdots,\beta_n\}$ 和 $\{\alpha_1,\alpha_2,\cdots,\alpha_n\}$ 等价. 这就说明了 $\alpha_1,\alpha_2,\cdots,\alpha_n$ 线性无关的充要条件是 $\beta_1,\beta_2,\cdots,\beta_n$ 线性无关.
    \item 记 $B_1=\{e_{11},e_{12},e_{21},e_{22}\},B_2=\{g_1,g_2,g_3,g_4\}$.\begin{enumerate}
        \item 设 $k_1g_1+k_2g_2+k_3g_3+k_4g_4=O$,可得 $k_1=k_2=k_3=k_4=0$,所以 $g_1,g_2,g_3,g_4$ 线性无关,从而是 $M_2(\mathbf{R})$ 的一组基.
        \item 由 $M_2(\mathbf{R}) \cong \mathbf{R^4}$,所以 $\{e_{11},e_{12},e_{21},e_{22}\}$ 可以表示为 $\mathbf{R^4}$ 中的自然基 $\{e_1,e_2,e_3,e_4\}$,而 $\{g_1,g_2,g_3,g_4\}$ 可表示为 $\{(1,0,0,0)^{\mathbf{T}},(1,1,0,0)^{\mathbf{T}},(1,1,1,0)^{\mathbf{T}},(1,1,1,1)^{\mathbf{T}}\}$.
        
        于是,由 \[\begin{pmatrix}g_1 & g_2 & g_3 & g_4\end{pmatrix}=\begin{pmatrix}e_{11} & e_{12} & e_{21} & e_{22}\end{pmatrix}C\]
        得 \[\begin{pmatrix}e_{11} & e_{12} & e_{21} & e_{22}\end{pmatrix}=\begin{pmatrix}g_1 & g_2 & g_3 & g_4\end{pmatrix}C^{-1}\]
        所以基 $B_2$ 变为 $B_1$ 的变换矩阵为 $C^{-1}=\begin{pmatrix}1 & -1 & 0 & 0 \\ 0 & 1 & -1 & 0 \\ 0 & 0 & 1 & -1 \\ 0 & 0 & 0 & 1\end{pmatrix}$.
        \item 考虑从 $A^2=A$ 中选取较为简单的矩阵,例如由
        \[\begin{pmatrix}a & b \\ 0 & 0\end{pmatrix}^2=\begin{pmatrix}a^2 & ab \\ 0 & 0\end{pmatrix}=\begin{pmatrix}a & b \\ 0 & 0\end{pmatrix}\]
        取 $a=1,b=0$ 或 $1$,得 $A_1=\begin{pmatrix}1 & 0 \\ 0 & 0\end{pmatrix},A_2=\begin{pmatrix}1 & 1 \\ 0 & 0\end{pmatrix}$

        类似地,可取 $A_3=\begin{pmatrix}0 & 0 \\ 0 & 1\end{pmatrix},A_4=\begin{pmatrix}0 & 0 \\ 1 & 1\end{pmatrix}$.

        这就取出了一组满足 $A^2=A$ 的线性无关的 $\{A_1,A_2,A_3,A_4\}$,是 $M_2(\mathbf{R})$ 的一组基 $B_3$.
        \item 先记 $B_2$ 变为 $B_3$ 的变换矩阵为 $D$,即 \[\begin{pmatrix}A_1 & A_2 & A_3 & A_4\end{pmatrix}=\begin{pmatrix}g_1 & g_2 & g_3 & g_4\end{pmatrix}D\]\
        按题 $(2)$ 中所述,此时有 \[\begin{pmatrix}1 & 1 & 0 & 0 \\ 0 & 1 & 0 & 0 \\ 0 & 0 & 0 & 1 \\ 0 & 0 & 1 & 1\end{pmatrix}=\begin{pmatrix}1 & 1 & 1 & 1 \\ 0 & 1 & 1 & 1 \\ 0 & 0 & 1 & 1 \\ 0 & 0 & 0 & 1\end{pmatrix}D\]
        由于上式右端已知矩阵的逆矩阵为上面的 $C^{-1}$,所以在上式两边左乘 $C^{-1}$,可得 \[D = \begin{pmatrix}1 & 0 & 0 & 0 \\ 0 & 1 & 0 & -1 \\ 0 & 0 & -1 & 0 \\ 0 & 0 & 1 & 1\end{pmatrix}\]
        由于矩阵 $A$ 关于 $B_2$ 的坐标为 $(1,1,1,1)^{\mathbf{T}}$,所以 $A$ 关于 $B_3$ 的坐标为
        \[Y=D^{-1}X=\begin{pmatrix}1 \\ 3 \\ -1 \\ 2\end{pmatrix}.\]
    \end{enumerate}
    \item \begin{enumerate}
        \item 初等变换即可.
        \item 同上.
        \item 矩阵 $A$ 秩为 $r$ 可写作 $A=P\begin{pmatrix}E_r & 0 \\ 0 & 0\end{pmatrix}Q = P(E_{11}+E_{22}+\cdots+E_{rr})Q$($E_r$ 是 $r\times r$ 的单位矩阵,$E_{ii}$ 是 $n\times n$ 的只有第 $i$ 行 $i$ 列的这个元素为 $1$,其他元素均为 $0$ 的矩阵). 
        每个 $PE_{ii}Q$ 都是秩为 $1$ 的矩阵,故得证.
        \item 记 $r(A)=r$,把 $A$ 写成 $P\begin{pmatrix}E_r & 0 \\ 0 & 0\end{pmatrix}Q$ 的形式. 构造 $B=Q^{-1}\begin{pmatrix}E_r & 0 \\ 0 & 0\end{pmatrix}P^{-1}$ 可以发现其满足条件,故得证.
    \end{enumerate}
    \item $r(BC)\leq r(B) \leq 1$,得证. 
    
    反之,若 $A$ 是秩为 $1$ 的 $3\times 3$ 矩阵,则存在可逆矩阵 $P,Q$ 使得 $A=P^{-1}E_{11}Q^{-1}$,其中 $E_{11}=\begin{pmatrix}1 & 0 & 0 \\ 0 & 0 & 0 \\ 0 & 0 & 0\end{pmatrix}=\begin{pmatrix}1 \\ 0 \\ 0\end{pmatrix}\begin{pmatrix}1 & 0 & 0\end{pmatrix}$.
    则取 $B=P^{-1}\begin{pmatrix}1 \\ 0 \\ 0\end{pmatrix},C=\begin{pmatrix}1 & 0 & 0\end{pmatrix}Q^{-1}$,有 $A=BC$,证毕.
    \item \begin{enumerate}
        \item $r(\alpha \alpha^{\mathbf{T}})\leq r(\alpha) = 1,r(\beta \beta^{\mathbf{T}})\leq r(\beta) = 1$. 由 $r(A+B) \leq r(A)+r(B)$ 得 $r(A)=r(\alpha \alpha^{\mathbf{T}}+\beta \beta^{\mathbf{T}}) \leq r(\alpha)+r(\beta)=2.$
        \item 若 $\alpha,\beta$ 均为 $\mathbf{0}$ 向量,显然. 否则假设 $\alpha$ 不为 $0$,则由于两向量线性相关,必有确定的 $k$ 使得 $\beta = k\alpha$,把 $\beta$ 用 $\alpha$ 表示之后易证.
    \end{enumerate}
    \item $r(A)=r$ 则 $AX=0$ 的解空间维数 $\mathrm{dim}N(A) = n-r$. 由 $r(A)+r(B)=k$ 得 $r(B)=k-r \leq n-r=\mathrm{dim}N(A)$. 要求 $AB=O$,说明 $B$ 的列向量均为 $AX=0$ 的解,那么只需要选择合适的列向量组拼接成 $B$ 即可(这一定能做到,因为 $B$ 维数不会超过解空间维数).
    \item 由于 $A$ 是 $m\times n$ 矩阵,$r(A)=m$,可知对于矩阵 $A$ 做初等列变换,可使其前 $m$ 列变为单位矩阵,后 $n-m$ 列变为全 $0$ 列.
    因此,存在 $n$ 阶可逆矩阵 $P$ 使得
    \[AP=\begin{pmatrix}E_m & O_{m\times (n-m)}\end{pmatrix}\]
    于是\[AP(AP)^{\mathrm{T}} = \begin{pmatrix}E & O\end{pmatrix} \begin{pmatrix}E \\ O\end{pmatrix}=E_m\]
    所以存在 $B=(PP^{\mathrm{T}}A^{\mathrm{T}})$ 为 $n\times m$ 矩阵,使 $AB=E$.
    \item 利用 $A,B$ 的相抵标准形. 存在 $n$ 阶可逆矩阵 $P_1,Q_1,P_2,Q_2$ 使得
    \[P_1AQ_1=\begin{pmatrix}E_{r_A} & O \\ O & O\end{pmatrix},P_2BQ_2=\begin{pmatrix}O & O \\ O & E_{r_B}\end{pmatrix}\]
    于是 \[AQ_1=P_1^{-1}\begin{pmatrix}E_{r_A} & O \\ O & O\end{pmatrix},P_2B=\begin{pmatrix}O & O \\ O & E_{r_B}\end{pmatrix}Q_2^{-1}\]
    所以 \[AQ_1P_2B=P_1^{-1}\begin{pmatrix}E_{r_A} & O \\ O & O\end{pmatrix}\begin{pmatrix}O & O \\ O & E_{r_B}\end{pmatrix}Q_2^{-1}=O\]
    取 $M=Q_1P_2$ 即可.
    \item \begin{enumerate}
        \item 易证,此处略去.
        \item 注意到 $B$ 的列向量均为 $AX=0$ 的解,设 $AX=0$ 的基础解系为 $\alpha_1,\cdots,\alpha_t(t=n-r)$,则易知
        \[B_{11}=(\alpha_1,0,\cdots,0),B_{12}=(0,\alpha_1,\cdots,0),\cdots,B_{1n}=(0,0,\cdots,\alpha_1),\]
        \[B_{21}=(\alpha_2,0,\cdots,0),B_{22}=(0,\alpha_2,\cdots,0),\cdots,B_{2n}=(0,0,\cdots,\alpha_2),\]
        \[\vdots\]
        \[B_{t1}=(\alpha_t,0,\cdots,0),B_{t2}=(0,\alpha_t,\cdots,0),\cdots,B_{tn}=(0,0,\cdots,\alpha_t)\]
        为 $S(A)$ 的一组基,故 $\mathrm{dim}S(A)=n(n-r)$.
    \end{enumerate}
\end{enumerate}

\centerline{\heiti C组}
\begin{enumerate}
    \item 对 $\begin{pmatrix}E_n & A' \\ A & E_s\end{pmatrix}$ 利用打洞原理有
    \[\begin{pmatrix}E_n-A'A & O \\ O & E_s\end{pmatrix} \leftarrow \begin{pmatrix}E_n & A' \\ A & E_s\end{pmatrix} \rightarrow \begin{pmatrix}E_n & O \\ O & E_s-AA'\end{pmatrix}\]
    所以 $r\begin{pmatrix}E_n-A'A & O \\ O & E_s\end{pmatrix}=r\begin{pmatrix}E_n & O \\ O & E_s-AA'\end{pmatrix}$,即 $s+r(E_n-A'A)=n+r(E_s-AA')$,即
    \[r(E_n-A'A)-r(E_s-AA')=n-s.\]
    \item \begin{enumerate}
        \item 由 \[\begin{pmatrix}A & 0 \\ 0 & B\end{pmatrix}\rightarrow \begin{pmatrix}A & AC \\ 0 & B\end{pmatrix}\rightarrow \begin{pmatrix}A & AC+BD \\ 0 & B\end{pmatrix}=\begin{pmatrix}A & E \\ 0 & B\end{pmatrix}\]
        \[\rightarrow \begin{pmatrix}0 & E \\ -AB & B\end{pmatrix}\rightarrow \begin{pmatrix}0 & E \\ AB & 0\end{pmatrix}\]
        可得.
    \item 用分块矩阵的方法,我们知道 
    \[\begin{pmatrix}A & O \\ O & B\end{pmatrix}\rightarrow \begin{pmatrix}A & O \\ A & B\end{pmatrix}\rightarrow \begin{pmatrix}A & A \\ A & A+B\end{pmatrix}\]
    结合 $AB=BA$,我们知道
    \[\begin{pmatrix}A & A \\ A & A+B\end{pmatrix}\begin{pmatrix}A+B & O \\ -A & E\end{pmatrix}=\begin{pmatrix}AB & A \\ O & A+B\end{pmatrix}\]
    于是
    \[r(A)+r(B)=r\begin{pmatrix}A & O \\ O & B\end{pmatrix}=r\begin{pmatrix}A & A \\ A & A+B\end{pmatrix}\ge \begin{pmatrix}AB & A \\ O & A+B\end{pmatrix}\ge r(AB)+r(A+B)\] 
    \end{enumerate}
    \item 略有超纲,使用贝祖定理,
    \[\exists u(x),v(x),u(x)f_1(x)+v(x)f_2(x)=1\]
    \[r\begin{pmatrix}f_1(A) & O \\ O & f_2(A)\end{pmatrix}=r\begin{pmatrix}f_1(A) & f_1(A)u(A)+f_2(A)v(A) \\ O & f_2(A)\end{pmatrix}=r\begin{pmatrix}f_1(A) & E \\ O & f_2(A)\end{pmatrix}\] 
    \[=r\begin{pmatrix}f_1(A) & E \\ -f_2(A)f_1(A) & O\end{pmatrix}=r\begin{pmatrix}O & E \\ f(A) & O\end{pmatrix}\]
    \item 由于 $A$ 是列满秩矩阵,$B$ 是行满秩矩阵,知存在可逆矩阵 $P_{3\times 3},Q_{2\times 2}$ 使得
    \[A=P\begin{pmatrix}E_2 \\ O\end{pmatrix},B=\begin{pmatrix}E_2 & O\end{pmatrix}Q\]
    于是 \[BA=\begin{pmatrix}E_2 & O\end{pmatrix}QP\begin{pmatrix}E_2 \\ O\end{pmatrix}\]
    由 $(AB)^2=9AB$ 有 \[P\begin{pmatrix}E_2 \\ O\end{pmatrix}\begin{pmatrix}E_2 & O\end{pmatrix}QP\begin{pmatrix}E_2 \\ O\end{pmatrix}\begin{pmatrix}E_2 & O\end{pmatrix}Q=9P\begin{pmatrix}E_2 \\ O\end{pmatrix}\begin{pmatrix}E_2 & O\end{pmatrix}Q\]
    即 \[\begin{pmatrix}E_2 \\ O\end{pmatrix}BA\begin{pmatrix}E_2 & O\end{pmatrix}=9\begin{pmatrix}E_2 \\ O\end{pmatrix}\begin{pmatrix}E_2 & O\end{pmatrix}\]
    也就是 \[\begin{pmatrix}BA & O \\ O & O\end{pmatrix}=\begin{pmatrix}9E_2 & 0 \\ 0 & 0\end{pmatrix}\]
    所以 $BA=9E_2$.
    \item 本题求核空间困难,但只需要求维数,我们考虑求像空间之后求出像空间维数,然后用维数公式求解.
    
    取 $F^{n\times p}$ 的自然基 $\{e_{11},e_{12},\cdots,e_{np}\}$($e_{ij}$ 表示仅有第 $i$ 行第 $j$ 列的元素为 $1$,其他均为 $0$ 的矩阵)

    则 $\mathrm{Im}\ \sigma=\mathrm{span}(\sigma(e_{11}),\cdots,\sigma(e_{np}))$.

    取 $A$ 的列向量,写成 $A=\begin{pmatrix}\alpha_1,\alpha_2,\cdots,\alpha_n\end{pmatrix}$,则 $\sigma(e_{ij})$ 可排列如下:
    \[(\alpha_1,0,\cdots,0),(0,\alpha_1,\cdots,0),\cdots,(0,0,\cdots,\alpha_1)\]
    \[(\alpha_2,0,\cdots,0),(0,\alpha_2,\cdots,0),\cdots,(0,0,\cdots,\alpha_2)\]
    \[\cdots\]
    \[(\alpha_n,0,\cdots,0),(0,\alpha_n,\cdots,0),\cdots,(0,0,\cdots,\alpha_n)\]
    由于 $r(A)=r$,故 $\alpha_1,\alpha_2,\cdots,\alpha_n$ 的极大线性无关组有 $r$ 个向量,不妨设为 $\alpha_1,\alpha_2,\cdots,\alpha_r$. 则下列向量:
    \[(\alpha_{r+1},0,\cdots,0),(0,\alpha_{r+1},\cdots,0),\cdots,(0,0,\cdots,\alpha_{r+1})\]
    \[\cdots\]
    \[(\alpha_n,0,\cdots,0),(0,\alpha_n,\cdots,0),\cdots,(0,0,\cdots,\alpha_n)\]
    均可以被其他向量线性表出. 观察除了上述向量的剩下的向量,可以发现这 $r\times p$ 个向量线性无关,从而 $\mathrm{dim}(\mathrm{Im}\ \sigma) = r\times p$.

    故由维数公式,得 $\mathrm{dim}(\mathrm{Ker}\ \sigma) = \mathrm{dim}F^{n\times p}-\mathrm{dim}(\mathrm{Im} \ \sigma) = (n-r)p$.
\end{enumerate}

\clearpage

\chapter{行列式(I)}

\section{行列式的几种定义}
很多教材采用``逆序数''定义行列式(感兴趣的同学可以参考丘维声《高等代数》等教材),但是本教材未提及,因此
我们作为复习课也不展开描述.本教材使用公理化定义(使用一些规则描述)并讲解了
递归式定义(按行(列)展开).

\subsection{公理化定义}
\begin{definition} \label{def:11:公理化定义}
    数域$\mathbf{F}$上的一个$n$阶行列式是取值于$\mathbf{F}$的$n$个$n$维向量
    $\alpha_1,\alpha_2,\ldots,\alpha_n \in \mathbf{F}^n$的一个函数,且$\forall \alpha_i,\beta_i \in \mathbf{F}^n$
    和$\forall \lambda \in \mathbf{F}$,满足下列规则:
    \begin{enumerate}
        \item(齐性)$D(\alpha_1,\ldots,\lambda\alpha_i,\ldots,\alpha_n)=\lambda D(\alpha_1,\ldots,\alpha_i,\ldots,\alpha_n)$;

        \item(加性,与 1 合称线性性)

        $D(\alpha_1,\ldots,\alpha_i+\beta_i,\ldots,\alpha_n)=D(\alpha_1,\ldots,\alpha_i,\ldots,\alpha_n)+D(\alpha_1,\ldots,\beta_i,\ldots,\alpha_n)$;

        \item(反对称性)$D(\alpha_1,\ldots,\alpha_i,\ldots,\alpha_j,\ldots,\alpha_n)=-D(\alpha_1,\ldots,\alpha_j,\ldots,\alpha_i,\ldots,\alpha_n)$;

        \item(规范性)$D(e_1,e_2,\ldots,e_n)=1$.
    \end{enumerate}
\end{definition}
在公理化定义中,我们将行列式定义为一个满足特定的运算性质的从列向量组合到数的函数.
事实上,公理化定义从是逆序数定义可以推导出的行列式的运算性质,教材采用这种定义避开了繁琐的说明.
除此之外,我们不难看出公理化定义可以形象地理解为对$n$维空间中体积的定义,
对几何意义感兴趣的同学可以参考 \href{https://b23.tv/BV1ys411472E}{3b1b《线性代数的本质》系列视频}相关内容.
\begin{example} \label{ex:11:公理化定义}
    使用\autoref{def:11:公理化定义} 验证下述命题的正确性:
    \begin{enumerate}
        \item 若行列式有一列为零向量,则行列式的值等于0.

        \item 若行列式有两列元素相同,则行列式的值等于0.

        \item 若行列式有两列元素成比例,则行列式的值等于0.

        \item 对行列式做倍加列变换,行列式的值不变.

        \item 若$\alpha_1,\alpha_2,\ldots,\alpha_n$线性相关,则$D(\alpha_1,\alpha_2,\ldots,\alpha_n)=0$.
    \end{enumerate}
\end{example}

\begin{example} \label{ex:11:公理化定义2}
    设向量$\alpha_1,\alpha_2,\beta_1,\beta_2$为三维列向量,又$A=(\alpha_1,\alpha_2,\beta_1),B=(\alpha_1,\alpha_2,\beta_2)$,
    且$|A|=3$,$|B|=2$,求$|2A+3B|$.
\end{example}

\subsection{递归式定义}
首先需要回顾余子式和代数余子式的概念:
\begin{definition} \label{def:11:余子式}
    在$n$阶行列式$D=|a_{ij}|_{n \times n}$中,去掉元素$a_{ij}$所在的第$i$行和第$j$列的所有元素
    而得到的$n-1$阶行列式称为元素$a_{ij}$的\keyterm{余子式}[minor],记作$M_{ij}$,并把数$A_{ij}=(-1)^{i+j}M_{ij}$
    称为元素$a_{ij}$的\keyterm{代数余子式}[cofactor].
\end{definition}
注意,虽然余子式和代数余子式在名称中含有式,但实际上他们是一个值.实际上行列式也称为``式'',但这些``式''
只是形状上有个形式,实际上只是一个值.
\begin{example} \label{ex:11:cofactor}
    根据\autoref{def:11:余子式} 计算行列式$\begin{vmatrix}
        2 & 1 & 3 \\
        -1 & 0 & 2 \\
        1 & 5 & -2
    \end{vmatrix}$每个元素的余子式和代数余子式.
\end{example}

接下来我们便可以给出递归式定义:
\begin{definition} \label{def:11:递归式定义}
    设$D=|a_{ij}|_{n \times n}$,则
    \begin{gather}
        \label{eq:11:递归式定义1}
        D=\sum_{k=1}^{n}a_{kj}A_{kj}=a_{1j}A_{1j}+a_{2j}A_{2j}+\cdots+a_{nj}A_{nj} \quad j=1,2,\ldots,n \\
        \label{eq:11:递归式定义2}
        D=\sum_{k=1}^{n}a_{ik}A_{ik}=a_{i1}A_{i1}+a_{i2}A_{i2}+\cdots+a_{in}A_{in} \quad i=1,2,\ldots,n
    \end{gather}
\end{definition}
其中$A_{ij}$即为\autoref{def:11:余子式} 给出的代数余子式,\autoref{eq:11:递归式定义1} 称为$D$对第$j$列的展开式,\autoref{eq:11:递归式定义2} 称为$D$对第$i$行的展开式.这一定义与公理化定义的
等价性不难证明.事实上,这一定义被称为递归式定义的原因是显然的(如果在程序设计课程中已经学习过递归的概念),它使用$n-1$阶行列式定义$n$阶行列式,我们对任意$n$阶行列式
都可以递归展开到1阶,从而得到最终行列式计算结果.
\begin{example} \label{ex:11:递归式定义}
    利用\autoref{def:11:递归式定义} 计算\autoref{ex:11:cofactor} 中的行列式,可以行列展开均使用并在上述公式中选取不同$i$和$j$以熟悉\autoref*{def:11:递归式定义},并注意体会递归式定义的含义.
\end{example}

递归式定义有一个相关的结论如下:
\begin{theorem}
    $n$阶行列式$D=|a_{ij}|_{n \times n}$的某一行(列)元素与另一行(列)相应元素的代数余子式
    的乘积之和等于0,即
    \begin{gather}
        \label{eq:11:递归式定义3}
        \sum_{k=1}^{n}a_{kj}A_{ki}=a_{1j}A_{1i}+a_{2j}A_{2i}+\cdots+a_{nj}A_{ni}=0 \quad j \neq i \\
        \label{eq:11:递归式定义4}
        \sum_{k=1}^{n}a_{jk}A_{ik}=a_{j1}A_{i1}+a_{j2}A_{i2}+\cdots+a_{jn}A_{in}=0 \quad j \neq i
    \end{gather}
\end{theorem}
我们若将行列式第$j$列元素替换为第$i$列元素,那么上述\autoref{eq:11:递归式定义3} 根据\autoref{def:11:递归式定义} 就是在求替换后的行列式,
并且有两列元素相同的行列式为0,我们便可以轻松地证明\autoref*{eq:11:递归式定义3},同理也可证明\autoref*{eq:11:递归式定义4}. 同学们可能
对\crefrange*{eq:11:递归式定义1}{eq:11:递归式定义4} 式繁杂的下标感到陌生,因此安排了\crefrange*{ex:11:公理化定义2}{ex:11:递归式定义} 希望大家熟悉这些公式.
\begin{example} \label{ex:11:递归式定义2}
    对\autoref{ex:11:递归式定义} 中的矩阵验证\autoref{def:11:公理化定义} 的正确性.
\end{example}
这一节中行列式是按照一行(列)展开的,若按多行(列)展开则需要相应的 Laplace 定理,感兴趣的同学可以了解.
\subsection{行列式的常用性质}
设$A,B \in \mathbf{F}^{n \times n}$,$k \in \mathbf{F}$,则
\begin{enumerate}
    \item 一般情况下,$|A \pm B| \neq |A|\pm|B|$;

    \item $|kA|=k^n|A|$;

    \item $A$可逆$\iff |A| \neq 0$;

    \item 初等矩阵行列式(注意初等矩阵不分行列,左乘右乘区分初等行列变换):$|E_{ij}|=-1,\enspace |E_i(c)|=c,\enspace |E_{ij}(k)|=1$;

    \item 利用4中的结论可以得到$|A^\mathrm{T}|=|A|$:

    \item 利用4中的结论可以得到$|AB|=|A||B|$,$|A^k|=|A|^k$;

    \item 利用4中的结论(求出初等矩阵逆矩阵行列式)可以得到若$A$可逆,则$|A^{-1}|=|A|^{-1}$.
\end{enumerate}

以上性质都可以基于定义或上述其他性质得到,下面介绍的性质需要用到``打洞法''(分块矩阵初等变换)
来证明:

\begin{enumerate}
    \item $\begin{vmatrix}
        A & O \\ O & B
    \end{vmatrix} = \begin{vmatrix}
        A & O \\ C & B
    \end{vmatrix} = \begin{vmatrix}
        A & D \\ O & B
    \end{vmatrix} = |A||B|,\ \begin{vmatrix}
        O & A \\ B & C
    \end{vmatrix} = (-1)^{kr}|A||B|$;

    \item 当$A$可逆时,有$\begin{vmatrix}
        A & B \\ C & D
    \end{vmatrix} = |A||D-CA^{-1}B|$,当$D$可逆时,有
    $\begin{vmatrix}
        A & B \\ C & D
    \end{vmatrix} = |D||A-BD^{-1}C|$,当$B$可逆时,有
    $\begin{vmatrix}
        A & B \\ C & D
    \end{vmatrix} = (-1)^{mn}|B||C-DB^{-1}A|$,当$C$可逆时,有
    $\begin{vmatrix}
        A & B \\ C & D
    \end{vmatrix} = (-1)^{mn}|C||B-AC^{-1}D|$;

    \item 设$A$和$B$分别是$n \times m$和$m \times n$矩阵,则$|E_n \pm AB|=|E_m \pm BA|$,且
    $|\lambda E_n \pm AB|=\lambda^{n-m}|\lambda E_m \pm BA|(n \geqslant m)$.
\end{enumerate}

还有一部分由这些性质可以推导的其他性质将出现在C组习题中供参考.这部分主要是技巧性内容,可以选择性完成.

\section{行列式的基本运算}
本节内容按照往年经验不是考试重点,实际上行列式这章在往年单独出现的
频率较低,一般都在求解特征值或者作为判断可逆等情况下作为结论使用,但是我们仍然需要掌握基本的行列式计算方法,
至少教材中给出的例题需要熟悉.

一般而言行列式的计算方法有根据定义求解(包括逆序数定义(包括低阶行列式直接计算公式)、公理化定义和递归式定义(或 Laplace 定理))、
高斯消元法化为上三角矩阵求解、拆分法(大拆分法、小拆分法)求解、加边法(升阶法)求解、特殊行列式求解
(如``箭形''行列式,循环行列式,Vandermonde 行列式)、递推法求解、数学归纳法求解(这两种方法一般用于大对角形)、
打洞法求解以及利用特征值求解等,具体方法我们将在下一讲中作为拓展展开讲解.

事实上,范德蒙行列式有着广泛的应用,在此我们证明\autoref{thm:4:覆盖定理}的有限维情形作为一个例子:
\begin{example}\label{ex:11:行列式证明覆盖定理}
    设$V_1,V_2,\ldots,V_s$是有限维线性空间$V$的$s$个非平凡子空间,证明:$V$中至少存在一个向量
    不属于$V_1,V_2,\ldots,V_s$中的任何一个,即$V_1 \cup V_2 \cup \cdots \cup V_s\subsetneq V.$
\end{example}
\begin{proof}
    设$\dim V=n$,设$\vec{\alpha_1},\vec{\alpha_2},\cdots,\vec{\alpha_n}$为$V$的一组基,构造向量组$\{\vec{\beta_k}\}$中每个元素满足
    \[\vec{\beta_k}=\vec{\alpha_1}+k\vec{\alpha_2}+\cdots+k^{n-1}\alpha_n,k=1,2,3,\cdots\]
    任取上述向量组中的$n$个向量$\vec{\beta_{k_1}},\vec{\beta_{k_2}},\cdots,\vec{\beta_{k_n}}$,其中
    $k_1<k_2<\cdots<k_n$,则有
    \[(\vec{\beta_{k_1}},\vec{\beta_{k_2}},\cdots,\vec{\beta_{k_n}})=(\vec{\alpha_1},\vec{\alpha_2},\cdots,\vec{\alpha_n})C\]
    其中
    \[C=\begin{pmatrix}
        1 & 1 & \cdots & 1 \\
        k_1 & k_2 & \cdots & k_n \\
        \vdots & \vdots &  & \vdots \\
        k_1^{n-1} & k_2^{n-1} & \cdots & k_n^{n-1}
    \end{pmatrix}.\]
    则$|C|$是一个范德蒙行列式,由范德蒙行列式的性质可知$|C| \neq 0$,因此$C$可逆,又由于
    $\vec{\alpha_1},\vec{\alpha_2},\cdots,\vec{\alpha_n}$是$V$的一组基,因此
    $\vec{\beta_{k_1}},\vec{\beta_{k_2}},\cdots,\vec{\beta_{k_n}}$线性无关,从而
    向量组$\{\vec{\beta_k}\}$中任意$n$个向量均构成$V$的一组基.

    由于$V_1,V_2,\ldots,V_s$是$V$的非平凡子空间,因此每个子空间最多包含$\{\vec{\beta_k}\}$中$n-1$个向量,
    进而$V_1\cup V_2\cup\ldots\cup V_s$只包含$\{\vec{\beta_k}\}$中有限个向量,
    所以必然存在一个向量$\vec{\beta}_j$使得$\vec{\beta}_j \notin V_1\cup V_2\cup\ldots\cup V_s$.
\end{proof}

\section{伴随矩阵}
伴随矩阵是一个重要的概念,其性质都比较经典,而且往年也有考察.
\begin{definition}
    称矩阵$A^*=\begin{pmatrix}
        A_{11} & A_{21} & \cdots & A_{n1} \\
        A_{12} & A_{22} & \cdots & A_{n2} \\
        \vdots & \vdots & \ddots & \vdots \\
        A_{1n} & A_{2n} & \cdots & A_{nn}
    \end{pmatrix}$为$A$的\keyterm{伴随矩阵}[adjugate matrix],其中$A_{ij}$是元素$a_{ij}$的代数余子式.
\end{definition}
我们要特别注意伴随矩阵代数余子式的下标与通常矩阵下标不一致,与转置下标一致.
伴随矩阵具有以下几个重要性质,请各位同学掌握其证明并理解:
\begin{example} \label{ex:11:伴随矩阵}
    证明下列关于$n$阶矩阵$A$的伴随矩阵$A^*$的性质:
    \begin{enumerate}
        \item $AA^*=A^*A=|A|E$,若$A$可逆,则有$A^{-1}=|A|^{-1}A^*$,$A^*=|A|A^{-1}$,$(A^*)^{-1}=(A^{-1})^*=|A|^{-1}A$.

        \item $r(A^*)=\begin{cases}
        n & r(A)=n \\ 1 & r(A)=n-1 \\ 0 & r(A) < n-1
    \end{cases}$.

        \item $|A^*|=|A|^{n-1}$,无论$A$是否可逆.

        \item $(AB)^*=B^*A^*$,$(A^\mathrm{T})^*=(A^*)^\mathrm{T}$,$(kA)^*=k^{n-1}A^*$,要求掌握$A$可逆时的证明,
        若不可逆则需要使用第二节习题C组中对角占优的推论证明.

        \item $(A^*)^*=|A|^{n-2}A$,$|(A^*)^*|=|A|^{(n-1)^2}$,无论$A$是否可逆(本题结论可以推广到更多重的伴随矩阵).

        \item 对正整数$k$,$(A^k)^*=(A^*)^k$.
    \end{enumerate}
\end{example}

在计算行列式时若出现伴随矩阵,我们经常使用\autoref{ex:11:伴随矩阵} 中的1,3进行计算.

使用伴随矩阵求逆矩阵是一种矩阵求逆的方式,我们通过一个简单的例子复习伴随矩阵的基本定义和性质:
\begin{example}
    判断矩阵$\begin{pmatrix}
        1 & 2 & 3 \\ 2 & 1 & 2 \\ 1 & 3 & 3
    \end{pmatrix}$是否可逆. 若可逆,利用伴随矩阵求其逆矩阵.
\end{example}

\section{Cramer法则}
本节内容大家了解即可,要了解Cramer法则的内容及其与线性方程组的关系.教材中还有部分几何的内容,
感兴趣的同学可以了解,当然一般而言几何部分不会考察.
\begin{theorem} \keyterm{Cramer法则} \label{thm:11:Cramer}
    对线性方程组
    \begin{gather}
        \label{eq:11:线性方程组1}
        \left\{ \begin{array}{rcl}
            a_{11}x_1+a_{12}x_2+\cdots+a_{1n}x_n&=&0 \\
            a_{21}x_1+a_{22}x_2+\cdots+a_{2n}x_n&=&0 \\
            &\vdots& \\
            a_{n1}x_1+a_{n2}x_2+\cdots+a_{nn}x_n&=&0
        \end{array} \right.
        \\
        \label{eq:11:线性方程组2}
        \left\{ \begin{array}{rcl}
            a_{11}x_1+a_{12}x_2+\cdots+a_{1n}x_n&=&b_1 \\
            a_{21}x_1+a_{22}x_2+\cdots+a_{2n}x_n&=&b_2 \\
            &\vdots& \\
            a_{n1}x_1+a_{n2}x_2+\cdots+a_{nn}x_n&=&b_n
        \end{array} \right.
    \end{gather}

    令$D=\begin{vmatrix}
        a_{11} & a_{12} & \cdots & a_{1n} \\
        a_{21} & a_{22} & \cdots & a_{2n} \\
        \vdots & \vdots & \ddots & \vdots \\
        a_{n1} & a_{n2} & \cdots & a_{nn}
    \end{vmatrix},D_1=\begin{vmatrix}
        b_1 & a_{12} & \cdots & a_{1n} \\
        b_2 & a_{22} & \cdots & a_{2n} \\
        \vdots & \vdots & \ddots & \vdots \\
        b_n & a_{n2} & \cdots & a_{nn}
    \end{vmatrix},\ldots,D_n=\begin{vmatrix}
        a_{11} & a_{12} & \cdots & b_1 \\
        a_{21} & a_{22} & \cdots & b_2 \\
        \vdots & \vdots & \ddots & \vdots \\
        a_{n1} & a_{n2} & \cdots & b_n
    \end{vmatrix}$,其中$D$称为系数行列式.
    \begin{enumerate}
        \item 方程组 \ref{eq:11:线性方程组1} 只有零解$\iff D \neq 0$,有非零解(无穷多解)$\iff D=0$,即$r(A)<n$;

        \item 方程组 \ref{eq:11:线性方程组2} 有唯一解$\iff D \neq 0$,此时$x_i=\dfrac{D_i}{D}\enspace(i=1,2,\ldots,n)$,
        当$D=0$时,方程组 \ref{eq:11:线性方程组2} 要么无解,要么有无穷多解.
    \end{enumerate}
\end{theorem}
我们可以用 Cramer 法则求解线性方程组,但要注意只有方程个数与未知数个数相等时才能使用,
并且需要系数行列式不为0.
\begin{example}
    求解方程组$\begin{cases}
        x_1+x_2+x_3=1 \\
        a_1x_1+a_2x_2+a_3x_3=0 \\
        a_1^2x_1+a_2^2x_2+a_3^2x_3=0
    \end{cases}$,其中$a_1,a_2,a_3$两两不等.
\end{example}

\section{行列式的秩}
\subsection{行列式的秩}
首先我们需要给出矩阵的子式、主子式的定义,然后给出相关的顺序主子式的定义.
\begin{definition}
    矩阵$A=(a_{ij})_{n \times n}$的任意$k$行($i_1<i_2<\cdots<i_k$行)和
    任意$k$列($j_1<j_2<\cdots<j_k$列)的交点上的$k^2$个元素排成的行列式
    \[\begin{vmatrix}
        a_{i_1j_1} & a_{i_1j_2} & \cdots & a_{i_1j_k} \\
        a_{i_2j_1} & a_{i_2j_2} & \cdots & a_{i_2j_k} \\
        \vdots & \vdots & \ddots  & \vdots \\
        a_{i_kj_1} & a_{i_kj_2} & \cdots & a_{i_kj_k}
    \end{vmatrix}\]
    称为矩阵$A$的一个$k$阶子式,若子式等于0则称$k$阶零子式,否则称非零子式.
    当$A$为方阵且$i_t=j_t\enspace(t=1,2,\ldots,k)$(即选取相同行列)时,称为$A$的
    $k$阶\keyterm{主子式}[principal minor]. 若$i_t=j_t=t\enspace(t=1,2,\ldots,k)$,称为$A$的$k$阶\keyterm{顺序主子式}[leading principal minor]
    (取前$k$行$k$列的左上角主子式).
\end{definition}
\begin{example}
    写出矩阵$\begin{pmatrix}
        1 & 5 & -2 \\ 2 & 3 & 4 \\ -1 & -3 & 0
    \end{pmatrix}$的所有一阶、二阶子式、主子式和顺序主子式.
\end{example}
接下来我们给出行列式的秩的定义.
\begin{definition}
    矩阵$A$的非零子式的最高阶数$r$称为$A$的行列式秩.
\end{definition}
\raggedright 其含义为$A$至少有一个$r$阶子式不为0,但所有$r+1$阶子式均为0.根据教材定理:
\begin{theorem}
    矩阵$A$的秩$r(A)=r \iff A$的行列式的秩为$r$.
\end{theorem}
\raggedright 我们可以得到上一个专题中矩阵的秩的等价定义.
\begin{definition}
    矩阵$A$的非零子式的最高阶数$r$称为矩阵$A$的秩,记为$r(A)$.
\end{definition}
\begin{example}
    利用定义求矩阵$\begin{pmatrix}
        1 & 1 & -1 & 3 \\ 1 & 2 & 1 & 1 \\ 2 & 3 & 0 & 4
    \end{pmatrix}$的行列式秩.
\end{example}
\subsection{关于秩的总结}
本学期我们一共学习了四个秩的概念:向量组的秩,线性映射的秩,矩阵的秩和行列式的秩.
事实上,我们在矩阵三秩相等性的证明中通过维数公式以及正交补等统一了向量组的秩(行秩、列秩的定义基于向量组的秩)
和线性映射的秩(矩阵的秩的定义基于线性映射的秩),在行列式一章中统一了矩阵的秩和行列式的秩
(利用行列式不为0与可逆的等价性). 至此我们统一了这四个秩,我们了解到虽然线性映射的秩、矩阵的秩、
行列式的秩的定义各不相同,但本质都在于向量组的秩(回顾线性映射的秩的定义,矩阵三秩相等以及行列式子式
不为0与矩阵可逆的关系,矩阵可逆与向量组的秩的关系). 这给我们的启示是上述提到的概念都可以互相转化考虑.
例如考虑可逆时,我们可以考虑行、列向量是否线性无关/矩阵对应的线性映射是否可逆/行列式是否为0等.
虽然说起来很简单,但是实际做题的时候很多同学还是容易思维局限,因此我们需要将这些概念的统一性放在重要的位置.

\vspace{2ex}
\centerline{\heiti \Large 内容总结}

\vspace{2ex}

\centerline{\heiti \Large 习题}
\vspace{2ex}
{\kaishu }
\begin{flushright}
    \kaishu

\end{flushright}
\centerline{\heiti A组}
\begin{enumerate}
    \item
\end{enumerate}
\centerline{\heiti B组}
\begin{enumerate}
    \item
\end{enumerate}
\centerline{\heiti C组}
\begin{enumerate}
    \item
\end{enumerate}

\chapter{行列式计算进阶}

\vspace{2ex}
\centerline{\heiti \Large 内容总结}

\vspace{2ex}

\centerline{\heiti \Large 习题}
\vspace{2ex}
{\kaishu }
\begin{flushright}
    \kaishu

\end{flushright}
\centerline{\heiti A组}
\begin{enumerate}
    \item
\end{enumerate}
\centerline{\heiti B组}
\begin{enumerate}
    \item
\end{enumerate}
\centerline{\heiti C组}
\begin{enumerate}
    \item
\end{enumerate}

\chapter{朝花夕拾}

\section{线性方程组解的一般理论}
本节内容非常重要,一方面这是教材前六章的一大目标,即研究线性方程组不同解的情况
的来由;另一方面在于,这一节的内容在考试中经常出现,因此希望引起重视.

注意:本章在内容排布上与教材略有区别,但教材内容都会涉及.目的在于希望大家
不是死记硬背结论,而是能够遇到变式的定理都能理解并给出证明.如果希望先按照教材思路
回顾,实际上教材本章核心就在于定理6.1-6.3,应熟练掌握其证明并深刻理解其内涵.
\subsection{线性方程组解的一般理论}
\begin{theorem} \textbf{\heiti 线性方程组有解的充要条件}

    线性方程组有解的充分必要条件是其系数矩阵与增广矩阵有相同的秩.
\end{theorem}
定理的证明非常简单,将方程组视为$x_1\vec{\beta}_1+x_2\vec{\beta}_2+\cdots+x_n\vec{\beta}_n=\vec{b}$,
则有解的条件为$\vec{b}$可以被$\vec{\beta}_1,\ldots,\vec{\beta}_n$线性表示,这等价于
向量组$(\vec{\beta}_1,\ldots,\vec{\beta}_n)$与$(\vec{\beta}_1,\ldots,\vec{\beta}_n,\vec{b})$等价,故定理成立.

\begin{theorem} \label{thm:13:方程组解}
    当方程组有解时(注意这个前提),以下定理成立:
    \begin{enumerate}
        \item 如果它的系数矩阵$A$的秩等于未知量的数目$n$,则方程组有唯一解;

        \item 如果$A$的秩小于$n$,则方程组有无穷多个解.
    \end{enumerate}
\end{theorem}
这一定理有一个推论:齐次线性方程组有非零解的充要条件是:它的系数矩阵的秩小于未知量的数目(对于方阵即为行列式一章描述的,有非零解充分必要条件为其行列式为0).
\autoref{thm:13:方程组解} 对应齐次线性方程组即为教材定理6.1的推论,对于非齐次的情况,注意本定理前提是方程组有解.证明时将方程组化为简化阶梯矩阵即可.
\subsection{齐次线性方程组解的一般理论}
对于齐次线性方程组$A\vec{X}=0$,我们有:
\begin{theorem}
    齐次线性方程组$A\vec{X}=0$的解空间为$\mathbf{R}^n$的子空间.
\end{theorem}
请回顾证明子空间的一般方法.在确认其为线性空间后,我们来研究该线性空间的基本性质.首先是由此引出的关于基础解系的概念.
基础解系即为齐次线性方程组解空间的一组基,且这组基的每一个线性组合都是该方程组的解、
然后我们来研究这一空间的维数:
\begin{theorem}
    矩阵$A \in \mathbf{M}_{m \times n}(\mathbf{F})$,若$r(A) = r$,则该齐次线性方程组解空间维数为$n - r$.
\end{theorem}
该定理即为教材定理6.1,证明使用维数公式(教材定理3.2).本定理改写为类似于维数公式的形式即为$r(A) + \dim N(A) = n$.
其中$N(A)$表示$A\vec{X}=0$的解空间.
\begin{example}
    若$n$元齐次线性方程组$A\vec{X} = 0$的解都是$B\vec{X} = 0$的解. 证明:$r(B) \leqslant r(A)$.
\end{example}
\begin{theorem}
    $^*$齐次线性方程组解空间的正交补是由方程组行向量为基张成的线性空间.
\end{theorem}
该定理需要用到正交补的概念,没有学习的班级可以参考教材2.9节简单理解.实际上在矩阵的秩的讲义中已经提到了这一定理.
本定理中考虑了行向量张成的行空间,以往我们考虑列空间更多.

\subsection{非齐次线性方程组解的一般理论}
对于非齐次线性方程组
\begin{equation} \label{eq:13:非齐次}
    x_1\vec{\beta}_1+x_2\vec{\beta}_2+\cdots+x_n\vec{\beta}_n=\vec{b}
\end{equation}
我们将n元齐次线性方程组
\begin{equation} \label{eq:13:齐次}
    x_1\vec{\beta}_1+x_2\vec{\beta}_2+\cdots+x_n\vec{\beta}_n=0
\end{equation}
称为其导出组,则我们有:
\begin{theorem}
    如果$n$元非齐次线性方程组有解,则它的解集$U=\{\gamma_0+\eta \mid \eta \in W\}$.
\end{theorem}
其中$\gamma_0$为\autoref{eq:13:非齐次} 的一个解(称为特解),$W$为\autoref{eq:13:齐次} 的解空间(\autoref*{eq:13:齐次} 的解称为通解).
对于通解+特解,我们可以想象一个3元非齐次线性方程$ax + by + cz = d$ 和齐次线性方程$ax + by + cz = 0$.
非齐次线性方程的解显然对应一个不过原点的平面,而齐次则过原点.
我们便可以认为是齐次线性方程解平面沿着特解对应的向量平移到非齐次线性方程的解平面,这便是这一结论的几何解释.同时我们可以得到下述结论:

1. $n$元非齐次线性方程组 \ref*{eq:13:非齐次} 的两个解的差是它的导出组 \ref*{eq:13:齐次} 的一个解.

2. $n$元非齐次线性方程组 \ref*{eq:13:非齐次} 的一个解与它的导出组 \ref*{eq:13:齐次} 的一个解之和仍是非齐次线性方程组 \ref*{eq:13:非齐次} 的一个解.

这两个性质证明比较简单,实际上根据上述几何描述形象理解也不困难.上述定理与性质对应教材定理6.3,可以参看.
\begin{example}
    设$n$阶矩阵$A$的行列式$|A|\neq 0$,记$A$的前$n-1$列形成的矩阵为$A_1$,$A$的第$n$列为$b$,
    问:线性方程组$A_1\vec{X}=b$是否有解?
\end{example}

\section{理论应用}
我们首先来看四个最为经典的问题:
\begin{example}
    利用线性方程组解的一般理论,证明以下命题:
    \begin{enumerate}
        \item 设$A,B$分别是$m \times n$和$n \times s$矩阵,且$AB=O$,证明:$r(A)+r(B)\leqslant n$;

        \item 设$A$是$m \times n$实矩阵,证明:$r(A^\mathrm{T}A)=r(A)$;

        \item 设$A,B$分别是$m \times n$和$n \times s$矩阵,则$r(AB)\leqslant\min\{r(A),r(B)\}$;

        \item $A^2=A \iff r(A)+r(E-A)=n$,$A^2=E \iff r(A+E)+r(A-E)=n$.
    \end{enumerate}
\end{example}
实际上,我们解决此类问题,很多时候等式都需要拆为小于等于和大于等于同时成立进行证明,经常利用
维数公式变形的齐次线性方程组解的一般理论,将问题转化为对像与核空间的研究,然后利用包含关系
(复杂的题目可能涉及子空间交与和的维数公式)以及已知的简单秩不等式进行证明.
可能部分题目较为困难,但至少请掌握上面例题中的情况.

\section{线性方程组拓展题型}
\subsection{含参数的线性方程组问题}

此类问题一般考察对于含参数的线性方程组,参数取值如何时有解/无解/有唯一解等.
此类问题一般有两种解法,一种是直接对系数矩阵做高斯消元法,另一种是考察系数矩阵行列式
是否为0(特别是系数矩阵行列式为$n$阶特殊行列式时).当然考察系数矩阵行列式的方法不是
通用的,因为有时候参数不在系数矩阵,或者区分非齐次线性方程组无解/有无穷解的情况.
我们来看一个简单的例子:
\begin{example}
    讨论下面方程组的解的情况,并在有解的情况下求解:\[\begin{cases}
        x_1+x_2-x_3=1 \\ 2x_1+3x_2+kx_3=3 \\ x_1+kx_2+3x_3=2
    \end{cases}\]
\end{example}
\subsection{线性方程组同解问题}
两个线性方程组同解实际上有两种情况:
\begin{enumerate}
    \item 两线性方程组都无解;

    \item 两线性方程组都有解且有相同的解集.
\end{enumerate}

我们来看两线性方程组同解的充要条件:
\begin{theorem}
    $n$元齐次线性方程组 $A_{m_1 \times n}\vec{X}=0$与 $B_{m_2 \times n}\vec{X}=0$同解的
    充要条件是$r\begin{pmatrix}
        A \\ B
    \end{pmatrix}=r(A)=r(B)$.
\end{theorem}
\begin{theorem}
    $n$元非齐次线性方程组 $A_{m_1 \times n}\vec{X}=b$与 $B_{m_2 \times n}\vec{X}=d$同解的
    充要条件是
    \begin{enumerate}
        \item $r(A)\neq r(A,b)$且$r(B)\neq r(B,d)$;或
        \item $r\begin{pmatrix}
            A & b \\ B & d
        \end{pmatrix}=r\begin{pmatrix}
            A \\ B
        \end{pmatrix}=r(A)=r(A,b)=r(B)=r(B,d)$.
    \end{enumerate}
\end{theorem}

这两个定理的证明比较简单,作为练习. 这两个定理可以用于两含参方程组同解问题的解决方法,当然应用前需要简要说明
以下这个定理.
\begin{example}
    已知方程组\begin{gather*}\begin{cases}
            x_1+2x_2+3x_3=0 \\ 2x_1+3x_2+5x_3=0 \\ x_1+x_2+ax_3=0
        \end{cases}
        \\
        \begin{cases}
            x_1+bx_2+cx_3=0 \\ 2x_1+b^2x_2+(c+1)x_3=0
        \end{cases}
    \end{gather*}
    同解,求$a,b,c$的值.
\end{example}

\subsection{线性方程组公共解问题}

公共解即为两线性方程组解集的交集,我们从齐次和非齐次讨论有公共解的条件:
\begin{theorem}
    对于$n$元齐次线性方程组 (1) $A_{m_1 \times n}\vec{X}=0$与 (2) $B_{m_2 \times n}\vec{X}=0$有
    \begin{enumerate}
        \item (1) 与 (2) 有非零公共解的充要条件是$r\begin{pmatrix}
                A \\ B
            \end{pmatrix}<n$;

        \item 设$\eta_1,\eta_2,\ldots,\eta_s\enspace(s=n-r(B))$是 (2) 的基础解系,则
        (1) 与 (2) 有非零公共解的充要条件是$A\eta_1,A\eta_2,\ldots,A\eta_s$线性相关;

        \item 设$\gamma_1,\gamma_2,\ldots,\gamma_t\enspace(t=n-r(A))$是(1) 的基础解系,
        $\eta_1,\eta_2,\ldots,\eta_s\enspace(s=n-r(B))$是 (2) 的基础解系,则(1) 与 (2) 有非零公共解的充要条件是
        $\gamma_1,\gamma_2,\ldots,\gamma_t,\eta_1,\eta_2,\ldots,\eta_s$线性相关.
    \end{enumerate}
\end{theorem}
\begin{theorem}
    对于$n$元非齐次线性方程组(1) $A_{m_1 \times n}\vec{X}=b$与 (2) $B_{m_2 \times n}\vec{X}=d$,若(1) 与 (2) 都有解,则
    \begin{enumerate}
        \item (1) 与 (2) 有公共解的充要条件是$r\begin{pmatrix}
                A \\ B
            \end{pmatrix}=r\begin{pmatrix}
                A & b \\ B & d
            \end{pmatrix}$;

        \item 若$r(B)=s$,且$\eta_1,\eta_2,\ldots,\eta_{n-s+1}$是 (2) 的$n-s+1$个线性无关的解,则
        (1) 与 (2) 有公共解的充要条件是$b$是$A\eta_1,A\eta_2,\ldots,A\eta_{n-s+1}$的凸组合,即
        存在数$k_1,k_2,\ldots,k_{n-s+1}$使得
        \[b=k_1A\eta_1+k_2A\eta_2+\cdots+k_{n-s+1}A\eta_{n-s+1},\]
        其中$k_1+k_2+\ldots+k_{n-s+1}=1$;

        \item 若$r(A)=t$,$r(B)=s$,$\gamma_1,\gamma_2,\ldots,\gamma_{n-t+1}$是(1) 的$n-t+1$个线性无关的解,
        $\eta_1,\eta_2,\ldots,\eta_{n-s+1}$是 (2) 的$n-s+1$个线性无关的解,则(1) 与 (2) 有公共解的充要条件是
        存在数$k_1,k_2,\ldots,k_{n-t+1}$和$l_1,l_2,\ldots,l_{n-s+1}$使得
        \[k_1\gamma_1+k_2\gamma_2+\cdots+k_{n-t+1}\gamma_{n-t+1}-l_1\eta_1-l_2\eta_2-\cdots-l_{n-s+1}\eta_{n-s+1}=0\]
        其中$k_1+k_2+\cdots+k_{n-t+1}=1$,$l_1+l_2+\cdots+l_{n-s+1}=1$.
    \end{enumerate}
\end{theorem}
以上两个定理的证明作为练习.当然这两个定理无需记忆,只需要通过证明理解其含义即可.下面我们看一个简单的例子:
\begin{example}
    设四元齐次线性方程组(1) 为\[\begin{cases}
        2x_1+3x_2-x_3=0 \\ x_1+2x_2+x_3-x_4=0
    \end{cases}\]已知另一个四元齐次线性方程组 (2) 的基础解系为
    \[\alpha_1=(2,-1,a+2,1)^\mathrm{T},\enspace\alpha_2=(-1,2,4,a+8)^\mathrm{T}\]
    \begin{enumerate}
        \item 求方程组 (1) 的一个基础解系;

        \item 当$a$为何值时,方程组 (1) 和 (2) 有非零公共解,并求出非零公共解.
    \end{enumerate}
\end{example}

\subsection{线性方程组反问题}

此类问题即已知方程组的解,要给出原方程组.对于齐次的情形,如果大家还记得系数矩阵行向量空间与解空间的
正交补关系,此类题目是容易解决的.对于非齐次线性方程组,则先利用通解得到其导出齐次线性方程组,然后将
特解代入得到需要求的非齐次线性方程组.
\begin{example}
    已知$\alpha_1=(1,2,-1,0,4)^\mathrm{T},\enspace\alpha_2=(-1,3,2,4,1)^\mathrm{T},\enspace\alpha_3=(2,9,-1,4,13)^\mathrm{T}$,
    且有$W=\spa(\alpha_1,\alpha_2,\alpha_3)$.
    \begin{enumerate}
        \item 求以$W$为解空间的一个齐次线性方程组;

        \item 求以$W'=\{\eta+\alpha \mid \alpha\in W\}$为解集的一个非齐次线性方程组,其中$\eta=(1,2,1,2,1)^\mathrm{T}$.
    \end{enumerate}
\end{example}

\vspace{2ex}
\centerline{\heiti \Large 内容总结}

\vspace{2ex}

\centerline{\heiti \Large 习题}
\vspace{2ex}
{\kaishu 即使我说二二得四,三三见九,也没有一字不错。
这些既然都错,则绅士口头的二二得七,三三见千等等,自然就不错了。}
\begin{flushright}
    \kaishu
    ——鲁迅,《朝花夕拾》
\end{flushright}
\centerline{\heiti A组}
\begin{enumerate}
    \item
\end{enumerate}
\centerline{\heiti B组}
\begin{enumerate}
    \item
\end{enumerate}
\centerline{\heiti C组}
\begin{enumerate}
    \item
\end{enumerate}

\chapter{史海拾遗}

\vspace{2ex}
\centerline{\heiti \Large 内容总结}

\vspace{2ex}

\centerline{\heiti \Large 习题}
\vspace{2ex}
{\kaishu }
\begin{flushright}
    \kaishu

\end{flushright}
\centerline{\heiti A组}
\begin{enumerate}
    \item 
\end{enumerate}
\centerline{\heiti B组}
\begin{enumerate}
    \item
\end{enumerate}
\centerline{\heiti C组}
\begin{enumerate}
    \item
\end{enumerate}

\chapter{多项式}

\section{多项式的定义}

我们从多项式的定义开始.对于函数 $p:\mathbf{F}\to\mathbf{F}$,若存在
$a_0,\ldots,a_m\in\mathbf{F}$使得对任意$z\in\mathbf{F}$有
$p(z)=a_0+a_1z+\cdots+a_mz^m$,则称函数$p$为系数在$\mathbf{F}$中的多项式.

对于一个定义,我们很关心它是否能带来一些简便,系数的唯一性便是一个很好的使得研究简便的性质.
回顾之前证明一个向量在一组线性无关向量下的坐标唯一的证明,我们需要零向量在线性无关向量组
下的坐标为0这一条件.在多项式中,这一定理转变为
\begin{theorem}
    设$a_0,\ldots,a_m\in\mathbf{F}$,若对任意$z\in\mathbf{F}$有$a_0+a_1z+\cdots+a_mz^m=0$,
    则$a_0=\cdots=a_m=0$.
\end{theorem}
基于此我们可以得到多项式的系数必然唯一,否则两相等多项式之差为0却可以有非零系数.
(本质上将)$1,x,x^2,\ldots$视为多项式构成的线性空间的一组基即可)

\section{零点与因式}

接下来我们研究多项式的零点与因式,从而可以引出多项式的分解.
我们称$s\in\mathbf{F}[x]$为多项式$p\in\mathbf{F}[x]$的因式,如果
存在多项式$q\in\mathbf{F}[x]$使得$p=sq$.但很多时候并不能整除,因此我们需要引入
多项式的带余除法:
\begin{theorem}
    设$p,s\in\mathbf{F}[x]$且$s\neq 0$,则存在唯一的多项式$q,r\in\mathbf{F}[x]$,
    使得$p=sq+r$,且$\deg r<\deg s$.
\end{theorem}
这一定理实际上是数的带余除法的延伸,证明是基本的,可参考教材4.8.
\begin{example}
    设多项式$f(x)$被$(x-1),(x-2),(x-3)$除后,余式分别为$4,8,16$.求$f(x)$被$(x-1)(x-2)(x-3)$除后的余式.
\end{example}
基于此我们可以研究多项式零点和因式的性质:
\begin{theorem}
    设$p\in\mathbf{F}[x]$.
    \begin{enumerate}
        \item 若$\lambda\in\mathbf{F}$,则$p(\lambda)=0$当且仅当存在多项式
            $q\in\mathbf{F}[x]$使得对每个$z\in\mathbf{F}$均有$p(z)=(z-\lambda)q(z)$;

        \item 若$p$是$m\enspace(m \geqslant 0)$次多项式,则$p$在$\mathbf{F}$上最多有$m$个互不相等的零点.
    \end{enumerate}
\end{theorem}
1 的证明基于带余除法,2只需基于1即可,都非常基本.我们希望2的最多能够在复数域的情况下取到,
这需要接下来这一基本而伟大的定理——代数学基本定理作为支撑:
\begin{theorem} \textbf{\heiti 代数学基本定理} \label{thm:14:代数学基本定理}
    非常数复多项式在复平面上必有零点.
\end{theorem}

代数基本定理最简单直接的证明来源于复分析中的刘维尔定理或柯西积分公式,感兴趣的同学可以学习,
这里只需要承认这一定理.基于此我们可以进行多项式的分解,我们分复数域和实数域进行讨论:
\begin{theorem} \label{thm:14:多项式分解}
    设$p\in\mathbf{F}[x]$是非常数多项式,则$p$可以唯一分解(不计因式的次序)为
    \begin{enumerate}
        \item $(\mathbf{F}=\mathbf{C})\quad p(z)=c(z-\lambda_1)\cdots(z-\lambda_m)$,
        其中$c,\lambda_1,\ldots,\lambda_m\in\mathbf{C}$;

        \item $(\mathbf{F}=\mathbf{R})\quad p(x)=c(x-\lambda_1)\cdots(x-\lambda_m)
        (x^2+b_1x+c_1)\cdots(x^2+b_Mx+c_M)$,其中$c,\lambda_1,\ldots,\lambda_m,b_1,\ldots,b_M,
        c_1,\ldots,c_M\in\mathbf{R}$,并且对每个$j$均有$b_j^2<4c_j$.
    \end{enumerate}
\end{theorem}
1 的证明基于代数学基本定理是简单的,唯一性也较为显然. 2的证明需要教材4.15和4.16的支持,
剩余的证明也是基本的.
\begin{example}
    证明:每个奇数次的实系数多项式都有实的零点.
\end{example}
\begin{example}
    设$p\in\mathbf{F}[x]$且$q\neq 0$.证明:$c$是$f(x)$的$k\enspace(k\geqslant 1)$重根的充要条件为
    \[f(c)=f'(c)=\cdots=f^{(k-1)}(c),\enspace f^{(k)}(c)\neq 0.\]
\end{example}
除此之外需要提及的是,若上述定理中$c=1$,则多项式最高次数的项的系数为1,则称这一多项式为\keyterm{首一多项式}[monic polynomial].

\section{整除与互素}

提到因式我们很容易想到类似于整数的最大公因数的定义,这里我们依次引入整除、公因式和最大公因式的概念:
\begin{definition}
    设$p,q\in\mathbf{F}[x]$且$q\neq 0$,则$q$整除$p$或$p$能被$q$整除(记为$q \mid p$)当且仅当
    $p$除以$q$的余式为0.
\end{definition}
\begin{example}
    设$p,q\in\mathbf{F}[x]$,证明:$p^2 \mid q^2\iff p \mid q$.
\end{example}
\begin{definition}
    在$\mathbf{F}[x]$中,若$s \mid p$且$s \mid q$,则称$s$是$p$和$q$的一个公因式.若$p$和$q$的公因式$s$
    满足对$p$和$q$的任一公因式$s'$都有$s' \mid s$,则称$s$是$p$和$q$的一个最大公因式.
\end{definition}
故我们可以看出,当$p$和$q$不为0时,最大公因式即为次数最大的公因式.相应的,我们也有最小公倍式的定义,这也类似于
整数中的最小公倍数,我们不再赘述.

类似于整数中的辗转相除法(或称欧几里得算法),对于多项式我们有如下结论:
\begin{theorem}\label{thm:14:欧几里得算法}
    设$p,q\in\mathbf{F}[x]$,存在它们的一个最大公因式$s$,则存在$u,v\in\mathbf{F}[x]$
    使得\[s=up+vq.\]
\end{theorem}
证明与欧几里得算法的构造是类似的,感兴趣的同学可以回顾初等数论的知识.在本节中我们更重视$p,q$最大公因式
为1的情况,我们称之为\keyterm{互素}[coprime],我们可以得到一个多项式互素的充要条件:
\begin{theorem}\label{thm:14:裴蜀定理}
    设$p,q\in\mathbf{F}[x]$,则$p$和$q$互素的充要条件是存在$u,v\in\mathbf{F}[x]$
    使得\[up+vq=1.\]
\end{theorem}
这一定理称之为\keyterm{裴蜀定理}[B\'ezout's Lemma],必要性根据\autoref{thm:14:欧几里得算法} 可以直接得到,充分性可以设$p$和$q$
的公因式为$s$,于是等式两边可以同时约去$s$,这表明必有$s \mid 1$,故$s=1$,$p,q$必然互素.
\begin{example}
    证明:$\mathbf{F}[x]$中两个次数大于0的多项式没有公共复根的充要条件是它们互素.
\end{example}

\vspace{2ex}
\centerline{\heiti \Large 内容总结}

\vspace{2ex}

\centerline{\heiti \Large 习题}
\vspace{2ex}
{\kaishu }
\begin{flushright}
    \kaishu

\end{flushright}
\centerline{\heiti A组}
\begin{enumerate}
    \item
\end{enumerate}
\centerline{\heiti B组}
\begin{enumerate}
    \item
\end{enumerate}
\centerline{\heiti C组}
\begin{enumerate}
    \item
\end{enumerate}

\chapter{不变子空间}

\section{不变子空间的定义}
在介绍本节内容前,我们需要首先对``算子''这一名词进行解释.
\begin{definition}
    向量空间到其自身的线性映射称为\keyterm{算子}[operator].
\end{definition}
以上是《线性代数应该这样学》对于算子的定义,本章中出现算子一词也默认为此含义.
实际上,狭义的算子指从一个函数空间到另一个函数空间(或其自身)的映射,例如
微积分中学习的梯度,散度以及 Laplace 算子等.本书中采用的是广义的定义,将算子
这一定义延伸至向量空间.

需要注意的是,Done Right 喜欢用抽象的算子作为研究对象,一般的高等代数则更喜欢具象的矩阵,
实际上二者是完全统一的,很多定理虽然用算子描述,实际上也有相应的矩阵版本.因为实际上矩阵$A$可以视为
算子$T\alpha=A\alpha$在自然基下的矩阵,由此可以做到统一.

在我们接下来的叙述中,我们希望将算子在基下的矩阵表示尽可能简单.在下面的内容中,有三个关键概念是
要经常提及的:算子,矩阵以及多项式,接下来的主干内容就是围绕这三者之间的关系展开.我可以在此画一个
三角形,各边连线上的内容大家可以在阅读过程中自己补全.

\begin{figure}[h]
    \centering
    \begin{tikzpicture}
        \node (A) at (0,0) {矩阵};
        \node (B) at (3,0) {多项式};
        \node (C) at (1.5,2) {算子};
        \draw[thick] (A) -- (B) -- (C) -- (A) -- cycle;
    \end{tikzpicture}
\end{figure}

我们在上一章中讨论了算子可对角化的充要条件.但事实上存在大量不可对角化
的算子,我们也希望得到其最简单的矩阵表示形式.在定理\ref{可对角化充要条件} % TODO 引用
中我们可以看到,$V$上算子$T$可对角化当且仅当$V$可以分解为$T$的特征子空间的
直和:\[V=V_{\lambda_1}\oplus V_{\lambda_2}\oplus\cdots\oplus V_{\lambda_s},\]
其中$\lambda_1,\lambda_2,\ldots,\lambda_s$为$T$的所有不同本征值.我们注意到
$\alpha\in V_{\lambda_i},\enspace T\alpha=\lambda_i\alpha\in V_{\lambda_i}$,这启示
我们可以将$V$分解为具有这一性质的子空间的直和来研究矩阵的简化形式.满足这一性质的
空间对于我们的研究非常重要,我们需要给予它一个定义:
\begin{definition}
    设$T\in \mathcal{L}(V)$,若$V$的子空间$U$满足$\forall \alpha\in U,\enspace T\alpha\in U$,
    则称$U$是$T$的\keyterm{不变子空间}[invariant subspace],简称为$T$-子空间.
\end{definition}
即不变子空间中的每一个向量在算子作用后仍在这一空间中.为了研究在某一子空间下映射的性质,
我们还需要引入映射的限制的概念:
\begin{definition}
    设$g:A\to B$是一个映射,在$A$的子集$A_0$上定义$f:A_0\to B$满足$f(a)=g(a),\enspace\forall a\in A_0$,
    则称$f$为$g$关于集合$A_0$的限制映射,记为$f=g\vert_{A_0}$.
\end{definition}
即限制映射就是将原映射的定义域进行收缩,但原定义域上的函数值保持不变.
事实上,若映射为线性映射,限制的集合为线性映射的不变子空间,则这一映射限制在这一空间上成为算子,
我们称其为\keyterm{限制算子}[restriction operator].

如果$U$是$T$的不变子空间,那么$T$还可以诱导出商空间$V/U$上的一个线性变换$T/U$,满足
$(T/U)(v+U)=Tv+U$,其中$v\in V$,称之为\keyterm{商算子}[quotient operator].

这一定义的线性性容易验证,这里需要提及的是合理性(即是否是良定义(well-defined)的).
事实上,对于一个映射,其合理性在于原像集合中的一个元素只能映射到像集中的唯一一个值
(否则不符合映射的定义).商算子的出发空间元素是等价类,因此如果出现$v+U=w+U$但$Tv+U\neq Tw+U$
的情况,这一定义描述的就不是映射,因此不是良定义.但我们可以验证这一映射是良定义的,详见教材106页.
\begin{example}
    设$T\in \mathcal{L}(V,W)$,定义$\tilde{T}:(V/(\ker T))\to W$如下:
    \[\tilde{T}(v+\ker T)=Tv.\]
    \begin{enumerate}
        \item $\tilde{T}$是良定义的,且是$(V/(\ker T))$到$W$上的线性映射;

        \item $\tilde{T}$是单射;

        \item $\im \tilde{T}=\im T$;

        \item $V/(\ker T)$同构于$\im T$.
    \end{enumerate}
\end{example}
\begin{example}
    设$T\in \mathcal{L}(V)$,证明:
    \begin{enumerate}
        \item $T/(\im T)=0$;

        \item $T/(\ker T)$是单射$\iff \ker T\cap\im T=\{0\}$.
    \end{enumerate}
\end{example}
教材例5.3给出了四个常见的不变子空间的例子,分别是两个平凡子空间和映射的像与核.教材8.20还给出了$p$
为多项式时,$\ker p(T)$和$\im p(T)$也为$T$的不变子空间.但有时我们可能会遇到
更为复杂的情形,如下面的例子:
\begin{example}
    设$V$是$n$维复向量空间,$T\in \mathcal{L}(V)$,若$T$有$n$个互异的本征值,求$T$的所有不变子空间的个数.
\end{example}
\begin{example}
    设$\mathbf{F}$为一数域,算子$T$定义为
    \[T(a,b)=(a,b)\begin{pmatrix}
        1 & -1 \\ 2 & 2
    \end{pmatrix}\]
    证明:
    \begin{enumerate}
        \item 当$\mathbf{F}=\mathbf{R}$时,$\mathbf{R}^2$无$T$的非零真不变子空间;

        \item 当$\mathbf{F}=\mathbf{C}$时,$\mathbf{C}^2$有$T$的非零真不变子空间.
    \end{enumerate}
\end{example}
除此之外,还有一些问题我们将在讨论若当标准形时进行讨论.

\section{特征值与特征向量}
从本章起,我们主要研究线性变换(而非一般线性映射)和方阵的性质.
\subsection{特征值与特征向量的定义与求解}
首先介绍线性变换和矩阵的特征值与特征向量的概念:
\begin{definition}
    设$\sigma$是线性空间$V(\mathbf{F})$上的一个线性变换,如果存在数$\lambda\in\mathbf{F}$
    和非零向量$\xi\in V$使得$\sigma(\xi)=\lambda\xi$,则称数$\lambda$为$\sigma$的一个\keyterm{特征值}[eigenvalue],
    并称非零向量$\xi$为$\sigma$属于其特征值$\lambda$的\keyterm{特征向量}[eigenvector].
\end{definition}
必须注意特征向量为非零向量,否则零向量对任意$\lambda$都满足上面定义,从而失去``特征''的含义.
但是特征值可以为0,此时实际上就是线性变换的零空间.

特征值与特征向量的几何意义在于,某一线性变换的特征向量在经过变换后得到的向量与原先向量共线.

我们称关于同一个特征值$\lambda$的所有特征向量构成的集合记为$V_\lambda=\{\xi \mid \sigma(\xi)=\lambda\xi,\enspace\xi\in V\}$,
称为$\sigma$关于其特征值$\lambda$的特征子空间.
\begin{example}
    证明:$V_\lambda$是$V$的子空间.
\end{example}
实际上,$\sigma(\xi)=\lambda\xi$等价于$(\lambda I-\sigma)(\xi)=0$,故特征子空间就是线性变换
$\lambda I-\sigma$的核,核中必有非零向量(特征向量非零),故$\lambda I-\sigma$在$\lambda$为特征值时
必不满秩,故我们可以通过$|\lambda E-A|=0$求解特征值,其中$A$为$\sigma$在某组基下的矩阵.
对于特征向量的求解,求出$(\lambda E-A)X=0$的非零解就是特征向量在基下的坐标.

上面是线性变换的特征值与特征向量的定义,我们在考试中更一般地会遇到下述矩阵的特征值与特征向量的定义:
\begin{definition}
    设矩阵$A\in \mathbf{M}_n(\mathbf{F})$,如果存在数$\lambda\in\mathbf{F}$和非零向量$X\in\mathbf{F}^n$使得
    $AX=\lambda X$,则称数$\lambda$为$A$的一个特征值,称非零向量$X$为$A$属于其特征值$\lambda$的特征向量.
\end{definition}
\begin{example}
    设$A=\begin{pmatrix}
        1 & -1 & 0 \\ 2 & 0 & 1 \\ 1 & a & 0
    \end{pmatrix}$,且存在非零向量$\alpha$使得$A\alpha=2\alpha$,求$\alpha$.
\end{example}
下面我们说明这两个定义的关系.实际上,假设$A$是$\sigma$在基$\alpha_1,\ldots,\alpha_n$下的表示矩阵,且
$\xi=(\alpha_1,\ldots,\alpha_n)X$,我们有
\begin{align*}
    \sigma(\xi)=\lambda\xi &\iff \sigma(\alpha_1,\ldots,\alpha_n)X=\lambda(\alpha_1,\ldots,\alpha_n)X \\
                           &\iff (\alpha_1,\ldots,\alpha_n)AX=(\alpha_1,\ldots,\alpha_n)(\lambda X) \\
                           &\iff AX=\lambda X
\end{align*}
因此$\lambda$同时是线性变换和矩阵的特征值,与基的选取无关,但$X$与基的选取有关.

我们在之前已经分析了求解特征值的方法,即求解$f(\lambda)=|\lambda E-A|$的根. 我们称其为矩阵$A$的特征多项式.
其$k$重根称为$k$重特征值(或称代数重数),对应的特征子空间维数称几何重数.

我们展开特征多项式得到以下定理:
\begin{theorem}
    对于$n$级矩阵$A=(a_{ij})$,记
    \[f(\lambda)=|\lambda E-A|=a_0\lambda^n+a_1\lambda^{n-1}+\cdots+a_{n-1}\lambda+a_n.\]
    则$a_0=1$,$a_n=(-1)^n|A|$,且$a_k$等于所有$k$级主子式之和乘以$(-1)^k$.
\end{theorem}
这一定理的证明无需掌握,并且关于特征多项式的进一步讨论也将在线性代数II中涉及.这里我们主要掌握两个特例,
即由韦达定理,我们有$\displaystyle\sum_{i=1}^{n}\lambda_i=\displaystyle\sum_{i=1}^{n}a_{ii}$,
$\displaystyle\prod_{i=1}^{n}\lambda_i=|A|$,
即特征值按重数求和为矩阵的迹(即矩阵对角线元素之和),特征值按重数求积为矩阵行列式.
这一结论在解决某些问题时有一定作用.

\subsection{特征值的基本性质}
关于特征值,我们有如下基本性质,证明较为基本,可以自行完成:

设$\lambda$是线性空间$V(\mathbf{F})$上的线性变换$\sigma$的特征值,$\xi$是$\sigma$属于$\lambda$的特征向量,则

1. $k\lambda$是$k\sigma$的特征值,$\lambda^m$是$\sigma^m$的特征值,且$\xi$仍是相应特征向量.

2. 若$f(x)=a_nx^n+a_{n-1}x^{n-1}+\cdots+a_1x+a_0$是$\mathbf{F}$上的多项式,则$f(\sigma)(\xi)=f(\lambda)\xi$.

3. 设$\lambda$是$n$阶矩阵$A$的特征值,$A$可逆,则$\lambda^{-1}$是$A^{-1}$的特征值,$|A|\lambda^{-1}$是$A$的伴随矩阵
$A^*$的特征值,且特征向量不变.
\begin{example}
    回答以下问题:
    \begin{enumerate}
        \item 设$A$为三阶矩阵,$A^2-A-2E=O$,$|A|=2$,求$|A^*+3E|$;

        \item 设$A$为三阶矩阵,其特征值为$1,-2,-1$,求$|A|$,$A^*+3E$的特征值,$(A^{-1})^2+2E$的特征值
            以及$|A^2-A+E|$;

        \item 设$\alpha=(1,0,-1)^\mathrm{T}$,且$A=\alpha\alpha^\mathrm{T}$,求$|6E-A^n|$;

        \item 设$A$为三阶矩阵,其特征值为$-1,-1,5$,求$A_{11}+A_{22}+A_{33}$;

        \item 设$A$为三阶实对称矩阵,$A^2=A$且$r(A)=2$,求$|A+2E|$.
    \end{enumerate}
\end{example}

下面一个例子也是重要的结论,实际上在行列式专题已给出类似结论,但我们现在从特征值角度考虑这一结论:
\begin{example}
    回答以下两个问题:
    \begin{enumerate}
        \item 设$A,B$均为$n$阶矩阵,证明:$\lambda\neq 0$是$AB$的特征值,则$\lambda$也是$BA$的特征值;

        \item 设$A\in \mathbf{M}_{m\times n}(\mathbf{C}),\enspace B\in \mathbf{M}_{n\times m}(\mathbf{C})$,证明:
        \[ \begin{pmatrix}
            AB & O \\ B & O
        \end{pmatrix}\sim\begin{pmatrix}
            O & O \\ B & BA
        \end{pmatrix} \]
        并由此推出$AB$和$BA$非零特征值相同,且$m=n$时有$|\lambda E-AB|=|\lambda E-BA|$.
    \end{enumerate}
\end{example}
不难发现2是1的推广.下面这一例子也是一些经典的结论,应当熟悉.
\begin{example}
    对下列矩阵$A$的特征值,能做出怎样的断言?
    \begin{enumerate}
        \item $A$可逆/$A$不可逆/$E+A$可逆/$4E+A$不可逆;

        \item $\det(E-A^2)=0$;

        \item $AA^\mathrm{T}=A^\mathrm{T}A=E$(正交)/$A^2=E$(对合)/$A^2=A$(幂等)/$A^k=0$(幂零);

        \item $A=\lambda_0E+B$($\lambda_0$为常数,且已知$B$的$n$个特征值为$\lambda_1,\lambda_2,\ldots,\lambda_n$);

        \item $A$为对角块矩阵,即$A=\diag(A_1,A_2,\ldots,A_m)$(与线代II中不变子空间有关).
    \end{enumerate}
\end{example}
在上一小节我们讨论了特征值之和为迹的事实,实际上关于迹以及相关的幂零矩阵的讨论在专题三中已有涉及,
现在可以回过头去再看一些性质的证明.除此之外,我们还可以给幂零矩阵一个等价的定义:
\begin{theorem}
    一个方阵为幂零矩阵当且仅当其所有特征值均为0.
\end{theorem}
这一定理``仅当''部分证明比较基本,只需用到上述某一特征值性质即可,但``当''的部分无需掌握证明,需要用到线性代数II的
Hamilton-Cayley 定理(事实上很多更进一步的讨论都要基于这一定理).
\subsection{特征向量的基本性质}
这一部分的定理与下一节中可对角化的等价条件直接相关,实际上有了本节的定理,可对角化条件是很显然的.
\begin{theorem}
    线性映射$\sigma$的不同特征值$\lambda_1,\ldots,\lambda_m$对应的特征向量$\xi_1,\ldots,\xi_m$线性无关.
\end{theorem}
\begin{theorem}
    线性映射$\sigma$的不同特征值$\lambda_1,\ldots,\lambda_m$对应的特征子空间$V_{\lambda_1},\ldots,V_{\lambda_m}$的和是直和,
    即$\dim(V_{\lambda_1}+V_{\lambda_2}+\cdots+V_{\lambda_m})=\displaystyle\sum_{j=1}^{m}\dim V_{\lambda_j}$.
\end{theorem}
以上两个定理的证明可以参考教材定理7.7及其推论,实际上二者等价,只需证出其中一个,另一个就是显然的.
两个定理有如下推论:
\begin{enumerate}
    \item 若$\lambda_1,\ldots,\lambda_m$是线性映射$\sigma$互异的特征值,则$V_{\lambda_i}\cap\sum\limits_{j\neq i}V_{\lambda_j}=\{0\}
    \enspace(i=1,\ldots,m)$,则一个特征向量不能属于多个特征值.这一推论来源于直和的一个等价条件,专题一的习题中有涉及.

    \item $\sigma$的不同特征值$\lambda_1,\ldots,\lambda_m$对应的特征子空间$V_{\lambda_1},\ldots,V_{\lambda_m}$的基向量
    合在一起构成的向量组线性无关,且是$V_{\lambda_1}+V_{\lambda_2}+\cdots+V_{\lambda_m}$的基.
\end{enumerate}

接下来这个定理说明了代数重数和几何重数之间的关系:
\begin{theorem}
    $n$维线性空间$V(\mathbf{F})$的线性变换$\sigma$的每个特征值$\lambda_i$的重数(代数重数)大于等于其特征子空间$V_{\lambda_i}$的维数
    (几何重数).
\end{theorem}
这一定理的证明比较复杂,版本也很多,有兴趣的同学可以了解.

\vspace{2ex}
\centerline{\heiti \Large 内容总结}

\vspace{2ex}

\centerline{\heiti \Large 习题}
\vspace{2ex}
{\kaishu }
\begin{flushright}
    \kaishu

\end{flushright}
\centerline{\heiti A组}
\begin{enumerate}
    \item
\end{enumerate}
\centerline{\heiti B组}
\begin{enumerate}
    \item
\end{enumerate}
\centerline{\heiti C组}
\begin{enumerate}
    \item
\end{enumerate}

\chapter{相似标准形}

\section{相似的定义与性质}
我们早在线性映射矩阵表示专题中提到这一定理:
\begin{theorem}
    \textbf{\heiti 基的选择对映射矩阵的影响}

    设线性变换$\sigma \in \mathcal{L}(V,V)$,$B_1=\{\alpha_1,\ldots,\alpha_n\}$和$B_2=\{\beta_1,\ldots,\beta_n\}$
    是线性空间的$V(\mathbf{F})$的两组基,基$B_1$变为基$B_2$的变换矩阵为$C$,如果$\sigma$在基$B_1$下的矩阵为$A$,
    则$\sigma$关于基$B_2$所对应的矩阵为$C^{-1}AC$.
\end{theorem}
定理相关习题在相关章节有介绍,教材也有例题.这一定理研究同一个映射在不同基下表示矩阵之间的关系.
我们将具有如上性质的两个矩阵的关系称为相似的,规范定义如下:
\begin{definition}
    若对于$A,B\in \mathbf{M}_n(\mathbf{F})$,存在可逆矩阵$C\in \mathbf{M}_n(\mathbf{F})$, 使得
    $C^{-1}AC=B$,则称$A$相似于$B$,记作$A\sim B$.
\end{definition}
相似矩阵有以下基本性质,证明较为基本,请自行完成:
\begin{enumerate}
    \item 相似是一种等价关系;两矩阵相似必相抵(秩相等);

    \item $A\sim B$可以得到$A^\mathrm{T}\sim B^\mathrm{T}$,$A^m\sim B^m$,更一般地,对于任意多项式$f(x)$都有$f(A)\sim f(B)$,且
    若$B=P^{-1}AP$,有$f(B)=P^{-1}f(A)P$.除此之外还有$A^*\sim B^*$,若$A,B$可逆,有$A^{-1}\sim B^{-1}$,$A^*\sim B^*$;

    \item $A_1\sim B_1$,$A_2\sim B_2$不一定有$A_1+A_2\sim B_1+B_2$,只有当$P^{-1}A_1P=B_1,P^{-1}A_2P=B_2$时
    (即相同的过渡矩阵$P$)才有$P^{-1}(A_1+A_2)P=B_1+B_2$;

    \item 若$A_1\sim B_1$,$A_2\sim B_2$,则有
    \[ \begin{pmatrix}
        A_1 & O \\ O & A_2
    \end{pmatrix}\sim\begin{pmatrix}
        B_1 & O \\ O & B_2
    \end{pmatrix}\]

    \item 相似矩阵有相同的特征多项式(逆命题不成立),即$A\sim B$有$|\lambda E-A|=|\lambda E-B|$,从而有相同的
    迹,行列式,特征值,但特征向量不一定相同;

    \item 与幂等矩阵相似的仍幂等,与对合矩阵相似的仍对合,与幂零矩阵相似的仍幂零
    (但与正交矩阵相似的不一定正交,但与正交矩阵正交相似的是正交矩阵).
\end{enumerate}
\begin{example}
    证明:两个可对角化的同阶矩阵特征值相同(包括重数)等价于它们相似.对于不可对角化的矩阵来说,这一结论还成立吗?
\end{example}
\begin{example}
    (教材定理$4.10$推广)设$P^{-1}AP=B$,证明:$A,B$分别属于同一特征值$\lambda$的特征向量$X$和$Y$满足$Y=P^{-1}X$.
\end{example}
一些题目可能需要判断矩阵是否相似,实际上我们有如下基本方法:
\begin{enumerate}
    \item 定义法:找到$P$使得$P^{-1}AP=B$即可,这一般是$A,B$没给出具体矩阵的做法,例如上面的性质证明;
    \item 我们也可以先计算两者特征多项式是否相等(即特征值是否一致),若不一致则一定不相似,得到结论,若一致且均为实对称矩阵则相似,
    否则不一定相似.于是对于这种特征值一致的情况,我们进行对角化,情况如下:
    \begin{enumerate}[label=(\arabic*)]
        \item 若两矩阵均可对角化,则两矩阵相似(上述例题结论);
        \item 若一个矩阵可对角化,另一个矩阵不可对角化,则一定不相似;
        \item 若两个矩阵都不可对角化,不一定相似.需要两矩阵各个特征值的几何重数(即各个特征子空间维数)都一致才相似,否则不相似
        (了解结论即可,具体原因线性代数II会涉及).
    \end{enumerate}
\end{enumerate}
\begin{example}
    设$A,B\in M_n(\mathbf{F})$,证明:若$A$可逆,则$AB\sim BA$.
\end{example}
\begin{example}
    设$A=\begin{pmatrix}
        0 & 0 & 1 \\ 0 & 1 & 0 \\ 1 & 0 & 0
    \end{pmatrix},B=\begin{pmatrix}
        -1 & 0 & 0 \\ 0 & 0 & 1 \\ 0 & -1 & 2
    \end{pmatrix}$,判断$A$与$B$是否相似.
\end{example}
最后我们谈一个拓展内容,我们考虑矩阵方程$AX-XB=O$,若$A,B$都是$n$阶方阵且$X$可逆,则$A$与$B$相似,所以
这一矩阵方程的解空间的维数实际上刻画了$A$与$B$的相似程度.我们有如下结论,不要求掌握,也不要求证明,了解即可:
\begin{theorem}
    设$A,B$分别为数域$P$上$n$阶、$m$阶方阵,则$A,B$有$r$个两两不等的公共特征值,则矩阵方程$AX-XB=O$有秩为
    $r$的矩阵解.反之,若数域为复数域,矩阵方程$AX-XB=O$有秩为$r$的矩阵解,则$A,B$至少有$r$个公共的特征值
    (计重数).
\end{theorem}
由此,复数域上$n$阶、$m$阶方阵$A,B$的矩阵方程$AX=XB$只有零解的充要条件是$A,B$没有公共特征值.

\section{对角矩阵}
\subsection{可对角化的条件}
我们知道,矩阵可对角化意味着矩阵相似于一个对角矩阵,即存在可逆矩阵$P$使得$P^{-1}AP=\Lambda$,其中
$\Lambda=\diag(\lambda_1,\lambda_2,\ldots,\lambda_n)$为对角矩阵.

将$P^{-1}AP=\Lambda$变形为$AP=P\Lambda$,并将矩阵$P$按列分块为$P=(X_1,X_2,\ldots,X_n)$,则有
$A(X_1,X_2,\ldots,X_n)=(X_1,X_2,\ldots,X_n)\diag(\lambda_1,\lambda_2,\ldots,\lambda_n)$,
利用分块矩阵乘法我们有$AX_j=\lambda_jX_j\enspace(X_j\neq 0,\enspace j=1,2,\ldots,n)$.

通过上述过程我们容易证明$n$维空间上线性映射可对角化当且仅当有$n$个线性无关的特征向量.并且这一过程也是我们
求解对角化问题的基本方法.综合上述推导以及上一节中2.3小节的定理,我们有如下结论:
\begin{theorem}
    设$V$是数域$\mathbf{F}$上的$n$维线性空间,$\sigma$是$V$上的线性变换,$\lambda_1,\lambda_2,\ldots,\lambda_s\in\mathbf{F}$
    是$\sigma$的所有互异特征值,则以下条件等价:
    \begin{enumerate}
        \item $\sigma$可对角化;

        \item $\sigma$有$n$个线性无关的特征向量,它们构成$V$的一组基;

        \item $V=V_{\lambda_1}\oplus V_{\lambda_2}\oplus\cdots\oplus V_{\lambda_s}$;

        \item $n=\dim V_{\lambda_1}+\dim V_{\lambda_2}+\cdots+\dim V_{\lambda_s}$;

        \item $\sigma$每个特征值的代数重数等于几何重数.
    \end{enumerate}
\end{theorem}
实际上对于矩阵我们有对应的定理,此处不再赘述.我们有一个推论,若有$n$个互不相同的特征值则一定能对角化,
这是2的直接推论,但反之不成立,有多重特征值的矩阵也可能可以对角化,只要满足上述条件.

实际上由特征值的性质,我们容易知道数域$\mathbf{F}$上矩阵$A$可对角化,则$A^*$可对角化,对于数域$\mathbf{F}$上
任意多项式$f(x)$,$f(A)$也可对角化,且$A$可逆时,$A^{-1}$也可对角化.
\begin{example}
    证明$r$阶上三角矩阵$(r>1)$
    \[J_0=\begin{pmatrix}
        \lambda_0 & 1 &  &  \\
          & \lambda_0 & \ddots &  \\
          &  & \ddots &  1 \\
          &  &  &  \lambda_0
    \end{pmatrix}\]
    不与对角阵相似.
\end{example}
\begin{example}
    设$A=(a_{ij})_{n\times n}$是上三角矩阵.
    \begin{enumerate}
        \item 求$A$的全部特征值;

        \item 若$A$主对角元互不相等,证明:$A$与对角阵相似;

        \item 若$n$个主对角元相等且$A$不为对角矩阵,证明:$A$不与对角阵相似.
    \end{enumerate}
\end{example}
\begin{example}
    设$\alpha$和$\beta$是$\mathbf{R}^n\enspace (n>1)$中两个列向量,$A=\alpha\beta^\mathrm{T}\neq O$.
    \begin{enumerate}
        \item 求$A$的特征值;

        \item 证明:$\alpha^\mathrm{T}\beta=0\iff A$不可对角化.
    \end{enumerate}
\end{example}
最后需要说明一点,如果一个矩阵可对角化,那么我们可以将其表示为$A=P\Lambda P^{-1}$,其中
$P$可逆(即所谓特征值分解).实际上相抵、相合都有类似的表示思想,在解决一些题目时是重要的.
\begin{example}
    设$n$阶实对称矩阵$A$的特征值$\lambda_i\geqslant 0\enspace(i=1,\ldots,n)$.证明:存在特征值都是非负数的实对称矩阵
    $B$使得$A=B^2$(本题可推广为多次幂).
\end{example}
\begin{example}
    设三阶矩阵$A$的特征值为$\lambda_1=-2,\lambda_2=1,\lambda_3=2$,对应的特征向量分别为
    $\alpha_1=(1,1,0)^\mathrm{T},\alpha_2=(1,0,1)^\mathrm{T},\alpha_3=(1,1,1)^\mathrm{T}$,求矩阵$A$.
\end{example}
\subsection{对角化问题的一般解法}
下面我们总结一下求解对角化问题的基本方法:对于一个$n$阶可对角化矩阵$A$,求变换矩阵$P$使得$P^{-1}AP=\Lambda$,步骤如下:
\begin{enumerate}
    \item 求出$A$的所有不同特征值;

    \item 求出$A$在不同特征值下的特征子空间的基;

    \item 将这组基按列排列成矩阵$P$.
\end{enumerate}

这一过程的合理性在本小节开头就有叙述.下面我们来看一个基本的例子:
\begin{example}
    设$A=\begin{pmatrix}
        2 & 2 & 0 \\ 8 & 2 & a \\ 0 & 0 & 6
    \end{pmatrix}$相似于对角矩阵,求常数$a$,并求可逆矩阵$P$使得$P^{-1}AP$为对角矩阵.
\end{example}
除此之外,我们还可以利用对角化求解矩阵的幂的问题,在专题三中已经介绍,此处不再赘述.

\subsection{实对称矩阵对角化}
这一部分内容因为涉及正交的概念所以有班级未提及,因此我们只能回顾不涉及正交的部分.
教材定义7.7给出了共轭矩阵的概念,下方给出了大量的性质,此处不再赘述.我们的重点在于以下两个定理:
\begin{theorem}
    实对称矩阵的特征值都是实数.
\end{theorem}
这一定理的证明应当掌握,特别是如何证明实数的方法(即共轭等于自身).
\begin{theorem}
    实对称矩阵一定可以相似对角化.
\end{theorem}
这一定理证明只需要讲教材定理7.13除去正交即可.以上两个定理十分重要,是我们接下来讨论以及解决一些问题的基础.
\begin{example}
    已知$A$是实反对称矩阵,证明:
    \begin{enumerate}
        \item $A$的特征值必为0或纯虚数;

        \item $E-A^2$是可逆矩阵.
    \end{enumerate}
\end{example}
当然要注意的一点是,因为无法涉及正交的内容,本节习题中所有的对角化问题都无需进行 Schmidt 正交化,只需要像
上一小节介绍的方法那样求出一般的可逆矩阵即可.
\subsection{幂等矩阵}
若$n$阶方阵$A$满足$A^2=A$,则$A$称为幂等矩阵.幂等矩阵具有如下基本性质,请自行证明:
\begin{enumerate}
    \item $A$是幂等矩阵等价于$r(A)+r(A-E)=n$;

    \item $A$为幂等矩阵则一定可对角化,特征值为0和1,其中1的重数等于$r(A)$;

    \item $A$是幂等矩阵时,$r(A)=\tr(A)$;

    \item 所有秩为1迹也为1的矩阵均为幂等矩阵.
\end{enumerate}
实际上,幂等矩阵还有很多其他的性质,我们可以回到映射的角度去理解这一矩阵,
例如其与投影变换的等价性(与像空间、核空间有关,可以自行证明).
\begin{example}
    设$A$,$B$为两个$n$阶幂等矩阵,证明:
    \begin{enumerate}
        \item $A+B$为幂等矩阵当且仅当$AB=BA=O$;

        \item $A-B$为幂等矩阵当且仅当$AB=BA=B$;

        \item 若$AB=BA$,则$AB$为幂等矩阵. 反之,若$AB$为幂等矩阵,是否必有$AB=BA$;

        \item 若$E-A-B$可逆,则$r(A)=r(B)$.
    \end{enumerate}
\end{example}

\section{上三角矩阵}

虽然对角矩阵十分简洁,但很可惜很多算子都不存在如此简洁的矩阵表示.我们考虑更为普遍但也能
保持良好性质的情况,上三角矩阵一定是一个好的突破口:
\begin{theorem}\label{thm:16:上三角矩阵等价条件}
    设$V$是有限维复向量空间,$T\in \mathcal{L}(V)$,则
    \begin{enumerate}
        \item $T$关于$V$的某组基有上三角矩阵,记为$A$;

        \item $T$可逆的充要条件是$A$的主对角元均不为0;

        \item $T$的本征值恰为$A$的主对角元.
    \end{enumerate}
\end{theorem}
除此之外,在第三章中我们也提到上三角矩阵相乘结果中对角线上元素是原矩阵对角线上
对应元素相乘的结果,其逆的对角线上元素是原矩阵对角线对应元素的逆.以上性质表明,
上三角矩阵是所有算子都可以在某组基下得到的且有良好性质的矩阵类型.

在Done Right的体系中,这一定理的第一条需要基于以下命题:
\begin{theorem}
    设$T\in \mathcal{L}(V)$,且$v_1,v_2,\ldots,v_n$是$V$的基,则以下条件等价:
    \begin{enumerate}
        \item $T$关于$v_1,v_2,\ldots,v_n$的矩阵是上三角的;

        \item 对每个$j=1,\ldots,n$有$Tv_j\in\spa(v_1,\ldots,v_j)$;

        \item 对每个$j=1,\ldots,n$有$\spa(v_1,\ldots,v_j)$在$T$下不变.
    \end{enumerate}
\end{theorem}
这一定理给出了上三角矩阵的几个充要条件,教材中关于\autoref{thm:16:上三角矩阵等价条件}(1)的
两种证明在选定研究对象空间后核心都是利用等价条件进一步证明,(2)中也用到相关的结论.
当然\autoref{thm:16:上三角矩阵等价条件}(1)的证明也可以基于上一学期的方法,各位同学可以在习题中
尝试完成证明,基本思想与证明实对称矩阵可正交对角化的分块矩阵方法类似.
\begin{example}
    设$V$是$n$维复向量空间. $T\in \mathcal{L}(V)$,证明:对任意的正整数$r\enspace(1\leqslant r\leqslant n)$,$T$有$r$维不变子空间.
\end{example}
基于这一例子的结论,结合商空间,我们也可以推导出后续要讲解的哈密顿-凯莱定理,感兴趣的读者可以了解.

接下来我们要讨论一个第六章习题中涉及但未归纳的问题,即算子/矩阵可交换的性质.我们有如下定理:
\begin{theorem}
    设$V$为$n$维复向量空间,$S,T\in \mathcal{L}(V)$,$ST=TS$,则
    \begin{enumerate}
        \item $S$的每个本征空间都是$T$的不变子空间;

        \item $S,T$有公共的本征向量.
    \end{enumerate}
\end{theorem}
将这一定理的算子改为矩阵实际上是等价的.这一定理的证明实际上第六章的例题或习题中都有,可以参考.
接下来我们希望应用这两个定理解决下面的问题:
\begin{example}
    设$V$为$n$维复向量空间,$S,T\in \mathcal{L}(V)$,$ST=TS$,则
    \begin{enumerate}
        \item 若$S$有$s$个不同的本征值,则$S,T$至少有$s$个公共且线性无关的本征向量;

        \item 存在$V$的一组基,使得$S$和$T$在这组基下的矩阵均为上三角矩阵.
    \end{enumerate}
\end{example}
这一习题的结论告诉我们:算子可交换对应于同时上三角化.例中 2 的结论如果换为矩阵表述应当是:
设$A,B$是复数域上的两个$n$阶矩阵,且$AB=BA$,则存在可逆矩阵$P$使得$P^{-1}AP$和$P^{-1}BP$
同时为上三角矩阵.

\section{分块对角矩阵}
\subsection{零空间的性质 \quad 幂零矩阵}
% TODO 零空间 / 核空间
% TODO 引用
在\ref{本征空间与对角矩阵}节的末尾,我们提到了对于不可对角化的算子得到相对简单的标准形
的思路:因为此时本征向量不足以构成一组基,原空间无法被分解为一维不变子空间的直和,也就无法被
分解为本征空间的直和.但注意到本征空间实际上是$\ker (\lambda I-T)$,而我们在
\ref{像与核高级结论}节中提到了关于零空间可以随着算子幂次增加而增加的结论,因此我们可以考虑
利用这一结论将本征空间扩张,从而使得扩张后的本征空间的直和为原空间.我们将关于零空间的这些结论
总结为如下定理:
\begin{theorem}\label{thm:16:零空间性质}
    设$T\in \mathcal{L}(V)$,则有
    \begin{enumerate}
        \item $\{0\}=\ker T^0\subset\ker T^1\subset\cdots\subset
        \ker T^k\subset\ker T^{k+1}\subset\cdots$;

        \item 设$m$是非负整数使得$\ker T^m=\ker T^{m+1}$,则
        \[\ker T^m=\ker T^{m+1}=\ker T^{m+2}=\ker T^{m+3}=\cdots\]

        \item 令$n=\dim V$,则$\ker T^n=\ker T^{n+1}=\ker T^{n+1}=\cdots$.
    \end{enumerate}
\end{theorem}
这一定理对应教材8.2-8.4,证明环环相扣,并且8.3的证明有一定的趣味性.接下来我们继续对\ref{像与核高级结论}节中
另一结论进行解读.我们有$T^2=T$时,$V=\ker T\oplus\im T$,虽然这一幂等的条件不是必要的,
但仍然存在很多矩阵无法满足这一等式,但我们有存在正整数$m$使得$V=\ker T^m\oplus\im T^m,\enspace\forall T\in \mathcal{L}(V)$.
实际上,我们可以取$m=\dim V$,这就是教材8.5的结论.证明方式请回顾\ref{直和}一节的两种方法,显然第二种更为简单直接.
这一结论实际上也给予我们理由相信,扩张后的本征空间的直和也可以被证明等于原空间.

基于上面零空间的讨论,为了后面小节的研究,我们将讲解幂零算子与幂零矩阵的相关准备知识.
\begin{definition}
    一个算子称为\keyterm{幂零}[nilpotent]的,如果它的某个幂等于 0.
\end{definition}
% TODO 引用
幂零矩阵的定义类似,参考\ref{矩阵的迹}一节.实际上,在定理\ref{幂零矩阵性质}中我们已经介绍了
部分内容,但现在的谈论角度与背景有差异,因此我们重述这一定理:
\begin{theorem} \label{thm:16:幂零算子性质}
    设$N\in \mathcal{L}(V)$是幂零的,则
    \begin{enumerate}
        \item $N$的所有本征值均为0(等价定义);

        \item $N^{\dim V}$=0;

        \item $V$有一组基使得$N$关于这组基的矩阵对角线和对角线下方元素均为0(等价定义);

        \item $N\pm I$可逆.
    \end{enumerate}
\end{theorem}
其中1的等价性其中一边需要使用后文的知识,因此我们放在后续章节的习题中.
2和3见教材8.18和8.19,而4可直接由3或者利用本征值$\lambda$与$T-\lambda I$可逆之间的关系得到,
3的等价性另一半见教材8.A习题12.

\subsection{广义本征空间与分块对角矩阵}
上一节中我们已经讨论了不可对角化算子获得简化矩阵的一般思想,即试图利用零空间增长的性质扩张
本征空间,使得扩张后的本征空间(称为广义本征空间)的直和为原空间.下面我们给出严谨定义:
\begin{definition}
    设$T\in \mathcal{L}(V)$,$\lambda\in\mathbf{F}$是$T$的本征值,若向量$v\neq 0$且存在正整数$j$使得
    $(T-\lambda I)^jv=0$,则称$v$为$T$对应于$\lambda$的\keyterm{广义本征向量}[generalized eigenvector].
    $T$对应于$\lambda$的全体广义本征向量与0向量构成的集合称为$T$相应于$\lambda$的\keyterm{广义本征空间}[generalized eigenspace],记为$G(\lambda,T)$.
\end{definition}
注意我们不定义广义本征值,因为若$\lambda$原先不是本征值,则对于任意的$j$,$(T-\lambda I)^j$
仍为双射.

实际上,根据\autoref{thm:16:零空间性质},我们有$G(\lambda,T)=\ker (T-\lambda I)^{\dim V}$.
需要补充说明的是,此处引入两个概念称为代数重数(或称重数)和几何重数,其中$\lambda$的代数重数定义为
广义本征空间的维数,几何重数定义为本征空间的维数.实际上第六章我们有类似的定义,我们将在多项式一节中
讲解它们的关联.

我们接下来的目标转向证明广义本征空间的和为直和且和为原空间,这样就能和定理\ref{本征常识}与
定理\ref{算子对角化}思路对应.
\begin{theorem} \label{thm:16:广义本征性质}
    设$V$是有限维的,$T\in \mathcal{L}(V)$.用$\lambda_1,\ldots,\lambda_m$表示$T$的所有互异本征值.
    \begin{enumerate}
        \item $T$对应于不同本征值的广义本征向量线性无关;

        \item $T$不同本征值对应的广义本征空间的和为直和,且$V=G(\lambda_1,T)\oplus\cdots\oplus
        G(\lambda_m,T)$;

        \item $V$有一个由$T$的广义本征向量组成的基;

        \item 每个$G(\lambda_i,T)$在$T$下都是不变的;

        \item 每个$(T-\lambda_j I)\vert_{G(\lambda_j,T)}$都是幂零的.
    \end{enumerate}
\end{theorem}
注意这一定理适用于复向量空间上的任一算子.上述定理分别对应于教材8.13,8.21和8.23.其中(1)的证明具有一定的
技巧性,上一学期也有类似的思想,但此定理更多的处理.(2)是本命题的核心,证明使用数学归纳法,略显繁杂.
(3)-(5)证明比较基本,并且让我们可以得到如下矩阵标准形:
\begin{theorem}
    设$V$是复向量空间,$T\in \mathcal{L}(V)$.设$\lambda_1,\cdots,\lambda_m$是$T$的所有互不相同的本征值,重数分别为
    $d_1,\cdots,d_m$,则$V$有一组基使得$T$关于这组基的有分块对角矩阵
    \[\begin{pmatrix}
        A_1 &  & O \\  & \ddots &  \\ O &  & A_m
    \end{pmatrix}\]
    其中每个$A_j$都是如下所示的$d_j\times d_j$上三角矩阵
    \[A_j=\begin{pmatrix}
        \lambda_j &  & * \\  & \ddots &  \\ O &  & \lambda_j
    \end{pmatrix}\]
\end{theorem}
这一定理的证明基于\autoref{thm:16:广义本征性质} 以及幂零算子性质\autoref{thm:16:幂零算子性质} 是简单的.
由此我们得到了一个相比于上三角矩阵更为简单,并且所有算子都可以获得的标准形.
\begin{example}
    设$T\in \mathcal{L}(\mathbf{C}^3)$定义为
    \[T(z_1,z_2,z_3)=(6z_1+3z_2+4z_3,6z_2+2z_3,7z_3),\]求一组基使其有分块对角矩阵并写出对应的分块对角矩阵.
\end{example}
\begin{example}
    设$T,S\in \mathcal{L}(V)$可逆,证明:$T$和$S^{-1}TS$有相同的本征值,且重数也相同.
\end{example}
在进入下一个话题前,我们先简单介绍算子平方根的概念,这一概念在之后内积空间算子会进一步说明.
\begin{definition}
    我们称算子$T\in \mathcal{L}(V)$的平方根是满足$R^2=T$的算子$R\in \mathcal{L}(V)$.
\end{definition}
关于复向量空间,我们有如下两个结论:
\begin{theorem} \label{thm:16:幂零平方根}
    设$V$是复向量空间.
    \begin{enumerate}
        \item 设$N\in \mathcal{L}(V)$幂零,则$(I+N)$有平方根;

        \item 若$T\in \mathcal{L}(V)$可逆,则$T$有平方根.
    \end{enumerate}
\end{theorem}
定理对应教材8.31和8.33. 1的证明基于$\sqrt{1+x}$的泰勒展开,并应用幂零矩阵的定义,表明我们可以计算出这一平方根.
我们不是第一次看到使用泰勒展开的情况,在\ref{矩阵的幂}一节的求逆的分式思想中使用了$\cfrac{1}{1-x}$的泰勒展开.
2 的证明实际上基于 1以及\autoref{thm:16:广义本征性质},是很简单明了的.
\begin{example}
    定义$N\in \mathcal{L}(\mathbf{F}^5)$为
    \[N(x_1,x_2,x_3,x_4,x_5)=(2x_2,3x_3,-x_4,4x_5,0)\]
    求$(I+N)$的一个平方根.
\end{example}

\vspace{2ex}
\centerline{\heiti \Large 内容总结}

\vspace{2ex}

\centerline{\heiti \Large 习题}
\vspace{2ex}
{\kaishu }
\begin{flushright}
    \kaishu

\end{flushright}
\centerline{\heiti A组}
\begin{enumerate}
    \item
\end{enumerate}
\centerline{\heiti B组}
\begin{enumerate}
    \item
\end{enumerate}
\centerline{\heiti C组}
\begin{enumerate}
    \item
\end{enumerate}

\phantomsection
\section*{19 多项式的进一步讨论}
\addcontentsline{toc}{section}{19 多项式的进一步讨论}

\vspace{2ex}

\centerline{\heiti A组}
\begin{enumerate}
    \item
\end{enumerate}

\centerline{\heiti B组}
\begin{enumerate}
    \item
\end{enumerate}

\centerline{\heiti C组}
\begin{enumerate}
    \item
\end{enumerate}

\clearpage

\chapter{若当标准形}
在之前的叙述中我们说明了复向量空间中算子都有分块对角矩阵,且每个分块都具有上三角矩阵的形式.本节我们希望每个分块能获得
比上三角矩阵更多0的形式.实际上,这一形式我们之前已经引入,并且也讨论了其极小多项式,本节中我们讨论如何得到这种形式的
矩阵,了解算子和矩阵若当标准形的求法以及若当标准形对应基的求法,并讲解其应用.

\section{若当标准形的存在与形式}
在\autoref{thm:16:广义本征性质} (5) 中我们提到过$T-\lambda_jI$限制在$\lambda_j$对应的广义本征空间上是幂零算子.
因此我们可以先研究幂零算子的若当标准形,然后加入简单的数量矩阵即可.

我们仍然沿着 Done Right 的思路,尽管本节的定理证明有一种看了几行就没兴趣再看下去的美感.我们需要注意的是,Done Right 以及其它
高等代数教材中关于若当的推导都十分繁杂,因此这并不是我们考虑的重点(当然建议读者阅读证明过程以加强理解).我们的核心在于掌握
这些结论并且能够计算出若当标准形,并运用若当标准形解决一些问题.在意推导的读者可以考虑学习抽象代数,运用模论推导若当标准形,
更加直接.
\begin{theorem} \label{thm:18:若当基存在}
    设$N\in \mathcal{L}(V)$是幂零的,则存在向量$v_1,\ldots,v_n$和非负整数$m_1,\ldots,m_n$使得
    \begin{enumerate}
        \item $N^{m_1}v_1,\ldots,Nv_1,v_1,\ldots,N^{m_n}v_n,\ldots,Nv_n,v_n$是$V$的基;

        \item $N^{m_1+1}v_1=\cdots=N^{m_n+1}v_n=0$.
    \end{enumerate}
\end{theorem}
1 中的向量组可以分为$n$个由$v_i\enspace(i=1,\ldots,n)$生成的$N$-强循环子空间,即$N^{m_i}v_i,\ldots,Nv_i,v_i$.
实际上,1 中的向量排列顺序以及2中的等于0的条件的目的都是使得$N$在这组基下的表示矩阵为分块对角矩阵,
每一块的大小为$(m_i+1)\times(m_i+1)$,形如\[\begin{pmatrix}
    0 & 1 &  & 0 \\  & \ddots & \ddots &  \\  &  &  \ddots & 1 \\ 0 &  &  & 0
\end{pmatrix}\]即为若当块$J_{m_i+1}(0)$.

我们更进一步,将数量矩阵代回.由于上述若当块是$(T-\lambda_iI)\vert_{G(\lambda_i,T)}$的若当标准形,是在\autoref{thm:18:若当基存在}
中的基下的矩阵,而我们知道,$\lambda_iI$在任意一组基下的矩阵都是对角线元素均为$\lambda_j$的对角矩阵,因此我们可以得到以下结论:
\begin{theorem}
    设$T\in \mathcal{L}(V)$,则$T$在\autoref{thm:18:若当基存在} 给出的基下的矩阵表示为分块对角矩阵,且每个对角块都是
    $(m_i+1)\times(m_i+1)\enspace(i=1,\ldots,n)$的矩阵,且具有形式\[\begin{pmatrix}
        \lambda_i & 1 &  & 0 \\  & \ddots & \ddots &  \\  &  &  \ddots & 1 \\ 0 &  &  & \lambda_i
    \end{pmatrix}\]而整体的分块对角矩阵即为若当形矩阵.
\end{theorem}

接下来我们需要描述矩阵的若当标准形.实际上,算子在某组基下有若当标准形与矩阵有相似标准形为若当标准形是等价的.我们考虑$P^{-1}AP=J$,
其中$P$为过渡矩阵,$J$为$A$的若当标准形.我们可以将$A$视为$T(\alpha)=A\alpha$在自然基下的矩阵,于是$T$在\autoref{thm:18:若当基存在}
给出的基(记为$B$)下的表示矩阵的求解方式就是$T(B)=(B)J$,将$B$组成的矩阵记作$P$,由$T$的定义可知,$T(B)=(B)J$等同于$AP=PJ$,即
$P^{-1}AP=J$,故如果我们要求矩阵相似于其若当标准形的过渡矩阵,问题转化为求解若当基然后排列成矩阵即可.于是我们下面将要介绍如何将基和
若当标准形具体地求出来.

\section{若当标准形的求解}
在上一小节的最后,我们将求解若当标准形的目标转化为了求解若当基,我们希望有更加算法化的方式去实现这一目标.为此,我们先引入一些记号.
我们记$G_j(\lambda,T)=\ker (T-\lambda I)^j$,根据零空间增长以及极小多项式的结论,我们知道当$i<j$时有
$G_i(\lambda,T)\subseteq G_j(\lambda,T)$,并且当$T-\lambda I$的次数为极小多项式中该本征值对应因式的次数后,零空间会停止增长.

我们可以优先考虑幂零算子$N\in \mathcal{L}(V)$,最后再考虑非幂零的情况应当给我们的算法加什么样的步骤.我们的目标是求出一组基形如
$N^{m_1}v_1,\ldots,Nv_1,v_1,\ldots,N^{m_n}v_n,\ldots,Nv_n,v_n$,这里有两组未知量,我们依次来说明如何求解.在求解之前,我们需要引入一个
定理表明下述方法的合理性:
\begin{theorem}
    设$T\in \mathcal{L}(V)$,若$V$中向量$v\in G_j(\lambda,T)\backslash G_{j-1}(\lambda,T)$,则
    \begin{enumerate}
        \item 对任意的$i<j$,有$(T-\lambda I)^iv\in G_{j-i}(\lambda,T)\backslash G_{j-i-1}(\lambda,T)$;

        \item $v,(T-\lambda I)v,\ldots,(T-\lambda I)^{j-1}v$线性无关.
    \end{enumerate}
\end{theorem}
定理的证明是基本的,我们留作习题.这一定理对于下述算法中若当基的阶梯形排列的合理性是必要的.
\begin{enumerate}
    \item \textbf{\heiti 求解 $m_1,\ldots,m_n$}

    我们首先确定幂次参数,这一参数确定后若当形矩阵就确定了,因为若当形矩阵中每个若当块的大小是$(m_i+1)\times(m_i+1)(i=1,\cdots,n)$.
    \begin{enumerate}
        \item 我们假设$m_1\geqslant\cdots\geqslant m_n$,并在接下来的步骤中尝试将若当基重排成如图格式:

        \begin{figure}[h]
            \centering
            \begin{tikzpicture}
                \foreach[count=\i] \len in {2, 3, 3, 5, 7, 7}
                    \draw (0, -1.2*\i+1.2) -- (1.2*\len, -1.2*\i+1.2);

                \foreach \i in {0,...,2}
                    \draw (1.2*\i, 0) -- (1.2*\i, -2.4);
                \draw (3.6, -1.2) -- (3.6, -2.4);

                \foreach \i in {0,...,5}
                    \draw (1.2*\i, -3.6) -- (1.2*\i, -6.0);
                \foreach \i in {6,...,7}
                    \draw (1.2*\i, -4.8) -- (1.2*\i, -6.0);

                \foreach \i in {1,...,2} {
                    \node at (1.2*\i-0.6, -0.6) {$v_{\i}$};
                    \node at (1.2*\i-0.6, -5.4) {$N^{m_{\i}}v_{\i}$};
                }
                \foreach \i in {1,...,3}
                    \node at (1.2*\i-0.6, -1.8) { $Nv_{\i}$};
                \node at (7.8, -5.4) { $N^{m_n}v_n$};
                \node at (1.8, -3.0) {$\cdots\cdots$};

                \node[anchor=west] at (2.4, -0.6) {$G_{m_1+1}(0,T) \backslash G_{m_1}(0,T)$};
                \node[anchor=west] at (3.6, -1.8) {$G_{m_1}(0,T) \backslash G_{m_1-1}(0,T)$};
                \node[anchor=west] at (6.0, -4.2) {$G_2(0,T) \backslash G_1(0,T)$};
                \node[anchor=west] at (8.4, -5.4) {$G_1(0,T)$};
            \end{tikzpicture}
        \end{figure}

        即第一排将若当基中在$G_1(0,T)$的向量排列,即在$N$作用一次后就等于0的向量;第二排将所有若当基中在
        $G_2(0,T)\backslash G_1(0,T)$的向量排列,即在$N$作用一次后不等于0但作用两次等于0的向量,以此类推.
        由于假设$m_1\geqslant\cdots\geqslant m_n$,这个图呈阶梯形.
        \item 接下来要将上图填满,首先要确定求解$G_j(0,T)$及其维数(因为幂零算子本征值为0)直到
        $j$等于$N$极小多项式的次数(即幂零指数,或当$G_j(0,T)=V$时),因为此后零空间不可能继续增加,阶梯形
        也就不会再延伸.在求出维数后阶梯形状也即确定,因为各层向量个数确定了.例如假设11维空间中的映射满足
        $G_1(0,T)$,$G_2(0,T)$,$G_3(0,T)$的维数分别为5,9,11.这说明从底至上向量个数依次为5,4($=9-5$),2($=11-9$).
        \item 基于上面的求解,这时我们就可以确定若当块的阶数$m_i+1\enspace(i=1,\ldots,n)$,因为这一阶梯中第$i$列的高度实际上
        就是$m_i+1$(因为每一列是$v_i,Nv_i,\ldots,N^{m_i}v_i$).将这些若当块拼起来就得到了幂零算子的若当标准形.
    \end{enumerate}
    \item \textbf{\heiti 求解 $v_1,\ldots,v_n$}

    本节内容按照以往的情况在考试中不要求,但为了保证讲义的完整性,我们将求若当基的方法也进行描述.
    \begin{enumerate}[label=(\arabic*)]
        \item 我们在前述内容中求出了各个$G_j(0,T)$,接下来我们需要利用这些向量将之前确定形状的阶梯内容填满.
        我们首先将阶梯最上方的向量确定,实际上就是利用求出的$G_{m_1+1}(0,T)$(也就是$V$)和$G_{m_1}(0,T)$求出
        二者之差.似乎很简单,但若仔细思索便会发现线性空间的差并不一定好求.举一个简单的例子,设
        $G_{m_1+1}(0,T)=\{\alpha_1,\alpha_2,\alpha_3,\alpha_4\}$,$G_{m_1}(0,T)=\{\beta_1,\beta_2\}$.
        这其中出现的所有向量可能都完全不一样,所以作差并不容易.但我们有一种好方法,如果我们每次从$G_{m_1+1}(0,T)$
        中挑选两个向量和$G_{m_1}(0,T)$中的两个向量放在一起,如果这四个向量线性无关,这就说明这两个挑出的向量就是作差的结果.
        原因在于这相当于$G_{m_1}(0,T)$直和这两个向量长成的空间后得到了$G_{m_1+1}(0,T)$.如果四个向量线性相关,这说明挑选
        的向量有在$G_{m_1}(0,T)$中的.
        \item 接下来继续计算第二行中的向量.实际上算出第一行后第二行中部分向量就已经确定了,例如图上的$Nv_1$和$Nv_2$.我们这时
        用类似的方法求解$G_{m_1}(0,T)$和$G_{m_1-1}(0,T)$的差,如图只需要确定一个向量,但这一个向量的确定除了要像(a)中一样
        每次从$G_{m_1}(0,T)$中选择一个与$G_{m_1-1}(0,T)$的基一起判断线性相关性外,还需要确保这个向量和已经求出的$Nv_1$和$Nv_2$
        是线性无关的,因为它们构成$G_{m_1}(0,T)$的一组基.

        总结一下,处于$G_j(0,T)\backslash G_{j-1}(0,T)$对应的行的需要补充的向量$v$应当满足如下三个条件:
        \begin{enumerate}[label=(\roman*)]
            \item $v\in G_j(0,T)$;

            \item $v\notin G_{j-1}(0,T)$(通过加入$G_{j-1}(0,T)$)的基保证线性无关判断);

            \item $v$与同一行中左边已求出的向量线性无关.
        \end{enumerate}
        \item 最后,我们将所有求出的基按照若当基原先的排列顺序重新组合即可.如果求矩阵相似于若当标准形的过渡矩阵,则按顺序
        按列摆放即可.
    \end{enumerate}
\end{enumerate}
需要注意的是,我们求解$G_j(0,T)$时实际上都是要用到矩阵形式进行高斯消元的,所以虽然说是基于算子,但过程中基本都是矩阵运算.
\begin{example}
    求矩阵\[\begin{pmatrix}
        2 & 3 & 0 & -1 & 2 & -2 \\ -1 & 0 & 2 & 1 & -1 & -2 \\
        1 & 3 & 2 & 0  & 1 & -4 \\ 5 & 6 & -1 & -2 & 5 & -3 \\
        3 & 3 & -1 & -2 & 3 & -1 \\ 1 & 3 & 2 & 0 & 1 & -4
    \end{pmatrix}\]的若当标准形以及相应的过渡矩阵(提示:这一矩阵是幂零指数为3的幂零矩阵).
\end{example}
对于一般的非幂零算子,我们需要首先利用第六章中求解特征多项式$f(\lambda)=|\lambda I-A|$的零点的方法求出所有本征值,
然后求出各个不变子空间(这一过程实际上也把将来要求的$G_j(0,T)$进行了求解),然后求各个不变子空间上幂零算子
$(T-\lambda_j)\vert_{G(\lambda_j,T)}$的若当标准形,即我们在$G(\lambda_j,T)$上执行上面所说的算法,注意最后的若当块对角线上为对应的本征值.
对于矩阵$A$,我们将其视为$T(\alpha)=A\alpha$在自然基下的矩阵,然后利用算子的方式即可.
同样地,虽然我们是对算子进行描述,但是我们求零空间仍然需要基于矩阵,所以我们需要在求出不变子空间后得到分块对角矩阵,
然后对各个块进行$A_i-\lambda_jI$的操作化为幂零矩阵然后进行计算.
\begin{example}
    设$A=\begin{pmatrix}
        2 & 1 & 1 \\ -2 & -1 & -2 \\ 1 & 1 & 2
    \end{pmatrix}$,求$A$的若当标准形$J$和矩阵$P$,使得$P^{-1}AP=J$.
\end{example}

\section{若当标准形的另一求法*}\label{sect:18:若当标准形的另一求法}
本节我们利用另一种方法得到若当标准形的另一种求解方式.这一方式从算法上说更为简便,结论也更为直白,但定理的证明思路与 Done Right 有较大差别.
如果感兴趣可以参考\textbf{丘维声《高等代数》},在这里我们不详细展开,只阐述结论,供有兴趣的读者了解.如果考试中要求若当标准形,
如果老师采用教材的思路讲解,建议使用之前的方法.如果老师提到了这一角度,那么应当也是可以使用的.
\begin{theorem}
    设$V$是$n$维复向量空间,$T\in \mathcal{L}(V)$,$\lambda_1,\ldots,\lambda_m$为其互异的本征值,则主对角元为$\lambda_j$的若当块的个数$N_j$为
    \[N(j)=n-r(T-\lambda_jI)\]
    其中$t$级若当块$J_t(\lambda_jI)$的个数$N_j(t)$为
    \[N_j(t)=r(T-\lambda_jI)^{t+1}+r(T-\lambda_jI)^{t-1}-2r(T-\lambda_jI)^t\]
    其中$t$应当小于等于极小多项式中因式$\lambda-\lambda_j$的幂次,$j=1,\ldots,m$.这个若当形矩阵$A$称为$T$的若当标准形,
    除去若当块的排列次序外,其若当标准形唯一.
\end{theorem}
实际上对于$n$阶矩阵有类似的定理,我们不再赘述.当我们求解矩阵的若当标准形时,
如果我们要求解过渡矩阵,也有简单的方法.要求$P$使得$P^{-1}AP=J$,则有$AP=PJ$.假定$P$为$n$阶矩阵,我们可以设$P=(X_1,\ldots,X_n)$,
剩下的任务就是解方程了.因此这种方法非常简单,缺陷在于绕开了若当基这一本质的问题.
\begin{example}
    利用上述方法求解矩阵\[\begin{pmatrix}
        2 & 3 & 2 \\ 1 & 8 & 2 \\ -2 & -14 & -3
    \end{pmatrix}\]的若当标准形以及对应的过渡矩阵.
\end{example}
最后我们需要提到一点,根据上面的叙述,若当标准形在不考虑若当块的排列顺序的情况下是唯一的.因此任一复数域上矩阵均有唯一的若当标准形(相似标准形),
因此我们可以知道,两矩阵相似的一个充要条件是两矩阵有相同的若当标准形(不考虑若当块的排列顺序).这一点仅根据 Done Right 的存在性证明是很难得到的.

\section{若当标准形的应用}
在最后一节中我们主要讨论若当标准形的应用.这里的应用主要针对于习题方面的应用.在实际中,如果是计算机专业的同学,可能
若当标准形的实用价值不大,它常用在求解一阶微分方程组用于电路理论或者计算数学的部分话题.

在不变子空间一节中我们提到,我们可以利用若当标准形求解不变子空间,如下面的例子:
\begin{example}
    设$V$为$n$维复向量空间,$T\in \mathcal{L}(V)$,$T$在基$\varepsilon_1,\ldots,\varepsilon_n$下的矩阵是一个若当块,证明:
    \begin{enumerate}
        \item $V$中包含$\varepsilon_1$的不变子空间只有$V$自身;

        \item $V$中任意非零不变子空间都包含$\varepsilon_n$;

        \item $V$不能分解为两个非平凡的不变子空间的直和;

        \item $V$中有且仅有$n+1$个不变子空间,它们分别是
        \[\{0\},\spa(\varepsilon_n),\spa(\varepsilon_{n-1},\varepsilon_n),\ldots,\spa(\varepsilon_1,\ldots,\varepsilon_{n-1},\varepsilon_n)\]
    \end{enumerate}
\end{example}
因此我们如果能将算子在一组基下表示为若当块,我们就可以很快地写出其不变子空间.

若当标准形的另一个应用在于我们可以利用它计算矩阵的幂,因为若当块的幂的计算是简单的:
\[J_k(a)^n=(aE+J_k(0))^n=a^nE+C_n^1a^{n-1}J_k(0)+\cdots+C_n^nJ_k(0)^n\]
同时我们也知道$J_k(0)^k=0$(幂零矩阵),所以利用若当标准形求解矩阵的幂是简单的.
\begin{example}
    设$T\in \mathcal{L}(V)$,$v_1,\ldots,v_n$为$T$的若当基,描述$T^2$在这组基下的矩阵.
\end{example}
接下来我们讨论若当标准形应用于矩阵分解的情形.我们首先讨论平方根分解,这一点在之前有提及,但此处我们希望从矩阵的角度讨论这一问题.
\begin{theorem}
    在复数域上,设$a\neq 0$,则$J_n(a)$有平方根.
\end{theorem}
我们可以考虑$J_n(\sqrt{a})^2$与$J_n(a)$的关系来证明这一命题.注意,这里的$a\neq 0$是必须的,因为我们不难证明如下定理:
\begin{theorem}
    当$n\geqslant 2$时,$J_n(0)$不存在平方根.
\end{theorem}
直接使用反证法即可.接下来我们考虑之前已经证明的\autoref{thm:16:幂零平方根} (2),即可逆算子一定有平方根,我们现在可以使用若当标准形的方式
证明,因为可逆算子特征值均不为0,因此每个若当块对角线上都不为0,均有平方根,故而得证.
\begin{example}
    定义$T\in \mathcal{L}(\mathbf{C}^3)$为$T(z_1,z_2,z_3)=(z_2,z_3,0)$.证明不存在
    $S\in \mathcal{L}(\mathbf{C}^3)$使得$S^2=T$.
\end{example}
本题为2020年期末考试最后一题,分值25.实际上只要想到幂零这一要点是很容易的,但若没有思考到位
则很容易走偏而失分.教材8.17给出了两种常见的幂零算子,一种是本题这一类型,另一个是微分算子,
应当熟练掌握.

除此之外,利用若当标准形,我们还可以有以下分解:
\begin{example}
    已知$A$是复数域上的$n$阶方阵,证明:
    \begin{enumerate}
        \item 存在可对角化的矩阵$B$和幂零矩阵$C$,使得$A=B+C$,且$BC=CB$;

        \item 存在复数域上的对称矩阵$B,C$,使得$A=BC$,并且可以指定$B,C$中任何一个为可逆矩阵.
    \end{enumerate}
\end{example}
这一命题 1 证明比较基本,2 只需了解即可,证明较为繁杂.

接下来我们讨论若当标准形与极小多项式之间的关系.实际上,我们早在\autoref{thm:17:若当形矩阵极小多项式} 中求解了
若当形矩阵的极小多项式.我们来看一个例子进行应用:
\begin{example}
    设$A$为$n$阶方阵且极小多项式次数为$n$,则$A$的若当标准形中各个若当块的主对角线元素互不相同.
\end{example}
从此例中我们可以产生一个直觉,即极小多项式的次数等于特征多项式时每个特征值对应若当块只有一个,若极小多项式某个
特征值对应的因式次数下降,那么将会使得若当块分为多个(出现多个对角线上元素相同的若当块),当次数下降到1时则均为
一阶若当块(一阶矩阵).这样的叙述十分抽象,因为 Done Right 并没有选择讲解如何从这一角度求解若当标准形,感兴趣的同学
可以参考关于$\lambda$-矩阵的行列式因子、不变因子、初等因子三因子理论,从这一理论出发可以理解之前\autoref{sect:18:若当标准形的另一求法}
小节的结论.

当然,我们仅基于直觉也可以得到一些常用的结论,如下面的例子:
\begin{example}
    回答以下两个问题:
    \begin{enumerate}
        \item 设$N\in \mathcal{L}(V)$幂零,证明:$N$的极小多项式是$z^{m+1}$,其中$m$是$N$的若当标准形中紧位于对角线
        上方的直线上连续出现的1的最大个数;

        \item 设$p,q\in\mathbf{C}[x]$是具有相同零点的首一多项式,$q$是$p$的多项式倍,证明:存在
        $T\in \mathcal{L}(\mathbf{C}^{\deg q})$使得$T$的特征多项式为$q$且极小多项式为$p$.
    \end{enumerate}
\end{example}

\vspace{2ex}
\centerline{\heiti \Large 内容总结}

\vspace{2ex}

\centerline{\heiti \Large 习题}
\vspace{2ex}
{\kaishu }
\begin{flushright}
    \kaishu

\end{flushright}
\centerline{\heiti A组}
\begin{enumerate}
    \item
\end{enumerate}
\centerline{\heiti B组}
\begin{enumerate}
    \item
\end{enumerate}
\centerline{\heiti C组}
\begin{enumerate}
    \item
\end{enumerate}

\chapter{内积空间}

\section{内积和范数}

\subsection{内积和范数的定义及性质}

前面研究的所有空间都是线性空间,只注重于线性结构,忽视了向量的度量性质,如向量的长度、夹角等. 但度量性质恰是在分析、几何问题中不可缺少的. 故从此章起,我们引入度量的概念,将线性空间推广为内积空间.

\term{内积}\index{neiji@内积 (inner product)}的引入始于我们曾在高中研究过的 $\mathbf{R}^{2}$ 与 $\mathbf{R}^{3}$ 上的向量点积,范数则是始于向量的长度概念. 内积即是点积性质的推广,本质上就是一个函数,它把 $ V $ 中元素的每个有序对 $(u, v)$ 都映射成一个数$ \langle u, v \rangle \in \mathbf{F}$,并且具有以下性质:

\begin{enumerate}
    \item 正定性:$\forall v \in V, \enspace \langle v, v \rangle \geqslant 0, \enspace \langle v, v \rangle = 0 \iff v = \vec{0}$;

    \item 第一个位置的加性:$\forall u, v, w \in V, \enspace \langle u + v, w \rangle = \langle u, w \rangle + \langle v, w \rangle$;

    \item 第一个位置的齐性:$\forall \lambda \in \mathbf{F}, \enspace \forall u, v \in V, \enspace \langle \lambda u, v \rangle = \lambda \langle u, v \rangle$;

    \item 共轭对称性:$\forall u, v \in V, \enspace \langle u, v \rangle = \overline{\langle v, u \rangle}$.
\end{enumerate}

每个实数都等于它的复共轭,所以在处理实向量空间时,共轭对称性实际上转变为对称性,即:$\forall u, v \in V, \enspace \langle u, v \rangle = \langle v, u \rangle$.

而由以上定义,我们可以快速得到以下性质:

\begin{enumerate}
    \item 对于每个取定的 $u \in V$,将 $ v $ 变为 $\langle v, u \rangle$ 的函数是 $ V $ 到 $\mathbf{F}$ 的线性映射.

    \item $\forall u \in V, \enspace \langle \vec{0}, u \rangle = \langle u, \vec{0} \rangle = 0$.

    \item $\forall u, v, w \in V, \enspace \langle u, v + w \rangle = \langle u, v \rangle + \langle u, w \rangle$.

    \item $\forall \lambda \in \mathbf{F}, \enspace \forall u, v \in V, \enspace \langle u, \lambda v \rangle = \overline{\lambda} \langle u, v \rangle$.
\end{enumerate}

其实从以上的定义与性质可以发现,实内积空间上的内积与我们之后要提到的双线性函数有着密不可分的联系——实线性空间上的正定对称双线性函数实际上就是该空间上的一个内积,在此先按下不表.

内积定义完成后,便可由该内积确定一个相应的范数:对于 $v \in V$,$v$ 的\term{范数}\index{fanshu@范数 (norm)}(记作 $ \lVert v \rVert $)定义为 $ \lVert v \rVert = \sqrt{\langle v, v \rangle}$. 并且具有以下性质:

\begin{enumerate}
    \item $\forall v \in V, \enspace \left\lVert v \right\rVert = 0 \iff v = \vec{0}$.

    \item $\forall v \in V, \enspace \forall \lambda \in \mathbf{F}, \enspace \left\lVert \lambda v \right\rVert  = \left\lvert \lambda \right\rvert \lVert v \rVert$.
\end{enumerate}

上述性质留给读者自证,从中我们也能发现一个普遍原理:处理范数的平方通常比直接处理范数更容易.

以下给出几个内积和范数的示例:

\begin{example} \label{ex:23:内积和范数}
    \begin{enumerate}
        \item $\mathbf{F}^{n}$ 上的欧几里得内积定义为:
              \[\left\langle (w_1, \ldots, w_n), (z_1, \ldots, z_n)\right\rangle = w_1\overline{z_1} + \cdots + w_n\overline{z_n} = \vec{w}\overline{\vec{z}}^{\mathrm{T}}\]
              对应范数:
              \[\left\lVert (z_1, \ldots, z_n) \right\rVert  = \sqrt{\lvert z^2_1 \rvert + \cdots + \lvert z^2_n \rvert}\]

        \item \label{item:23:内积和范数:2}
              定义在 $ \left[-1, 1\right] $ 上的连续实值函数构成的向量空间可定义内积如下:
              \[\left\langle f, g\right\rangle = \int_{-1}^1f(x)g(x)\,\mathrm{d}x\]
              对应范数:
              \[\left\lVert f \right\rVert = \sqrt{\int_{-1}^1(f(x))^2\,\mathrm{d}x}\]
    \end{enumerate}
\end{example}

\subsection{正交的定义 \quad 基于正交的性质}

以下给出一个关键定义:

\begin{definition} \index{zhengjiao@正交 (orthogonal)}
    两个向量 $u, v \in V$ 称为\term{正交的},如果 $\langle u, v\rangle = 0$.
\end{definition}

该定义中向量的次序是无关紧要的,因为 $\langle u, v\rangle = 0 \iff \langle v, u\rangle = 0$.

那么正交的定义关键在何处呢?以下给出 $\mathbf{R}^{n}$ 空间上夹角的定义以供理解(证明良定义需要用到 Cauchy-Schwarz 不等式):

\begin{definition}
    设 $u, v \in \mathbf{R}^{n}$,则 $u, v$ 的夹角 $ \theta $ 为$ \theta = \arccos \dfrac{\langle u, v\rangle}{\lVert u \rVert \lVert v \rVert}$.
\end{definition}

那么我们可以发现,当两向量正交时,它们的夹角就是 $\dfrac{\pi}{2}$,也就是我们在几何中常说的垂直,它能将我们导向一些重要的定理.

让我们从一些简单的结果开始研究正交性,比如正交性与 $ \vec{0} $ 的关系:

\begin{enumerate}
    \item $ \vec{0} $ 正交与 $V$ 中的任意向量.

    \item $ \vec{0} $ 是 $V$ 中唯一一个与自身正交的向量.
\end{enumerate}

然后是熟悉的勾股定理在内积空间上的推广:

\begin{theorem}
    设 $u, v$ 是 $V$ 中的正交向量,则 $\lVert u + v \rVert^2 = \lVert u \rVert^2 + \lVert v \rVert^2 $.
\end{theorem}

注意勾股定理的逆定理仅在实内积空间上成立.

借助于正交的性质,我们能够简化很多与内积相关的计算,进而会很自然的思考这样一个问题:一个向量能否分解两个互相正交的向量?

从而便引进了正交分解:

\begin{theorem}
    设 $u, v \in V$ 且 $v \neq \vec{0}$. 令 $ c = \dfrac{\langle u, v\rangle}{\lVert v \rVert^2}, \enspace w = u - \dfrac{\langle u, v\rangle}{\lVert v \rVert^2}v$. 则 $\langle w, v\rangle = 0$且 $u = cv + w$.
\end{theorem}

而通过正交分解,我们可以证明一个数学中最重要的不等式(之一):\term{Cauchy-Schwarz 不等式}.

\begin{theorem}[Cauchy-Schwarz 不等式] \index{Cauchy@Cauchy-Schwarz 不等式 (Cauchy-Schwarz inequality)}
    设 $u, v \in V$. 则 $\left\lvert \left\langle u, v\right\rangle \right\rvert \leqslant \lVert u \rVert\lVert v \rVert$. 等号成立当且仅当 $u, v$ 之一是另一个的标量倍.
\end{theorem}

也可以通过引入参数,利用二次三项式的判别式证明.

借助 Cauchy-Schwarz 不等式,我们可以得到三角不等式:

\begin{theorem}
    设 $u, v \in V$. 则 $\lVert u, v \rVert \leqslant \lVert u \rVert + \lVert v \rVert$. 等号成立当且仅当 $u, v$ 之一是另一个的非负标量倍.
\end{theorem}

其几何解释就是俗称的三角形两边之和小于第三边.

另一个与几何相关的结论就是平行四边形恒等式:

\begin{theorem}
    设 $u, v \in V$. 则 $ \lVert u + v \rVert^{2} + \lVert u - v \rVert^{2} = 2(\lVert u \rVert^{2} + \lVert v \rVert^{2})$.
\end{theorem}

其几何解释为任意的平行四边形两对角线的长度的平方和等于四边长度的平方和.

以下为另外几个与内积有关的恒等式,我们会在证明正规算子和自伴算子的性质时运用到它们:

\begin{example}
    证明下列式子成立:
    \begin{enumerate}
        \item $\mathbf{F} = \mathbf{R}$ 时:
              \begin{gather}
                  \label{eq:23:内积和范数的性质1}
                  \langle u, v\rangle = \frac{1}{4}\left( \lVert u + v \rVert^2 - \lVert u - v \rVert^2\right) \\
                  \label{eq:23:内积和范数的性质2}
                  \langle Tu, v\rangle + \langle Tv, u\rangle = \frac{1}{2}\left(\langle T(u + v), u + v\rangle - \langle T(u - v), u - v\rangle\right)
              \end{gather}

        \item $\mathbf{F} = \mathbf{C}$ 时:
              \begin{gather}
                  \label{eq:23:内积和范数的性质3}
                  \langle u, v\rangle = \frac{1}{4}\left((\lVert u + v \rVert^2 - \lVert u - v \rVert^2) + \i(\lVert u + \i v \rVert^2 - \lVert u - \i v \rVert^2)\right) \\
                  \label{eq:23:内积和范数的性质4}
                  \begin{aligned}
                      \langle Tu, v\rangle ={} & \frac{1}{4}  ((\langle T(u + v), u + v\rangle - \langle T(u - v), u - v\rangle)     \\
                                               & + \i(\langle T(u + \i v), u + \i v\rangle + \langle T(u - \i v), u - \i v\rangle)).
                  \end{aligned}
              \end{gather}
    \end{enumerate}

\end{example}

\section{标准正交基}

这一节我们将沿着正交的路径接着往下走,看看如果整个向量组乃至一组基都是单位化的且互相正交的话,会有怎样的性质. 我们也会讲解如何获取这样一组基的算法,并介绍 Riesz 表示定理,其揭示了线性泛函和内积的深刻联系.

\begin{definition} \index{zhengjiao!biaozhun@标准正交 (orthonormal), 规范正交}
    如果一个向量组的每个向量范数都是 1 且与其他向量正交则称这个向量组是\term{标准正交}(规范正交)的.
\end{definition}

在本书的剩余部分中我们都称此性质为标准正交.

由以上定义,我们得出:$ V $ 上的向量组 $ e_1, \ldots , e_n $ 是标准正交的,如果
\[ \langle e_j, e_k \rangle = \delta _{jk} = \begin{cases}
        1 & j = k    \\
        0 & j \neq k
    \end{cases} \]

标准正交组的优势在于处理其线性组合的范数很方便.

\begin{theorem}
    若 $e_1, \ldots , e_m$ 是 $ V $ 中的标准正交向量组,则对 $\forall a_1, \ldots, a_m \in \mathbf{F}$ 均有
    \[ \lVert a_1e_1 + \cdots + a_ne_n\rVert^2 = \lvert a_1 \rvert^2 + \cdots + \lvert a_n \rvert^2.\]
\end{theorem}

反复使用勾股定理即可证明. 该定理也有一个重要推论:

\begin{theorem}
    任何标准正交向量组都是线性无关的.
\end{theorem}

令其线性组合为 $ \vec{0} $ 即证.

\begin{example} \label{ex:22:标准正交组}
    设 $e_1, \ldots , e_m$ 是 $ V $ 的标准正交组. 设 $ v \in V $. 证明
    \[ \lVert v \rVert^2 = \lvert \langle v, e_1\rangle \rvert^2 + \cdots + \lvert \langle v, e_m\rangle \rvert^2 \]
    当且仅当 $ v \in \spa(e_1, \ldots , e_m)$.
\end{example}

既然标准正交组都是线性无关的,很自然我们就会想到在线性空间中最有用的线性无关组:基. 也就有了标准正交基的定义.

\begin{definition}
    $ V $ 的标准正交基是 $ V $ 中的标准正交向量组构成的基.
\end{definition}

而由向量组确定为基的等价条件,易知长度为 $\dim V$ 的标准正交向量组都是$ V $ 的标准正交基.

标准正交基的优势就在于表出向量的表出系数可以提前确定.

\begin{theorem}
    设 $e_1, \ldots e_n$ 是 $ V $ 标准正交基且 $ v \in V$. 则
    \begin{equation} \label{eq:23:Fourier展开}
        v = \langle v, e_1 \rangle e_1 + \cdots + \langle v, e_n \rangle e_n
    \end{equation}
    且
    \[ \lVert v \rVert^2 = \lvert \langle v, e_1\rangle \rvert^2 + \cdots + \lvert \langle v, e_m\rangle \rvert^2. \]
\end{theorem}

此为\autoref{ex:22:标准正交组} 的特例. \autoref{eq:23:Fourier展开} 也被称为 $ v $ 的 Fourier 展开,其中每个系数 $ \langle v, e_j \rangle $ 被称为 $ v $ 的 Fourier 系数.

标准正交基的性质十分美妙,但我们取出一组基使得其恰好是标准正交基是十分困难的,所幸前人已经研究出了一套算法,可以将所有线性无关组转变为标准正交组,且张成空间相同.

\begin{theorem}[Gram-Schmidt 过程] \index{Gram@Gram-Schmidt 过程 (Gram-Schmidt process)}
    设 $v_1, \ldots ,v_n$ 是 $ V $ 中的线性无关向量组. 设 $e_1 = \dfrac{v_1}{\lVert v_1 \rVert}$. 对于 $ j = 2, \ldots , m$,定义 $ e_j $ 如下:
    \[ e_j = \frac{v_j - \langle v_j, e_1 \rangle e_1 - \cdots - \langle v_j, e_{j - 1} \rangle e_{j - 1} }{\lVert v_j - \langle v_j, e_1 \rangle e_1 - \cdots - \langle v_j, e_{j - 1} \rangle e_{j - 1} \rVert}\]
    则 $e_1, \ldots , e_m $ 是 $ V $ 中的标准正交组,使得对 $ j = 1, \ldots , m $ 有
    \[ \spa(v_1, \ldots, v_j) = \spa(e_1, \ldots, e_j) \]
\end{theorem}

证明前半部分使用归纳法,后半部分证明两向量组等价即可.

让我们简单运用一下 Gram-Schmidt 过程.
\begin{example}
    求 $\mathbf{R}[x]_2$ 的一组标准正交基,内积定义为 $\langle p, q \rangle = \displaystyle\int_{-1}^1 p(x)q(x)\,\mathrm{d}x$.
\end{example}

不难发现,Gram-Schmidt 过程实际上可以分成两部分:
\begin{enumerate}
    \item 正交化:定义 $ u_1 = v_1 , \enspace u_j = v_j - \langle v_j, e_1 \rangle e_1 - \cdots - \langle v_j, e_{j - 1} \rangle e_{j - 1}, \enspace j = 2, \ldots , m$,此时 $u_1, \ldots , u_m$ 已经互相正交.

    \item 单位化:$ e_j = \dfrac{u_j}{\lVert u_j \rVert} , \enspace j = 1, \ldots , m$,从而有 $\lVert e_j \rVert = 1, \enspace j = 1, \ldots , m$
\end{enumerate}

Gram-Schmidt 过程可以说是线性代数计算较为困难的方面之一,也是应试经常考察的方面,需要多加注意.

借助 Gram-Schmidt 过程,显然我们可以得到以下结论:
\begin{enumerate}
    \item 每个有限维内积空间都有标准正交基;

    \item 设 $ V $ 是有限的. 则 $ V $ 中每个标准正交向量组都可以扩充成 $ V $ 的标准正交基.
\end{enumerate}
也可以得到这样一个定理.

\begin{theorem}[Schur 定理] \label{thm:23:Schur} \index{Schur@Schur 定理 (Schur's triangularization theorem)}
    设 $ V $ 是有限维的复内积空间,且 $ T \in \mathcal{L}(V) $,则 $ T $ 关于 $ V $ 的某个标准正交基具有上三角矩阵.
\end{theorem}

证明并不复杂,只需要结合\autoref{thm:20:上三角矩阵存在} 和 Gram-Schmidt 过程即可. 虽然十分浅显,但它迈出了我们在内积空间上算子简化表示的第一步,在更进一步的结论中我们会运用到它.

在此我们先打住,回忆一下内积的定义,其本质上就是一个函数,它把 $ V $ 中元素的每个有序对 $(u, v)$ 都映射成一个数$ \langle u, v \rangle \in \mathbf{F}$,而我们也很熟悉一类把 $ V $ 中元素映射成一个数的函数,即所谓的线性泛函. 那么这两者之间是否存在着某种联系?我们先借助几个例子观察一下.

\begin{example}
    \begin{enumerate}
        \item 定义如下的函数 $\varphi : \mathbf{F}^{3} \rightarrow \mathbf{F}$
              \[\varphi(z_1, z_2, z_3) = 2z_1 - 5z_2 + z_3\]
              是 $\mathbf{F}^{3}$ 上的线性泛函. 我们可以将其写成以下形式:$ \forall z \in \mathbf{F}^{3}$,
              \[\varphi(z) = \langle z, u\rangle\]
              其中 $u = (2, -5, 1)$.

        \item 定义如下的函数 $\varphi$: $\mathbf{R}[x]_2 \rightarrow \mathbf{R}$
              \[\varphi(p) = \int_{-1}^1 p(t)(\cos(\pi t))\,\mathrm{d}t\]
              是 $\mathbf{R}[x]_2$ 上的线性泛函, 此处的内积为\autoref{ex:23:内积和范数} \ref*{item:23:内积和范数:2} 中定义的. 但以下的事实并不那么显然:$ \exists u \in \mathbf{R}[x]_2$, 使得$\forall p \in \mathbf{R}[x]_2$ 均有 $ \varphi (p) = \langle p, u\rangle $.
    \end{enumerate}
\end{example}

可以发现,若是固定内积的第二个位置上的向量,内积就等同于一个线性泛函. 即对于确定的 $ u \in V , \enspace \varphi(v) = \langle v, u \rangle$ 就是一个线性泛函. 下面的定理揭示了这两者的关系,其指出 $ V $ 上所有的线性泛函都是这种形式:
\begin{theorem}[Riesz 表示定理] \label{thm:23:Riesz} \index{Riesz@Riesz 表示定理 (Riesz representation theorem)}
    设 $ V $ 是有限维的且 $ \varphi $ 是 $ V $ 上的线性泛函,则存在唯一的向量$u \in V$ 使得对 $\forall v \in V$ 均有 $ \varphi(v) = \langle v, u\rangle $.
\end{theorem}

\begin{proof}
    存在性:设 $e_1, \ldots , e_n$ 是 $ V $ 上的一组标准正交基,则对 $\forall v \in V $ 均有
    \begin{align*}
        \varphi (v) & = \varphi(\langle v, e_1 \rangle e_1 + \cdots + \langle v, e_n \rangle e_n )           \\
                    & = \langle v, e_1 \rangle \varphi (e_1) + \cdots + \langle v, e_n \rangle \varphi (e_n) \\
                    & = \langle v, \overline{\varphi(e_1)}e_1 + \cdots + \overline{\varphi(e_n)}e_n \rangle
    \end{align*}
    故取
    \[ u = \overline{\varphi(e_1)}e_1 + \cdots + \overline{\varphi(e_n)}e_n, \]
    对 $\forall v \in V$ 都有 $\varphi(v) = \langle v, u \rangle $.

    唯一性:设 $ u_1, u_2 \in V $ 使得对 $\forall v \in V $ 均有
    \[\varphi(v) = \langle v, u_1 \rangle = \langle v, u_2 \rangle.\]
    则对 $\forall v \in V $ 均有
    \[ 0 = \langle v, u_1 \rangle - \langle v, u_2 \rangle = \langle v, u_1 - u_2 \rangle\]
    取 $ v = u_1 - u_2 $ 可得 $ u_1 - u_2 = 0 $,即 $ u_1 = u_2 $,唯一性得证.
\end{proof}

Riesz 表示定理不仅证明了内积和线性泛函的联系,也给出了求解向量 $ u $ 的公式使其满足$ \forall v \in V $,使得 $ \varphi(v) = \langle v, u\rangle $. 具体来说,就是
\[ u = \overline{\varphi(e_1)}e_1 + \cdots + \overline{\varphi(e_n)}e_n, \]
而根据 Riesz 表示定理,我们知道 $ u $ 只依赖于线性泛函 $ \varphi $,所以选取任意一组$ V $ 上的标准正交基都会计算出相同的结果.

按以往的经验,该节的内容常在考试中作为大题单独考察.
\begin{example}
    定义在 $ V = \mathbf{R}^3 $ 上的运算
    \[ \langle \vec{x}, \vec{y} \rangle_V = x_1 y_1 + x_2 y_2 + (x_2 + x_3)(y_2 + y_3) \]
    其中 $ \vec{x} = (x_1, x_2, x_3),\enspace \vec{y} = (y_1, y_2, y_3) $.
    \begin{enumerate}
        \item 验证 $ \langle \cdot, \cdot \rangle_V $ 是 $ \mathbf{R}^3 $ 上的一个内积;

        \item 求 $ \mathbf{R}^3 $ 在 $ \langle \cdot, \cdot \rangle_V $ 下的一组标准正交基;

        \item 求 $ \vec{\beta} \in V $ 使得 $ \forall \vec{x} \in V,\enspace x_1 + 2x_2 = \langle \vec{x}, \vec{\beta} \rangle_V $.
    \end{enumerate}
\end{example}

\section{正交补}

本节的内容更偏向于几何方向,将带领大家了解空间的正交补,以及一种特殊的映射:正交投影. 并介绍一下极小化问题及一点应用.

\subsection{正交补 \quad 正交投影}

\begin{definition} \index{zhengjiao!bu@补 (orthogonal complement)}
    设 $ U $ 是 $ V $ 的子集,则 $ U $ 的\term{正交补}(记作 $ U^{\perp } $)是由 $ V $ 中与 $ U $ 的每个向量都正交的那些向量组成的集合:
    \[U^{\perp } = \{ v \in V \mid \forall u \in U, \enspace \langle v, u\rangle = 0\}\]
\end{definition}

例如,若 $ U $ 是 $ \mathbf{R}^{3} $ 中的直线,则 $ U^{\perp } $ 是垂直于 $ U $ 且包含原点的平面. 若 $ U $ 是 $ \mathbf{R}^{3} $ 中的平面,则 $ U^{\perp } $ 是垂直于 $ U $ 且包含原点的直线.

正交补具有如下的基本性质:
\begin{enumerate}
    \item 若 $ U $ 是 $ V $ 的子集(注意使用的是子集),则 $ U^{\perp }$ 是 $ V $ 的子空间.

    \item $ \{ \vec{0} \}^{\perp } = V $.

    \item $ V^{\perp } = \{ \vec{0} \} $.

    \item 若 $ U $ 是 $ V $ 的子集,则 $ U \cap U^{\perp } \subset \{ \vec{0} \}$.

    \item 若 $ U $ 和 $ W $ 均为 $ V $ 的子集且 $ U \subset W $,则 $ W^{\perp } \subset U^{\perp }$.
\end{enumerate}

那么根据之前的几何示例,我们不难猜测,如果 $ U $ 上升成为了一个子空间,那么就可以诱导一个自然的直和分解.

\begin{theorem}
    设 $ U $ 是 $ V $ 的有限维子空间,则 $ V = U \oplus U^{\perp } $.
\end{theorem}

由该直和分解,我们可以推出两个结论:
\begin{enumerate}
    \item 若 $ V $ 是有限维的且 $ U $ 是 $ V $ 的子空间,则$ \dim U^{\perp }= \dim V - \dim U$

    \item 设 $ U $ 是 $ V $ 的有限维子空间,则 $ U = (U^{\perp})^{\perp} $
\end{enumerate}
第二条的证明还是具有一定技巧性的,希望读者仔细品味.

除去这两个结论,该直和分解为我们定义正交投影奠定了基础:
\begin{definition} \index{zhengjiao@touying@投影 (orthogonal projection)}
    设 $ U $ 是 $ V $ 的有限维子空间. 定义 $ V $ 到 $ U $ 上的\term{正交投影}
    为如下算子$ P_U \in \mathcal{L} (V)$:对 $ v \in V $ 将其写成 $ v = u + w $,其中$ u \in U $ 且 $ w \in U^{\perp }$,则 $ P_U v = u $.
\end{definition}

正交投影的性质相当之多,不过大部分是成对刻画以及简单的推理,在此仅作罗列不加证明:

设 $ U $ 是 $ V $ 的有限维子空间且 $ v \in V$. 则
\begin{enumerate}
    \item $ P_U \in \mathcal{L} (V) $;

    \item 对 $ \forall u \in U$ 均有 $ P_U u = u $;

    \item 对 $ \forall w \in U^{\perp}$ 均有 $ P_U w = \vec{0} $;

    \item $ \im P_U = U$;

    \item $ \ker P_U = U^{\perp}$;

    \item $ v - P_U v \in U^{\perp}$;

    \item \label{item:23:正交投影性质:7}
          $ {P_U}^{2} = P_U$;

    \item $ \lVert P_U v\rVert \leqslant \lVert v \rVert $;

    \item 对 $ U $ 的每个规范正交基 $e_1, \ldots , e_m$ 均有 $ P_U v = \langle v, e_1 \rangle e_1 + \cdots + \langle v, e_m \rangle e_m$.
\end{enumerate}
其中由 \ref*{item:23:正交投影性质:7} 我们知道正交投影存在一种矩阵表示是幂等的,且进一步可以证明在实空间上则是对称幂等的,这一性质在卡方分布中有着运用,此处仅仅介绍一下.

\begin{example}
    设 $ U $ 是实内积空间 $ V $ 的一个有限维子空间. 证明:正交投影 $ P_U $ 具有以下性质:
    \[\langle P_U u, v\rangle = \langle u, P_U v\rangle, \enspace \forall u, v \in V \]
\end{example}

这个例子的意义可能你暂时还没法理解,但等你学习完正规算子与自伴算子后,再结合这个例子,你会发现实内积空间上的正交投影存在一种对称幂等的矩阵表示形式是显然的.

\subsection{极小化问题}

我们常会遇到这样的一种问题:给定 $ V $ 的子空间 $ U $ 和点 $ v \in V $,求点$ u \in U $ 使得 $ \lVert v - u \rVert $ 最小. 下面的定理表明,取 $ u = P_U v$即可解决此极小化问题.

\begin{theorem}
    设 $ U $ 是 $ V $ 的有限维子空间,$ v \in V $ 且 $ u \in U $. 则
    \[\lVert v - P_U v \rVert \leqslant \lVert v - u \rVert. \]
    等号成立当且仅当 $ u = P_U v $.
\end{theorem}

\subsubsection*{$^*$ 极小化问题的应用:最小二乘解}

在许多实际问题中我们需要研究一个变量 $ y $ 和其他一些变量 $ x_1, \ldots , x_n $之间的依赖关系. 经过实际观测和分析,假定 $ y $ 与 $ x_1, \ldots , x_n $ 之间呈线性关系,即
\[ y = k_1 x_1 + \cdots + k_n x_n, \]
其中系数 $ k_1, \ldots , k_n $ 是未知的,为确定它们,需要观测数据 $ m $ 次,即测得 $ m $ 组数:
\begin{center}
    \begin{tabular}{ccccc}
        $ y $      & $ x_1 $    & $ x_2 $    & $ \cdots $ & $ x_n $    \\
        \hline
        $ b_1 $    & $ a_{11} $ & $ a_{12} $ & $ \cdots $ & $ a_{1n}$  \\
        $ \vdots $ & $ \vdots $ & $ \vdots $ & $ \ddots $ & $ \vdots $ \\
        $ b_m $    & $ a_{m1} $ & $ a_{m2} $ & $ \cdots $ & $ a_{mn}$
    \end{tabular}
\end{center}
如果观测是绝对精准的,那么只需要测量 $ m = n $ 次,通过线性方程组即可解得 $ k_1, \ldots , k_n $. 但是任何观测都会有误差,这样就会需要更多的观测次数,即 $ m > n $,得到如下的线性方程组
\[ \begin{cases} \begin{aligned}
            a_{11}k_1 + \cdots + a_{1n}k_n & = b_1,          \\
            a_{21}k_1 + \cdots + a_{2n}k_n & = b_2,          \\
                                           & \vdotswithin{=} \\
            a_{m1}k_1 + \cdots + a_{mn}k_n & = b_m.          \\
        \end{aligned} \end{cases} \]
中,方程个数 $ m $ 大于未知数个数 $ n $,该线性方程组可能无解. 于是我们的目标转为寻找一组数 $ c_1, \ldots , c_n $,使得
\begin{align*}
                    & \sum_{ i = 1 }^{m} (a_{i1}c_1 + \cdots + a_{in}c_n - b_i )^{2}                                                    \\
    \leqslant \quad & \sum_{ i = 1 }^{m} (a_{i1}k_1 + \cdots + a_{in}k_n - b_i )^{2}, \enspace \forall k_1, \ldots , k_n \in \mathbf{R}
\end{align*}
此时我们把 $ (c_1, \ldots , c_n)^{\mathrm{T}} $ 称为该线性方程组的最小二乘解.

鉴于上式两侧平方和的形式类似于欧几里得空间 $ \mathbf{R}^{m} $ 下范数的平方,该问题可被转化为一个极小化问题. 该向量的第 $ i $ 个分量为
\[a_{i1}c_1 + \cdots  + a_{in}c_n - b_i, \enspace i = 1, \ldots , m\]
将线性方程组的系数矩阵记为 $ \mathbf{A} $,其列向量组记为 $ \vec{\alpha} _1 , \ldots , \vec{\alpha} _n $,行向量组记为 $ \vec{\gamma} _1 , \ldots , \vec{\gamma} _m $. 令
\[ \vec{x} = (k_1, \ldots , k_n)^{\mathrm{T}}, \enspace \vec{\beta} = (b_1, \ldots , b_n)^{\mathrm{T}}, \enspace \vec{\alpha} = (c_1, \ldots , c_n)^{\mathrm{T}}\]
将分量形式合写为
\[ \begin{pmatrix}
        \vec{\gamma} _1\vec{\alpha} - b_1 \\
        \vec{\gamma} _2\vec{\alpha} - b_2 \\
        \vdots                            \\
        \vec{\gamma} _m\vec{\alpha} - b_m \\
    \end{pmatrix}
    =
    \begin{pmatrix}
        \vec{\gamma} _1\vec{\alpha} \\
        \vec{\gamma} _2\vec{\alpha} \\
        \vdots                      \\
        \vec{\gamma} _m\vec{\alpha} \\
    \end{pmatrix}
    -
    \begin{pmatrix}
        b_1    \\
        b_2    \\
        \vdots \\
        b_m    \\
    \end{pmatrix}
    = \mathbf{A}\vec{\alpha} - \vec{\beta} \]

设 $ U = \spa(\vec{\alpha} _1 , \ldots , \vec{\alpha} _n)$,易知 $ \mathbf{A}\vec{\alpha} \in U , \enspace \mathbf{A}\vec{x} \in U , \enspace \forall k_1, \ldots , k_n \in \mathbf{R}$

从而条件转化为
\begin{align*}
               & \lVert \mathbf{A}\vec{\alpha} - \vec{\beta} \rVert^{2} \leqslant \lVert \mathbf{A}\vec{x} - \vec{\beta} \rVert^{2}, \enspace \vec{x} \in \mathbf{R}^{n} \\
    \iff \quad & P_U \vec{\beta} = \mathbf{A}\vec{\alpha}                                                                                                                \\
    \iff \quad & \vec{\beta} - \mathbf{A}\vec{\alpha} \in U^{\perp}                                                                                                      \\
    \iff \quad & \langle \vec{\beta} - \mathbf{A}\vec{\alpha}, \vec{\alpha} _j \rangle = 0, \enspace j = 1, \ldots , n                                                   \\
    \iff \quad & {\vec{\alpha} _j}^{\mathrm{T}}(\vec{\beta} - \mathbf{A}\vec{\alpha}) = 0, \enspace j = 1, \ldots , n                                                    \\
    \iff \quad & \mathbf{A}^{\mathrm{T}}(\vec{\beta} - \mathbf{A}\vec{\alpha}) = \vec{0}                                                                                 \\
    \iff \quad & \mathbf{A}^{\mathrm{T}}\mathbf{A}\vec{\alpha} = \mathbf{A}^{\mathrm{T}}\vec{\beta}
\end{align*}

由于
\begin{gather*}
    r(\mathbf{A}^{\mathrm{T}}\mathbf{A}, \mathbf{A}^{\mathrm{T}}\vec{\beta})= r (\mathbf{A}^{\mathrm{T}}(\mathbf{A}, \vec{\beta})) \leqslant r(\mathbf{A}^{\mathrm{T}}) = r(\mathbf{A}^{\mathrm{T}}\mathbf{A}) \\
    r(\mathbf{A}^{\mathrm{T}}\mathbf{A}, \mathbf{A}^{\mathrm{T}}\vec{\beta}) \geqslant r(\mathbf{A}^{\mathrm{T}}\mathbf{A})
\end{gather*}

因此$r(\mathbf{A}^{\mathrm{T}}\mathbf{A}, \mathbf{A}^{\mathrm{T}}\vec{\beta}) = r(\mathbf{A}^{\mathrm{T}}\mathbf{A})$,由我们很久之前学习的关于非齐次线性方程组有解的条件,可以得出 $ \mathbf{A}^{\mathrm{T}}\mathbf{A}\vec{x} = \mathbf{A}^{\mathrm{T}}\vec{\beta}$ 一定有解. 故求线性方程组 $\mathbf{A}\vec{x} = \vec{\beta}$ 的最小二乘解转化为求线性方程组 $ \mathbf{A}^{\mathrm{T}}\mathbf{A}\vec{x} = \mathbf{A}^{\mathrm{T}}\vec{\beta}$ 的解.

\vspace{2ex}
\centerline{\heiti \Large 内容总结}

本章是内积空间的基础,首先通过点积引入内积这一最基本的概念,然后相应地定义了范数. 然后沿着正交的路径拾级而上,从正交的定义、性质,到标准正交的向量组,标准正交基,最后到了正交的子空间:正交补. 此外,在标准正交基部分中我们顺着前人的思路,成功掌握了求标准正交基的方法,即 Gram-Schmidt 过程,也通过 Riesz 表示定理寻找到了内积的凭依:它就是我们曾学习过的线性泛函,只不过换了一种形式. 另外还有一些可能并非应试重点考察但我希望你了解一下的内容,它们往往与之后的章节或是其他的课程有着一些现阶段不容易想见的联系,如极小化的应用等. 但这正是数学的美妙之处,不是吗?

\vspace{2ex}
\centerline{\heiti \Large 习题}

\vspace{2ex}
{\kaishu }
\begin{flushright}
    \kaishu

\end{flushright}

\centerline{\heiti A组}
\begin{enumerate}
    \item
\end{enumerate}

\centerline{\heiti B组}
\begin{enumerate}
    \item 设 $ (e_1, \ldots , e_m) $ 是复内积空间 $ V $ 的一个标准正交组,证明:$ \forall v \in V $,均有
          \[ \sum_{j = 1}^{m} \lvert \langle v, e_j \rangle \rvert^2 \leqslant \lVert v \rVert^2, \]
          等号成立当且仅当 $ v = \displaystyle\sum_{j = 1}^{m} \langle v, e_j \rangle e_j $. 这个不等式被称为 Bessel 不等式.
\end{enumerate}

\centerline{\heiti C组}
\begin{enumerate}
    \item
\end{enumerate}

\chapter{内积空间上的算子(I)}

\section{正交矩阵和酉矩阵}

\section{正定矩阵}

\vspace{2ex}
\centerline{\heiti \Large 内容总结}

\vspace{2ex}

\centerline{\heiti \Large 习题}
\vspace{2ex}
{\kaishu }
\begin{flushright}
    \kaishu

\end{flushright}
\centerline{\heiti A组}
\begin{enumerate}
    \item
\end{enumerate}
\centerline{\heiti B组}
\begin{enumerate}
    \item
\end{enumerate}
\centerline{\heiti C组}
\begin{enumerate}
    \item
\end{enumerate}

\chapter{内积空间上的算子(II)}

\section{自伴算子和正规算子}

\section{谱定理}

\vspace{2ex}
\centerline{\heiti \Large 内容总结}

\vspace{2ex}

\centerline{\heiti \Large 习题}
\vspace{2ex}
{\kaishu }
\begin{flushright}
    \kaishu

\end{flushright}
\centerline{\heiti A组}
\begin{enumerate}
    \item
\end{enumerate}
\centerline{\heiti B组}
\begin{enumerate}
    \item
\end{enumerate}
\centerline{\heiti C组}
\begin{enumerate}
    \item
\end{enumerate}

\chapter{二次型}

\section{二次型的定义}
\begin{definition}
    $n$个元$x_1,x_2,\ldots,x_n$的二次齐次多项式
    \begin{align*}
        f(x_1,x_2,\ldots,x_n) &= \sum_{i=1}^{n}a_{ii}x_i^2+\sum\limits_{1\leqslant i<j\leqslant n}2a_{ij}x_ix_j \\
                              &= a_{11}x_1^2+a_{22}x_2^2+\cdots+a_{nn}x_n^2 \\
                              & \quad +2a_{12}x_1x_2+\cdots+2a_{1n}x_1x_n+2a_{23}x_2x_3+\cdots+2a_{n-1,n}x_{n-1}x_n
    \end{align*}
    称为数域$\mathbf{F}$上的$n$元二次型(简称\keyterm{二次型}[quadratic form]).
\end{definition}
本学期研究的主要是实二次型.若令$a_{ij}=a_{ji}(1\leqslant i<j\leqslant n)$,则二次型可表示为
\[f(x_1,x_2,\ldots,x_n)=\sum_{i=1}^{n}\sum_{j=1}^{n}a_{ij}x_ix_j=X^\mathrm{T}AX\]
其中$X=(x_1,x_2,\ldots,x_n)^\mathrm{T}\in\mathbf{R}^n$,$A=(a_{ij})_{n\times n}$为实对称矩阵,
并称对称矩阵$A$为二次型$f(x_1,x_2,\ldots,x_n)$的矩阵.

注意,二次型实际上是一个$\mathbf{R}^n\to\mathbf{R}$的函数,所以本质上代入$x_1,\ldots,x_n$后就是一个实数,写成矩阵形式
我们也可以发现矩阵相乘结果为$1\times 1$矩阵,即一个实数,因此不必把二次型想得过于复杂.

同时需要注意,二次型对应矩阵一定是对称矩阵.实际上一个形如$f(x_1,x_2,\ldots,x_n)=\displaystyle\sum_{i=1}^{n}\displaystyle\sum_{j=1}^{n}a_{ij}x_ix_j$
的函数可以对应的矩阵是很多的,但我们要求$a_{ij}=a_{ji}$才能得到二次型对应的矩阵.
\begin{example}
    已知二次型
    \[f(X)=(x_1,x_2,x_3,x_4)\begin{pmatrix}
        1 & 2 & 3 & -4 \\ 3 & 2 & 1 & 4 \\ -4 & 3 & -7 & 2 \\ 0 & -6 & 8 & 4
    \end{pmatrix}\begin{pmatrix}
        x_1 \\ x_2 \\ x_3 \\ x_4
    \end{pmatrix}\]
    写出二次型$f(X)$的矩阵.
\end{example}
\begin{example}
    回答以下问题:
    \begin{enumerate}
        \item 已知$A$是一个$n$阶矩阵,则$A$为反对称矩阵的充要条件是对任意$n$元列向量$X$都有$X^\mathrm{T}AX=0$;

        \item 若二次型$f(x_1,x_2,\ldots,x_n)=X^\mathrm{T}AX$对任意$n$元列向量$X$都有$f(x_1,x_2,\ldots,x_n)=0$,证明:$A=O$;

        \item 设二次型$f(x_1,x_2,\ldots,x_n)=X^\mathrm{T}AX$,$g(x_1,x_2,\ldots,x_n)=X^\mathrm{T}BX$,证明:若$f(x_1,x_2,\ldots,x_n)=
              g(x_1,x_2,\ldots,x_n)$,则$A=B$.
    \end{enumerate}
\end{example}

\section{矩阵相合的定义与性质}
\begin{definition}
    我们称$n$阶矩阵$A$相合于$B$(记作$A\simeq B$),如果存在可逆矩阵$C$使得$B=C^\mathrm{T}AC$.
\end{definition}
矩阵相合(合同)有如下基本性质:
\begin{enumerate}
    \item 合同是等价关系;合同不同于相似,是与数域有关的;合同要求$C$必须可逆,因此是一种特殊的相抵;

    \item $A\simeq B$一般不能得到$A^m\simeq B^m$(但是$A,B$为实对称矩阵时可以),但如果可逆,我们有
    $A^{-1}\simeq B^{-1}$,同时如果$A_1\simeq A_2,B_1\simeq B_2$,则有$\begin{pmatrix}
        A_1 & O \\ O & B_1
    \end{pmatrix}\simeq\begin{pmatrix}
        A_2 & O \\ O & B_2
    \end{pmatrix}$;

    \item $A\simeq B$表明$A$可以每次做相同的初等行列变换得到$B$,反之亦然.这实际上就是初等变换法求相合标准形
    的基本原理,详见教材260页小字部分,感兴趣同学可以了解,一般不会要求使用这一方法.
\end{enumerate}

\begin{example}
    设$A\simeq B$,$C\simeq D$,且它们都是$n$阶实对称矩阵,问:$A+C\simeq B+D$ 是否成立.
\end{example}
\begin{example}
    判断:矩阵相似是否一定合同?矩阵合同是否一定相似?对于实对称矩阵上述论断又是否正确呢?
    正确请说明理由,不正确请举出反例.
\end{example}
实际上,教材中引入合同与二次型使用了双线性函数这一概念,实际上与双线性函数的度量矩阵有关,感兴趣的同学可以了解,
但这部分属于小字,考试一般不做考查要求.

\section{二次型标准形的定义与求解}
实际上二次型可以视为一个空间曲线/曲面方程,我们希望这些方程化为标准形式,有助于我们讨论一些问题.
由于实二次型对应矩阵为实对称矩阵,实对称矩阵一定可以相似对角化,故有下面的定理:
\begin{theorem}
    任意二次型$f(X)=X^\mathrm{T}AX$总可以通过可逆的线性变换$X=PY$(其中$P$可逆)化为标准形,
    即$f(X)=X^\mathrm{T}AX\xlongequal{X=PY}Y^\mathrm{T}(P^\mathrm{T}AP)Y=d_1y_1^2+d_2y_2^2+\cdots+d_ny_n^2$.
\end{theorem}
一般而言,我们有三种方法求解二次型标准形,分别为正交变换法,配方法和初等变换法.正交变换法由于涉及正交因此不作要求,
初等变换法之前已经提及并且较为复杂,不推荐优先使用.因此我们接下来主要使用配方法.

注意,求二次型标准形不应使用之前求相似标准形的一般方法,因为只有正交矩阵才能保证$P^{-1}=P^\mathrm{T}$,一般矩阵无法保证.
当然实际上求得的对角矩阵都是由特征值按重数排列而成的,只是矩阵$P$不合要求,应当做 Schmidt 正交化.

配方法的思想非常简单,就是利用配方消除混合乘积项,将二次型表示成几个平方和的形式,最后通过坐标变换$X=CY$(又称仿射变换,其中$C$可逆)
化标准形.
\begin{example}
    用配方法把三元二次型
    \[f(x_1,x_2,x_3)=2x_1^2+3x_2^2+x_3^2+4x_1x_2-4x_1x_3-8x_2x_3\]
    化为标准形,并求所用的坐标变换$X=CY$即变换矩阵$C$.
\end{example}
配方法是合理的,因为$X=CY$,其中$C$可逆,则$X^\mathrm{T}AX=Y^\mathrm{T}(C^\mathrm{T}AC)Y$,配方法使得$C^\mathrm{T}AC$
为对角矩阵,因此可以得到相合标准形.但是这种方法不能用来求相似对角化,原因仍然是$C^{-1}=C^\mathrm{T}$需要$C$为正交矩阵,但
坐标变换矩阵不一定满足.所以一定要区分好求解相似、相合标准形使用的方法,不能因为题目经常给的是实对称矩阵而混淆,只有正交变换法
是通用的,因为正交矩阵满足$P^{-1}=P^\mathrm{T}$使得相似、相合的定义统一.

注意:有的同学可能知道正交变换法的具体操作流程,如果能保证计算正确且题目不强制配方法时可以使用,但是历年考试经常出现部分题目
求解特征值时三次方程解不出的情况,此时一定要立刻醒悟,转向配方法解决问题.

\section{相合规范形 \quad 惯性定理}
事实上,一个二次型通过正交变换标准化得到的对角矩阵对角线上元素为特征值按重数排列的结果,但是使用配方法、初等变换法则不一定,
甚至配方方式或者初等变换顺序不同都会产生不同的对角矩阵,因此相合标准形不唯一.但我们知道,相抵标准形唯一,相似标准形不考虑
排列组合因素也是唯一的,因此我们也需要统一相合标准形.

我们不难发现,任一对角矩阵一定相合于$\diag(1,\ldots,1,-1,\ldots,-1,0,\ldots,0)$(我们很容易写出对应的可逆变换矩阵),
我们称这一相合标准形为相合规范形,其中+1的个数称为矩阵的正惯性指数,-1的个数称为矩阵的负惯性指数.并且由于变换矩阵可逆,根据
相抵标准形的结论,我们有原矩阵$A$的秩$r(A)$等于这一对角矩阵的秩,于是也等于正负惯性指数之和.显然,$A$可逆时,其相合规范形
主对角元没有0.

但我们没有说明一个矩阵的相合规范形是否唯一,实际上这就是下面惯性定理的结果:
\begin{theorem}
    实对称矩阵的相合规范形唯一.
\end{theorem}
这一定理有很多等价表述,例如实对称矩阵正、负惯性指数唯一,或者实对称矩阵相合标准形中对角线上正、负、零的个数唯一.
或者实对称矩阵特征值中正、负、零的个数唯一等.
这一定理的证明方法比较经典,最关键的一步在于代入数值导出矛盾,代入的方法是在两种表达的正负号分界线前后分别置0,使得
两种表达形式一个大于0,一个小于等于0.
\begin{example}
    解答如下问题:
    \begin{enumerate}
        \item 设$n$元二次型$f(x_1,x_2,\ldots,x_n)=l_1^2+\cdots+l_p^2-l_{p+1}^2-\cdots-l_{p+q}^2$,其中$l_i\enspace (i=1,2,\ldots,p+q)$
            是关于$x_1,x_2,\ldots,x_n$的一次齐次式. 证明:$f(x_1,x_2,\ldots,x_n)$的正惯性指数$\leqslant p$,负惯性指数$\leqslant q$;

        \item 已知$A$为$m$阶实对称矩阵,$C$为$m\times n$实矩阵,证明:$C^\mathrm{T}AC$的正负惯性指数
            分别小于等于$A$的正负惯性指数.
    \end{enumerate}
\end{example}
\begin{example}
    确定二次型$f(x_1,x_2,\ldots,x_{10})=x_1x_2+x_3x_4+x_5x_6+x_7x_8+x_9x_{10}$的秩以及正、负惯性指数.
\end{example}

惯性定理的``惯性''二字与物理中的惯性有关,实际上透露着某种不变性. 根据惯性定理,我们有如下结论:
\begin{enumerate}
    \item 我们可以按相合关系对全体$n$阶实对称矩阵分类,因为实对称矩阵相合意味着规范形唯一,我们可以按照$+1$、$-1$、0个数的不同
        划分为$\dfrac{(n+1)(n+2)}{2}$个等价类(相抵、相似也是等价关系,可以思考划分等价类的方式与个数);

    \item 实数域上两个实对称矩阵相合的充要条件是它们有相同的正负惯性指数,两个对角矩阵相合的充要条件是对角线上正、负、零个数相同.
\end{enumerate}
注:复数域上两个对称矩阵相合的充要条件是它们的秩相同(可以思考其证明),例如$E_n$和$-E_n$在复数域上相合,但实数域上不相合.
\begin{example}
    设$A=\begin{pmatrix}
        1 & 2 & 0 \\ 2 & 1 & 0 \\ 0 & 0 & 3
    \end{pmatrix},\enspace B=\begin{pmatrix}
        -2 & 0 & 0 \\ 0 & 2 & 1 \\ 0 & 1 & 2
    \end{pmatrix}$,判断$A$与$B$是否相合.
\end{example}

\section{标准形的应用}
我们在本学期讨论了三种标准形,即相抵标准形,相似标准形和相合标准形,实际上它们之间的关系我们已经讨论,
即相似一定相抵,相合一定相抵,但相似和相合互相没有包含关系.本节我们考虑一些基于矩阵分解的问题,利用之前所学的
相抵标准形、相似标准形、相合标准形的分解解决一些问题.本节内容可以选择性掌握.

首先看一个关于幂等矩阵的例题,需要用到相抵标准形、相似标准形的分解:
\begin{example}
    解答以下两个问题:
    \begin{enumerate}
        \item 证明:任意一个方阵都可以分解成一个可逆矩阵和一个幂等矩阵的乘积;

        \item 已知$A$是一个秩为$r$的$n$级非零矩阵,证明:$A$为幂等矩阵的充要条件是存在列满秩的$n\times r$矩阵$B$和行满秩的
            $r\times n$矩阵$C$使得$A=BC$且$CB=E_r$.
    \end{enumerate}
\end{example}
下面是一个利用相合标准形进行分解的例子:
\begin{example}
    (与正交有关)证明:每个秩为$r$的$n(r<n)$阶实对称矩阵均可表示为$n-r$个秩为$n-1$的实对称矩阵的乘积.
\end{example}

\vspace{2ex}
\centerline{\heiti \Large 内容总结}

\vspace{2ex}

\centerline{\heiti \Large 习题}
\vspace{2ex}
{\kaishu }
\begin{flushright}
    \kaishu

\end{flushright}
\centerline{\heiti A组}
\begin{enumerate}
    \item
\end{enumerate}
\centerline{\heiti B组}
\begin{enumerate}
    \item
\end{enumerate}
\centerline{\heiti C组}
\begin{enumerate}
    \item
\end{enumerate}

\chapter{极分解与奇异值分解}

\vspace{2ex}
\centerline{\heiti \Large 内容总结}

\vspace{2ex}

\centerline{\heiti \Large 习题}
\vspace{2ex}
{\kaishu }
\begin{flushright}
    \kaishu

\end{flushright}
\centerline{\heiti A组}
\begin{enumerate}
    \item
\end{enumerate}
\centerline{\heiti B组}
\begin{enumerate}
    \item
\end{enumerate}
\centerline{\heiti C组}
\begin{enumerate}
    \item
\end{enumerate}

\chapter{实空间上的算子}

\vspace{2ex}
\centerline{\heiti \Large 内容总结}

\vspace{2ex}

\centerline{\heiti \Large 习题}
\vspace{2ex}
{\kaishu }
\begin{flushright}
    \kaishu

\end{flushright}
\centerline{\heiti A组}
\begin{enumerate}
    \item
\end{enumerate}
\centerline{\heiti B组}
\begin{enumerate}
    \item
\end{enumerate}
\centerline{\heiti C组}
\begin{enumerate}
    \item
\end{enumerate}

\chapter{行列式(II)}

\vspace{2ex}
\centerline{\heiti \Large 内容总结}

\vspace{2ex}

\centerline{\heiti \Large 习题}
\vspace{2ex}
{\kaishu }
\begin{flushright}
    \kaishu

\end{flushright}
\centerline{\heiti A组}
\begin{enumerate}
    \item
\end{enumerate}
\centerline{\heiti B组}
\begin{enumerate}
    \item
\end{enumerate}
\centerline{\heiti C组}
\begin{enumerate}
    \item
\end{enumerate}

\chapter{线性代数与解析几何基础}

解析几何很大程度上是线性代数发展的初衷,在研究点线面以及几何体时,将集体的几何问题抽象化为代数问题使其方便解决与计算,即是解析几何的主要思想.
本节我们将会从线性代数的角度探究解析几何的一些基本概念与方法.此在线性代数课程的考察中也会有少部分的解析几何内容,但内容较浅,主要考察点、直线、平面等之间的关系.
\section{欧几里得空间}
在前面的学习中我们已经较为全面地学习了内积空间的相关知识,而在解析几何中,我们在更多情况下会研究\keyterm{欧几里得空间}[Euclidean Space]下的问题.
\begin{definition}[欧几里得空间]
    欧几里得空间(欧氏空间)是一个有限维实内积空间.
\end{definition}
同学们可能对欧氏空间的几何直观更为熟悉.当欧氏空间的维数为$2$或$3$时,我们可以用熟悉的平面直角坐标系与空间直角坐标系来描述欧氏空间中的向量,并用点积作为向量的内积.
\section{欧氏空间上的运算}
我们也已经基本掌握了模、内积、夹角等在内积空间中的基本概念,在此我们引入一些在先前的学习中接触较少的概念.
\begin{definition}
    \keyterm{点积}[dot product]是在三维欧氏空间中对两个向量的运算,用$\vec{a}\cdot\vec{b}$表示.两向量点积得到的数值等于两向量模长的乘积与两向量夹角的余弦的乘积.
\end{definition}
特别的,三维欧氏空间中的向量点积$(a_1,a_2,a_3)\cdot(b_1,b_2,b_3)$可以表示为$$a_1b_1+a_2b_2+a_3b_3$$
由点积的计算,我们可以很方便地得到两向量夹角的余弦,即$$\cos\theta=\frac{\vec{a}\cdot\vec{b}}{|\vec{a}||\vec{b}|}$$
\begin{definition}
    \keyterm{叉乘}[cross product]是在三维欧氏空间中对两个向量的运算,用$\vec{a}\times\vec{b}$表示.两向量叉乘得到的向量垂直于两向量,方向遵循右手定则,其模长为两向量的模的乘积与两向量夹角的正弦的乘积.
\end{definition}
由定义可知,叉乘仅在三维欧氏空间中有定义,且叉乘的结果是一个向量,而不是一个数.
关于叉乘向量的计算有另一种更常用的用行列式表示的计算方法,即
$$(a_1,a_2,a_3)\times(b_1,b_2,b_3)=\begin{vmatrix}
    \vec{i}&\vec{j}&\vec{k}\\
    a_1&a_2&a_3\\
    b_1&b_2&b_3
\end{vmatrix}$$
其中$i$,$j$,$k$为三维欧氏空间的自然基.

在解析几何中,叉乘的一个重要应用是求解与两向量垂直的向量.
\begin{definition}[向量的混合积]
    \keyterm{混合积}[mixed product]是三维欧氏空间中对三个向量的运算,用$[\vec{a},\vec{b},\vec{c}]$表示,等价于$(\vec{a}\times\vec{b})\cdot\vec{c}$.
\end{definition}
混合积的几何意义是以$\vec{a}$、$\vec{b}$和$\vec{c}$为邻边的平行六面体的体积,可以用行列式表示为
$$[(a_1,a_2,a_3),(b_1,b_2,b_3),(c_1,c_2,c_3)]=\begin{vmatrix}
    a_1&a_2&a_3\\
    b_1&b_2&b_3\\
    c_1&c_2&c_3
\end{vmatrix}$$
其应用之一是可以用来判断三个向量是否共面.
\section{点、直线、平面的表示}
一个点在欧氏空间中可以用一个向量来表示.在三维欧氏空间中,我们可以用三个实数来表示一个点的坐标.

\subsection{平面的方程}
平面是欧氏空间中的一个基本几何对象,我们有多种代数方法来表示平面.

平面的一般方程是平面的一种最基本的表示方法,即$Ax+By+Cz+D=0$.
平面的一般方程十分简洁,但是我们很难由此方程得到平面的几何性质,因此我们还需要考虑其他的表示方法.
例如,一个平面由平面上一点与平面上两个不共线的向量来表示.假设已知平面上一点$P(x_0,y_0,z_0)$和平面上两个不共线的向量$\vec{u}=(a,b,c)$和$\vec{v}=(d,e,f)$,则平面上的任意一点$Q(x,y,z)$都满足$\vec{PQ}$与$\vec{u}$和$\vec{v}$线性相关,即
$$\vec{PQ}=k_1\vec{u}+k_2\vec{v}$$
化为坐标形式即为
$$\begin{cases}
    x=x_0+k_1a+k_2d\\
    y=y_0+k_1b+k_2e\\
    z=z_0+k_1c+k_2f
\end{cases}$$
这就是平面的参数方程,其中$k_1$、$k_2$是参数.

此外,平面还可以由平面上一点和平面的法向量来表示.假设已知平面上一点$P(x_0,y_0,z_0)$和平面的法向量$\vec{n}=(A,B,C)$,则平面上的任意一点$Q(x,y,z)$都满足向量$\vec{PQ}$与$\vec{n}$垂直,即点积为$0$.
由此可得其方程为$$A(x-x_0)+B(y-y_0)+C(z-z_0)=0$$这种表示方法称为\keyterm{点法式}[point-normal form].

我们发现这跟平面的一般方程十分相似,实际上,我们可以直接通过平面的一般方程得到平面的法向量.

在得到一张由其他方式表示的平面时,我们往往也会将其转化为一般式或点法式,以便于我们计算其与其他几何对象的关系.
例如,得到一个由平面上一点与平面上两不共线的向量表示的平面,则可以通过对两个向量做叉乘运算得到平面的法向量,从而得到平面的点法式.

\begin{example}
    若已知一个平面上有三点$A(1,2,0)$,$B(0,1,-1)$,$C(1,1,1)$,求该平面的一般方程.
\end{example}

\subsection{直线的方程}
直线在欧氏空间中也是一个基本对象,同样有多种代数方法可以表示直线.

首先直线可以用某两张平面的交表示.假设有两相交平面的方程,联立可得直线方程
$$\begin{cases}
    A_1x+B_1y+C_1z+D_1=0\\
    A_2x+B_2y+C_2z+D_2=0
\end{cases}$$
即为直线的一般方程.这种联立方程的表示方法最为基本,但是不够简洁,大多情况下也不够直观.
所以更多情况下我们希望在表示中可以直观体现直线的一些特征.因此,可以用直线上的一个点和直线的方向(即方向向量)来确定一条直线.

假设已知直线上的一点$A_0(x_0,y_0,z_0)$和直线的方向向量$\vec{l}(a,b,c)$,则直线上的任意一点$A(x,y,z)$都满足$\vec{AA_0}$与$\vec{l}$平行,用具体的方程则表示为
$$\frac{x-x_0}{a}=\frac{y-y_0}{b}=\frac{z-z_0}{c}$$
其中$a$,$b$,$c$不为零.这种表示方法称为\keyterm{点向式}[point-direction form].

如果我们对上述式子进行替换,令$$t=\frac{x-x_0}{a}=\frac{y-y_0}{b}=\frac{z-z_0}{c}$$
则可得
$$\begin{cases}
    x=x_0+at\\
    y=y_0+bt\\
    z=z_0+ct
\end{cases}$$
这样就得到了直线的参数方程,其中$t$为参数.

当然还有以两点确定一条直线的表示方法,我们可以轻松地算出直线的方向向量,然后用点向式或参数方程来表示.最后可以得出方程
$$\frac{x-x_1}{x_2-x_1}=\frac{y-y_1}{y_2-y_1}=\frac{z-z_1}{z_2-z_1}$$

那么如何实现从一般方程到点向式或参数方程的转换呢?最简单的方法是求解线性方程组再用两点表示或者参数表示,但是这样的方法比较麻烦,事实上我们可以利用法向量进行转换.
假设两平面的一般方程为$A_1x+B_1y+C_1z+D_1=0$与$A_2x+B_2y+C_2z+D_2=0$,则可以得到两平面的法向量分别为$\vec{n_1}=(A_1,B_1,C_1)$、$\vec{n_2}=(A_2,B_2,C_2)$,
因为该直线在两张平面内,所以直线与两个法向量都垂直,所以$\vec{n_1}\times\vec{n_2}$即为直线的方向向量.再求出一般方程的一个解(即直线上一点)即可得到直线的点向式与参数方程.

\section{平面与直线间的位置关系}
对于三维欧氏空间中的几何对象,我们主要需要研究平行、相交与重合等关系.我们可以通过平面与直线的方程来判断.
\subsection{线与线的位置关系}
线与线之间的位置关系判断主要依靠它们的方向向量.如果两条直线的方向向量平行,则两条直线平行或重合,此时再判断两直线是否存在公共点,若联立方程有解,说明两直线重合,否则两条直线平行.
如果两条直线的方向向量不平行,则还需要判断两条直线是否共面,若共面则说明两条直线相交,否则两条直线异面.此时以两直线方程联立方程组,若有解则说明存在交点,否则说明两条直线异面.

\begin{example}
    已知直线$L_1=\begin{cases}
        x+y+z-1=0\\
        x-2y+2=0
    \end{cases}$,$L_2=\begin{cases}
        x=2t\\
        y=t+a\\
        z=bt+1
    \end{cases}$,试确定$a$,$b$的值使得$L_1$,$L_2$是:
    \begin{enumerate}
        \item 平行直线
        \item 异面直线
    \end{enumerate}
\end{example}

\subsection{线与面的位置关系}
线与面的位置关系首先需要判断线的方向向量与平面的法向量的关系.如果方向向量与法向量平行,则说明线与面垂直.
如果两者垂直,则说明该直线与平面平行或者在平面内,只需再判断直线上的点是否在平面内即可.

此外还有一些对于平面不同表示形式的方法.例如,假设已知直线的方向向量与平面上两个不平行的向量,则可以对这三个向量做混合积,如果混合积为零,则说明三个向量共面,即直线与平面平行或者在平面内.

\subsection{面与面的位置关系}
面与面的位置关系主要依靠两个平面的法向量来判断.如果两个平面的法向量平行,则说明两个平面平行或重合,再判断两平面是否存在公共点.
若两法向量垂直,则两平面也垂直.
\vspace{2ex}
\centerline{\heiti \Large 内容总结}
这里关于解析几何的部分浅尝辄止,只是简单地介绍了一些基本的概念与方法,希望能够帮助大家对解析几何有一个简单的初步认识.
在线性代数课程中可能的相关考察基本也仅限于点、线、面之间的关系,方程的联立、求解等等,或许大家在未来其他课程的学习中可以学到更多相关的知识.
\vspace{2ex}

\centerline{\heiti \Large 习题}
\vspace{2ex}
{\kaishu }
\begin{flushright}
    \kaishu

\end{flushright}
\centerline{\heiti A组}
\begin{enumerate}
    \item
\end{enumerate}
\centerline{\heiti B组}
\begin{enumerate}
    \item
\end{enumerate}
\centerline{\heiti C组}
\begin{enumerate}
    \item
\end{enumerate}

\chapter{线性代数与多元微积分}

\section{向量函数的导数}

\section{行列式的导数}

\section{雅可比行列式}

\vspace{2ex}
\centerline{\heiti \Large 内容总结}

\vspace{2ex}

\centerline{\heiti \Large 习题}
\vspace{2ex}
{\kaishu }
\begin{flushright}
    \kaishu

\end{flushright}
\centerline{\heiti A组}
\begin{enumerate}
    \item
\end{enumerate}
\centerline{\heiti B组}
\begin{enumerate}
    \item
\end{enumerate}
\centerline{\heiti C组}
\begin{enumerate}
    \item
\end{enumerate}

\chapter{线性代数与统计学}

在统计学中,当我们研究多个随机变量之间的相关关系时,我们将会见到大量熟悉的线性代数知识.
本节的目标便是希望选取几个经典且基本的统计学中使用线性代数中概念与方法的例子帮助读者
在学习统计学的过程中看见线性代数不会感到陌生.

\section{多元正态分布}


\section{马尔科夫链}
最后我们介绍随机过程中运用线性代数的重要的例子——马尔科夫链.本小节使用线性代数的角度
不同于前面小节侧重于二次型等方面,我们将会探讨

\vspace{2ex}
\centerline{\heiti \Large 内容总结}

\vspace{2ex}

\centerline{\heiti \Large 习题}
\vspace{2ex}
{\kaishu 在终极的分析中,一切知识都是历史;在抽象的意义下,一切科学都是数学;
在理性的基础上,所有的判断都是统计学.}
\begin{flushright}
    \kaishu
    ——C.R.Rao,《统计与真理》
\end{flushright}
\centerline{\heiti A组}
\begin{enumerate}
    \item
\end{enumerate}
\centerline{\heiti B组}
\begin{enumerate}
    \item
\end{enumerate}
\centerline{\heiti C组}
\begin{enumerate}
    \item
\end{enumerate}


\backmatter
\pdfbookmark[0]{索引}{index}
\printindex

\end{document}
