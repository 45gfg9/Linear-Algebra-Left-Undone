\chapter{多项式的进一步讨论}

\section{特征多项式\ 哈密顿-凯莱定理}
接下来的内容我们希望讨论一类特殊的多项式,将算子代入后多项式值为0. 这样的多项式我们称为算子的
\textbf{零化多项式},讨论这一内容可以让我们从其他角度进一步理解矩阵标准形.
\begin{definition}
	设$T\in L(V)$,若$p\in\mathcal{P}(\mathbf{F})$使得$p(T)=0$,则称$p$为$T$的一个零化多项式.
\end{definition}
矩阵的零化多项式有类似的定义(后面多项式都有类似的定义,将算子替换为矩阵即可,并且基于算子的定理都可以推广到算子
对应的矩阵).接下来我们讨论一类多项式,我们在第六章实际上已经提到过,但我们这里给出一种新的理解:
\begin{definition}
	设$V$是复向量空间,$T\in L(V)$.令$\lambda_1,\cdots,\lambda_m$表示$T$的所有互异本征值,
	重数分别为$d_1,\cdots,d_m$,则多项式$$(z-\lambda_1)^{d_1}\cdots(z-\lambda_m)^{d_m}$$
	称为$T$的特征多项式.
\end{definition}
根据这一定义,$T$的特征多项式的次数为$\dim V$(由定理\ref{广义本征性质}(2)可得),
并且其零点恰为$T$的全部本征值.回忆定理\ref{特征多项式1}对特征多项式的定义,$f(\lambda)=|\lambda I-A|$,
且多项式的$k$重根即为$k$重本征值,本征值的重数也就称为代数重数.
\begin{example}
	设$V$是复向量空间,$V_1,\cdots,V_m$都是$V$的非零子空间使得$V=V_1\oplus\cdots\oplus V_m$.设$T\in L(V)$,每个
	$V_j$在$T$下不变.对每个$j$,令$p_j$表示$T|_{V_j}$的多项式.证明:$T$的特征多项式为$p_1\cdots p_m$.
\end{example}
实际上,根据定理\ref{多项式分解},我们知道这两个特征多项式的定义统一等价于两个代数重数的定义统一.教材在
10.25中直接证明了特征多项式的统一,然后我们可以得出代数重数定义统一,即广义本征空间重数等于本征值作为特征多项式
$|\lambda I-A|=0$的零点重数.

在上面的讨论中我们依据广义本征空间的维数定义特征多项式,接下来我们希望变换思路,利用第六章中的特征多项式的定义
推导广义本征空间的相关结论,这一正一逆的思路在某种程度上也体现出两个研究体系的等价性.

下面我们引入哈密顿-凯莱定理,我们的动机是获得零化多项式.观察矩阵$A=\begin{pmatrix}
	1 & 2 \\ 0 & -1
\end{pmatrix}$,我们容易验证$A^2-I=0$,因此$\lambda^2-1$是$A$的一个零化多项式,同时我们发现这是
$A$的特征多项式,因此我们可以猜想,是否对于所有的矩阵都有特征多项式是零化多项式的结论,事实上,这就是
著名的\textbf{哈密顿-凯莱定理}:
\begin{theorem}
	设$V$是复向量空间,$T\in L(V)$.令$q$表示$T$的特征多项式,则$q(T)=0$.
\end{theorem}
基于done right的体系,我们可以很轻松地利用广义本征空间分解证明这一定理.一般的高等代数教材引入$\lambda$矩阵
等不直观的方式证明,十分繁杂,这也体现了done right这一思路的优越性.

这一定理在解题时可能的应用在于求解一些矩阵多项式/求矩阵的逆/求矩阵的幂,都是偏向于技术性的,因此我们不在此展开.

接下来我们利用这一定理继续我们逆向推导的过程.我们的目标同样是找到能在直和后得到原空间
的不变子空间的定义方式.实际上我们可以参照以下结论:
\begin{theorem}\label{多项式分解与零空间直和}
	设$T\in L(V)$,且在$\mathcal{P}(\mathbf{F})$中有$p=p_1p_2$,且$(p_1,p_2)=1$,则有
	$$\textup{null }p(T)=\textup{null }p_1(T)\oplus\textup{null }p_2(T).$$
\end{theorem}
定理的证明我们留作习题,需要基于定理\ref{裴蜀定理}.我们需要将这一定理推广到因式更多的情况,
证明只需要依照\ref{多项式分解与零空间直和}然后进行数学归纳法即可:
\begin{theorem}\label{多项式分解与零空间直和2}
	设$T\in L(V)$,且在$\mathcal{P}(\mathbf{F})$中有$p=p_1p_2\cdots p_s$,且$p_1,p_2,\cdots,p_s$两两互素,
	则有$$\textup{null }p(T)=\textup{null }p_1(T)\oplus\textup{null }p_2(T)\oplus\cdots\oplus\textup{null }p_s(T).$$
\end{theorem}
这一定理表明,将多项式分解为互素的多项式乘积,原多项式作用于算子的零空间等于分解后各个互素因式作用于算子的零空间的直和.
我们结合哈密顿-凯莱定理,如果$p$是$T$的特征多项式,故$p(T)=0$,则$\textup{null }p(T)$就是全空间$V$.接下来我们对第六章中
定义的特征多项式分解为互素因式乘积,有
$$p(\lambda)=|\lambda I-A|=(\lambda-\lambda_1)^{r_1}(\lambda-\lambda_2)^{r_2}\cdots(\lambda-\lambda_m)^{r_m},$$
其中$A$为$T$在某组基下的矩阵.能进行这样的分解的原因在于第六章中结论告诉我们特征多项式是首一的,且零点就是算子本征值,
根据定理\ref{多项式分解}可以知道这样的分解存在且$\lambda_1,\cdots,\lambda_m$为$T$的所有互异本征值,$r_1,\cdots,r_m$为本征值
作为特征多项式零点的重数.然后由于分解的因式显然是两两互素的,因此根据定理\ref{多项式分解与零空间直和2},我们有
$$\textup{null }p(T)=V=\textup{null }(T-\lambda_1I)^{r_1}\oplus\cdots\oplus\textup{null }(T-\lambda_mI)^{r_m},$$
这或许就是一种巧合,我们从多项式的角度也推导出了和广义本征空间相近的结论,并且这里将广义本征空间定义中$(T-\lambda I)$的幂次
降低了.除此之外,$\textup{null }(T-\lambda_iI)^{r_i}(i=1,2,\cdots,m)$也是$T$的不变子空间,因此我们也能得到分块对角矩阵的标准形,
并且定理\ref{广义本征性质}的其他结论也可以基于此得到,此处不再赘述.

总结一下上述两种推导思路,其核心都在于寻找合适的不变子空间的直和分解,使得算子在直和分解获得的基下的矩阵是分块对角矩阵.
done right的思想基于从本征空间到广义本征空间的自然扩张(基本原理是零空间增长),这保持了不变子空间的性质并且使得所有算子
都有这样的直和分解,由此进一步得到哈密顿-凯莱定理.一般高等代数的思想在于首先引入$\lambda$矩阵理论得到哈密顿-凯莱定理,然后基于
定理\ref{多项式分解与零空间直和2}的直和分解得到合适的不变子空间直和分解.我们可以看出,done right的思路相对而言更加直观,但其缺陷
也是显然的,很多具象的结论以及计算方式都在一般高等代数的思路中相对更为直接.所以done right向我们展示了高等代数中核心结论如何用更
符合直觉的方式得到,实际应用时原先等价的技术手段很多时候仍是不可或缺的.
\begin{example}
	设$T\in L(V)$,$p(z)=a_nx^n+\cdots+a_1x\in\mathcal{P}(\mathbf{F})$是$T$的一个零化多项式,其中$a_1\neq 0$,证明:
	$$V=\textup{null }T\oplus\textup{range }T.$$
\end{example}
下面我们希望将广义本征空间定义中$(T-\lambda I)$的幂次进一步降低到下界(即找到最低的幂次使得零空间停止增长),这需要引入极小多项式
的概念.

\section{极小多项式及其性质}
\begin{definition}
	设$T\in L(V)$,则$T$的极小多项式是唯一一个使得$p(T)=0$的次数最小的首一多项式.
\end{definition}
这一定义的合理性需要下述定理保证:
\begin{theorem}
	设$T\in L(V)$,则存在唯一一个次数最小的首一多项式$p$使得$p(T)=0$.
\end{theorem}
事实上,根据$1,T,T^2,\cdots,T^{(\dim V)^2}$的线性相关性(因为$L(V)$空间是$(\dim V)^2$维的)可以知道极小多项式的最低次数最多为
$(\dim V)^2$,这一点早在第二章中就有涉及.而根据哈密顿-凯莱定理,特征多项式$p$是$\dim V$次的且$p(T)=0$,故极小多项式的最高次数的
上界进一步下降到$\dim V$.以上讨论都可以证明上述定理的存在性,唯一性的证明只需利用基于多项式带余除法的套路即可(这种基于次数的唯一性
证明是常见的,请务必掌握).

如果需要计算极小多项式,我们可以给出一个算法化的描述.对于$m=1,2,\cdots$,我们相继考虑线性方程组
$$a_0M(I)+a_1M(T)+\cdots+a_{m-1}M(T^{m-1})+M(T^m)=0,$$
直到这一方程组有一个解$a_0,a_1,\cdots,a_{m-1}$,此时$a_0,a_1,\cdots,a_{m-1},1$即为极小多项式的次数.
\begin{example}\label{求极小多项式}
	求矩阵$A=\begin{pmatrix}
		0 & 0 & 0 \\ 1 & 0 & 2 \\ 2 & 1 & -1
	\end{pmatrix}$和$B=\begin{pmatrix}
		2 & 2 & 1 \\ 0 & 2 & -1 \\ 0 & 0 & -3
	\end{pmatrix}$的最小多项式.
\end{example}
下面我们给出一些简单算子/矩阵的极小多项式:

1. 幂零算子:$N\in L(V)$且$N^l=0$,但$N^{l-1}\neq 0$($l$称为幂零指数),极小多项式为$\lambda^l$;

2. 幂等算子:$T\in L(V)$且$T^2=T$,极小多项式为$\lambda^2-\lambda$或$\lambda$或$\lambda-1$;

3. 对合算子:$T\in L(V)$且$T^2=I$,极小多项式为$\lambda^2-1$或$\lambda+1$或$\lambda-1$;

4. 引入\textbf{若当块}.若域$\mathbf{F}$上的一个$r$级矩阵形如$$\begin{pmatrix}
	a & 1 & 0 & \cdots & 0 & 0 \\
	0 & a & 1 & \cdots & 0 & 0 \\
	\vdots & \vdots & \vdots &  & \vdots & \vdots \\
	0 & 0 & 0 & \cdots & a & 1 \\
	0 & 0 & 0 & \cdots & 0 & a
\end{pmatrix},$$
则称其为一个$r$级若当块(1级显然就是1阶矩阵),记作$J_r(a)$,其中$a$是对角线上元素.不难得到其
极小多项式等于特征多项式$(\lambda-a)^r$.

我们利用多项式的带余除法以及哈密顿-凯莱定理可以得到下述简单的结论:
\begin{theorem}
	设$T\in L(V)$.
	
	\textup{(1)}$q\in\mathcal{P}(\mathbf{F})$,则$q(T)=0$当且仅当$q$是$T$的极小多项式的多项式倍;

	\textup{(2)}设$\mathbf{F}=\mathbf{C}$,则$T$的特征多项式是$T$的极小多项式的多项式倍.
\end{theorem}
在例\ref{求极小多项式}中我们不难发现,两个矩阵的极小多项式和特征多项式根一致,实际上这是对任意算子(或矩阵)
都成立的结论:
\begin{theorem}\label{极小与特征根相同}
	设$T\in L(V)$,则$T$的极小多项式的零点恰好是$T$的本征值,即极小多项式与特征多项式在$\mathbf{F}$中有
	相同的根(重数可以不同).
\end{theorem}
证明这一定理注意要证明两个方向,其一是极小多项式的零点都是特征多项式的零点,这只需要基于特征多项式是极小多项式的倍式;
另一方面是特征多项式的零点(本征值)都是极小多项式的零点,这也是简单的,只需要利用极小多项式是零化多项式以及本征值的基本性质即可.

这一定理是非常重要的,它关系到下一小节关于多项式和标准形关系的讨论.并且例题\ref{求极小多项式}也可以基于此有更快的解法.
除此之外,我们还可以得到一个推论,即相似的矩阵有相同的极小多项式.因为上面的结论基于算子,相似矩阵对应的算子是一致的.

\section{多项式与标准形的应用}
在最后一小节我们尝试将两种描述算子的角度(标准形和多项式)联系起来,主要的桥梁就是上一小节中讨论的极小多项式.在前文讨论
特征多项式诱导的不变子空间分解时,我们将广义本征空间定义中需要求零空间的算子幂次降低,而依据\ref{极小与特征根相同}以及
特征多项式是极小多项式的倍式可知,这一幂次还可以进一步降低:
\begin{theorem}\label{极小多项式与分解}
	设$T\in L(V)$,$T$的极小多项式为$p=(\lambda-\lambda_1)^{s_1}\cdots(\lambda-\lambda_m)^{s_m}$,则有
	$$\textup{null }p(T)=V=\textup{null }(T-\lambda_1I)^{s_1}\oplus\cdots\oplus\textup{null }(T-\lambda_mI)^{s_m}.$$
\end{theorem}
并且我们知道,极小多项式的因式次数无法继续降低,否则不为零化多项式,因此它也给出了广义本征空间定义中需要求零空间的算子的幂次为
何值时,零空间会停止增长,并且这是一个下界,基于此我们更进一步地理解了极小多项式因式次数的含义.

实际上我们也可以逆向思考,如果我们已知空间的不变子空间分解,我们应当如何求解极小多项式.实际上这一结论是很直观的,答案是各个不变子空间
的极小多项式的最小公倍式.证明我们留作习题,基本思想是利用限制算子说明全空间上的极小多项式是各个不变子空间的极小多项式的倍式.
这一结论的应用或许并不直接,但如果我们考虑算子在不变子空间直和分解下的分块对角矩阵,那么这一分块对角矩阵的极小多项式实际上就等于
各个分块的极小多项式的最小公倍式.
\begin{example}\label{若当形矩阵极小多项式}
	我们在此继续引入\textbf{若当形矩阵},即由若干个若当块组成的分块对角矩阵.设$A$为若当形矩阵,
	$A=\textup{diag}(J_{r_1}(a),\cdots,J_{r_s}(a),J_{t_1}(b),\cdots,J_{t_m}(b))$,其中$r_1\le\cdots\le r_s$,
	$t_1\le\cdots\le t_m$,则$A$的极小多项式$p$为$(\lambda-a)^{r_1},\cdots,(\lambda-a)^{r_s},(\lambda-b)^{t_1},(\lambda-b)^{t_m}$
	的最小公倍式,即为$(\lambda-a)^{r_s}(\lambda-b)^{t_m}$.实际上,这一结论还可以进一步推广,但描述较为繁杂,读者只需从此例理解基本思想即可.
\end{example}
在定理\ref{极小多项式与分解}中我们了解了极小多项式中因子幂次与广义本征空间的关联.加入极小多项式的各个因式的次数均为1,这与可对角化算子的不变子空间
分解是一致的!因此我们可以得到下面的结论:
\begin{theorem}
	设$T\in L(V)$,$T$可对角化当且仅当$T$的极小多项式能分解成不同的一次因式的乘积.
\end{theorem}
这给出了算子可对角化的另一等价条件,基于此,第六章中给出矩阵多项式判断可对角化的习题都可以“秒杀”,例如幂等矩阵、对合矩阵可对角化,但幂零矩阵除非自身
为0否则一定不可对角化,高于1阶的若当块矩阵一定不可对角化,包含高于1阶的若当块矩阵的若当形矩阵也一定不可对角化.

我们也可从矩阵的角度来理解.例如幂等矩阵$A$满足$A^2=A$,根据多项式诱导的不变子空间分解,我们很容易得到$A$为幂等矩阵的充要条件为$r(A)+r(A-E)=n$,
其它对合矩阵等情况各位同学也可以自己写出等价条件,虽然形式上可以千变万化,但实质就是多项式诱导的不变子空间分解.

除此之外,联系多项式互素分解与不变子空间分解的对应关系,这也表明算子可对角化当且仅当其各个广义本征空间就是其本征空间,即满足代数重数等于几何重数.
或者说$T$的每个广义本征向量都是其本征向量.
\begin{example}
	设$T\in L(V)$,若$T$可对角化,则对于$T$的任意非平凡不变子空间$U$,都有$T|_U$可对角化.
\end{example}
\begin{example}
	已知某个实对称矩阵$A$的特征多项式为$\lambda^5+3\lambda^4-6\lambda^3-10\lambda^2+21\lambda-9$,求$A$的极小多项式.
\end{example}
\begin{example}
	设$V$为$n$阶方阵构成的线性空间,$T\in L(V)$,$\forall A\in V, T(A)=2A-3A^{\mathrm{T}}$.

	\textup{(1)}求$T$的特征值;

	\textup{(2)}证明:$T$可对角化.
\end{example}
我们需要补充说明一点,虽然矩阵相似不随数域改变而改变,但可对角化与数域有关.例如实矩阵$A$的极小多项式为$\lambda^3-1$,在它在实数域上无法分解为互素
一次因式的乘积,复数域上则可以,这表明$A$在实数域上不可对角化,但在复数域上可以.

\vspace{2ex} 
\centerline{\heiti \Large 内容总结}

\vspace{2ex} 

\centerline{\heiti \Large 习题}
\vspace{2ex} 
{\kaishu }
\begin{flushright}
    \kaishu

\end{flushright}
\centerline{\heiti A组}
\begin{enumerate}
	\item 
\end{enumerate}
\centerline{\heiti B组}
\begin{enumerate}
	\item 
\end{enumerate}
\centerline{\heiti C组}
\begin{enumerate}
	\item 
\end{enumerate}