\chapter{内积空间上的算子(II)}

\section{正交矩阵和酉矩阵}

本节我们将唤醒一些沉睡的记忆,如果你已经忘了过渡矩阵或矩阵的相似,可以移步到前面的章节再回顾一下. 如果你还在这的话,那么坐稳,我们马上开始. 

\vspace{2ex}

\subsection{定义 \quad Schur 定理}

为了更好地引进正交矩阵和酉矩阵,我们有必要把共轭转置说的更清楚些. 
共轭转置有着以下的运算性质,虽然都是看起来很显然的事情,此处还是稍稍赘述一下:

设有矩阵 $ A $, $ B $ 和数 $ \lambda \in \mathbf{C}$,则

\begin{enumerate}    
    \item $ (\overline{A + B})^{\mathrm{T}} = \overline{A}^{\mathrm{T}} + \overline{B}^{\mathrm{T}} $;
    
    \item $ \overline{(AB)}^{\mathrm{T}} = \overline{B}^{\mathrm{T}} \overline{A}^{\mathrm{T}} $;
    
    \item $ (\overline{\lambda A})^{\mathrm{T}} = \overline{\lambda} \enspace \overline{A}^{\mathrm{T}} $;
    
    \item $ \overline{\overline{A}^{\mathrm{T}}}^{\mathrm{T}} = A $. 
\end{enumerate}

共轭转置说清楚后,便可以由此定义正交矩阵和酉矩阵. 

\begin{definition} \keyterm{酉矩阵} \keyterm{正交矩阵} 
    在复数域(实数域)上,矩阵 $ A $ 满足 $ \overline{A}^{\mathrm{T}} A = E $( $ {A}^{\mathrm{T}} A = E $ ),
    则矩阵 $ A $ 被称为酉矩阵(正交矩阵) 
\end{definition}

而如何刻画正交矩阵和酉矩阵的性质呢?下面的一个定理揭示了其与标准正交基的关系,可以从中窥得一些性质.  

\begin{theorem}
    设 $ (e_1, e_2, \ldots , e_n) $ 是复(实)内积空间 $ V $ 上的标准正交基,$ (f_1, f_2, \ldots , f_n) $ 是 $ V $ 上的一组基,
    从 $ (e_1, e_2, \ldots , e_n) $ 到 $ (f_1, f_2, \ldots , f_n) $ 的过渡矩阵为 $ A $. 则 $ (f_1, f_2, \ldots , f_n) $ 是标准正交基的
    充要条件是 $ A $ 为酉矩阵(正交矩阵). 
\end{theorem}

以下仅针对复内积空间的情况进行证明. 

\begin{proof}
    由过渡矩阵的定义,$ (f_1, f_2, \ldots , f_n) $ = $ (e_1, e_2, \ldots , e_n)A $,$ A = (a_{ij})_{n \times n} $. 

    由矩阵乘法的运算,可以得到
    \[ f_i = \sum_{j = 1}^{n} a_{ji}e_j , \enspace f_k = \sum_{j = 1}^{n} a_{jk}e_j. \]

    对两者做内积,有
    \[
    \langle f_i, f_k \rangle = \left\langle \sum_{j = 1}^{n} a_{ji}e_j, \sum_{j = 1}^{n} a_{jk}e_j \right\rangle
    = \sum_{j = 1}^{n} a_{ji}\overline{a_{jk}} 
    \]

    注意到 $ a_{ji}, j = 1, \ldots , n $ 是 $ A^{\mathrm{T}} $ 的第 $ i $ 行的元素,
    $ \overline{a_{jk}}, j = 1, \ldots , n $ 是 $ \overline{A} $ 的第 $ k $ 列的元素.
   
    定义 $ B = A^{\mathrm{T}}\overline{A} = (b_{ik})_{n \times n} $,则 $ \langle f_i, f_k \rangle = b_{ik} $. 

    必要性:如果 $ (f_1, f_2, \ldots , f_n) $ 是一组标准正交基,则
    \[
        b_{ik} = \langle f_i, f_k \rangle = 
        \begin{cases}
            1, & i = k \\
            0, & i \neq k 
        \end{cases}    
    \]

    由此可知 $ B = E $, $ \overline{B} = \overline{A}^{\mathrm{T}} A = \overline{E} = E $,即 $ A $ 是酉矩阵. 
    
    充分性:将必要性证明推理过程倒写即可. 

\end{proof}

如果这条定理中的 $ (e_1, e_2, \ldots , e_n) $ 取为该空间的自然基,就会有 $ (f_1, f_2, \ldots , f_n) = A $,我们便可以不太严谨地
得到如下的这个结论

\begin{theorem}
    矩阵 $ A $ 是酉矩阵(正交矩阵)等价于其列向量构成标准正交基. 
\end{theorem}

证明是平凡的,就交给你自己验证了. 

那提到了过渡矩阵,我们也就不得不提与之息息相关的一个等价关系——相似了. 相信你已经回忆起来,相似实际上是同一个算子在不同基下的矩阵表示之间的关系,
实现这个变化正是依赖于两组基之间的过渡矩阵,那么在内积空间上,我们最想看的就是一个算子在标准正交基下的矩阵表示形式能简化成什么样子,这就是内积空间上算子表示的主线. 

在此之前,让我们定义两个特殊一点的相似关系:

\begin{definition}
    \begin{enumerate}

        \item \keyterm{酉相似}:复内积空间上,若 $ B = P^{-1}AP = \overline{P}^{\mathrm{T}}AP $,
        则称矩阵 $ A $ 与矩阵 $ B $ 酉相似. 

        \item \keyterm{正交相似}:实内积空间上,若 $ B = P^{-1}AP = {P}^{\mathrm{T}}AP $,
        则称矩阵 $ A $ 与矩阵 $ B $ 正交相似. 
    \end{enumerate}
\end{definition}

有了两个相似关系,我们就可以开始我们简化算子在标准正交基上的矩阵表示的第一步了. 

\begin{theorem} \keyterm{Schur 定理} 
    设 $ V $ 是有限维的复内积空间,且 $ T \in \mathcal{L}(V) $,则 $ T $ 关于 $ V $ 的某个标准正交基具有上三角矩阵. 
\end{theorem}

证明并不复杂,只需要结合\ref{任何算子都关于其一组基有上三角矩阵}和 Gram-Schmidt 过程即可. % TODO 引用

\subsection{等距同构}

虽然我们成功定义了正交矩阵和酉矩阵,但这个定义总感觉怪怪的?它仅仅描述了满足如此性质的矩阵是正交矩阵和酉矩阵,
但似乎并没有触及到更深层、更本质的事物,这显然是不能让我们满意的. 因此,我们决定回到与之对应的算子上去看一看. 

那么由之前一章,我们知道,算子与其伴随在同一组基下的矩阵表示是互为共轭对称的,所以设对应的算子是 $ S $,
则其应该满足 $ S^*S = I $. 那么这个性质能将我们导向何处呢?

考虑两侧同时作用向量 $ u $,再与 向量 $ v $ 做内积,那么我们得到了如下的式子:
\[ \langle S^*Su, v \rangle = \langle u, v \rangle. \]
再结合伴随的定义,稍微做个变换,就有了下面这个美妙的结果:
\[ \langle Su, Sv \rangle = \langle u, v \rangle. \]
也就是说,这个算子 $ S $ 同时作用在两个向量上的话不改变它们的内积. 
更进一步的话,如果取 $ v = u $,我们就能得到最终的结果:
\[ \lVert Su \rVert = \lVert u \rVert \]
算子 $ S $ 保持范数. 

\begin{definition}
    \keyterm{等距同构} 算子 $ S \in \mathcal{L}(V) $ 称为等距同构,如果 $ \forall v \in V $,
    都有 $ \lVert Su \rVert = \lVert u \rVert $.  
\end{definition}

注意我们这里虽然使用了共轭对称,但是从伴随的角度上来说复内积空间和实内积空间其实是一样的,
也就是说等距同构的概念在这两类空间上是一致的,只不过刻画上会有所差距,之后会有所介绍. 此外,
也常称实内积空间上的等距同构为正交算子,复内积空间上的等距同构称为酉算子. 

让我们看道简单的例题加深一下对等距同构的印象. 
\begin{example}
    设 $ \lambda_1, \ldots , \lambda_n $ 都是模为 1 的标量,
    $ e_1, \ldots , e_n $ 是 $ V $ 的标准正交基,$ S \in \mathcal{L}(V) $ 
    满足 $ Se_j = \lambda_je_j $,证明 $ S $ 是等距同构. 
\end{example}

\section{正定矩阵}

\vspace{2ex}
\centerline{\heiti \Large 内容总结}

\vspace{2ex}

\centerline{\heiti \Large 习题}
\vspace{2ex}
{\kaishu }
\begin{flushright}
    \kaishu

\end{flushright}
\centerline{\heiti A组}
\begin{enumerate}
    \item
\end{enumerate}
\centerline{\heiti B组}
\begin{enumerate}
    \item
\end{enumerate}
\centerline{\heiti C组}
\begin{enumerate}
    \item
\end{enumerate}
