\chapter{线性空间的运算}

在前述章节中我们对(有限维)线性空间中的基本概念以及研究的基本问题进行了了解.事实上,很多时候我们
还需要研究不同线性空间进行运算后得到的新集合的性质,本节我们将详细展开讨论这一问题.

\section{线性空间的交、并、和}
\begin{definition}
    设$W_1,W_2$是线性空间$V(\mathbf{F})$的两个子空间,则
    \begin{align*}
    W_1 \cap W_2&=\{\alpha \mid \alpha\in W_1 \text{ 且 } \alpha\in W_2\} \\
    W_1 \cup W_2&=\{\alpha \mid \alpha\in W_1 \text{ 或 } \alpha\in W_2\} \\
    W_1 + W_2&=\{\alpha_1+\alpha_2 \mid \alpha_1\in W_1,\enspace\alpha_2\in W_2\}
    \end{align*}
    分别称为$W_1$和$W_2$的交、并、和.
\end{definition}

交与并的定义实际上与集合交与并的定义类似,而和的定义可能有些许反直觉.我们可以通过一个例子来体会
为什么要定义子空间的和.
\begin{example}\label{ex:4:子空间运算}
    在$\mathbf{R}^3$中,我们设
    \[\vec{\alpha_1}=(0,0,1),\vec{\alpha_2}=(0,1,0),\vec{\alpha_3}=(1,0,0).\]
    令$\mathbf{R}^3$子空间$W_1=\spa(\vec{\alpha_1},\vec{\alpha_2})$,
    $W_2=\spa(\vec{\alpha_1},\vec{\alpha_3})$,
    则$W_1$实际上是$yOz$平面,$W_2$是$xOz$平面,因此我们根据交与并的概念(实际上就是集合取交集
    和并集)得到$W_1 \cap W_2=\spa(\vec{\alpha_1})$(即$z$坐标轴).
    
    进一步考察并集,事实上显然$W_1 \cup W_2$得到的集合表示$xOz$和$yOz$平面上
    所有的点.事实上我们发现,$W_1 \cup W_2$得到的集合关于向量加法、数乘运算并不封闭,例如只需取
    $\vec{\alpha_2}+\vec{\alpha_3}=(0,1,1)$就不在$W_1 \cup W_2$中,因此不再是$\mathbf{R}^3$的子空间.
    
    接下来我们考察二者之和.事实上$W_1+W_2=\mathbf{R}^3$.原因在于
    \begin{enumerate}
        \item $\forall \vec{\beta}\in W_1 + W_2$,由子空间和的定义可知有$\vec{\beta}=\vec{\beta_1}+\vec{\beta_2}$,
        其中$\vec{\beta_1}\in W_1\subset \mathbf{R}^3$,$\vec{\beta_2}\in W_2\subset \mathbf{R}^3$,
        由于$\mathbf{R}^3$是线性空间,其中元素关于加法运算封闭,因此
        $\vec{\beta}=\vec{\beta_1}+\vec{\beta_2}\in \mathbf{R}^3$,即$W_1+W_2\subset \mathbf{R}^3$;

        \item $\mathbf{R}^3$中任一向量$(x,y,z)$总能写成$(x,y,z)=(0,y,z)+(x,0,0)$的形式,其中$(0,y,z)$在$W_1$中,
        $(x,0,0)$在$W_2$中,因此根据子空间和的定义可知$\mathbf{R}^3\subset W_1 + W_2$成立.
    \end{enumerate}
    综上,我们得到$W_1+W_2=\mathbf{R}^3$.
\end{example}

从上面证明$W_1+W_2=\mathbf{R}^3$的过程中我们可以提炼出证明子空间的和等于某一空间的一般方法:本质而言仍然是证明
集合相等,因此证明两边包含即可.证明子空间的和属于某一空间是平凡的,如上述证明的第一部分;第二部分证明某一空间属于子空间和
只需要将该空间中任意向量都可分解为各个子空间中向量的和即可.

事实上,根据\autoref{ex:4:子空间运算}我们发现,子空间$W_1$和$W_2$的交与和仍然是线性空间,但是它们的并不是线性空间.
事实上,我们可以证明如下定理:
\begin{theorem}
    设$W_1,W_2$是线性空间$V(\mathbf{F})$的两个子空间,则
    \begin{enumerate}
        \item $W_1 \cap W_2$是$V$的子空间;
        \item $W_1 + W_2$是$V$的子空间;
        \item $W_1 \cup W_2$为$V\text{ 的子空间 } \iff W_1 \subseteq W_2$或$W_2 \subseteq W_1 \iff W_1 \cup W_2=W_1+W_2$.
    \end{enumerate}
\end{theorem}

定理前两条的证明见教材74页,第三条我们留作习题供读者练习,因为在考试中有出现过.
前两条还可以进行推广,即$V$的有限个子空间的交与和仍然是$V$的子空间.

除此之外,这一定理也告诉我们为什么需要研究子空间的和而更少研究子空间的并:因为子空间的和仍然是线性空间.直观理解实际上就是
和的定义中出现了两个子空间的向量的加法,而构成子空间的核心就是运算封闭,因此这一定义为子空间的和仍构成子空间提供了保证,
因此这一定义也是十分自然的.

\section{覆盖定理}
\begin{theorem}\label{thm:4:覆盖定理}
    设$V_1,V_2,\ldots,V_s$是线性空间$V$的$s$个非平凡子空间,证明:$V$中至少存在一个向量
    不属于$V_1,V_2,\ldots,V_s$中的任何一个,即$V_1 \cup V_2 \cup \cdots \cup V_s\subsetneq V.$
\end{theorem}

这一定理称为覆盖定理,它表明任何一个线性空间都不能被自身有限个非平凡子空间通过并得到.初看可能有些不够自然,
但我们可以从简单的几何意义获得直观的理解:有限条直线的并不可能是一个平面.下面我们利用数学归纳法进行证明.

\begin{proof}
    \begin{enumerate}
        \item 当$s=2$时,由于$V_1,V_2$是非平凡子空间,因此$V$中存在$\vec{\alpha}\notin V_1$,若$\vec{\alpha}\notin V_2$,
        则结论已经成立.若$\vec{\alpha}\in V_2$,由$V_2$非平凡知存在$\vec{\beta}\notin V_2$,我们考虑$\vec{\alpha}+\vec{\beta}$
        和$2\vec{\alpha}+\vec{\beta}$,则必有这两个向量都不属于$V_2$(否则有$\beta\in V_1$),并且这两个向量也不能同时属于$V_1$
        (否则两个向量相减等于$\vec{\alpha}$也属于$V_1$,矛盾),这就说明这两个向量中至少有一个既不在$V_1$中也不在$V_2$中,因此结论成立.
        \item 对于$s>2$,假设命题对$s-1$个子空间成立,即$V$中存在向量$\vec{\alpha}\notin V_1\cup V_2\cup\ldots\cup V_{s-1}$,
        若$\vec{\alpha}\notin V_s$,则结论成立.若$\vec{\alpha}\in V_s$,由$V_s$非平凡知存在$\vec{\beta}\notin V_s$,我们考虑
        $\vec{\alpha}+\vec{\beta},2\vec{\alpha}+\vec{\beta},\cdots,s\vec{\alpha}+\vec{\beta}$,则与归纳基础中同样的原因,
        必有这$s$个向量都不属于$V_s$,且这$s$个向量中不可能存在两个向量同属于一个$V_i(i=1,2,\cdots,s-1)$,因此这$s$个向量中至少有一个
        不在$V_1\cup V_2\cup\ldots\cup V_s$中,因此结论成立.
    \end{enumerate}
\end{proof}

本质而言$s>2$的情况就是将$s-1$个子空间的并视为一个整体,然后套用$s=2$的情况证明.
若将这一定理的条件限制在$V$为有限维线性空间,我们也可以利用范德蒙行列式的方法证明,详见\autoref{ex:11:行列式证明覆盖定理}.
事实上覆盖定理在习题中也有出现,例如教材91-92页第8、9题,都是覆盖定理的直接证明.我们下面再给出一个例子供读者应用覆盖定理:
\begin{example}
    设$V_1,V_2,\ldots,V_s$是线性空间$V$的$s$个非平凡子空间,证明:存在$V$的一组基$\alpha_1,\alpha_2,\ldots,\alpha_n$
    都不在$V_1,V_2,\ldots,V_s$中.
\end{example}

\section{维数公式}
\begin{theorem}
    设$W_1,W_2$是线性空间$V(\mathbf{F})$的两个子空间,则
    \[\dim W_1+\dim W_2=\dim(W_1+W_2)+\dim(W_1\cap W_2).\]
\end{theorem}
上式称为子空间的维数公式,区别于下一专题中的线性映射基本定理的维数公式.这一定理的证明思想非常重要,因此此处我们给出证明.

\begin{proof}
    设$\dim W_1=s$,$\dim W_2=t$,$\dim(W_1\cap W_2)=r$.设$W_1\cap W_2$的一组基为$\alpha_1,\alpha_2,\ldots,\alpha_r$,
    则可以扩充为$W_1$的一组基,记为$\alpha_1,\alpha_2,\ldots,\alpha_r,\beta_1,\ldots,\beta_{s-r}$;
    也可以扩充为$W_2$的一组基,记为$\alpha_1,\alpha_2,\ldots,\alpha_r,\gamma_1,\ldots,\gamma_{t-r}$.
    则我们有
    \[W_1+W_2=\spa(\alpha_1,\cdots,\alpha_r,\beta_1,\cdots,\beta_{s-r},\gamma_1,\cdots,\gamma_{t-r})\]
    (如果对这一步有疑问可以回顾\autoref{ex:4:子空间运算}中的证明)
    由此,我们要证$\dim (W_1+W_2)=s+t-r$,只需证$\alpha_1,\cdots,\alpha_r,\beta_1,\cdots,\beta_{s-r},\gamma_1,\cdots,\gamma_{t-r}$
    线性无关.为此,我们设
    \begin{equation}\label{eq:4:维数公式证明1}
        a_1\alpha_1+\cdots+a_r\alpha_r+b_1\beta_1+\cdots+b_{s-r}\beta_{s-r}+c_1\gamma_1+\cdots+c_{t-r}\gamma_{t-r}=0,
    \end{equation}
    即
    \begin{equation}\label{eq:4:维数公式证明2}
        a_1\alpha_1+\cdots+a_r\alpha_r+b_1\beta_1+\cdots+b_{s-r}\beta_{s-r}=-c_1\gamma_1-\cdots-c_{t-r}\gamma_{t-r}.
    \end{equation}
    显然,\autoref{eq:4:维数公式证明2}等号两端的向量分别属于$W_1$和$W_2$,因此它们都属于$W_1\cap W_2$,因此都可以被$W_1\cap W_2$
    的基线性表示,即
    \[-c_1\gamma_1-\cdots-c_{t-r}\gamma_{t-r}=d_1\alpha_1+\cdots+d_r\alpha_r,\]
    即
    \begin{equation}\label{eq:4:维数公式证明3}
        c_1\gamma_1+\cdots+c_{t-r}\gamma_{t-r}+d_1\alpha_1+\cdots+d_r\alpha_r=0.
    \end{equation}
    由于$\alpha_1,\cdots,\alpha_r,\gamma_1,\cdots,\gamma_{t-r}$是$W_2$的基,因此\autoref{eq:4:维数公式证明3}所有系数都为0,即
    $c_1=\cdots=c_{t-r}=d_1=\cdots=d_r=0$.代入\autoref{eq:4:维数公式证明2}后,由于$\alpha_1,\cdots,\alpha_r,\beta_1,\cdots,\beta_{s-r}$
    是$W_1$的基,因此可得$a_1=\cdots=a_r=b_1=\cdots=b_{s-r}=0$,因此,代入\autoref{eq:4:维数公式证明1}后可知
    $\alpha_1,\cdots,\alpha_r,\beta_1,\cdots,\beta_{s-r},\gamma_1,\cdots,\gamma_{t-r}$必定线性无关
    (因为根据前述证明所有系数只能为0),故得证.
\end{proof}

总结而言,这一定理证明用到的思想就是``设小扩大''.我们设出最小空间$V_1\cap V_2$的基,然后分别扩充为$V_1$和$V_2$
的基,然后观察要证明的等式和已知的联系,然后利用\autoref{eq:4:维数公式证明2}构造等式两边属于不同空间的向量这一技巧即可.
下面是一个证明思想类似的例子,需要用到矩阵的相关知识,暂未学到的同学可以先略过本题:
\begin{example}
    已知$A,B$分别是数域$\mathbf{F}$上的$s \times k$和$k \times n$矩阵,$X$是$n \times 1$
    的列向量. 证明:所有满足$ABX=0$的$BX$构成一个线性空间$V$,且$\dim V = r(B) - r(AB)$.
\end{example}

\section{线性空间的直和}
我们将来证明或者利用和空间时,很多时候都是利用和空间定义进行向量分解.我们特别重视分解唯一时的情形,因为这对我们的研究
很有帮助,这时的和即为直和.严谨而言,我们有如下定义:
\begin{definition}
    设$W_1,W_2$是线性空间$V(\mathbf{F})$的两个子空间. 若$W_1 \cap W_2=\{0\}$,则$W_1+W_2$叫做
    $W_1$与$W_2$的\keyterm{直和}[direct sum],记作$W_1\oplus W_2$.此时称$W_1,W_2$为\keyterm{互补子空间}[complementary subspaces],
    或$W_1$是$W_2$的补空间,或$W_2$是$W_1$的补空间.
\end{definition}

直和有以下等价的命题,我们证明或者利用直和都可以任意选择:
\begin{theorem}
    对于子空间$W_1,W_2$,下列命题等价:
    \begin{enumerate}[label=(\arabic*)]
        \item $W_1+W_2$是直和,即$W_1 \cap W_2=\{0\}$;

        \item $W_1+W_2$中的每个向量$\alpha$的分解式$\alpha=\alpha_1+\alpha_2\enspace(\alpha_1\in W_1,\enspace\alpha_2\in W_2)$唯一;

        \item 零向量的分解式$0=\alpha_1+\alpha_2 \enspace(\alpha_1\in W_1,\enspace\alpha_2\in W_2)$仅当$\alpha_1=\alpha_2=0$时成立;

        \item $\dim (W_1+W_2)=\dim W_1+\dim W_2$.
    \end{enumerate}
\end{theorem}

定理的证明是基本的,可以参考教材76页.在实际运用中我们要非常熟悉这些等价条件,因为都可能使用到.

我们也可以定义有限个子空间的直和,即$V=W_1\oplus W_2\oplus\cdots\oplus W_n \iff W_i \cap \sum\limits_{j \neq i}W_j=\{0\}$,
即每个子空间与其余子空间的和的交都是$\{0\}$.
等价命题也是上述定理的推广,例如唯一分解、0的分解以及维数公式推广,此处不再赘述,详见教材76页.
除此之外,我们还有一个与多空间直和相关的定理:
\begin{theorem}
    若$V=V_1\oplus V_2$,$V_1=V_{11}\oplus\cdots\oplus V_{1s}$,$V_2=V_{21}\oplus\cdots\oplus V_{2t}$,则
    \[V=V_{11}\oplus\cdots\oplus V_{1s}\oplus V_{21}\oplus\cdots\oplus V_{2t}\]
\end{theorem}
这一定理的证明是很简单的,实际上利用零向量分解唯一即可.

在习题中我们证明直和一般有两种思路,一种是先证和,再证直和,我们来看一个例子:
\begin{example}
    数域$\mathbf{F}$上所有$n$阶方阵组成的线性空间$V=\mathbf{M}_n(\mathbf{F})$,$V_1$表示所有对称矩阵
    组成的集合,$V_2$表示所有反对称矩阵组成的集合. 证明:$V_1,V_2$都是$V$的子空间,且$V=V_1\oplus V_2$.
\end{example}
还有一种证明$V=V_1\oplus V_2$的方式是先令$W=V_1+V_2$,先证明和为直和(即交为$\{0\}$)再证$W=V$即可,
下面是一个例子:
\begin{example}
    设$A$是数域$\mathbf{F}$上的一个$n$阶可逆方阵,$A$的前$r$个行向量组成的矩阵为$B$,后$n-r$个
    行向量组成的矩阵为$C$,$n$元线性方程组$BX=0$与$CX=0$的解空间分别为$V_1,V_2$. 证明:$\mathbf{F}^n=V_1\oplus V_2$.
\end{example}

最后我们要提醒读者注意的是,有限维线性空间的一个子空间的补空间并不唯一,如下面的例子:
\begin{example}
    在$\mathbf{R}^3$中,$W_1=\spa(\vec{\alpha_1})$,则其补空间根据直和的维数公式可知为2,记为
    $W_2=\spa(\vec{\alpha_2},\vec{\alpha_3})$.实际上只需要$\vec{\alpha_1},\vec{\alpha_2},\vec{\alpha_3}$线性无关即可,
    事实上这样的选择是有无穷种的,因为$W_1$本质表示一条直线,故$W_2$是不包含$W_1$且不与$W_1$平行的平面即可,
    这样$\vec{\alpha_2},\vec{\alpha_3}$是$W_2$任意一组基都可以.
\end{example}

\vspace{2ex}
\centerline{\heiti \Large 内容总结}

\vspace{2ex}

\centerline{\heiti \Large 习题}
\vspace{2ex}
{\kaishu }
\begin{flushright}
    \kaishu

\end{flushright}
\centerline{\heiti A组}
\begin{enumerate}
    \item
\end{enumerate}
\centerline{\heiti B组}
\begin{enumerate}
    \item 设$W_1,W_2$是线性空间$V(\mathbf{F})$的两个子空间,则
    $W_1 \cup W_2$为$V\text{ 的子空间 } \iff W_1 \subseteq W_2$或$W_2 \subseteq W_1 \iff W_1 \cup W_2=W_1+W_2$.

\end{enumerate}
\centerline{\heiti C组}
\begin{enumerate}
    \item
\end{enumerate}
