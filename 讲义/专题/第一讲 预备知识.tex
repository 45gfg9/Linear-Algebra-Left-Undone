\chapter{预备知识}

\indent 线性代数作为大学的第一门数学课,预修要求并不高.我们默认读者具有基本的
高中数学知识,因此关于集合、映射以及向量的基本知识我们不在此赘述.这一讲
我们将从基本代数结构开始,以便后续线性空间的引入,然后我们将介绍本书中常见的概念——等价类
和最常用的算法之一——高斯消元法.

\section{基本代数结构}
我们选择从基本代数结构谈起,因为在以往的实践中我们深切地体会到直接引入线性空间的跳跃.因此我们希望从更具象的例子开始,
首先引入“代数结构”这一基本概念,然后在下一节中自然地引出线性空间的定义.

我们首先考察一个简单的例子:实数集$\mathbf{R}$,它是一个集合.在初中我们便知道,在$\mathbf{R}$上我们可以定义加法和乘法
两种运算.本质而言,运算是一种映射(或者更通俗而言,函数):
\begin{align*}
	+: & \mathbf{R}\times\mathbf{R}\to\mathbf{R}\\
	   & (a,b)\mapsto a+b\\
	\times: & \mathbf{R}\times\mathbf{R}\to\mathbf{R}\\
	   & (a,b)\mapsto a\times b
\end{align*}
它们以两个实数作为函数的自变量,函数值也是一个实数.或许看到这里还是有些许迷茫,但如果我们回忆映射的基本定义$f:A\to B$,
$a\mapsto f(a)$,并将加法乘法写成$+(2,3)=5,\times(2,3)=6$,想必就会恍然大悟($+$和$\times$实际上就是函数名,函数做的事情
就是输入两个自变量然后进行加法/乘法运算得到函数值).

在上述讨论中,我们所做的事情很简单,就是给定一个集合,然后在这一集合的元素之间定义运算.实际上这就是代数系统的定义:
\begin{definition}
	一般地,我们把一个非空集合$X$和在$X$上定义的若干代数运算$f_1,\cdots,f_k$组成的系统称为代数系统(简称代数系),记作
	$\langle X:f_1,\cdots,f_k\rangle$.
\end{definition}

特别注意的是,代数系统上定义的运算必须保证封闭性,也就是运算后的结果必须仍然在集合$X$中.

不难理解,代数系统其中蕴含的性质与其中定义的运算具有的性质是关联很大的.我们仍然以实数域为例,介绍在代数学中关心的几个运算性质.
我们首先讨论实数域上的加法运算,以下性质对于任意$a,b,c\in\mathbf{R}$都成立:
	
	1. 结合律:$(a+b)+c=a+(b+c)$;
	
	2. 单位元:存在一个元素$0$,使得$a+0=0+a=a$;
	
	3. 逆元:对于任意$a$,存在一个元素$-a$,使得$a+(-a)=(-a)+a=0$(0为单位元);
	
	4. 交换律:$a+b=b+a$.

对于乘法运算(可记为$\cdot$或$\times$),单位元一般记为1(更一般的可以记为$e$),逆元记为$a^{-1}$.
事实上,我们可以给出更多的例子:
\begin{example}
	验证:

	(1)代数系统$\langle \mathbf{R}\backslash\{0\}:\circ\rangle$定义的一般乘法运算

	(2)代数系统$\langle \mathbf{R}^2:+\rangle$上定义的平面向量的加法

	均满足上述四条运算性质.
\end{example}

事实上,我们可以对上面的定义做进一步的抽象.我们可以忽略集合中元素的差异(元素可以是实数,也可以是上述例子中
的平面向量等),同时也可以忽略运算定义的差异,只关心运算作用于集合元素的性质.
对于一般的代数系统$\langle G:\circ\rangle$,我们有如下定义:
\begin{definition}\label{群定义}
	若运算$\circ$满足结合律,则称代数系统$\langle G:\circ\rangle$为半群;若在半群基础上存在单位元,则称之为含幺半群;
	若在含幺半群基础上每个元素存在逆元,则称之为群;若在群的基础上运算还满足交换律,则称之为交换群,也称为Abel群.
\end{definition}

定义\ref{群定义}给出了我们本节第一个要讨论的代数结构——群的定义.简而言之,代数结构就是在集合上定义具有某些特定性质的运算
后得到的一类代数系统.事实上,教材中42-44页给出了大量抽象的例子有助于同学们理解上述一系列群的定义,同时也能使读者体会到
“在集合上定义运算”的方式的多样与抽象.

事实上,在很多集合上我们不仅可以定义一种运算,也可以定义两种甚至更多运算,在代数结构中我们仅讨论最多两种运算的情况.事实上,
我们最开始的实数集合定义加法和乘法的例子便可以引入一个新的代数结构——域:
\begin{definition}
	我们称代数系统$\langle F:+,\circ\rangle$为一个域,如果

	1. $\langle F:+\rangle$是交换群,其单位元记作0;
	
	2. $\langle F\backslash\{0\}:\circ\rangle$是交换群;
	
	3. 运算$\circ$对$+$满足左、右分配律,即
	$$a\circ(b+c)=a\circ b+a\circ c,$$
	$$(b+c)\circ a=b\circ a+c\circ a.$$
\end{definition}

显然,实数域$\mathbf{R}$上定义一般的实数加法和乘法后构成一个域.

当然,还有一种代数结构对于$\circ$运算的要求有所降低,但也有广泛的应用,这就是环:
\begin{definition}
	我们称代数系统$\langle R:+,\circ\rangle$为一个环,如果
	
	1. $\langle R:+\rangle$是交换群,其单位元记作0;

	2. $\langle R:\circ\rangle$是半群;

	3. 运算$\circ$对$+$满足左、右分配律,即
	$$a\circ(b+c)=a\circ b+a\circ c,$$
	$$(b+c)\circ a=b\circ a+c\circ a.$$
	
	若关于$\circ$存在单位元,则称之为含幺环,若进一步每个非0($+$运算单位元)元素关于$\circ$都有逆元,则称之为除环.
	另外,若上述定义中$\circ$运算满足交换律,则称为交换环,结合上述除环和交换环两个定义,我们可以发现,交换除环即为域.
\end{definition}

我想大部分读者都会对这样的抽象的原因表示不解,如果这个问题无法解答,我想在下一章直接引入抽象的线性空间更会引发同学们对于
“学了这个有什么用”的怀疑.我们可以举一些不那么贴切但具象的例子来说明这其中的意义.读者高中阶段想必大都经受过解析几何的摧残,
大家在拿到题目时总会首先观察到题目属于“定点”、“定值”或是“极值”等问题,大家将自动与自己做题的经验或技巧匹配用于解答这几类
问题.同理,在研究一个特定的代数系统(例如定义了加法和乘法的实数域)的性质时,我们可以首先将其归类为群、环或是域等,然后
我们只需要利用群环域各自的性质来研究这个代数系统的性质,而不需要再去研究这个代数系统的具体定义.在这一过程中我们实现了问题的
“归约”,即将一个复杂的问题转化为一个简单的更为抽象的问题,正如将解决上千道解析几何问题转化为研究几种题型的技巧.这一“归约”
的思想在将来的学习生活中我们将经常遇见,在实际中例如投资股票时我们可以将投资转化为提高投资组合的期望收益而尽力降低方差(风险)
的求取极值的问题,在理论中,例如在计算理论的学习中我们会学习更为形式化的对问题的归约,这在算法复杂性研究中是基础的思想.对于
这类抽象问题感兴趣的同学不妨可以选择数学科学学院的抽象代数等课程,或是阅读本讲义的“后继”教程
\href{https://frightenedfoxcn.github.io/notes/series/alg-for-cs/}{《写给计算机系学生的代数》}作进一步的了解.

当然,这段描述因为涉及的知识容量较大,大概无法说服每一个读者.但我们会在学习线性空间、线性映射的过程中不断重复这些思想,直到
读者具备的知识容量足够时,一定会恍然大悟其中的奥妙.

\section{复数域的引入}
本书前半段讨论的框架是实数域、复数域都适用的,当然为了简化,我们的例子大都来源于实数.从多项式一讲开始,我们便会开始强调
实数域和复数域结论的不同,因此我们有必要在此引入复数域.

我们都知道,

\section{等价关系}
我们首先从(二元)关系这一概念入手,因为等价类的划分需要基于等价关系.实际上,这里的二元关系和日常生活中的关系
是紧密相连的,例如将全人类作为谈论的背景集合,二元关系我们可以视为一个二元组,那么(小头爸爸,大头儿子)是符合
这一关系的,但(章鱼哥,海绵宝宝)显然不符合.因此我们可以将父子关系看作笛卡尔积集合人类$\times$人类的子集.更一般化的,
集合$A$中的关系可以由$A\times A$的子集$$\{(a,b)\ |\ a,b\in A,aRb\}$$
来刻画,其中$R$是这个关系本身(实质上是两个元素之间的某种性质),例如之前讨论的父子关系,或是数学中的大于、小于或同余等.
事实上,反过来,由$A\times A$的子集可以确定一个关系,例如我把全世界所有的父子组合放在这个集合中,那么这个集合就定义了
人类中的父子关系.

接下来我们要讨论一种特别的关系,即等价关系.它对关系$R$有一定的规定:
\begin{definition}
	集合$A$中关系若满足以下条件:

	(自反性)$aRa,\forall a\in A$;

	(对称性)若$aRb$,则$bRa$;

	(传递性)若$aRb,bRc$,则$aRc$,

	则称$R$为$A$的一个等价关系.
\end{definition}
基于等价关系我们进一步定义等价类.若$R$是集合$A$的一个等价关系且$a\in A$,则$A$中所有与$a$有关系$R$的元素集合
$$K_a=\{b\in A\ |\ bRa\}$$称为$a$所在的等价类,$a$称为这个等价类的代表元素,并记$\{K_a\}$为所有等价类为元素构成的集族.

我们可能需要一个例子来理解这些概念.我们不难证明,初等数论中的同余关系是一种等价关系,以模3同余为例,我们取整体集合为正整数集合,
对于3,它的等价类就是所有和3模3同余的元素集合,即所有3的倍数.同理,对于1,它所在的等价类就是模3余1的全体正整数,2所在的等价类是
全体模3余2的正整数.除此之外,我们还发现一个特点,即这三个等价类将原集合分成了三个无交集的子集,且它们的并集就是原集合,即这三个
等价类构成了原集合的一个\textup{分划}(即分为并为原集合且不相交的子集).这一结论对所有等价类都成立,是很直观的结论,其证明可以参考
离散数学等教材:
\begin{theorem}
	设$R$是集合$A$的等价关系,则由所有不同的等价类构成的子集族$\{K_a\}$是$A$的分划.反之,我们也可以基于分划在$A$中定义等价关系.
\end{theorem}
基于此我们可以定义\textup{商集}的概念:
\begin{definition}
	设$R$是集合$A$的等价关系,以关于$R$的等价类为元素的集族(实际上就是集合构成的集合)$\{K_a\}$称为$A$对$R$的商集,记为$A/R$.
	由$$\pi(a)=K_a,\forall a\in A$$定义的$A$到$A/R$上的映射$\pi$称为$A$到$A/R$上的自然映射.
\end{definition}
我们可以看到,自然映射$\pi$将$A$中的元素$a$映到自己所在的等价类$K_a$.

\section{高斯消元法}


\vspace{2ex} 
\centerline{\heiti \Large 内容总结}
\vspace{2ex} 

本讲为了后续章节讲述方便引入了一些基本概念和算法.尽管这是一门面向理工科应用的数学课,但我们仍然希望以最自然的方式引入
概念,而非填鸭式地轰炸,因此我们首先从大家最熟悉的

\centerline{\heiti \Large 习题}
\vspace{2ex} 
{\kaishu 我这门课很简单,只有简单的加减乘除四则运算,甚至除法都不太需要.}
\begin{flushright}
    \kaishu
	——浙江大学数学科学学院教授吴志祥
\end{flushright}
\centerline{\heiti A组}
\begin{enumerate}
	\item 
\end{enumerate}
\centerline{\heiti B组}
\begin{enumerate}
	\item 
\end{enumerate}
\centerline{\heiti C组}
\begin{enumerate}
	\item 
\end{enumerate}