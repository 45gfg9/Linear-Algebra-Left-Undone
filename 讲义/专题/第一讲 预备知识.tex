\chapter{预备知识}

\indent 线性代数作为大学的第一门数学课,预修要求并不高.我们默认读者具有基本的
高中数学知识,因此关于集合、映射以及向量的基本知识我们不在此赘述.这一讲
我们将从基本代数结构开始,以便后续线性空间的引入,然后我们将介绍本书中常见的概念——等价类
和最常用的算法之一——高斯消元法.

\section{基本代数结构}
我们选择从基本代数结构谈起,因为在以往的实践中我们深切地体会到直接引入线性空间的跳跃.因此我们希望从更具象的例子开始,
首先引入``代数结构''这一基本概念,然后在下一节中自然地引出线性空间的定义.

我们首先考察一个简单的例子:实数集$\mathbf{R}$,它是一个集合.在初中我们便知道,在$\mathbf{R}$上我们可以定义加法和乘法
两种运算.本质而言,运算是一种映射(或者更通俗而言,函数):

\begin{center}
    \begin{tabular}{rrcl}
        $+\enspace\colon$ & $\mathbf{R}\times\mathbf{R}$&$\to$&$\mathbf{R}$\\
                   & $(a,b)$&$\mapsto$&$a+b$\\
        $\times\enspace\colon$ & $\mathbf{R}\times\mathbf{R}$&$\to$&$\mathbf{R}$\\
                        & $(a,b)$&$\mapsto$&$a\times b$
    \end{tabular}
\end{center}

它们以两个实数作为函数的自变量,函数值也是一个实数.或许看到这里还是有些许迷茫,但如果我们回忆映射的基本定义$f:A\to B$,
$a\mapsto f(a)$,并将加法乘法写成$+(2,3)=5$,$\times(2,3)=6$,想必就会恍然大悟:$+$和$\times$实际上就是函数名,函数做的事情
就是输入两个自变量然后进行加法/乘法运算得到函数值.

在上述讨论中,我们所做的事情很简单,就是给定一个集合,然后在这一集合的元素之间定义运算.实际上这就是代数系统的定义:
\begin{definition}
    一般地,我们把一个非空集合$X$和在$X$上定义的若干代数运算$f_1,\cdots,f_k$组成的系统称为\keyterm{代数系统}[algebraic system](简称代数系),记作
    $\langle X : f_1,\ldots,f_k\rangle$.
\end{definition}

特别注意的是,代数系统上定义的运算必须保证封闭性,也就是运算后的结果必须仍然在集合$X$中.

不难理解,代数系统其中蕴含的性质与其中定义的运算具有的性质是关联很大的.我们仍然以实数域为例,介绍在代数学中关心的几个运算性质.
我们首先讨论实数域上的加法运算,以下性质对于任意$a,b,c\in\mathbf{R}$都成立:

\begin{enumerate}
    \item 结合律:$(a+b)+c=a+(b+c)$;

    \item 单位元:存在一个元素$0$,使得$a+0=0+a=a$;

    \item 逆元:对于任意$a$,存在一个元素$-a$,使得$a+(-a)=(-a)+a=0$(0为单位元);

    \item 交换律:$a+b=b+a$.
\end{enumerate}

对于乘法运算(可记为$\cdot$或$\times$),单位元一般记为1(更一般的可以记为$e$),逆元记为$a^{-1}$.
事实上,我们可以给出更多的例子:
\begin{example}\label{ex:1:abelgroup}
    \begin{enumerate}[label=(\arabic*)]
        \item 代数系统$\langle \mathbf{R}\backslash\{0\}:\circ\rangle$定义的一般乘法运算

        \item 代数系统$\langle \mathbf{R}^2:+\rangle$定义的平面向量的加法
    \end{enumerate}

    均满足上述四条运算性质.
\end{example}

事实上,我们可以对上面的定义做进一步的抽象.我们可以忽略集合中元素的差异(元素可以是实数,也可以是上述例子中
的平面向量等),同时也可以忽略运算定义的差异,只关心运算作用于集合元素的性质.
对于一般的代数系统$\langle G:\circ\rangle$,我们有如下定义:
\begin{definition} \label{def:1:group}
    若运算$\circ$满足结合律,则称代数系统$\langle G:\circ\rangle$为\keyterm{半群}[semigroup];若在半群基础上存在单位元,则称之为\keyterm{含幺半群}[monoid];
    若在含幺半群基础上每个元素存在逆元,则称之为\keyterm{群}[group];若在群的基础上运算还满足交换律,则称之为\keyterm{Abel群}[Abelian group],也称\keyterm{交换群}[commutative group].
\end{definition}

\autoref{def:1:group} 给出了我们本节第一个要讨论的代数结构——群的定义. 简而言之,代数结构就是在集合上定义具有某些特定性质的运算
后得到的一类代数系统.事实上,教材中42-44页给出了大量抽象的例子有助于同学们理解上述一系列群的定义,同时也能使读者体会到
``在集合上定义运算''的方式的多样与抽象.

事实上,在很多集合上我们不仅可以定义一种运算,也可以定义两种甚至更多运算,在代数结构中我们仅讨论最多两种运算的情况.事实上,
我们最开始的实数集合定义加法和乘法的例子便可以引入一个新的代数结构——域:
\begin{definition}
    我们称代数系统$\langle F:+,\circ\rangle$为一个\keyterm{域}[field],如果
    \begin{enumerate}
        \item $\langle F:+\rangle$是交换群,其单位元记作0;

        \item $\langle F\backslash\{0\}:\circ\rangle$是交换群;

        \item 运算$\circ$对$+$满足左、右分配律,即
        \begin{gather*}
            a\circ(b+c)=a\circ b+a\circ c \\
            (b+c)\circ a=b\circ a+c\circ a
        \end{gather*}
    \end{enumerate}
\end{definition}

显然,实数域$\mathbf{R}$上定义一般的实数加法和乘法后构成一个域.

当然,还有一种代数结构对于$\circ$运算的要求有所降低,但也有广泛的应用,这就是环:
\begin{definition}
    我们称代数系统$\langle R:+,\circ\rangle$为一个\keyterm{环}[ring],如果
    \begin{enumerate}
        \item $\langle R:+\rangle$是交换群,其单位元记作0;

        \item $\langle R:\circ\rangle$是半群;

        \item 运算$\circ$对$+$满足左、右分配律,即
        \begin{gather*}
            a\circ(b+c)=a\circ b+a\circ c \\
            (b+c)\circ a=b\circ a+c\circ a
        \end{gather*}
    \end{enumerate}

    若关于$\circ$存在单位元,则称之为\keyterm{含幺环}[ring with identity],若进一步每个非0($+$运算单位元)元素关于$\circ$都有逆元,则称之为\keyterm{除环}[division ring].
    另外,若上述定义中$\circ$运算满足交换律,则称为\keyterm{交换环}[commutative ring],结合上述除环和交换环两个定义,我们可以发现,交换除环即为域.
\end{definition}

我想大部分读者都会对这样的抽象的原因表示不解,如果这个问题无法解答,我想在下一章直接引入抽象的线性空间更会引发同学们对于
``学了这个有什么用''的怀疑. 我们可以举一些不那么贴切但具象的例子来说明这其中的意义. 读者高中阶段想必大都经受过解析几何的摧残,
大家在拿到题目时总会首先观察到题目属于``定点''、``定值''或是``极值''等问题,大家将自动与自己做题的经验或技巧匹配用于解答这几类
问题.同理,在研究一个特定的代数系统(例如定义了加法和乘法的实数域)的性质时,我们可以首先将其归类为群、环或是域等,然后
我们只需要利用群环域各自的性质来研究这个代数系统的性质,而不需要再去研究这个代数系统的具体定义.在这一过程中我们实现了问题的
``归约'',即将一个复杂的问题转化为一个简单的更为抽象的问题,正如将解决上千道解析几何问题转化为研究几种题型的技巧.这一``归约''
的思想在将来的学习生活中我们将经常遇见,在实际中例如投资股票时我们可以将投资转化为提高投资组合的期望收益而尽力降低方差(风险)
的求取极值的问题,在理论中,例如在计算理论的学习中我们会学习更为形式化的对问题的归约,这在算法复杂性研究中是基础的思想.对于
这类抽象问题感兴趣的同学不妨可以选择数学科学学院的抽象代数等课程,或是阅读本讲义的``后继''教程
\href{https://frightenedfoxcn.github.io/notes/series/alg-for-cs/}{《写给计算机系学生的代数》}作进一步的了解.

当然,这段描述因为涉及的知识容量较大,大概无法说服每一个读者.但我们会在学习线性空间、线性映射的过程中不断重复这些思想,直到
读者具备的知识容量足够时,一定能领会其中的奥妙.

\section{复数域的引入}
本书前半段讨论的框架是实数域、复数域都适用的,当然为了简化,我们的例子大都来源于实数.从多项式一讲开始,我们便会开始强调
实数域和复数域结论的不同,因此我们有必要在此引入复数域.

直观来看,实数位于数轴上,复数则分布在二维平面上,因此我们可以先考虑平面点集$\mathbf{R}^2$,并在其上定义加法和乘法运算使其
成为一个域.我们回顾高中学习的平面向量知识,我们记$\vec{e}_1=(1,0)$,$\vec{e}_2=(0,1)$,则$\mathbf{R}^2$上的任一向量
$\vec{u}=(x,y)$可写为$x\vec{e}_1+y\vec{e}_2$.此外,我们仍沿袭高中对向量长度的定义,即$\lvert\vec{u}\rvert=\sqrt{x^2+y^2}$.

在\autoref{ex:1:abelgroup} 中我们已经验证了$\mathbf{R}^2$上的向量加法满足Abel群的条件,因此我们只需要定义$\mathbf{R}^2$上的乘法
使得代数系统$\langle\mathbf{R}^2\backslash\{(0,0)\}:\circ\rangle$也为Abel群.这一乘法的构造需要满足一些自然的条件,同时
也能实现构成Abel群的要求.事实上,我们有如下定理:
\begin{theorem}\label{th:1:complexmult}
    平面点集$\mathbf{R^2}$上存在唯一的乘法$\circ$,满足
    \begin{enumerate}
        \item(单位元)$\vec{u}\circ\vec{e}_1=\vec{e}_1\circ\vec{u},\enspace\forall\vec{u}\in\mathbf{R^2}$;

        \item(长度可乘性)$\lvert\vec{u}\circ\vec{v}\rvert=\lvert\vec{u}\rvert\lvert\vec{v}\rvert$.
    \end{enumerate}

    此乘法满足交换律,且使得$\langle\mathbf{R}^2:+,\circ\rangle$成为域.
\end{theorem}

上述定理中第一个条件是非常自然的,因为在二维平面上,$\{(x,0),x\in\mathbf{R}\}$实际上就是实数轴,因此$\vec{e}_1=(1,0)$相当于
实数1,因此作为乘法单位元是非常自然的.第二条长度可乘则看起来没那么自然,但在接下来的证明中我们将会了解到其意义.

\begin{proof}
    对任意向量
\end{proof}

在\autoref{th:1:complexmult} 赋予的乘法下,$\langle\mathbf{R}^2:+,\circ\rangle$成为复数域$\mathbf{C}$.我们可将$\vec{e}_1$合理简记为1,
同时$\vec{e}_2$简记为$\mathrm{i}$,利用$\vec{e}_2^2=-\vec{e}_1$可知$\mathrm{i}^2=-1$,这与我们熟知的虚数单位的定义是统一的.在此记号的约定下,
$(a,b)$即为$a+b\mathrm{i}$.这一代数表示引入的相关概念,如实部、虚部、纯虚数,以及复数四则运算法则在高中阶段大家都已熟知,在此不再赘述.

非零复数$z=x+y\mathrm{i}$也可写为极坐标的形式,即$z=|z|(\cos\theta+\mathrm{i}\sin\theta)$,其中$|z|=\sqrt{x^2+y^2}$为复数
的平面表示的模长,$\theta\in\mathbf{R}$为连接原点与$z$的有向线段与$x$轴正方向的夹角(在相差$2\pi$整数倍的意义下唯一).我们称
$\theta$为复数$z$的辐角.

\section{等价关系}
我们首先从(二元)关系这一概念入手,因为等价类的划分需要基于等价关系.实际上,这里的二元关系和日常生活中的关系
是紧密相连的,例如将全人类作为谈论的背景集合,二元关系我们可以视为一个二元组,那么$(\text{小头爸爸}, \text{大头儿子})$是符合
这一关系的,但$(\text{章鱼哥}, \text{海绵宝宝})$显然不符合.因此我们可以将父子关系看作笛卡尔积集合$\text{人类}\times\text{人类}$的子集.更一般化的,
集合$A$中的关系可以由$A\times A$的子集
\[\{(a,b) \mid a,b\in A, \enspace a\,R\,b\}\]
来刻画,其中$R$是这个关系本身(实质上是两个元素之间的某种性质),例如之前讨论的父子关系,或是数学中的大于、小于或同余等.
事实上,反过来,由$A\times A$的子集可以确定一个关系,例如我把全世界所有的父子组合放在这个集合中,那么这个集合就定义了
人类中的父子关系.

接下来我们要讨论一种特别的关系,即等价关系.它对关系$R$有一定的规定:
\begin{definition}
    集合$A$中关系若满足以下条件:
    \begin{itemize}
        \item(自反性)$\forall a\in A, \enspace a\,R\,a$;

        \item(对称性)若$a\,R\,b$,则$b\,R\,a$;

        \item(传递性)若$a\,R\,b$,$b\,R\,c$,则$a\,R\,c$,
    \end{itemize}
    则称$R$为$A$的一个等价关系.
\end{definition}
基于等价关系我们进一步定义等价类.若$R$是集合$A$的一个等价关系且$a\in A$,则$A$中所有与$a$有关系$R$的元素集合
\[K_a=\{b\in A \mid b\,R\,a\}\]
称为$a$所在的等价类,$a$称为这个等价类的代表元素,并记$\{K_a\}$为所有等价类为元素构成的集族.

我们可能需要一个例子来理解这些概念.我们不难证明,初等数论中的同余关系是一种等价关系,以模3同余为例,我们取整体集合为正整数集合,
对于3,它的等价类就是所有和3模3同余的元素集合,即所有3的倍数.同理,对于1,它所在的等价类就是模3余1的全体正整数,2所在的等价类是
全体模3余2的正整数.除此之外,我们还发现一个特点,即这三个等价类将原集合分成了三个无交集的子集,且它们的并集就是原集合,即这三个
等价类构成了原集合的一个分划(即分为并为原集合且不相交的子集).这一结论对所有等价类都成立,是很直观的结论,其证明可以参考
离散数学等教材:
\begin{theorem}
    设$R$是集合$A$的等价关系,则由所有不同的等价类构成的子集族$\{K_a\}$是$A$的\keyterm{分划}[partition].反之,我们也可以基于分划在$A$中定义等价关系.
\end{theorem}
基于此我们可以定义商集的概念:
\begin{definition}
    设$R$是集合$A$的等价关系,以关于$R$的等价类为元素的集族(实际上就是集合构成的集合)$\{K_a\}$称为$A$对$R$的\keyterm{商集}[quotient set],记为$A/R$.
    由
    \[\pi(a) = K_a, \enspace \forall a\in A\]
    定义的$A$到$A/R$上的映射$\pi$称为$A$到$A/R$上的自然映射.
\end{definition}
我们可以看到,自然映射$\pi$将$A$中的元素$a$映到自己所在的等价类$K_a$.

\section{高斯消元法}
高斯消元法是考试中一定会考察的内容,无论是单独一个大题考察,还是嵌入在求解极大线性无关组等问题中.
注意单独考察解方程时,时间充足时建议将过程写完整,标明初等行变换的具体步骤,并且至少写出阶梯矩阵和行简化阶梯矩阵.
除此之外,需要保证计算中尽量减少错误,时间充足可以解完方程后将答案代入进行检查.

需要强调的是,不要认为本节内容很简单就放过了,实际上如果长期不计算高斯消元法很容易
造成眼高手低的窘境,因此希望各位同学熟悉高斯消元法的基本步骤并熟练应用.
\subsection{基本步骤}
一般的,对于一个由$m$个方程组成的$n$元(即变量数为$n$)线性方程组
\[ \left\{
\begin{array}{rcl}
    a_{11}x_1+a_{12}x_2+\cdots+a_{1n}x_n&=&b_1 \\
    a_{21}x_1+a_{22}x_2+\cdots+a_{2n}x_n&=&b_2 \\
    &\vdots& \\
    a_{m1}x_1+a_{m2}x_2+\cdots+a_{mn}x_n&=&b_m
\end{array}
\right. \]

将其系数排列成矩阵
\[
\begin{pmatrix}
    a_{11} & a_{12} & \cdots & a_{1n} \\
    a_{21} & a_{22} & \cdots & a_{2n} \\
    \vdots & \vdots & \ddots & \vdots \\
    a_{m1} & a_{m2} & \cdots & a_{mn}
\end{pmatrix}
\]
且记$\vec{b}=(b_1,b_2,\cdots,b_m)^\mathrm{T}$,若$\vec{b}=\vec{0}$则称此方程为齐次线性方程组,否则为非齐次线性方程组.
再将$n$个未知量记为$n$元列向量$\vec{X}=(x_1,x_2,\cdots,x_n)^\mathrm{T}$,我们便可以把方程组简记为
$A\vec{X}=\vec{b}$.

令$\vec{\beta}_i=(a_{1i},a_{2i},\cdots,a_{mi})^\mathrm{T}$,即方程组系数矩阵的某一列,
则方程组还可以记为$x_1\vec{\beta}_1+x_2\vec{\beta}_2+\cdots+x_n\vec{\beta}_n=\vec{b}$,这一形式将在之后多次见到.

在以上的记号下,我们可以将解线性方程组的过程转化为矩阵的初等行变换.高斯消元法的一般步骤如下:

\centerline{线性方程组$\overset{1}{\longrightarrow}$增广矩阵$\overset{2}{\longrightarrow}$阶梯矩阵$\overset{3}{\longrightarrow}$(行)简化阶梯矩阵$\overset{4}{\longrightarrow}$解}

\begin{itemize}
    \item 步骤1只需要将线性方程组转化为$(A, \vec{b})$的形式,得到左$n$列为系数矩阵,最右列为列向量$\vec{b}$的$n+1$列的增广矩阵;
    \item 步骤2是通过初等行变换后,得到教材P34(1-13)的形式的矩阵——阶梯矩阵.阶梯矩阵系数全零行在最下方,并且非零行中,在下方的行的第一个非零元素一定在上方行的右侧(每行第一个非零元素称主元素);
    \item 步骤3将主元素化1后将主元素所在列的其他元素均通过初等行变换化为0即可;
    \item 步骤4中,我们分三种情况讨论:
    \begin{enumerate}
        \item 有唯一解:没有全零行,且行简化阶梯矩阵对角线上全为1,其余元素均为0,此时可以直接写出解;
        \item 无解:出现矛盾方程,即系数为0的行的行末元素不为0,此时直接写无解即可;
        \item 有无穷解:非上述情况.此时设出自由未知量将其令为$k_1,k_2,\ldots$,然后代入增广矩阵对应的方程组即可.注意选取自由未知量时,选取没有主元素出现的列对应的未知量会与标准答案更贴近(如教材P33选取$x_2,x_5$),当然选择其他作为自由未知量也可以.
    \end{enumerate}
\end{itemize}

\vspace{2ex}
\centerline{\heiti \Large 内容总结}
\vspace{2ex}

本讲为了后续章节讲述方便引入了一些基本概念和算法.尽管这是一门面向理工科应用的数学课,但我们仍然希望以最自然的方式引入
概念,而非填鸭式地轰炸,因此我们首先从大家最熟悉的

\centerline{\heiti \Large 习题}
\vspace{2ex}
{\kaishu 我这门课很简单,只有简单的加减乘除四则运算,甚至除法都不太需要.}
\begin{flushright}
    \kaishu
    ——浙江大学数学科学学院教授吴志祥
\end{flushright}
\centerline{\heiti A组}
\begin{enumerate}
    \item
\end{enumerate}
\centerline{\heiti B组}
\begin{enumerate}
    \item
\end{enumerate}
\centerline{\heiti C组}
\begin{enumerate}
    \item
\end{enumerate}
