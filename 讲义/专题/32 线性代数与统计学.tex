\chapter{线性代数与统计学}

在统计学中,当我们研究多个随机变量之间的相关关系时,我们将会见到大量熟悉的线性代数知识.
本节的目标便是希望选取几个经典且基本的统计学中使用线性代数中概念与方法的例子帮助读者
在学习统计学的过程中看见线性代数不会感到陌生.

\section{多元正态分布}


\section{马尔科夫链}
最后我们介绍随机过程中运用线性代数的重要的例子——马尔科夫链.本小节使用线性代数的角度
不同于前面小节侧重于二次型等方面,我们将会探讨

\vspace{2ex}
\centerline{\heiti \Large 内容总结}

\vspace{2ex}

\centerline{\heiti \Large 习题}
\vspace{2ex}
{\kaishu 在终极的分析中,一切知识都是历史;在抽象的意义下,一切科学都是数学;
在理性的基础上,所有的判断都是统计学.}
\begin{flushright}
    \kaishu
    ——C.R.Rao,《统计与真理》
\end{flushright}
\centerline{\heiti A组}
\begin{enumerate}
    \item
\end{enumerate}
\centerline{\heiti B组}
\begin{enumerate}
    \item
\end{enumerate}
\centerline{\heiti C组}
\begin{enumerate}
    \item
\end{enumerate}
