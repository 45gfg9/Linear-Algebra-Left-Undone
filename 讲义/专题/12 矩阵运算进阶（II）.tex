\chapter{矩阵运算进阶(II)} \label{chap:矩阵运算进阶(II)}

本讲我们将讨论技巧性更强的一些内容,如特殊矩阵,矩阵可交换、求逆和求幂等. 大部分都是方法为主,理解的内容不多,但对于我们拿到问题有更好的洞察是很重要的.



\vspace{2ex}
\centerline{\heiti \Large 内容总结}

本讲我们介绍了矩阵运算的另一些技巧性较强的话题,包括有关特殊矩阵(对角矩阵、三角矩阵和基本矩阵等)的基本性质、矩阵可交换问题的讨论、矩阵求逆和求幂的进阶方法,希望读者能掌握其中的一些基本方法,对于熟练矩阵计算,解决一些实际问题有一定帮助.

\vspace{2ex}
\centerline{\heiti \Large 习题}

\vspace{2ex}
{\kaishu 尽管⼀批教授和教科书编者⽤关于矩阵的荒唐⾄极的计算内容掩盖了线性代数的简明性,但是鲜有与之相较更为初等的理论.}
\begin{flushright}
    ——Jean Dieudonne
\end{flushright}

\centerline{\heiti A组}
\begin{enumerate}
    \item 设方阵$A$满足$A^2-A-2E=O$,证明:
          \begin{enumerate}
              \item $A$和$E-A$都是可逆矩阵,并求它们的逆矩阵;

              \item $A+E$和$A-2E$不可能同时可逆.
          \end{enumerate}

    \item 若$A,B$为两个$n$阶矩阵且满足$A+B=AB$,证明:
          \begin{enumerate}
              \item $A-E$和$B-E$均可逆;

              \item $AB=BA$;

              \item $r(A)=r(B)$.
          \end{enumerate}
\end{enumerate}

\centerline{\heiti B组}
\begin{enumerate}
    \item 设$f(x)=1+x+\cdots+x^{m-1}$,$g(x)=1-x$,$A=\begin{pmatrix}
                  a & b \\ 0 & a
              \end{pmatrix}$,计算$f(A)g(A)$.

    \item 已知矩阵$A=\begin{pmatrix}
                  1 & 0 & 4 \\ 0 & 1 & 2 \\ 0 & 1 & 2
              \end{pmatrix}$,求证:所有与$A$可交换的矩阵构成$\mathbf{M}_3(\mathbf{R})$的一个子空间,并求子空间的一组基.

    \item 已知矩阵$A=\begin{pmatrix}
                  1 & 0 & 1 \\ 0 & 2 & 0 \\ 1 & 0 & 1
              \end{pmatrix}$,
          \begin{enumerate}
              \item 求所有与$A$可交换的矩阵;

              \item 若$AB+E=A^2+B$,求$B$.
          \end{enumerate}

    \item 设$A \in \mathbf{F}^{n \times n}$,令$C(A)=\{B \in \mathbf{F}^{n \times n} \mid AB=BA\}$.
          \begin{enumerate}
              \item 证明:$C(A)$为$\mathbf{F}^{n \times n}$的一个子空间;

              \item 求$C(E)$;

              \item 当$A$为对角线上元素互不相等的对角阵时,求$C(A)$的维数和一组基.
          \end{enumerate}

    \item 设$A$是$n$阶矩阵,$A^k=O$对某个正整数$k$成立,求证下列方阵可逆,并求它们的逆:
          \begin{enumerate}
              \item $E+A$;

              \item $E-A$;

              \item $E+A+\dfrac{1}{2!}A^2+\cdots+\dfrac{1}{(k-1)!}A^{k-1}$.
          \end{enumerate}

    \item 设$A=\begin{pmatrix}
        1 & \cdots & 1 \\
        \vdots & \ddots & \vdots \\
        1 & \cdots & 1
    \end{pmatrix}$,求:
    \begin{enumerate}
        \item 一个二次实系数多项式$f(x)=ax^2+bx$,使得$f(A)=O$;
        \item $A^100$;
        \item $(A+E)^3$;
        \item $(A+E)^{-1}$.
    \end{enumerate}
    % 新题,需要答案
\end{enumerate}

\centerline{\heiti C组}
\begin{enumerate}
    \item 已知数列$\{a_n\},\{b_n\}$满足$a_0=-1,\enspace b_0=3$,且
          \[\begin{cases}
                  a_n=3a_{n-1}+b_{n-1}+2^{n-1} \\
                  b_n=2a_{n-1}+4b_{n-1}+2^n
              \end{cases}\]
          求$\{a_n\},\{b_n\}$的通项公式.

    \item 证明以下两个命题:
          \begin{enumerate}
              \item 与矩阵$I=\begin{pmatrix}
                            0 & 1 &   &        &   \\
                              &   & 1 &        &   \\
                              &   &   & \ddots &   \\
                              &   &   &        & 1 \\
                            1 &   &   &        & 0
                        \end{pmatrix}$可交换的矩阵$A$都可以写成$I$的一个多项式,即$A=a_{11}E+a_{12}I+a_{13}I^2+\cdots+a_{1n}I^{n-1}$;

              \item 与矩阵$J=\begin{pmatrix}
                            0 & 1 &   &        &   \\
                              &   & 1 &        &   \\
                              &   &   & \ddots &   \\
                              &   &   &        & 1 \\
                              &   &   &        & 0
                        \end{pmatrix}$可交换的矩阵$A$都可以写成$J$的一个多项式,即$A=a_{11}E+a_{12}J+a_{13}J^2+\cdots+a_{1n}J^{n-1}$.
          \end{enumerate}
\end{enumerate}
