\chapter{线性代数与几何}

解析几何很大程度上是线性代数发展的初衷,在研究点线面以及几何体时,将集体的几何问题抽象化为代数问题使其方便解决与计算,即是解析几何的主要思想. 本节我们将会从线性代数的角度探究解析几何的一些基本概念与方法. 此在线性代数课程的考察中也会有少部分的解析几何内容,但内容较浅,主要考察点、直线、平面等之间的关系.

\section{欧几里得空间}

在前面的学习中我们已经较为全面地学习了内积空间的相关知识,而在解析几何中,我们在更多情况下会研究\term{欧几里得空间}\index{oujilidekongjian@欧几里得空间 (Euclidean space)}下的问题.
\begin{definition}{欧几里得空间}{}
    欧几里得空间(欧氏空间)是一个有限维实内积空间.
\end{definition}
同学们可能对欧氏空间的几何直观更为熟悉. 当欧氏空间的维数为 2 或 3 时,我们可以用熟悉的平面直角坐标系与空间直角坐标系来描述欧氏空间中的向量,并用点积作为向量的内积.

\section{欧氏空间上的运算}

我们也已经基本掌握了模、内积、夹角等在内积空间中的基本概念,在此我们引入一些在先前的学习中接触较少的概念.
\begin{definition}{点积}{} \index{dianji@点积 (dot product)}
    \term{点积}是在三维欧氏空间中对两个向量的运算,用$\vec{a}\cdot\vec{b}$表示. 两向量点积得到的数值等于两向量模长的乘积与两向量夹角的余弦的乘积.
\end{definition}
特别的,三维欧氏空间中的向量点积$(a_1,a_2,a_3)\cdot(b_1,b_2,b_3)$可以表示为\[a_1b_1+a_2b_2+a_3b_3\]
由点积的计算,我们可以很方便地得到两向量夹角的余弦,即\[\cos\theta=\frac{\vec{a}\cdot\vec{b}}{|\vec{a}||\vec{b}|}\]
\begin{definition}{叉乘}{} \index{chacheng@叉乘 (cross product)}
    \term{叉乘}是在三维欧氏空间中对两个向量的运算,用$\vec{a}\times\vec{b}$表示. 两向量叉乘得到的向量垂直于两向量,方向遵循右手定则,其模长为两向量的模的乘积与两向量夹角的正弦的乘积.
\end{definition}
由定义可知,叉乘仅在三维欧氏空间中有定义,且叉乘的结果是一个向量,而不是一个数. 关于叉乘向量的计算有另一种更常用的用行列式表示的计算方法,即
\[(a_1,a_2,a_3)\times(b_1,b_2,b_3)=\begin{vmatrix}
        \vec{i} & \vec{j} & \vec{k} \\
        a_1     & a_2     & a_3     \\
        b_1     & b_2     & b_3
    \end{vmatrix}\]
其中$\vec{i},\vec{j},\vec{k}$为三维欧氏空间的自然基.

在解析几何中,叉乘的一个重要应用是求解与两向量垂直的向量.
\begin{definition}{混合积}{} \index{hunheji@混合积 (mixed product)} \index{biaoliangsancongji@标量三重积 (scalar triple product)}
    \term{混合积}(或称\term{标量三重积},不同于\term{矢量三重积})是三维欧氏空间中对三个向量的运算,用$[\vec{a},\vec{b},\vec{c}]$表示,等价于$(\vec{a}\times\vec{b})\cdot\vec{c}$.
\end{definition}
混合积的几何意义是以$\vec{a},\vec{b},\vec{c}$为邻边的平行六面体的体积,可以用行列式表示为
\[[(a_1,a_2,a_3),(b_1,b_2,b_3),(c_1,c_2,c_3)]=\begin{vmatrix}
        a_1 & a_2 & a_3 \\
        b_1 & b_2 & b_3 \\
        c_1 & c_2 & c_3
    \end{vmatrix}\]
同时读者也不难验证 $ (\vec{a}\times\vec{b})\cdot\vec{c} = \vec{a}\cdot(\vec{b}\times\vec{c}) $. 其应用之一是可以用来判断三个向量是否共面.

\section{点、直线、平面的表示}

一个点在欧氏空间中可以用一个向量来表示. 在三维欧氏空间中,我们可以用三个实数来表示一个点的坐标.

\subsection{平面的方程}

平面是欧氏空间中的一个基本几何对象,我们有多种代数方法来表示平面.

平面的一般方程是平面的一种最基本的表示方法,即$Ax+By+Cz+D=0$. 平面的一般方程十分简洁,但是我们很难由此方程得到平面的几何性质,因此我们还需要考虑其他的表示方法. 例如,一个平面由平面上一点与平面上两个不共线的向量来表示. 假设已知平面上一点$P(x_0,y_0,z_0)$和平面上两个不共线的向量$\vec{u}=(a,b,c)$和$\vec{v}=(d,e,f)$,则平面上的任意一点$Q(x,y,z)$都满足$\overrightarrow{PQ}$与$\vec{u}$和$\vec{v}$线性相关,即
\[\overrightarrow{PQ}=k_1\vec{u}+k_2\vec{v}\]
化为坐标形式即为
\[\begin{cases}
        x=x_0+k_1a+k_2d \\
        y=y_0+k_1b+k_2e \\
        z=z_0+k_1c+k_2f
    \end{cases}\]
这就是平面的参数方程,其中$k_1,k_2$是参数.

此外,平面还可以由平面上一点和平面的法向量来表示. 假设已知平面上一点$P(x_0,y_0,z_0)$和平面的法向量$\vec{n}=(A,B,C)$,则平面上的任意一点$Q(x,y,z)$都满足向量$\overrightarrow{PQ}$与$\vec{n}$垂直,即点积为0. 由此可得其方程为\[A(x-x_0)+B(y-y_0)+C(z-z_0)=0\]这种表示方法称为\term{点法式}.

我们发现这跟平面的一般方程十分相似,实际上,我们可以直接通过平面的一般方程得到平面的法向量.

在得到一张由其他方式表示的平面时,我们往往也会将其转化为一般式或点法式,以便于我们计算其与其他几何对象的关系. 例如,得到一个由平面上一点与平面上两不共线的向量表示的平面,则可以通过求两向量的叉积得到平面的法向量,从而得到平面的点法式.

\begin{example}{}{}
    若已知一个平面上有三点$A(1,2,0),\enspace B(0,1,-1),\enspace C(1,1,1)$,求该平面的一般方程.
\end{example}

\subsection{直线的方程}

直线在欧氏空间中也是一个基本对象,同样有多种代数方法可以表示直线.

首先直线可以用某两张平面的交表示. 假设有两相交平面的方程,联立可得直线方程
\[\begin{cases}
        A_1x+B_1y+C_1z+D_1=0 \\
        A_2x+B_2y+C_2z+D_2=0
    \end{cases}\]
即为直线的一般方程. 这种联立方程的表示方法最为基本,但是不够简洁,大多情况下也不够直观. 所以更多情况下我们希望在表示中可以直观体现直线的一些特征. 因此,可以用直线上的一个点和直线的方向(即方向向量)来确定一条直线.

假设已知直线上的一点$A_0(x_0,y_0,z_0)$和直线的方向向量$\vec{l}=(a,b,c)$,则直线上的任意一点$A(x,y,z)$都满足$\overrightarrow{AA_0}$与$\vec{l}$平行,用具体的方程则表示为
\[\frac{x-x_0}{a}=\frac{y-y_0}{b}=\frac{z-z_0}{c}\]
其中$a,b,c$不为零. 这种表示方法称为\term{点向式}.

如果我们对上述式子进行替换,令\[t=\frac{x-x_0}{a}=\frac{y-y_0}{b}=\frac{z-z_0}{c}\]
则可得
\[\begin{cases}
        x=x_0+at \\
        y=y_0+bt \\
        z=z_0+ct
    \end{cases}\]
这样就得到了直线的参数方程,其中$t$为参数.

当然还有以两点确定一条直线的表示方法,我们可以轻松地算出直线的方向向量,然后用点向式或参数方程来表示. 最后可以得出方程
\[\frac{x-x_1}{x_2-x_1}=\frac{y-y_1}{y_2-y_1}=\frac{z-z_1}{z_2-z_1}\]

那么如何实现从一般方程到点向式或参数方程的转换呢?最简单的方法是求解线性方程组再用两点表示或者参数表示,但是这样的方法比较麻烦,事实上我们可以利用法向量进行转换. 假设两平面的一般方程为$A_1x+B_1y+C_1z+D_1=0$与$A_2x+B_2y+C_2z+D_2=0$,则可以得到两平面的法向量分别为$\vec{n}_1=(A_1,B_1,C_1),\enspace\vec{n}_2=(A_2,B_2,C_2)$,因为该直线在两张平面内,所以直线与两个法向量都垂直,所以$\vec{n}_1\times\vec{n}_2$即为直线的方向向量. 再求出一般方程的一个解(即直线上一点)即可得到直线的点向式与参数方程.

\section{平面与直线间的位置关系}

对于三维欧氏空间中的几何对象,我们主要需要研究平行、相交与重合等关系. 我们可以通过平面与直线的方程来判断.

\subsection{线与线的位置关系}

线与线之间的位置关系判断主要依靠它们的方向向量. 如果两条直线的方向向量平行,则两条直线平行或重合,此时再判断两直线是否存在公共点,若联立方程有解,说明两直线重合,否则两条直线平行. 如果两条直线的方向向量不平行,则还需要判断两条直线是否共面,若共面则说明两条直线相交,否则两条直线异面. 此时以两直线方程联立方程组,若有解则说明存在交点,否则说明两条直线异面.

\begin{example}{}{}
    已知直线$L_1=\begin{cases}
            x+y+z-1=0 \\
            x-2y+2=0
        \end{cases},\enspace L_2=\begin{cases}
            x=2t  \\
            y=t+a \\
            z=bt+1
        \end{cases}$,试确定$a,b$的值使得$L_1,L_2$是:
    \begin{enumerate}
        \item 平行直线;

        \item 异面直线.
    \end{enumerate}
\end{example}

\subsection{线与面的位置关系}

线与面的位置关系首先需要判断线的方向向量与平面的法向量的关系. 如果方向向量与法向量平行,则说明线与面垂直. 如果两者垂直,则说明该直线与平面平行或者在平面内,只需再判断直线上的点是否在平面内即可.

此外还有一些对于平面不同表示形式的方法. 例如,假设已知直线的方向向量与平面上两个不平行的向量,则可以对这三个向量做混合积,如果混合积为零,则说明三个向量共面,即直线与平面平行或者在平面内.

\subsection{面与面的位置关系}

面与面的位置关系主要依靠两个平面的法向量来判断. 如果两个平面的法向量平行,则说明两个平面平行或重合,再判断两平面是否存在公共点. 若两法向量垂直,则两平面也垂直.

\vspace{2ex}
\centerline{\heiti \Large 内容总结}

这里关于解析几何的部分浅尝辄止,只是简单地介绍了一些基本的概念与方法,希望能够帮助大家对解析几何有一个简单的初步认识. 在线性代数课程中可能的相关考察基本也仅限于点、线、面之间的关系,方程的联立、求解等等,或许大家在未来其他课程的学习中可以学到更多相关的知识.

\vspace{2ex}
\centerline{\heiti \Large 习题}

\vspace{2ex}
{\kaishu 我决心放弃那个仅仅是抽象的几何. 这就是说,不再去考虑那些仅仅是用来练思想的问题. 我这样做,是为了研究另一种几何,即目的在于解释自然现象的几何.}
\begin{flushright}
    \kaishu
    ——笛卡尔
\end{flushright}

\centerline{\heiti A组}
\begin{enumerate}
    \item
\end{enumerate}

\centerline{\heiti B组}
\begin{enumerate}
    \item
\end{enumerate}

\centerline{\heiti C组}
\begin{enumerate}
    \item
\end{enumerate}
