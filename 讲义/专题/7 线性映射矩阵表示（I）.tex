\chapter{线性映射矩阵表示(I)}

接下来的三讲我们将展开一个重要的承上启下的内容——线性映射矩阵表示,我们将前面逐步搭建的线性空间与线性映射的抽象
转变为具象的表达——矩阵,并深入理解矩阵的表示与之前抽象概念之间的关联.引入矩阵后,我们一方面可以成功将线性方程组(矩阵表达的形式)
解的本质理论的探究与之前所学习的线性空间、线性映射结合,从而迈出了里程碑式的一步;另一方面
有形的矩阵表达使得我们可以引入更多的计算技巧和工具,使我们未来的研究相对于前述章节而言更为具象.

本节我们将介绍矩阵和线性映射矩阵表示的定义,讨论如何利用线性映射的运算和矩阵表示来引入矩阵的三种基本运算:
加法、数乘和乘法,并探讨它们的基本性质.

\section{线性映射矩阵表示}
\begin{definition}
    域$\mathbf{F}$中的$m\times n$个元素$a_{ij}(i=1,\cdots,m,j=1,\cdots,n)$排成$m$行$n$列的矩形数表,称为
    域$F$上的一个$m\times n$矩阵,记作
    \[A=\begin{pmatrix}
        a_{11} & a_{12} & \cdots & a_{1n} \\
        a_{21} & a_{22} & \cdots & a_{2n} \\
        \vdots & \vdots & \ddots & \vdots \\
        a_{m1} & a_{m2} & \cdots & a_{mn}
    \end{pmatrix}\]
    或简记为$(a_{ij})_{m\times n}$,其中$a_{ij}$表示矩阵$A$的第$i$行第$j$列的元素.
\end{definition}

我们有一些常用的矩阵,例如零矩阵,即所有元素均为0的矩阵,通常记为$O$;单位矩阵也十分常见,它表示
对角线上元素为1,其余元素为0的矩阵,通常记为$E$.

除此之外,我们通常记域$\mathbf{F}$上的$m\times n$矩阵全体为$\mathbf{F}^{m\times n}$或$M_{m\times n}(\mathbf{F})$.
当$m=n$时矩阵称为方阵,域$\mathbf{F}$上全体$n$阶矩阵(或称$n$阶方阵)记为$\mathbf{F}^{n\times n}$或$M_n(\mathbf{F})$.

\begin{definition}\label{def:7:线性映射矩阵表示}
    设$B_1=\{\varepsilon_1,\varepsilon_2,\ldots,\varepsilon_n\}$是$V_1(F)$的基,$B_2=\{\alpha_1,\alpha_2,\cdots,\alpha_m\}$是$V_2(F)$的基.
    则线性映射$\sigma \in \mathcal{L}(V_1,V_2)$被它作用于基$B_1$的像
    \[\sigma(B_1)=\{\sigma(\varepsilon_1),\sigma(\varepsilon_2),\ldots,\sigma(\varepsilon_n)\}\]
    所唯一确定,而$\sigma(B_1)$是$V_2$的子空间,于是其中元素都可以被基$B_2$线性表示,即
    \[ \left\{
     \begin{array}{rcl}
        \sigma(\varepsilon_1)&=&a_{11}\alpha_1+a_{21}\alpha_2+\ldots+a_{m1}\alpha_m \\
        \sigma(\varepsilon_2)&=&a_{12}\alpha_1+a_{22}\alpha_2+\ldots+a_{m2}\alpha_m \\
        &\vdots& \\
        \sigma(\varepsilon_n)&=&a_{1n}\alpha_1+a_{2n}\alpha_2+\ldots+a_{mn}\alpha_m
     \end{array}
    \right. \]

    我们将$\sigma(B_1)=\{\sigma(\varepsilon_1),\sigma(\varepsilon_2),\ldots,\sigma(\varepsilon_n)\}$
    关于基$B_2$的坐标排列成矩阵$\mathbf{M}(\sigma)$,即
    \[\mathbf{M}(\sigma)=\begin{pmatrix}
        a_{11} & a_{12} & \cdots & a_{1n} \\
        a_{21} & a_{22} & \cdots & a_{2n} \\
        \vdots & \vdots & \ddots & \vdots \\
        a_{m1} & a_{m2} & \cdots & a_{mn}
    \end{pmatrix}\]
\end{definition}
更通俗来说,线性映射矩阵表示就是将线性映射在一组基上的像在另一组基下的坐标表示按列排列得到的结果.
这一整体过程我们也可以用如下记号表示:
\begin{equation}\label{eq:7:线性映射矩阵表示}
    (\sigma(\epsilon_1),\sigma(\epsilon_2),\ldots,\sigma(\epsilon_n))=(\alpha_1,\alpha_2,\ldots,\alpha_m)\mathbf{M}(\sigma).
\end{equation}

根据定义我们直接有如下简单的观察:
\begin{enumerate}
    \item 线性映射矩阵表示的结果是一个$m\times n$矩阵,其中$m$是到达空间的维数,$n$是出发空间的维数;
    \item 若$\sigma$在基下矩阵表示为$A=(a_{ij})_{m\times n}$,在出发空间的基的第$i$个向量在到达空间基下的坐标为
    $(a_{1i},a_{2i},\ldots,a_{mi})$,即矩阵$A$的第$i$列,或写为$\sigma(\varepsilon)=a_{1i}\alpha_1+a_{2i}\alpha_2+\ldots+a_{mi}\alpha_m$.
\end{enumerate}

\begin{example}\label{ex:7:矩阵表示1}
    已知$\sigma \in \mathcal{L}(\mathbf{R}^3,\mathbf{R}^3)$且$\sigma(x_1,x_2,x_3)=(x_1+x_2,x_1-x_3, x_2)$
    \begin{enumerate}[label=(\arabic*)]
        \item 求$\sigma$的像空间和核空间;

        \item 求$\sigma$关于$\mathbf{R}^3$自然基的矩阵.
    \end{enumerate}
\end{example}
\begin{solution}
    \begin{enumerate}[label=(\arabic*)]
        \item 求像空间和核空间的方法我们在之前已经介绍过,我们为了计算方便取$\mathbf{R}^3$的自然基$e_1,e_2,e_3$计算有:
        \[\im\sigma=\spa(\sigma(e_1),\sigma(e_2),\sigma(e_3))=\spa((1,1,0),(1,0,1),(0,-1,0))=\mathbf{R}^3\]
        对于核空间,解方程$\sigma(\alpha)=0$即可,我们也可以用更简洁的方式书写:
        \[\ker\sigma=\{(x_1,x_2,x_3)\mid \sigma(x_1,x_2,x_3)=(0,0,0)\}=\{(0,0,0)\}\]
        即方程只有零解,核空间可以记为$\ker\sigma=\{0\}$(只含零元的空间的一般记法).
        \item 我们根据\autoref{def:7:线性映射矩阵表示},我们应先写出$\sigma$在出发空间一组基(按题目要求是$\mathbf{R}^3$
        自然基)下的像,并将像表示为到达空间基(按题目要求是$\mathbf{R}^3$自然基)的线性组合,即
        \begin{gather*}
            \sigma(e_1)=(1,1,0)=e_1+e_2=(e_1,e_2,e_3)\begin{pmatrix}
                1 \\ 1 \\ 0
            \end{pmatrix} \\
            \sigma(e_2)=(1,0,1)=e_1+e_3=(e_1,e_2,e_3)\begin{pmatrix}
                1 \\ 0 \\ 1
            \end{pmatrix} \\
            \sigma(e_3)=(0,-1,0)=-e_2=(e_1,e_2,e_3)\begin{pmatrix}
                0 \\ -1 \\ 0
            \end{pmatrix}
        \end{gather*}
        接下来我们把坐标依次按列称矩阵就得到了本题需要求解的矩阵:
        \[\mathbf{M}(\sigma)=\begin{pmatrix}
            1 & 1 & 0 \\
            1 & 0 & -1 \\
            0 & 1 & 0
        \end{pmatrix}\]
    \end{enumerate}
\end{solution}

有趣的是,在结合我个人的学习经历以及过往辅学的经验后,我总结出了第二问的一种常见的错误解法,这里我需要加粗强调,下面这种
解法是\textbf{完全错误的!!!},这里展示这一解法是为了让读者将前面所学的知识完全厘清:

\begin{solution}
    (\textbf{错误解法!!!})$\sigma(x_1,x_2,x_3)=(x_1+x_2,x_1-x_3, x_2)=(x_1,x_2,x_3)\begin{pmatrix}
        1 & 1 & 0 \\
        1 & 0 & 1 \\
        0 & -1 & 0
    \end{pmatrix}$
\end{solution}

我们惊奇地发现,这一结果和我们前面得到的标准答案在向量的排列方式上发生了变化,即标准答案的1、2、3行变为了这里的1、2、3列,
我们需要强调两点:
\begin{enumerate}
    \item 为什么这种解法是错误的:我们可以直接比较\autoref{eq:7:线性映射矩阵表示}和这一解法中,\autoref{eq:7:线性映射矩阵表示}
    的等号左边是$n$个向量在$\sigma$下的像,而上述解法$\sigma(x_1,x_2,x_3)$只是$\sigma$在一个向量下的像,这显然是不一样的!!!
    同样,等号右边括号内\autoref{eq:7:线性映射矩阵表示}是到达空间的一组基,而上述解法中仍然只是一个向量.我们从未定义过这样解题
    的结果是什么,所以千万不能做这种无意义的事!!!

    容易导致混淆的原因可能在于$(x,y,z)$向量是排列成一行的,可能看起来和$(e_1,e_2,e_3)$有点相似,但如果我们将后者拆分成
    $((1,0,0),(0,1,0),(0,0,1))$,你还会混淆吗?

    \item 为什么会出现行列互换这样的错误:事实上
    \[\sigma(x,y,z)=\sigma(xe_1+ye_2+ze_3)=x\sigma(e_1)+y\sigma(e_2)+z\sigma(e_3)=(x,y,z)\begin{pmatrix}
        \sigma(e_1) \\ \sigma(e_2) \\ \sigma(e_3)
    \end{pmatrix},\]
    这里将$\sigma(e_1),\sigma(e_2),\sigma(e_3)$的结果按行排列成矩阵,而标准答案是将$\sigma(e_1),\sigma(e_2),\sigma(e_3)$
    在$\mathbf{R}^3$自然基下的坐标按列排列成矩阵,回忆$\mathbf{R}^n$向量在自然基下坐标是其本身这一性质,标准答案就是将
    $\sigma(e_1),\sigma(e_2),\sigma(e_3)$按列排列成矩阵,由此我们解释了行列互换发生的原因.
\end{enumerate}

这也就是为什么我强调读者不要参考教材102页例3求解像空间的方法来求解像空间——很容易导致这里矩阵表示犯这样的错误,
并且容易导致初学时无法区分求解像空间和线性映射矩阵表示的方法.在这里我必须再次强调:在没有完全熟练掌握这些概念和方法前,
不要乱用方法!!!

还需要强调的一点是,教材121页例2中介绍了旋转变换的矩阵表示,即$\mathbf{R}^2$中向量绕原点按逆时针方向旋转$\theta$角的变换
关于$\mathbf{R}^2$的自然基的矩阵表示为
\[\mathbf{M}(\sigma)=\begin{pmatrix}
    \cos\theta & -\sin\theta \\
    \sin\theta & \cos\theta
\end{pmatrix}\]
这一形式可以记忆,在之后会多次出现.

\section{$\mathcal{L}(V_1,V_2)$与矩阵空间的同构}
\subsection{矩阵的加法和数乘}
本节我们将完善上一讲中同构的例子的细节,即若$\dim V_1(\mathbf{F})=m$,$\dim V_2(\mathbf{F})=n$,则
$\mathcal{L}(V_1,V_2) \cong \mathbf{F}^{m \times n}$,其中$\mathbf{F}^{m \times n}$表示全体$m\times n$矩阵构成的线性空间.

要证明这一结论,首先要说明全体$m\times n$矩阵关于某种运算的确构成线性空间,这里的运算——即矩阵的加法和数乘还需要我们来定义.
我们有一个非常自然的想法——既然$V_1\to V_2$的全体线性映射关于线性映射加法和数乘构成线性空间,那么我们也许可以利用线性映射加法与数乘运算
的矩阵表示来定义加法和数乘运算.

我们首先回顾线性映射的加法和数乘运算:设$\sigma,\tau\in \mathcal{L}(V_1,V_2)$,规定$\sigma$与$\tau$之和及$\lambda$与
$\sigma$的数乘$\lambda\sigma$分别为
\begin{gather*}
    (\sigma+\tau)(\alpha)=\sigma(\alpha)+\tau(\alpha),\enspace\forall\alpha\in V_1 \\
    (\lambda\sigma)(\alpha)=\lambda(\sigma(\alpha)),\enspace\forall\alpha\in V_1
\end{gather*}

回顾线性映射的矩阵表示,我们实际上是要计算出线性映射在出发空间一组基下的像在到达空间一组基下的坐标然后按列排列.我们取
$V_1$的基$B_1=\{\varepsilon_1,\varepsilon_2,\ldots,\varepsilon_n\}$,$V_2$的基$B_2=\{\alpha_1,\alpha_2,\ldots,\alpha_m\}$,
假设$\sigma$和$\tau$在$B_1$和$B_2$下的矩阵分别为$A=(a_{ij})_{m\times n}$和$B=(b_{ij})_{m\times n}$,则
\begin{gather*}
    \sigma(\varepsilon_i)=a_{1i}\alpha_1+a_{2i}\alpha_2+\ldots+a_{mi}\alpha_m \\
    \tau(\varepsilon_i)=b_{1i}\alpha_1+b_{2i}\alpha_2+\ldots+b_{mi}\alpha_m.
\end{gather*}
因此
\[(\sigma+\tau)(\varepsilon_i)=(a_{1i}+b_{1i})\alpha_1+(a_{2i}+b_{2i})\alpha_2+\ldots+(a_{mi}+b_{mi})\alpha_m,i=1,2,\cdots,n\]
即$(\sigma+\tau)$矩阵表示$\mathbf{M}(\sigma+\tau)$的第$i$列元素为$A$和$B$的第$i$列对应元素相加.由于$i$是任取的,因此$(\sigma+\tau)$的矩阵表示
每一列都是$A$和$B$同一列对应元素相加,实际上对于整个矩阵而言就是矩阵相同位置元素相加,即
\begin{align*}
    \mathbf{M}(\sigma+\tau)&=\begin{pmatrix}
        a_{11}+b_{11} & a_{12}+b_{12} & \cdots & a_{1n}+b_{1n} \\
        a_{21}+b_{21} & a_{22}+b_{22} & \cdots & a_{2n}+b_{2n} \\
        \vdots & \vdots & \ddots & \vdots \\
        a_{m1}+b_{m1} & a_{m2}+b_{m2} & \cdots & a_{mn}+b_{mn}
    \end{pmatrix} \\
    &\triangleq\begin{pmatrix}
        a_{11} & a_{12} & \cdots & a_{1n} \\
        a_{21} & a_{22} & \cdots & a_{2n} \\
        \vdots & \vdots & \ddots & \vdots \\
        a_{m1} & a_{m2} & \cdots & a_{mn}
    \end{pmatrix}+\begin{pmatrix}
        b_{11} & b_{12} & \cdots & b_{1n} \\
        b_{21} & b_{22} & \cdots & b_{2n} \\
        \vdots & \vdots & \ddots & \vdots \\
        b_{m1} & b_{m2} & \cdots & b_{mn}
    \end{pmatrix} \\
    &=\mathbf{M}(\sigma)+\mathbf{M}(\tau).
\end{align*}
式中$\triangleq$表示定义,即定义矩阵加法为矩阵对应元素相加.同理,我们也可以通过线性映射的数乘
定义矩阵数乘运算如下:
\[\mathbf{M}(\lambda\sigma)=\begin{pmatrix}
    \lambda a_{11} & \lambda a_{12} & \cdots & \lambda a_{1n} \\
    \lambda a_{21} & \lambda a_{22} & \cdots & \lambda a_{2n} \\
    \vdots & \vdots & \ddots & \vdots \\
    \lambda a_{m1} & \lambda a_{m2} & \cdots & \lambda a_{mn}
\end{pmatrix}\triangleq\lambda\begin{pmatrix}
    a_{11} & a_{12} & \cdots & a_{1n} \\
    a_{21} & a_{22} & \cdots & a_{2n} \\
    \vdots & \vdots & \ddots & \vdots \\
    a_{m1} & a_{m2} & \cdots & a_{mn}
\end{pmatrix}=\lambda\mathbf{M}(\sigma).\]
事实上这非常符合我们对于矩阵加法和数乘的幻想,即矩阵加法就是对应元素相加,矩阵数乘就是对应元素乘以一个数.

在利用线性映射的加法和数乘定义了非常自然的矩阵加法和数乘后,我们需要验证$m\times n$矩阵全体关于这两种运算
构成线性空间.这里我们只需回顾线性空间运算的八条要求然后逐一验证即可,实际上非常简单,因此不在此赘述.

\subsection{同构的说明}
在上一小节中我们定义了矩阵的加法和数乘运算,也验证了全体$m\times n$矩阵关于这两种运算构成线性空间
$\mathbf{F}^{m\times n}$,接下来我们需要讨论的是对于$n$维线性空间$V_1$和$m$维线性空间$V_2$,
$\mathcal{L}(V_1,V_2)$与$\mathbf{F}^{m\times n}$的同构.
即我们需要定义一个线性双射$\varphi:\mathcal{L}(V_1,V_2)\to\mathbf{F}^{m\times n}$.事实上我们只需要
很自然地利用线性映射矩阵表示定义,即定义
\[\varphi(\sigma)=\mathbf{M}(\sigma)\]
,也就是说$\varphi$将线性映射$\sigma$映射为其矩阵表示.接下来需要验证$\varphi$是线性双射.
\begin{enumerate}
    \item 线性性是显然的,因为根据矩阵加法和数乘的定义,我们有
    \begin{gather*}
        \varphi(\sigma+\tau)=\mathbf{M}(\sigma+\tau)=\mathbf{M}(\sigma)+\mathbf{M}(\tau)=\varphi(\sigma)+\varphi(\tau) \\
        \varphi(\lambda\sigma)=\mathbf{M}(\lambda\sigma)=\lambda\mathbf{M}(\sigma)=\lambda\varphi(\sigma).
    \end{gather*}
    \item 双射也是显然的:
    \begin{enumerate}
        \item 对于单射性,我们考察$\varphi$的核空间$\ker\varphi$中的元素$\sigma$,即$\sigma$在基下的矩阵表示为零矩阵,
        那么$\sigma$必然为零映射,因为它将所有基映射为0,故必然将所有出发空间元素映射为0,因此核空间为$\{0\}$,单射成立;
        \item 对于满射性,我们需要为任意$m\times n$矩阵$(a_{ij})_{m\times n}$找到一个线性映射,使得这一矩阵为这一
        线性映射在基下的矩阵表示.事实上,给定基和矩阵表示,我们就知道了线性映射在出发空间的基下的像——因为给定到达空间的基和矩阵就给定了
        线性映射在出发空间的基在到达空间的基下的坐标.然后根据\autoref{thm:5:线性映射构造}知我们一定能找到这一映射,故满射性成立.
    \end{enumerate}
\end{enumerate}

由此我们证明了$\mathcal{L}(V_1,V_2)\cong\mathbf{F}^{m\times n}$.而我们很容易知道,$\mathbf{F}^{m\times n}$的维数为
$mn$,因为$m\times n$个只有一个元素为1,其余元素为0的$m\times n$矩阵显然是其一组基(线性无关和张成性质都是显然的),
由此我们也得到了$\dim\mathcal{L}(V_1,V_2)=mn$.

\section{线性映射的复合与矩阵乘法}
\subsection{矩阵乘法的定义与基本性质}
由于篇幅的限制,我们这里不再将教材中已有的内容重复.教材124-125页的推导通过线性映射的复合运算定义了矩阵的乘法运算.总结而言,
计算矩阵$A$和$B$的乘法时要求$A$的列数等于$B$的行数,且矩阵$A$与$B$
相乘结果中第$i$行第$j$列元素为矩阵$A$的第$i$行与矩阵$B$的第$j$列对应位置元素相乘后求和的结果,
即对于$A=(a_{ij})_{m \times n}$和$B=(b_{ij})_{n \times l}$,我们记矩阵$C=AB=(c_{ij})_{m \times l}$,有
\[c_{ij}=a_{i1}b_{1j}+a_{i2}b_{2j}+\cdots+a_{in}b_{nj}\enspace(i=1,\ldots,m,\enspace j=1,\ldots,l).\]

\begin{example}
    设$A=\begin{pmatrix}
        1 & 0 & -1 \\ 1 & 1 & -3
    \end{pmatrix}, B=\begin{pmatrix}
        0 & 3 \\ 1 & 2 \\ 3 & 1
    \end{pmatrix}$,求$AB$和$BA$.
\end{example}
\begin{solution}

\end{solution}

教材125页下方也介绍了几个矩阵运算的基本性质:
\begin{enumerate}[label=(\arabic*)]
    \item $(AB)C=A(BC)$(结合律)
    \item $\lambda(AB)=(\lambda A)B=A(\lambda B),\enspace \lambda \in \mathbf{F}$
    \item $A(B+C)=AB+AC$(左分配律)
    \item $(B+C)P=BP+CP$(右分配律)
\end{enumerate}
证明方法十分简单暴力:直接设出矩阵元素然后暴力计算证明等号两边对应位置(如第$i$行第$j$列元素)相等即可.
125页下方给出了结合律的证明,感兴趣的同学可以参考,实际上记住结论即可.

实际上,由矩阵加法和乘法满足的运算律可知,全体$n$阶方阵构成的集合$\mathbf{F}^{n\times n}$关于矩阵加法和乘法构成环.

在本节最后,我们有四个非常重要的问题需要仔细探讨:
\begin{enumerate}
    \item 在有了矩阵乘法的定义后,高斯消元法中我们将线性方程组简记为$AX=b$实际上是相当自然的,除此之外,我们将向量坐标
    表示为列向量的形式,例如
    \[\alpha=(\alpha_1,\alpha_2,\cdots,\alpha_n)\begin{pmatrix}
        x_1 \\ x_2 \\ \vdots \\ x_n
    \end{pmatrix}\]
    这也是符合矩阵乘法定义的一种习惯(虽然基一般不是数域$\mathbf{F}$中的元素).

    \item 事实上,在这里我们可以看出求解齐次线性方程组和求解线性映射核空间之间的关联.我们设$AX=0$的解为
    \[X=\begin{pmatrix}
        x_1 \\ x_2 \\ \vdots \\ x_n
    \end{pmatrix},\]
    由$AX=0$和线性映射矩阵表示,我们有
    \begin{equation}\label{eq:7:方程组与核空间1}
        (\sigma(\varepsilon_1),\sigma(\varepsilon_2),\ldots,\sigma(\varepsilon_n))\begin{pmatrix}
            x_1 \\ x_2 \\ \vdots \\ x_n
        \end{pmatrix}=(\alpha_1,\alpha_2,\ldots,\alpha_m)A\begin{pmatrix}
            x_1 \\ x_2 \\ \vdots \\ x_n
        \end{pmatrix}=0,
    \end{equation}
    即$x_1\sigma(\varepsilon_1)+x_2\sigma(\varepsilon_2)+\cdots+x_n\sigma(\varepsilon_n)=0$,即
    \begin{equation}\label{eq:7:方程组与核空间2}
        \sigma(x_1\varepsilon_1+x_2\varepsilon_2+\cdots+x_n\varepsilon_n)=0.
    \end{equation}
    由此可见,我们将齐次线性方程组的解空间与线性映射的核空间联系起来了,即线性映射的核空间中元素在一组基下的向量就是
    这一线性映射在这组基下的矩阵表示作为系数矩阵的线性方程组的解.这一联系将在朝花夕拾一讲中有更深入的讨论.

    \item 我们可以更进一步理解矩阵乘法.假设矩阵$A=(a_{ij})_{m\times n}$与$B=(b_{ij})_{n\times l}$相乘,我们有如下结论:
    \begin{enumerate}
        \item 乘积的第$k$列等于$A$乘以$B$的第$k$列,乘积的第$j$行等于$A$的第$j$行乘以$B$,这一点根据矩阵乘法计算方式显然;
        \item 乘积的每一列都是矩阵$A$各列的线性组合,每一行都是矩阵$B$各行的线性组合.我们简要说明前者,后者理由类似.我们考察
        乘积的每一列,由1可知乘积的第$k$列等于$A$乘以$B$的第$k$列,我们展开写乘积矩阵$C=(c_{ij})_{m\times l}$第$k$列的结果:
        \begin{gather*}
            c_{1k}=a_{11}b_{1k}+a_{12}b_{2k}+\cdots+a_{1n}b_{nk} \\
            c_{2k}=a_{21}b_{1k}+a_{22}b_{2k}+\cdots+a_{2n}b_{nk} \\
            \vdots \\
            c_{mk}=a_{m1}b_{1k}+a_{m2}b_{2k}+\cdots+a_{mn}b_{nk}
        \end{gather*}
        我们将上面的行进行组合,写成列向量形式,即
        \[\begin{pmatrix}
            c_{1k} \\ c_{2k} \\ \vdots \\ c_{mk}
        \end{pmatrix}=b_{1k}\begin{pmatrix}
            a_{11} \\ a_{21} \\ \vdots \\ a_{m1}
        \end{pmatrix}+b_{2k}\begin{pmatrix}
            a_{12} \\ a_{22} \\ \vdots \\ a_{m2}
        \end{pmatrix}+\cdots+b_{nk}\begin{pmatrix}
            a_{1n} \\ a_{2n} \\ \vdots \\ a_{mn}
        \end{pmatrix}\]
        由此我们将乘积的列表示成了矩阵$A$各列的线性组合.
    \end{enumerate}

    \item 之后我们会经常看见两种记号,即
    \begin{gather*}
        (\sigma(\varepsilon_1),\sigma(\varepsilon_2),\ldots,\sigma(\varepsilon_n))=(\alpha_1,\alpha_2,\ldots,\alpha_m)A, \\
        \sigma(\varepsilon_1,\varepsilon_2,\ldots,\varepsilon_n)=(\alpha_1,\alpha_2,\ldots,\alpha_m)A.
    \end{gather*}
    教材中两个记号是等价的,这只是记号上的差别,含义完全相同.但是在之后我们还会看到一个很特别的书写方式
    \[(\sigma(\varepsilon_1,\varepsilon_2,\ldots,\varepsilon_n))B=\sigma((\varepsilon_1,\varepsilon_2,\ldots,\varepsilon_n)B),\]
    教材不加解释地使用了这一等式,这容易导致读者的困惑,因此我们这里简要说明它们的确是等价的,从而接下来读者可以放心地自由使用这一结论.
    
    根据上述的第一个性质可知,我们只需要证明对$B$的某一列上式成立即可,因为乘法结果是列与列对应的.我们设$B$的第$k$列为
    \[B_k=\begin{pmatrix}
        b_{1k} \\ b_{2k} \\ \vdots \\ b_{nk}
    \end{pmatrix}\]
    则
    \begin{align*}
        (\sigma(\varepsilon_1,\varepsilon_2,\ldots,\varepsilon_n))\begin{pmatrix}
            b_{1k} \\ b_{2k} \\ \vdots \\ b_{nk}
        \end{pmatrix}&=(\sigma(\varepsilon_1),\sigma(\varepsilon_2),\ldots,\sigma(\varepsilon_n))\begin{pmatrix}
            b_{1k} \\ b_{2k} \\ \vdots \\ b_{nk}
        \end{pmatrix}\\&=b_{1k}\sigma(\varepsilon_1)+b_{2k}\sigma(\varepsilon_2)+\cdots+b_{nk}\sigma(\varepsilon_n)
        \\&=\sigma(b_{1k}\varepsilon_1+b_{2k}\varepsilon_2+\cdots+b_{nk}\varepsilon_n)
        \\&=\sigma((\varepsilon_1,\varepsilon_2,\ldots,\varepsilon_n)\begin{pmatrix}
            b_{1k} \\ b_{2k} \\ \vdots \\ b_{nk}
        \end{pmatrix})
    \end{align*}
    故得证.
\end{enumerate}

事实上矩阵乘法有很多和数的乘法重要的不同,我们在此特别指出供读者参考:
\begin{enumerate}
    \item 矩阵乘法不一定满足交换律(即$AB$不一定等于$BA$,事实上随手写两个矩阵,很大的概率就是不交换的).因此实数的完全平方公式代入矩阵
    不一定成立,即很多时候$(A+B)^2=A^2+AB+BA+B^2\neq A^2+2AB+B^2$;
    
    \item 但是注意数量矩阵(即对角线上元素都相等,其余均为0,单位矩阵是其特例)和任何矩阵相乘都是可交换的,这一点在矩阵求幂时很有用;
    \item $A\neq O$且$B\neq O$不能推出$AB\neq O$.例如线性方程组$AX = 0$有非零解,若$B$的各列均为方程非零解,则$AB = O$.
    \item 消去律也不一定满足:即$AB = AC$不一定$A = B$.原因在于$AB=AC \implies A(B-C)=O$,由(2)可知不一定$B = C$.
\end{enumerate}

\subsection{矩阵多项式}
我们在线性空间中已经介绍过,我们一般用$\mathbf{F}[x]_{m+1}$表示数域$\mathbf{F}$上的次数最高为$m$的多项式全体,其中的元素我们
一般记为
\[p(x)=a_mx^m+a_{m-1}x^{m-1}+\cdots+a_1x+a_0,a_i\in\mathbf{F}(i=1,2,\cdots,m).\]
事实上这里的自变量不一定需要是一个数,也可以是线性映射或者矩阵.例如线性映射$\sigma:V\to V$构成的$m$次多项式可以记为
\[p(\sigma)=a_m\sigma^m+a_{m-1}\sigma^{m-1}+\cdots+a_1\sigma+a_0I\]
其中$\sigma^i$表示$\sigma$复合$i$次,$I$表示恒等映射.我们很容易说明当$\sigma$在$V$的一组基下矩阵表示为$A$时,
$p(\sigma)$在同一组基下的矩阵表示为
\[p(A)=a_mA^m+a_{m-1}A^{m-1}+\cdots+a_1A+a_0E,\]
其中$E$表示单位矩阵.由此我们便得到了矩阵多项式的定义,我们有如下几点需要强调:
\begin{enumerate}
    \item 这里我们要求$\sigma$是线性变换(即出发空间和到达空间一致),事实上也并不必要,只需出发空间和到达空间维数相同即可,因为我们的目的
    是保证矩阵的幂次可以定义(即$A$和$A$可乘,因此$A$的行列数一致);
    \item 上面的定义隐含:$\sigma^0$为恒等变换,$A^0=E$为单位矩阵;
    \item $A^kA^m=A^{k+m},\enspace (A^k)^m=A^{km}$,其中$A$为方阵,$k,m$为任意整数. 负整数对应于逆矩阵的情况,下一讲会作进一步解释.
\end{enumerate}
\begin{example}
    展开矩阵多项式$(A+\lambda E)^n$.
\end{example}
\begin{solution}

\end{solution}

\begin{example}
    设$f(x),g(x) \in \mathbf{F}[x],\enspace A,B \in \mathbf{M}_n(\mathbf{F})$. 证明:
    \[f(A)g(A)=g(A)f(A)\]
    \begin{enumerate}
        \item 如果$AB=BA$,则$f(A)g(B)=g(B)f(A)$;

        \item 设$f(x)=1+x+\cdots+x^{m-1}$,$g(x)=1-x$,$A=\begin{pmatrix}
            a & b \\ 0 & a
        \end{pmatrix}$,计算$f(A)g(A)$.
    \end{enumerate}
\end{example}
\begin{solution}

\end{solution}

\section{一组例题}
在介绍了矩阵乘法后,我们可以进一步审视线性映射矩阵表示的定义.我们来看一组初学时可能混淆或者不理解的例子,从而加深
对概念的理解:
\begin{example}\label{ex:7:矩阵表示2}
    设$A=\begin{pmatrix}1 & 0 & 2 \\ -1 & 2 & 1 \\ 1 & 2 & 5\end{pmatrix}$为两个三维线性空间之间的线性映射$\sigma$对应的矩阵,
    求$\sigma$的像空间和核空间.
\end{example}
(注:本题没有给出线性映射出发空间和到达空间的基,读者可以任意假设)

\begin{solution}
求解像空间和核空间,仍然是原先介绍的方法,虽然本题没有给出线性映射的直接定义,但矩阵表示也能给我们足够的信息.
我们设这一矩阵表示的线性映射为$\sigma$,且
\[(\sigma(\varepsilon_1),\sigma(\varepsilon_2),\sigma(\varepsilon_3))=(\alpha_1,\alpha_2,\alpha_3)A\]
根据线性映射矩阵表示的定义,我们知道矩阵表示就是线性映射在出发空间一组基下的像在到达空间一组基下的坐标按列排列,因此
\begin{gather*}
    \sigma(\varepsilon_1)=\alpha_1-\alpha_2+\alpha_3 \\
    \sigma(\varepsilon_2)=2\alpha_2+2\alpha_3 \\
    \sigma(\varepsilon_3)=2\alpha_1+\alpha_2+5\alpha_3
\end{gather*}
因此$\im\sigma=\spa(\alpha_1-\alpha_2+\alpha_3,2\alpha_2+2\alpha_3,2\alpha_1+\alpha_2+5\alpha_3)$,然后求解
极大线性无关组即可,结果为$\im\sigma=\spa(\alpha_1-\alpha_2+\alpha_3,2\alpha_2+2\alpha_3)$

这里求解极大线性无关组的方法我们可以回忆\autoref{ex:3:转化为坐标},我们先将三个向量转化为到达空间基下坐标,然后求解极大线性无关组,
最后把基添加回来即可.实际上我们会发现,这里的三个坐标就是矩阵$A$的三个列向量(因为矩阵表示就是线性映射在出发空间一组基下的像
在到达空间一组基下的坐标按列排列),因此我们只需要求解矩阵$A$的列向量的极大线性无关组($(1,-1,1),(0,2,2)$)然后再将到达空间的基添加回来即可.

然后求解核空间,我们设$\sigma(\varepsilon)=0$,将$\varepsilon$写成出发空间基的表示后
事实上就是\autoref{eq:7:方程组与核空间2}的形式,我们已说明这一形式与
\autoref{eq:7:方程组与核空间1}等价,因此我们只需求解$AX=0$然后代回出发空间的基即可,
最终结果为$\ker\sigma=\spa(4\varepsilon_1+3\varepsilon_2-2\varepsilon_3)$.
\end{solution}

总结一下,此类题目求解像空间实际上就是求出矩阵列向量的极大线性无关组,然后记得把基添加回来.求解核空间只需
求解齐次线性方程组$AX=0$即可.

\begin{example}\label{ex:7:矩阵表示3}
    已知3阶矩阵$A=\begin{pmatrix}
        1 & 0 & 1 \\ 0 & -1 & 0 \\ -1 & 1 & -1
    \end{pmatrix}$. 定义$\mathbf{F}^{3 \times 3}$上的线性变换$\sigma(X)=AX,\enspace X \in \mathbf{F}^{3 \times 3}$.
    求$\sigma$的像和核.
\end{example}
\begin{solution}

\end{solution}

实际上,\autoref{ex:7:矩阵表示2}和\autoref{ex:7:矩阵表示3}都属于已知映射求像和核的题目,求解方法仍然是原先介绍的方法,
只是\autoref{ex:7:矩阵表示2}没有像\autoref{ex:7:矩阵表示1}或\autoref{ex:7:矩阵表示3}给出了线性映射的定义,而是给出矩阵表示,
但这也完全不影响我们的求解.
% TODO 编号系统 autoref

\vspace{2ex}
\centerline{\heiti \Large 内容总结}

\vspace{2ex}

\centerline{\heiti \Large 习题}
\vspace{2ex}
{\kaishu }
\begin{flushright}
    \kaishu

\end{flushright}
\centerline{\heiti A组}
\begin{enumerate}
    \item 设$\sigma: V_1\to V_2$是线性映射,证明:$\sigma(W_1)$和$\sigma^{-1}(W_2)$分别是$V_2$和$V_1$的子空间
\end{enumerate}
\centerline{\heiti B组}
\begin{enumerate}
    \item
\end{enumerate}
\centerline{\heiti C组}
\begin{enumerate}
    \item
\end{enumerate}
