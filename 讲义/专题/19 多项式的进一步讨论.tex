\chapter{多项式的进一步讨论}

在前面的讲解中我们讨论了如何利用核空间的性质将线性空间分解为若干个线性变换的不变子空间,得到广义特征子空间和分块对角矩阵的结论. 从本讲起我们希望变换一个角度,从多项式出发推导出这一结论,并由此出发深入讨论多项式与相似标准形的关联.

为了接下来讨论的方便,我们首先介绍一个接下来常用的一类特殊的多项式,它将线性变换(矩阵)代入后多项式值为0. 这样的多项式我们称为线性变换(矩阵)的零化多项式,我们的严谨定义如下:
\begin{definition}[零化多项式] \index{duoxiangshi!linghua@零化 (annihilating polynomial)}
    我们有线性变换和矩阵的零化多项式定义如下:
    \begin{enumerate}
        \item 设$T\in \mathcal{L}(V)$,若$p\in\mathbf{F}[x]$使得$p(T)=0$,则称$p$为$T$的一个\term{零化多项式};

        \item 设$A\in\mathbf{F}^{n\times n}$,若$p\in\mathbf{F}[x]$使得$p(A)=0$,则称$p$为$A$的一个零化多项式.
    \end{enumerate}
\end{definition}

一个显然的事情是,如果一个多项式是线性变换$T$的零化多项式$p$,那么它一定也是$T$在任意一组基下表示矩阵$A$的零化多项式,这是因为线性变换的多项式$p(T)=0$的矩阵表示是$p(A)=0$.

\section{特征多项式 \quad Hamilton-Cayley 定理}

接下来我们首先讨论线性变换的特征多项式,事实上我们在不变子空间一讲中实际上已经提到过矩阵的特征多项式,这里我们将给出线性变换的相关定义并讨论二者关联:
\begin{definition}[特征多项式] \index{duoxiangshi!tezheng@特征 (characteristic polynomial)}
    设$V$是复向量空间,$\sigma\in \mathcal{L}(V)$. 令$\lambda_1,\ldots,\lambda_m$表示$\sigma$的所有互异特征值,重数(即对应的广义特征子空间的维数)分别为$d_1,\ldots,d_m$,则多项式
    \begin{equation}\label{eq:21:线性变换特征多项式}
        p(z)=(z-\lambda_1)^{d_1}\cdots(z-\lambda_m)^{d_m}
    \end{equation}
    称为$\sigma$的\term{特征多项式}.
\end{definition}

根据这一定义,我们有两个直接的结论:
\begin{enumerate}
    \item $\sigma$的特征多项式的次数为$\dim V$,因为\autoref{thm:17:广义特征性质} \ref*{item:17:广义特征性质:2} 保证了$\displaystyle\sum_{i=1}^m d_i=\dim V$;

    \item $\sigma$的特征多项式的零点恰为$\sigma$的全部特征值,这是由上述定义决定的.
\end{enumerate}

\begin{example}
    设$V$是复向量空间,$V_1,\ldots,V_m$都是$V$的非零子空间使得$V=V_1\oplus\cdots\oplus V_m$. 设$\sigma\in \mathcal{L}(V)$,每个$V_j$在$\sigma$下不变. 对每个$j$,令$p_j$表示$\sigma|_{V_j}$的多项式. 证明:$\sigma$的特征多项式为$p_1\cdots p_m$.
\end{example}

\begin{proof}

\end{proof}

设$\sigma\in\mathcal{L}(V)$在任一组基下的表示矩阵为$A$,则$\sigma$和$A$的特征值是完全一样的,因此特征值与其重数都是一致的. 我们回忆不变子空间一讲关于矩阵特征多项式的\autoref{thm:18:特征多项式展开} 并做因式分解:
\begin{equation}\label{eq:21:矩阵特征多项式}
    f(\lambda)=|\lambda I-A|=(\lambda-\lambda_1)^{r_1}\cdots(\lambda-\lambda_m)^{r_m}
\end{equation}
由于\autoref{eq:21:线性变换特征多项式} 和\autoref{eq:21:矩阵特征多项式} 的$k$重根都表示$k$重特征值,且$\sigma$和$A$的特征值及其重数一致,因此我们可以得到$d_i=r_i(i=1,\ldots,m)$. 注意到$d_i$是基于广义特征子空间维数定义的代数重数,$r_i$是基于矩阵特征多项式解的重数定义的代数重数,因此两个代数重数的定义也在此处统一了. 且多项式的$k$重根即为$k$重特征值,特征值的重数也就称为代数重数. 因此此后我们不再区分两种特征多项式和两种代数重数的定义.

\begin{corollary}
    \begin{enumerate}
        \item 线性变换的特征多项式等于矩阵的特征多项式;
        \item 相似的矩阵有相同的特征多项式.
    \end{enumerate}
\end{corollary}

在上面的讨论中我们依据广义特征子空间的维数定义特征多项式,接下来我们希望变换思路,利用特征多项式的定义推导广义特征子空间的相关结论,从而将17讲开头提到的三角形中多项式的部分补全. 事实上,接下来要讨论的思路是一般的高等代数教材中惯用的思路,这一正一逆的思路在某种程度上也体现出两个研究体系的等价性.

我们首先要引入 Hamilton-Cayley 定理. 我们的动机是获得零化多项式,使我们向着目标前进. 观察矩阵$A=\begin{pmatrix}
        1 & 2 \\ 0 & -1
    \end{pmatrix}$,我们容易验证$A^2-I=0$,因此$\lambda^2-1$是$A$的一个零化多项式,同时我们发现这是$A$的特征多项式,因此我们可以猜想,是否对于所有的矩阵都有特征多项式是零化多项式呢?事实上,这就是著名的 \term{Hamilton-Cayley 定理}:
\begin{theorem}[Hamilton-Cayley 定理] \label{thm:21:HC} \index{Hamilton@Hamilton-Cayley 定理 (Cayley-Hamilton theorem)}
    设$V$是复向量空间,$\sigma\in \mathcal{L}(V)$. 令$q$表示$\sigma$的特征多项式,则$q(\sigma)=0$.
\end{theorem}
定理的证明我们将从前面讨论的三个相似标准形:对角矩阵,上三角矩阵和分块对角矩阵三个角度给出三个证明. 通过这三个证明我们可以体会到标准形与多项式背后的联系.

\begin{enumerate}
    \item 利用对角矩阵

          为了介绍这一角度的证明,我们将会回顾数学分析或拓扑学中学习的稠密性的定义:
          \begin{definition}

          \end{definition}

          \begin{lemma}

          \end{lemma}

          \begin{proof}

          \end{proof}

    \item 利用上三角矩阵

          \begin{proof}

          \end{proof}

    \item 利用分块对角矩阵

          \begin{proof}

          \end{proof}
\end{enumerate}

\section{特征多项式与标准形}
\subsection{特征多项式唯一分解与广义特征子空间}
接下来我们便可以利用这一定理继续我们逆向推导的过程. 我们的目标同样是找到能在直和后得到原空间的不变子空间的分解方式(即找到广义特征子空间). 实际上,我们可以利用\autoref{thm:17:裴蜀定理} 得到以下关键结论:
\begin{theorem} \label{thm:21:多项式分解与核空间直和}
    设$\sigma\in \mathcal{L}(V)$,且在$\mathbf{F}[x]$中有$p=p_1p_2$,且$p_1,p_2$互素,则有
    \[\ker p(\sigma)=\ker p_1(\sigma)\oplus\ker p_2(\sigma).\]
\end{theorem}

\begin{proof}

\end{proof}

为了得到广义特征子空间的定义,我们还需要将这一定理推广到因式更多的情况,证明只需要依照\autoref{thm:21:多项式分解与核空间直和} 然后进行数学归纳法即可,此处不再赘述:
\begin{theorem} \label{thm:21:多项式分解与核空间直和2}
    设$\sigma\in \mathcal{L}(V)$,且在$\mathbf{F}[x]$中有$p=p_1p_2\cdots p_s$,且$p_1,p_2,\ldots,p_s$两两互素,则有\[\ker p(\sigma)=\ker p_1(\sigma)\oplus\ker p_2(\sigma)\oplus\cdots\oplus\ker p_s(\sigma).\]
\end{theorem}

这一定理表明,将多项式分解为互素的多项式乘积,原多项式作用于线性变换的核空间等于分解后各个互素因式作用于线性变换的核空间的直和. 我们结合 Hamilton-Cayley 定理,如果$p$是$\sigma$的特征多项式,故$p(\sigma)=0$,则$\ker p(\sigma)$就是全空间$V$. 接下来我们将特征多项式分解为互素因式乘积,有
\[p(\lambda)=(\lambda-\lambda_1)^{r_1}(\lambda-\lambda_2)^{r_2}\cdots(\lambda-\lambda_m)^{r_m},\]
其中$\lambda_1,\ldots,\lambda_m$为$\sigma$的所有互异特征值,$r_1,\ldots,r_m$为特征值的重数. 然后由于分解的因式显然是两两互素的,因此根据\autoref{thm:21:多项式分解与核空间直和2},我们有
\[\ker p(\sigma)=V=\ker (\sigma-\lambda_1I)^{r_1}\oplus\cdots\oplus\ker (\sigma-\lambda_mI)^{r_m},\]
这或许就是一种巧合,我们从多项式的角度也推导出了和广义特征子空间相近的结论. 我们回顾\autoref{thm:17:广义特征性质} (2):
\[V=G(\lambda_1,\sigma)\oplus\cdots\oplus G(\lambda_m,\sigma),\]
其中$G(\lambda_i,\sigma)=\ker (\sigma-\lambda_iI)^{\dim V}$,这与上式的形式是类似的,但这里将广义特征子空间定义中$(\sigma-\lambda I)$由于核空间扩张所需的幂次降低了. 除此之外,$\ker (\sigma-\lambda_iI)^{r_i}\enspace(i=1,2,\ldots,m)$也是$\sigma$的不变子空间,因此我们也能得到分块对角矩阵的标准形,并且\autoref{thm:17:广义特征性质} 的其他结论也可以基于此得到,此处不再赘述.
\begin{example}
    设$\sigma\in \mathcal{L}(V)$,$p(z)=a_nx^n+\cdots+a_1x\in\mathbf{F}[x]$是$\sigma$的一个零化多项式,其中$a_1\neq 0$,证明:
    \[V=\ker \sigma\oplus\im\sigma.\]
\end{example}

\begin{proof}

\end{proof}

\subsection{初等因子分解与若当标准形}
事实上,学习了若当标准形后,我们已不满足于基于广义特征子空间的分解. 我们知道同一个特征值可能对应多个若当块,因此每个广义特征子空间还能继续分解,分解成可以得到若当块的由循环基张成的循环子空间的直和. 我们可以根据\autoref{cor:18:若当基存在}形式化地写出这一分解:设$V$是$\mathbf{C}$上的有限维线性空间,$T\in\mathcal{L}(V)$,设每个特征值$\lambda_i(i=1,\cdots,n)$对应的广义特征子空间$K_{\lambda_i}$可以由$s_i$组若当基$B_{i1},B_{i2},\cdots,B_{is_i}$张成,其中每一组基都可以表达为:
\[B_{ij}=\{(T-\lambda_iI)^{r_{ij}-1}v_{ij},(T-\lambda_iI)^{r_{ij}-2}v_{ij},\cdots,v_{ij}\},\]
事实上$r_{ij}$就是这组循环基的长度. 我们设由$B_{ij}$张成的循环子空间为$C_{ij}$,则有
\[K_{\lambda_i}=C_{i1}\oplus C_{i2}\oplus\cdots\oplus C_{is_i}=\mathop{\oplus}\limits_{j=1}^{s_i} C_{ij},\]
再根据\autoref{thm:17:广义特征性质}\ref{item:17:广义特征性质:2},我们有
\begin{equation} \label{eq:19:循环子空间分解}
    V=\mathop{\oplus}\limits_{i=1}^n K_{\lambda_i}=\mathop{\oplus}\limits_{i=1}^n\mathop{\oplus}\limits_{j=1}^{s_i} C_{ij}.
\end{equation}

简而言之,我们将$V$分解为了一系列循环子空间的直和,每个循环子空间的循环基对应一个若当块. 进一步地,我们也希望探究这一分解与多项式之间的联系. 这时我们需要回忆起\autoref{ex:14:多项式域扩张},帮助我们得到如下结论:
\begin{theorem} \label{thm:19:循环子空间同构于商空间}
    设$V$是$\mathbf{C}$上的有限维线性空间,$T\in\mathcal{L}(V)$. 设
    \[B_{ij}=\{(T-\lambda_iI)^{r_{ij}-1}v_{ij},(T-\lambda_iI)^{r_{ij}-2}v_{ij},\cdots,v_{ij}\}\]
    是$V$的一组循环基,$C_{ij}$是由$B_{ij}$张成的循环子空间. 则有
    \[C_{ij}\cong \mathbf{F}[\lambda]/((\lambda-\lambda_i)^{r_{ij}}).\]
\end{theorem}

事实上证明这一定理是显然的,因为根据\autoref{ex:14:多项式域扩张},我们知道$\mathbf{F}[\lambda]/((\lambda-\lambda_i)^{r_{ij}})$是一个$r_{ij}$维线性空间,与$C_{ij}$相同. 但为什么我们在这里选择这一特别的商空间呢?事实上,回顾\autoref{ex:14:多项式域扩张}的解答,我们发现,这一商空间的基是
\[\overline{(\lambda-\lambda_i)^{r_{ij}-1}},\overline{(\lambda-\lambda_i)^{r_{ij}-2}},\cdots,\overline{1},\]
其中$\overline{a}=a+(\lambda-\lambda_i)^{r_{ij}}$. 我们很容易发现这一基的形式与$B_{ij}$的形式完全是一一对应的:$(T-\lambda_iI)^kv_{ij}$对应$\overline{(\lambda-\lambda_i)^k}$,事实上,我们不难发现以下两个关系:
\begin{gather*}
    B_{ij}=\{p(T)v_{ij}\mid \deg p\leqslant r_{ij}-1\},\\
    \mathbf{F}[\lambda]/((\lambda-\lambda_i)^{r_{ij}})=\{\overline{p(\lambda)}\mid \deg p\leqslant r_{ij}-1\}.
\end{gather*}
因此我们能构造出一个非常自然的同构映射$\phi:C_{ij}\to \mathbf{F}[\lambda]/((\lambda-\lambda_i)^{r_{ij}})$,使得$\phi(p(T)v_{ij})=\overline{p(\lambda)},\enspace\forall \deg p\leqslant r_{ij}-1$.

我们可以做进一步的观察,对于限制在$C_{ij}$上的线性映射$T|_{C_{ij}}$,我们利用同构映射将其自然地变为$\mathbf{F}[\lambda]/((\lambda-\lambda_i)^{r_{ij}})$上的线性映射$T'$:由于$T|_{C_{ij}}(p(T)v_{ij})=T\cdot p(T)v_{ij}$,根据同构我们有$T'(p(\lambda))=\overline{\lambda\cdot p(\lambda)}$.

事实上,由于映射$T'$和$T|_{C_{ij}}$是完全由它们对应的线性空间的同构映射$\phi$对应的,因此我们很自然地能知道,$T'$在基$\overline{(\lambda-\lambda_i)^{r_{ij}-1}},\overline{(\lambda-\lambda_i)^{r_{ij}-2}},\cdots,\overline{1}$下的矩阵实际上与$T|_{C_{ij}}$在基$B_{ij}$下的矩阵是完全一致的,即是一个若当块. 如果读者不相信,我们将这一结果的计算性验证留作习题.

总而言之,经过上面一系列看似繁杂的推演,我们事实上是想说明$C_{ij}$和$\mathbf{F}[\lambda]/((\lambda-\lambda_i)^{r_{ij}})$的同构是非常自然的,我们完全可以将$\mathbf{F}[\lambda]/((\lambda-\lambda_i)^{r_{ij}})$也视为一个循环子空间,并且我们也将循环基中出现的$(T-\lambda_i I)^k$中的多项式$(\lambda-\lambda_i)^k$提取出来. 利用这一同构,我们可以将\autoref{eq:19:循环子空间分解}写为
\begin{equation} \label{eq:19:初等因子分解}
    V=\mathop{\oplus}\limits_{i=1}^n\mathop{\oplus}\limits_{j=1}^{s_i} \mathbf{F}[\lambda]/((\lambda-\lambda_i)^{r_{ij}}).
\end{equation}
注意,上式的等号实际上代表同构,在不引起歧义的情况下,我们可以这样简化书写. 事实上,这一分解直接引入了一个新的概念:初等因子(组).
\begin{definition}
    设$V$是$\mathbf{C}$上的有限维线性空间,$T\in\mathcal{L}(V)$. 我们称\autoref{eq:19:初等因子分解}中出现的多项式$(\lambda-\lambda_i)^{r_{ij}}$为$T$的一个\term{初等因子},并称$T$的全体初等因子为其\term{初等因子组}.
\end{definition}

对于像初等因子这样,是一次多项式(或者将来数域不再是复数域时的一般不可约多项式)的次方的形式的多项式,我们称其为准素多项式. 这是因为不可约多项式可以类比为整数中的素数,而初等因子只是这些``素多项式''的次方,因此只有单个``素多项式''为其因子,非常简单,并且在代数中也比较常见.

除此之外需要说明的是,引入初等因子组的一个重要目的就是希望说明它是我们判断两个矩阵是否相似的全系不变量. 事实上,这一句话出现了两个疑点:
\begin{enumerate}
    \item 什么是全系不变量?
    \item 我们上面定义的是线性映射的初等因子组,那么什么是矩阵的初等因子组呢?
\end{enumerate}

\begin{theorem} \label{thm:19:初等因子组是相似全系不变量}
    初等因子组是矩阵相似的全系不变量,即两个矩阵相似当且仅当它们有相同的初等因子组.
\end{theorem}

对于\autoref{eq:19:初等因子分解}给出的初等因子分解,我们知道它实际上与循环子空间分解是完全等价的,于是我们很自然地会有如下三个问题:
\begin{enumerate}
    \item 为什么叫初等因子分解?这是因为这样的分解已经是最细的分解了吗(即不能再进一步分解使其与另一种循环子空间分解同构吗)?
    \item 是否还有其它更粗粒度的分解与其它可能的循环子空间分解对应?即我们是否能结合一些初等因子,得到更大的循环子空间?是否有最粗粒度的分解?
    \item 我们知道在复数域上多项式是可以完全分裂成一次多项式的乘积的,但是在其它一些域,例如实数域或者有理数域上,多项式无法完全分裂,那么将会得到二次多项式或者其它形式的初等因子,此时初等因子分解是否还有意义?
\end{enumerate}
% TODO: 区分两种循环子空间
关于以上三个问题,我们都将在下一讲中给出解答. 事实上第二、第三个问题的回答分别对应着第一、第二有理标准形.

\section{极小多项式及其性质}
下面我们希望将广义特征子空间定义中$(\sigma-\lambda I)$的幂次进一步降低到下界(即找到最低的幂次使得核空间停止增长),这需要引入极小多项式的概念.
\begin{definition}
    我们有如下线性变换和矩阵的极小多项式定义:
    \begin{enumerate}
        \item 设$\sigma\in \mathcal{L}(V)$,则$\sigma$的极小多项式是唯一一个使得$p(\sigma)=0$的次数最小的首一多项式;

        \item 设$A\in\mathbf{F}^{n\times n}$,则$A$的极小多项式是唯一一个使得$p(A)=0$的次数最小的首一多项式.
    \end{enumerate}
\end{definition}
这一定义的合理性需要下述定理保证,我们只证明线性变换的角度,矩阵实际上只需要将定理和证明中的线性变换替换为矩阵即可:
\begin{theorem}\label{thm:21:极小多项式存在}
    设$\sigma\in \mathcal{L}(V)$,则存在唯一一个次数最小的首一多项式$p$使得$p(\sigma)=0$.
\end{theorem}

\begin{proof}

\end{proof}

这一定理的证明表明$V$上每个线性变换的极小多项式的次数最多为$(\dim V)^2$,而若$V$为复向量空间时,由 Hamilton-Cayley 定理我们知道极小多项式的次数最多为$\dim V$,事实上实空间也有这样的结论,我们将在实空间上的线性变换一讲中讨论.

如果需要计算极小多项式,我们可以给出一个算法化的描述. 对于$m=1,2,\ldots$,我们相继考虑线性方程组
\[a_0M(I)+a_1M(\sigma)+\cdots+a_{m-1}M(\sigma^{m-1})+M(\sigma^m)=0,\]
直到这一方程组有一个解$a_0,a_1,\ldots,a_{m-1}$,此时$a_0,a_1,\ldots,a_{m-1},1$即为极小多项式的次数.
\begin{example} \label{ex:21:最小多项式}
    求矩阵$A=\begin{pmatrix}
            0 & 0 & 0 \\ 1 & 0 & 2 \\ 2 & 1 & -1
        \end{pmatrix}$和$B=\begin{pmatrix}
            2 & 2 & 1 \\ 0 & 2 & -1 \\ 0 & 0 & -3
        \end{pmatrix}$的最小多项式.
\end{example}

\begin{solution}
    \begin{enumerate}
        \item

        \item
    \end{enumerate}
\end{solution}

下面我们给出一些简单线性变换/矩阵的极小多项式:
\begin{enumerate}
    \item 幂零线性变换:$N\in \mathcal{L}(V)$且$N^l=0$,但$N^{l-1}\neq 0$($l$称为幂零指数),极小多项式为$\lambda^l$;

    \item 幂等线性变换:$\sigma\in \mathcal{L}(V)$且$\sigma^2=\sigma$,极小多项式为$\lambda^2-\lambda$或$\lambda$或$\lambda-1$;

    \item 对合线性变换:$\sigma\in \mathcal{L}(V)$且$\sigma^2=I$,极小多项式为$\lambda^2-1$或$\lambda+1$或$\lambda-1$;

    \item 引入\term{若当块}\index{ruodangkuai@若当块 (Jordan block)}. 若域$\mathbf{F}$上的一个$r$级矩阵形如\[\begin{pmatrix}
                  a & 1 &        &   \\
                    & a & \ddots &   \\
                    &   & \ddots & 1 \\
                    &   &        & a
              \end{pmatrix}\]
          则称其为一个$r$级若当块(1级显然就是1阶矩阵),记作$J_r(a)$,其中$a$是对角线上元素. 不难得到其极小多项式等于特征多项式$(\lambda-a)^r$.
\end{enumerate}

\subsection{极小多项式的性质}
我们利用多项式的带余除法以及 Hamilton-Cayley 定理可以得到下述简单的结论:
\begin{theorem}
    设$\sigma\in \mathcal{L}(V)$.
    \begin{enumerate}
        \item $q\in\mathbf{F}[x]$,则$q(\sigma)=0$当且仅当$q$是$\sigma$的极小多项式的多项式倍;

        \item 设$\mathbf{F}=\mathbf{C}$,则$\sigma$的特征多项式是$\sigma$的极小多项式的多项式倍.
    \end{enumerate}
\end{theorem}

\begin{proof}
    \begin{enumerate}
        \item

        \item
    \end{enumerate}
\end{proof}

在\autoref{ex:21:最小多项式} 中我们不难发现,两个矩阵的极小多项式和特征多项式根一致,实际上这是对任意线性变换(或矩阵)都成立的结论:
\begin{theorem} \label{thm:21:极小多项式与特征多项式相同根}
    设$\sigma\in \mathcal{L}(V)$,则$\sigma$的极小多项式的零点恰好是$\sigma$的特征值,即极小多项式与特征多项式在$\mathbf{F}$中有相同的根(重数可以不同).
\end{theorem}

\begin{proof}

\end{proof}

这一定理是非常重要的,它关系到下一小节关于多项式和标准形关系的讨论,且\autoref{ex:21:最小多项式} 也可以基于此有更快的解法:

\begin{solution}
    \begin{enumerate}
        \item

        \item
    \end{enumerate}
\end{solution}

除此之外,我们还可以得到一个推论:
\begin{corollary}
    \begin{enumerate}
        \item 线性变换的极小多项式等于矩阵的极小多项式;
        \item 相似的矩阵有相同的极小多项式.
    \end{enumerate}
\end{corollary}

\begin{proof}

\end{proof}

\subsection{极小多项式与对角化}

在最后一小节我们尝试将两种描述线性变换的角度(标准形和多项式)联系起来,主要的桥梁就是上一小节中讨论的极小多项式. 在前文讨论特征多项式诱导的不变子空间分解时,我们将广义特征子空间定义中需要求核空间的线性变换幂次降低,而依据\autoref{thm:21:极小多项式与特征多项式相同根} 以及特征多项式是极小多项式的倍式可知,这一幂次还可以进一步降低:
\begin{theorem} \label{thm:21:极小多项式与分解}
    设$\sigma\in \mathcal{L}(V)$,$\sigma$的极小多项式为$p=(\lambda-\lambda_1)^{s_1}\cdots(\lambda-\lambda_m)^{s_m}$,则有
    \[\ker p(\sigma)=V=\ker (\sigma-\lambda_1I)^{s_1}\oplus\cdots\oplus\ker (\sigma-\lambda_mI)^{s_m}.\]
\end{theorem}

\begin{proof}

\end{proof}

通过\autoref{thm:21:极小多项式存在} 我们知道,极小多项式的因式次数无法继续降低,否则不为零化多项式,因此它也给出了广义特征子空间定义中需要求核空间的线性变换的幂次为何值时,核空间会停止增长,并且这是一个下界,基于此我们更进一步地理解了极小多项式因式次数的含义.

实际上我们也可以逆向思考,如果我们已知空间的不变子空间分解,我们应当如何求解极小多项式. 实际上这一结论是很直观的,答案是各个不变子空间的极小多项式的最小公倍式,严谨叙述如下:
\begin{theorem}
    设$\sigma\in\mathcal{L}(V)$,如果$V$能分解成$\sigma$的一些非平凡不变子空间的直和:
    \[V=U_1\oplus\cdots\oplus U_m,\]
    且$\sigma\vert_{U_i}$的极小多项式为$p_i$,则$\sigma$的极小多项式为
    \[p=\lcm(p_1,\ldots,p_m).\]
    其中$\lcm(p_1,\ldots,p_m)$表示$p_1,\ldots,p_m$的最小公倍式.
\end{theorem}

\begin{proof}

\end{proof}

这一结论的应用或许并不直接,但如果我们考虑线性变换在不变子空间直和分解下的分块对角矩阵,那么这一分块对角矩阵的极小多项式实际上就等于各个分块的极小多项式的最小公倍式.
\begin{example} \label{thm:21:若当形矩阵极小多项式}
    我们在此继续引入\term{若当形矩阵}\index{ruodangxingjuzhen@若当形矩阵 (Jordan matrix)},即由若干个若当块组成的分块对角矩阵. 设$A$为若当形矩阵,$A=\diag(J_{r_1}(a),\ldots,J_{r_s}(a),J_{t_1}(b),\ldots,J_{t_m}(b))$,其中$r_1\leqslant\cdots\leqslant r_s$,$t_1\leqslant\cdots\leqslant t_m$,则$A$的极小多项式$p$为$\lcm((\lambda-a)^{r_1},\ldots,(\lambda-a)^{r_s},(\lambda-b)^{t_1},(\lambda-b)^{t_m})$,即为$(\lambda-a)^{r_s}(\lambda-b)^{t_m}$. 实际上,这一结论还可以进一步推广,但描述较为繁杂,读者只需从此例理解基本思想即可.
\end{example}
在\autoref{thm:21:极小多项式与分解} 中我们了解了极小多项式中因子幂次与广义特征子空间的关联. 加入极小多项式的各个因式的次数均为1,这与可对角化线性变换的不变子空间分解是一致的!因此我们可以得到下面的结论:
\begin{theorem}
    设$\sigma\in \mathcal{L}(V)$,$\sigma$可对角化当且仅当$\sigma$的极小多项式能分解成不同的一次因式的乘积.
\end{theorem}

\begin{proof}

\end{proof}

这给出了线性变换可对角化的另一等价条件,基于此,不变子空间一讲中给出矩阵多项式判断可对角化的习题都可以``秒杀'',例如幂等矩阵、对合矩阵可对角化,但幂零矩阵除非自身为0否则一定不可对角化,高于1阶的若当块矩阵一定不可对角化,包含高于1阶的若当块矩阵的若当形矩阵也一定不可对角化.

我们也可从矩阵的角度来理解. 例如幂等矩阵$A$满足$A^2=A$,根据多项式诱导的不变子空间分解,我们很容易得到$A$为幂等矩阵的充要条件为$r(A)+r(A-E)=n$,其它对合矩阵等情况各位同学也可以自己写出等价条件,虽然形式上可以千变万化,但实质就是多项式诱导的不变子空间分解.

除此之外,联系多项式互素分解与不变子空间分解的对应关系,这也表明线性变换可对角化当且仅当其各个广义特征子空间就是其特征子空间,即满足代数重数等于几何重数. 或者说$\sigma$的每个广义特征向量都是其特征向量.
\begin{example}
    证明:设$\sigma\in \mathcal{L}(V)$,若$\sigma$可对角化,则对于$\sigma$的任意非平凡不变子空间$U$,都有$\sigma\vert_U$可对角化.
\end{example}

\begin{proof}

\end{proof}

\begin{example}
    已知某个实对称矩阵$A$的特征多项式为$\lambda^5+3\lambda^4-6\lambda^3-10\lambda^2+21\lambda-9$,求$A$的极小多项式.
\end{example}

\begin{solution}

\end{solution}

\begin{example}
    设$V$为$n$阶方阵构成的线性空间,$\sigma\in \mathcal{L}(V),\enspace \forall A\in V,\enspace \sigma(A)=2A-3A^{\mathrm{T}}$.
    \begin{enumerate}
        \item 求$\sigma$的特征值;

        \item 证明:$\sigma$可对角化.
    \end{enumerate}
\end{example}

\begin{solution}
    \begin{enumerate}
        \item

        \item
    \end{enumerate}
\end{solution}

我们需要补充说明一点,虽然矩阵相似不随数域改变而改变,但可对角化与数域有关. 例如实矩阵$A$的极小多项式为$\lambda^3-1$,在它在实数域上无法分解为互素一次因式的乘积,复数域上则可以,这表明$A$在实数域上不可对角化,但在复数域上可以.

最后我们给出一个特征多项式等于极小多项式的充分必要条件:
\begin{theorem} \label{thm:19:特征多项式等于极小多项式}
    设$V$是有限维线性空间,$T\in \mathcal{L}(V)$,则$T$的特征多项式等于极小多项式当且仅当$V$是循环子空间,即可以由一组循环基张成.
\end{theorem}
\begin{proof}

\end{proof}

\subsection{若当标准形与极小多项式}
接下来我们讨论若当标准形与极小多项式之间的关系. 实际上,我们早在\autoref{thm:21:若当形矩阵极小多项式} 中求解了若当形矩阵的极小多项式. 我们来看一个例子进行应用:
\begin{example}
    设$A$为$n$阶方阵且极小多项式次数为$n$,则$A$的若当标准形中各个若当块的主对角线元素互不相同.
\end{example}

\begin{proof}

\end{proof}

% 从此例中我们可以产生一个直觉,即极小多项式的次数等于特征多项式时每个特征值对应若当块只有一个,若极小多项式某个特征值对应的因式次数下降,那么将会使得若当块分为多个(出现多个对角线上元素相同的若当块),当次数下降到1时则均为一阶若当块(一阶矩阵). 这样的叙述十分抽象,因为《线性代数应该这样学》并没有选择讲解如何从这一角度求解若当标准形,感兴趣的同学可以参考关于$\lambda$-矩阵的行列式因子、不变因子、初等因子三因子理论,从这一理论出发可以理解之前\hyperref[sect:22:若当标准形的另一求法]{若当标准形的另一求法}小节的结论.

当然,我们仅基于直觉也可以得到一些常用的结论,如下面的例子:
\begin{example}
    回答以下两个问题:
    \begin{enumerate}
        \item 设$N\in \mathcal{L}(V)$幂零,证明:$N$的极小多项式是$z^{m+1}$,其中$m$是$N$的若当标准形中紧位于对角线上方的直线上连续出现的1的最大个数;

        \item 设$p,q\in\mathbf{C}[x]$是具有相同零点的首一多项式,$q$是$p$的多项式倍,证明:存在$\sigma\in \mathcal{L}(\mathbf{C}^{\deg q})$使得$\sigma$的特征多项式为$q$且极小多项式为$p$.
    \end{enumerate}
\end{example}

\begin{proof}
    \begin{enumerate}
        \item

        \item
    \end{enumerate}
\end{proof}

\vspace{2ex}
\centerline{\heiti \Large 内容总结}

\vspace{2ex}
\centerline{\heiti \Large 习题}

\vspace{2ex}
{\kaishu }
\begin{flushright}
    \kaishu

\end{flushright}

\centerline{\heiti A组}
\begin{enumerate}
    \item
\end{enumerate}

\centerline{\heiti B组}
\begin{enumerate}
    \item 证明:\autoref{thm:19:循环子空间同构于商空间}之后定义的线性映射$T'$在基$\overline{(\lambda-\lambda_i)^{r_{ij}-1}},\overline{(\lambda-\lambda_i)^{r_{ij}-2}},\cdots,\overline{1}$下的矩阵实际上就是一个若当块.
\end{enumerate}

\centerline{\heiti C组}
\begin{enumerate}
    \item
\end{enumerate}
