\chapter{朝花夕拾}

我为这一讲取了一个很有诗意的名字,用以说明这一节我们重在对往日所学知识的回忆.我们一路走来为了能进入这一讲
做了太多准备工作,包括一开始难以理解的抽象空间和映射,以及后面具象但充满技巧性的矩阵与行列式.但``吹尽狂沙
始到金'',这一讲希望读者跟随我们的脚步,回忆起这一路上学习的核心概念和定理,为我们线性方程组一般理论的讨论
画下一个完美的句点.

在解的一般理论中,我们将首先讨论有无解以及有解时唯一解和无穷解对应的情况,然后分别讨论齐次与非齐次线性方程组
解的结构分别具有什么特征.
除此之外,我们也将利用一般理论讨论一些秩有关的等式和不等式,也将讨论线性方程组中一些特殊的题型.愿读者在每一个
定理的证明和每一个例题的解答中,都能进一步体会之前所学的知识,加深理解,有所感悟.

\section{线性方程组解的一般理论}
\subsection{线性方程组解的一般理论}
\begin{theorem} \textbf{\heiti 线性方程组有解的充要条件} \label{thm:15:有解条件}

    线性方程组有解的充分必要条件是其系数矩阵与增广矩阵有相同的秩.
\end{theorem}
定理的证明非常简单,这里简要介绍思路:将方程组视为$x_1\vec{\beta}_1+x_2\vec{\beta}_2+\cdots+x_n\vec{\beta}_n=\vec{b}$
($\beta_i$就是系数矩阵的第$i$列),
则有解的条件为$\vec{b}$可以被$\vec{\beta}_1,\ldots,\vec{\beta}_n$线性表示,这等价于
向量组$(\vec{\beta}_1,\ldots,\vec{\beta}_n)$与$(\vec{\beta}_1,\ldots,\vec{\beta}_n,\vec{b})$等价,故定理成立.

\begin{theorem} \label{thm:15:方程组解}
    当方程组有解时(注意这个前提),以下定理成立:
    \begin{enumerate}
        \item 如果它的系数矩阵$A$的秩等于未知量的数目$n$,则方程组有唯一解;

        \item 如果$A$的秩小于$n$,则方程组有无穷多个解.
    \end{enumerate}
\end{theorem}
实际上,这一定理就是\autoref{thm:13:Cramer}结论的一部分,因此我们不再赘述其证明.实际上,通过上面两个定理我们首先了解了
线性方程组有无解的一般准则,然后讨论了有解前提下唯一解、无穷解对应于什么情况.事实上,有关线性方程组解的情况的讨论至此文
意已尽.无论是理论层面或是解决题目的方面,这两个定理都为我们提供了足量的信息.
\begin{example}
    设$n$阶矩阵$A$的行列式$|A|\neq 0$,记$A$的前$n-1$列形成的矩阵为$A_1$,$A$的第$n$列为$b$,
    问:线性方程组$A_1\vec{X}=b$是否有解?
\end{example}
\begin{solution}

\end{solution}

\subsection{齐次线性方程组解的一般理论}
接下来我们将分别针对齐次和非齐次线性方程组的情况展开关于解的结构性质的讨论.
回顾\autoref{ex:2:常见子空间}中的讨论,对于齐次线性方程组$A\vec{X}=0$,我们有:
\begin{theorem}
    齐次线性方程组$A\vec{X}=0$的解空间为$\mathbf{R}^n$的子空间.
\end{theorem}
这一结论告诉我们,齐次线性方程组解构成线性空间,这是一个重要的结构性结论.在确认其为线性空间后,
我们来研究该线性空间的基本性质.首先是由此引出的关于基础解系的概念.
基础解系即为齐次线性方程组解空间的一组基,且这组基的每一个线性组合都是该方程组的解、
然后我们来研究这一空间的维数:
\begin{theorem}\label{thm:15:齐次维数}
    矩阵$A \in \mathbf{M}_{m \times n}(\mathbf{F})$,若$r(A) = r$,则该齐次线性方程组解空间维数为$n - r$.
\end{theorem}
事实上,本定理可以改写为类似于维数公式的形式,即
\begin{equation}\label{eq:15:齐次维数公式}
    r(A) + \dim N(A) = n.
\end{equation}
其中$N(A)$表示$A\vec{X}=0$的解空间,
区别在于维数公式中$A$应当替换为线性映射$\sigma$.

我们令$A$是线性映射$\sigma$在出发空间和到达空间基下的矩阵表示,根据矩阵的秩的定义,$r(A)=r(\sigma)$;又根据
\autoref{eq:7:方程组与核空间2}的讨论,$\ker\sigma$和$N(A)$之间是坐标的一一对应关系,因此$\dim N(A)=\dim\ker\sigma$.
因此我们有\autoref{eq:15:齐次维数公式}等价于$r(\sigma)+\dim\ker\sigma=n$,实际上这就是线性映射基本定理,
因此正确性是毋庸置疑的,因此与之等价的\autoref{eq:15:齐次维数公式}也成立,证毕.

我们可以用\autoref{thm:15:齐次维数}解决很多问题,下面是一个最简单的例子:
\begin{example}
    若$n$元齐次线性方程组$A\vec{X} = 0$的解都是$B\vec{X} = 0$的解. 证明:$r(B) \leqslant r(A)$.
\end{example}
\begin{proof}
    
\end{proof}

实际上在前面的讨论中,无论是\autoref{thm:15:有解条件}的证明还是\autoref{thm:15:齐次维数}的结论,都与列向量
组成的线性空间有关,仿佛从未出现过行向量有关的定理.事实上,我们将在未来讨论了内积空间正交性后展开对行向量空间
的讨论,现在囿于概念上的缺乏无法叙述相关定理.

\subsection{非齐次线性方程组解的一般理论}
回顾\autoref{ex:2:常见子空间}中的讨论我们发现,非齐次线性方程组的解不构成线性空间,但我们可以尝试将其
与齐次线性方程组解空间联系起来研究.对于非齐次线性方程组
\begin{equation} \label{eq:15:非齐次}
    x_1\vec{\beta}_1+x_2\vec{\beta}_2+\cdots+x_n\vec{\beta}_n=\vec{b}
\end{equation}
我们将$n$元齐次线性方程组
\begin{equation} \label{eq:15:齐次}
    x_1\vec{\beta}_1+x_2\vec{\beta}_2+\cdots+x_n\vec{\beta}_n=0
\end{equation}
称为其导出组,则我们有:
\begin{theorem}
    如果$n$元非齐次线性方程组有解,则它的解集$U=\{\gamma_0+\eta \mid \eta \in W\}$.
\end{theorem}
其中$\gamma_0$为\autoref{eq:15:非齐次} 的一个解(称为特解),$W$为\autoref{eq:15:齐次} 的解空间(\autoref*{eq:15:齐次} 的解称为通解).
对于通解+特解的结构,如果读者在此之前学习了商空间一节,那么我们就会发现$U$实际上就是$W$的一个仿射子集.当然如果没有学习相关概念,
我们可以想象一个3元非齐次线性方程$ax + by + cz = d$ 和齐次线性方程$ax + by + cz = 0$.
非齐次线性方程的解显然对应一个不过原点的平面,而齐次则过原点.
我们便可以认为是齐次线性方程解平面沿着特解对应的向量平移到非齐次线性方程的解平面,这便是这一结论的几何解释.同时我们可以得到下述结论:
\begin{enumerate}
    \item $n$元非齐次线性方程组 \ref*{eq:15:非齐次} 的两个解的差是它的导出组 \ref*{eq:15:齐次} 的一个解;
    \item $n$元非齐次线性方程组 \ref*{eq:15:非齐次} 的一个解与它的导出组 \ref*{eq:15:齐次} 的一个解之和仍是非齐次线性方程组 \ref*{eq:15:非齐次} 的一个解.
\end{enumerate}
\begin{proof}
    事实上,\ref*{eq:15:非齐次}可以写为$AX=b(b\neq 0)$,其中$A$为系数矩阵,$\beta_i$为$A$的第$i$列,同理,\ref*{eq:15:齐次}可以写为$AX=0$.
    \begin{enumerate}
        \item 设$\gamma_1,\gamma_2$分别是非齐次线性方程组 \ref*{eq:15:非齐次} 的两个解,则
        \[A(\gamma_1-\gamma_2)=b-b=0.\]
        \item 设$\gamma_1$是非齐次线性方程组 \ref*{eq:15:非齐次} 的一个解,$\eta_1$是齐次线性方程组 \ref*{eq:15:齐次} 的一个解,则
        \[A(\gamma_1+\eta_1)=b+0=b.\]
    \end{enumerate}
\end{proof}
实际上根据上述几何描述形象理解这两个结论也不困难.下面我们将通过一些例子进一步探讨非齐次线性方程组解的
结构问题:
\begin{example}
    若$X_0$是$AX=b$的一个特解,$X_1,\cdots,X_p$是$AX=0$的基础解系,
	证明:
	\begin{enumerate}[label=(\arabic*)]
        \item $X_0,X_1,X_2,\cdots,X_p$线性无关;
        \item $X_0,X_0+X_1,X_0+X_2,\cdots,X_0+X_p$线性无关;
        \item $AX=b$的任一个解$X$可表示为
        \[X=k_0X_0+k_1(X_0+X_1)+k_2(X_0+X_2)\cdots+k_p(X_0+X_p),\]
        其中$k_0+k_1+k_2+\cdots+k_p=1$.
    \end{enumerate}
\end{example}
\begin{proof}
    
\end{proof}

\begin{example}
    设$A$为$s \times n$矩阵,且$r(A)=r$,证明:非齐次线性方程组$AX=b$至多存在$n-r+1$个线性无关的解向量.
\end{example}
\begin{proof}
    
\end{proof}

\section{理论应用}
本节我们将综合线性方程组解的一般理论和之前所学的知识讨论一些秩的等式/不等式问题.
我们首先来看四个最为经典的问题:
\begin{example}
    利用线性方程组解的一般理论,证明以下命题:
    \begin{enumerate}
        \item 设$A,B$分别是$m \times n$和$n \times s$矩阵,且$AB=O$,证明:$r(A)+r(B)\leqslant n$;

        \item 设$A$是$m \times n$实矩阵,证明:$r(A^\mathrm{T}A)=r(A)$;

        \item 设$A,B$分别是$m \times n$和$n \times s$矩阵,则$r(AB)\leqslant\min\{r(A),r(B)\}$;

        \item $A^2=A \iff r(A)+r(E-A)=n$,$A^2=E \iff r(A+E)+r(A-E)=n$.
    \end{enumerate}
\end{example}
\begin{proof}
    \begin{enumerate}
        \item 
        \item 
        \item 
        \item 
    \end{enumerate}
\end{proof}

实际上,我们解决此类问题,很多时候等式都需要拆为小于等于和大于等于同时成立进行证明,经常利用
维数公式变形的齐次线性方程组解的一般理论,将问题转化为对像与核空间的研究,然后利用包含关系
(复杂的题目可能涉及子空间交与和的维数公式)以及已知的简单秩不等式进行证明.
可能部分题目较为困难,但至少请掌握上面例题中的情况.
\begin{example}
    设$A^*$为矩阵$A$的伴随矩阵,证明:
    \[r(A^*)=\begin{cases}
        n & r(A)=n \\ 1 & r(A)=n-1 \\ 0 & r(A) < n-1
    \end{cases}.\]
\end{example}
\begin{proof}
    
\end{proof}

\section{线性方程组拓展题型}
本节我们将介绍与线性方程组有关的一些题型,可能与高中数学讨论``题型''的学习风格有些类似.
需要注意的是,除了含参问题外,其余问题我们都将分别从齐次和非齐次两个方面进行讨论,给出
问题的一般解法.但实际上这里给出的解法并非能直接套用到所有的题目中,在习题中我们会遇到更多
特别的题目.因此更重要的应当是理解解题思路,而不是死记硬背解题方法.

\subsection{含参数的线性方程组问题}
此类问题一般考察对于含参数的线性方程组,参数取值如何时有解/无解/有唯一解等.
本质而言,\autoref{thm:15:有解条件}和\autoref{thm:15:方程组解}完全可以解决这一问题.

事实上,利用\autoref{thm:15:方程组解}在有解情况下只需计算行列式判断非常方便,但
判断无解需要利用\autoref{thm:15:有解条件},其中线性相关性的判断通常仍然需要我们对
系数矩阵进行高斯消元法.我们来看一个简单的例子:
\begin{example}
    讨论下面方程组的解的情况,并在有解的情况下求解:\[\begin{cases}
        x_1+x_2-x_3=1 \\ 2x_1+3x_2+kx_3=3 \\ x_1+kx_2+3x_3=2
    \end{cases}\]
\end{example}
\begin{solution}

\end{solution}

\subsection{线性方程组同解问题}
两个线性方程组同解实际上有两种情况:
\begin{enumerate}
    \item 两线性方程组都无解(注意齐次没有这种情况,因为一定有零解);

    \item 两线性方程组都有解且有相同的解集.
\end{enumerate}

下面的定理给出了两线性方程组同解的充要条件.实际上,这两个定理的证明很值得作为练习综合运用所学知识:
\begin{theorem}
    $n$元齐次线性方程组 $A_{m_1 \times n}\vec{X}=0$与 $B_{m_2 \times n}\vec{X}=0$同解的
    充要条件是$r\begin{pmatrix}
        A \\ B
    \end{pmatrix}=r(A)=r(B)$.
\end{theorem}
\begin{proof}
    
\end{proof}

\begin{theorem}
    $n$元非齐次线性方程组 $A_{m_1 \times n}\vec{X}=b$与 $B_{m_2 \times n}\vec{X}=d$同解的
    充要条件是
    \begin{enumerate}
        \item $r(A)\neq r(A,b)$且$r(B)\neq r(B,d)$;或
        \item $r\begin{pmatrix}
            A & b \\ B & d
        \end{pmatrix}=r\begin{pmatrix}
            A \\ B
        \end{pmatrix}=r(A)=r(A,b)=r(B)=r(B,d)$.
    \end{enumerate}
\end{theorem}
\begin{proof}
    
\end{proof}

我们来看一个例子来运用上述定理:
\begin{example}
    已知方程组\begin{gather*}\begin{cases}
            x_1+2x_2+3x_3=0 \\ 2x_1+3x_2+5x_3=0 \\ x_1+x_2+ax_3=0
        \end{cases}
        \\
        \begin{cases}
            x_1+bx_2+cx_3=0 \\ 2x_1+b^2x_2+(c+1)x_3=0
        \end{cases}
    \end{gather*}
    同解,求$a,b,c$的值.
\end{example}
\begin{solution}

\end{solution}

\subsection{线性方程组公共解问题}
公共解即为两线性方程组解集的交集,我们从齐次和非齐次讨论有公共解的条件:
\begin{theorem}
    对于$n$元齐次线性方程组 (1) $A_{m_1 \times n}\vec{X}=0$与 (2) $B_{m_2 \times n}\vec{X}=0$有
    \begin{enumerate}
        \item (1) 与 (2) 有非零公共解的充要条件是$r\begin{pmatrix}
                A \\ B
            \end{pmatrix}<n$;

        \item 设$\eta_1,\eta_2,\ldots,\eta_s\enspace(s=n-r(B))$是 (2) 的基础解系,则
        (1) 与 (2) 有非零公共解的充要条件是$A\eta_1,A\eta_2,\ldots,A\eta_s$线性相关;

        \item 设$\gamma_1,\gamma_2,\ldots,\gamma_t\enspace(t=n-r(A))$是(1) 的基础解系,
        $\eta_1,\eta_2,\ldots,\eta_s\enspace(s=n-r(B))$是 (2) 的基础解系,则(1) 与 (2) 有非零公共解的充要条件是
        $\gamma_1,\gamma_2,\ldots,\gamma_t,\eta_1,\eta_2,\ldots,\eta_s$线性相关.
    \end{enumerate}
\end{theorem}
\begin{proof}
    
\end{proof}

\begin{theorem}
    对于$n$元非齐次线性方程组(1) $A_{m_1 \times n}\vec{X}=b$与 (2) $B_{m_2 \times n}\vec{X}=d$,若(1) 与 (2) 都有解,则
    \begin{enumerate}
        \item (1) 与 (2) 有公共解的充要条件是$r\begin{pmatrix}
                A \\ B
            \end{pmatrix}=r\begin{pmatrix}
                A & b \\ B & d
            \end{pmatrix}$;

        \item 若$r(B)=s$,且$\eta_1,\eta_2,\ldots,\eta_{n-s+1}$是 (2) 的$n-s+1$个线性无关的解,则
        (1) 与 (2) 有公共解的充要条件是$b$是$A\eta_1,A\eta_2,\ldots,A\eta_{n-s+1}$的凸组合,即
        存在数$k_1,k_2,\ldots,k_{n-s+1}$使得
        \[b=k_1A\eta_1+k_2A\eta_2+\cdots+k_{n-s+1}A\eta_{n-s+1},\]
        其中$k_1+k_2+\ldots+k_{n-s+1}=1$;

        \item 若$r(A)=t$,$r(B)=s$,$\gamma_1,\gamma_2,\ldots,\gamma_{n-t+1}$是(1) 的$n-t+1$个线性无关的解,
        $\eta_1,\eta_2,\ldots,\eta_{n-s+1}$是 (2) 的$n-s+1$个线性无关的解,则(1) 与 (2) 有公共解的充要条件是
        存在数$k_1,k_2,\ldots,k_{n-t+1}$和$l_1,l_2,\ldots,l_{n-s+1}$使得
        \[k_1\gamma_1+k_2\gamma_2+\cdots+k_{n-t+1}\gamma_{n-t+1}-l_1\eta_1-l_2\eta_2-\cdots-l_{n-s+1}\eta_{n-s+1}=0\]
        其中$k_1+k_2+\cdots+k_{n-t+1}=1$,$l_1+l_2+\cdots+l_{n-s+1}=1$.
    \end{enumerate}
\end{theorem}
\begin{proof}
    
\end{proof}

这两个定理看起来非常长,实则无需记忆(最多记住二者的第一条),只需要通过证明理解其含义即可.下面我们看一个简单的例子:
\begin{example}
    设四元齐次线性方程组(1) 为\[\begin{cases}
        2x_1+3x_2-x_3=0 \\ x_1+2x_2+x_3-x_4=0
    \end{cases}\]已知另一个四元齐次线性方程组 (2) 的基础解系为
    \[\alpha_1=(2,-1,a+2,1)^\mathrm{T},\enspace\alpha_2=(-1,2,4,a+8)^\mathrm{T}\]
    \begin{enumerate}
        \item 求方程组 (1) 的一个基础解系;

        \item 当$a$为何值时,方程组 (1) 和 (2) 有非零公共解,并求出非零公共解.
    \end{enumerate}
\end{example}
\begin{solution}

\end{solution}

\subsection{线性方程组反问题}

此类问题即已知方程组的解,要给出原方程组.我们仍按齐次与非齐次分开的思路讨论此类问题的一般解法.
这里我们之间通过例子来讲解方法:
\begin{example}
    已知$n$维列向量组$\alpha_1,\ldots,\alpha_s$线性无关,求一齐次线性方程组以
    $\alpha_1,\ldots,\alpha_s$为基础解系.
\end{example}
\begin{solution}

\end{solution}

\begin{example}
    设向量组$\alpha_1,\ldots,\alpha_s$线性无关,求一非齐次线性方程组$AX=b$,使其解集
    以$\alpha_1,\ldots,\alpha_s$为极大线性无关组.
\end{example}
\begin{solution}

\end{solution}

下面我们来看一个更为一般的例子来运用上面介绍的方法:
\begin{example}
    已知$\alpha_1=(1,2,-1,0,4)^\mathrm{T},\enspace\alpha_2=(-1,3,2,4,1)^\mathrm{T},\enspace\alpha_3=(2,9,-1,4,13)^\mathrm{T}$,
    且有$W=\spa(\alpha_1,\alpha_2,\alpha_3)$.
    \begin{enumerate}
        \item 求以$W$为解空间的一个齐次线性方程组;

        \item 求以$W'=\{\eta+\alpha \mid \alpha\in W\}$为解集的一个非齐次线性方程组,其中$\eta=(1,2,1,2,1)^\mathrm{T}$.
    \end{enumerate}
\end{example}
\begin{solution}

\end{solution}

\vspace{2ex}
\centerline{\heiti \Large 内容总结}

\vspace{2ex}

\centerline{\heiti \Large 习题}
\vspace{2ex}
{\kaishu 即使我说二二得四,三三见九,也没有一字不错。
这些既然都错,则绅士口头的二二得七,三三见千等等,自然就不错了。}
\begin{flushright}
    \kaishu
    ——鲁迅,《朝花夕拾》
\end{flushright}
\centerline{\heiti A组}
\begin{enumerate}
    \item 证明以下关于线性方程组解的理论的基本定理:

    第一组(齐次线性方程组解空间的一般理论)
    \begin{enumerate}
        \item 设矩阵 $A \in M_{m\times n}(\mathbf{F})$,若 $r(A)=r$,则齐次线性方程组 $AX=0$ 的解空间 $N(A)$ 是 $\mathbf{F}^n$ 的一个 $n-r$ 维子空间.
        
        \item 设 $A$ 为 $m \times n$ 矩阵,则
        \begin{enumerate}[label=(\arabic*)]
            \item 齐次线性方程组 $AX=0$ 只有零解等价于 $r(A)=n$;
            \item 齐次线性方程组 $AX=0$ 有非零解(无穷解)等价于 $r(A)<n$.
        \end{enumerate}
        
        \item 设 $A$ 为 $n$ 阶矩阵,则
        \begin{enumerate}[label=(\arabic*)]
            \item 齐次线性方程组 $AX=0$ 只有零解等价于 $|A|\neq 0$;
            \item 齐次线性方程组 $AX=0$ 有非零解(无穷解)等价于 $|A|=0$.
        \end{enumerate}

    第二组(非齐次线性方程组解空间的一般理论)
        
        \item 对于非齐次线性方程组 $AX=b$,下列命题等价:
        \begin{enumerate}[label=(\arabic*)]
            \item $AX=b$ 有解;
            \item $b \in R(A)$,即 $b$ 可被 $A$ 的列向量组线性表示;
            \item $r(A,b)=r(A)$,即增广矩阵的秩等于系数矩阵的秩.
        \end{enumerate}
        
    第三组(线性方程组解的结构的一般理论)
        
        \item 设 $X_1,\ X_2,\ \dots,\ X_s$ 为齐次线性方程组 $AX=0$ 的一组解,则 $k_1X_1+k_2X_2+\dots+k_sX_s$ 也为齐次线性方程组 $AX=0$ 的解,其中 $k_1,\ k_2,\ \dots,\ k_s$ 为任意常数.
        
        \item 设 $\eta_0$ 为非齐次线性方程组 $AX=b$ 的一个解,$X_1,\ X_2,\ \dots,\ X_s$ 为齐次线性方程组 $AX=0$ 的一组解,则 $k_1X_1+k_2X_2+\dots+k_sX_s+\eta_0$ 也为非齐次线性方程组 $AX=b$ 的解.
        
        \item 设 $\eta_1,\ \eta_2$ 为非齐次线性方程组 $AX=b$ 的两个解,则 $\eta_2-\eta_1$ 为齐次线性方程组 $AX=0$ 的解.
        
        \item 设 $X_1,\ X_2,\ \dots,\ X_s$ 为非齐次线性方程组 $AX=b$ 的一组解,则 $k_1X_1+k_2X_2+\dots+k_sX_s$ 也为非齐次线性方程组 $AX=b$ 的解的充分必要条件是 $k_1+k_2+\dots+k_s=1$.
        
        \item 设 $X_1,\ X_2,\ \dots,\ X_s$ 为非齐次线性方程组 $AX=b$ 的一组解,则 $k_1X_1+k_2X_2+\dots+k_sX_s$ 为齐次线性方程组 $AX=0$ 的解的充分必要条件是 $k_1+k_2+\dots+k_s=0$.
        
    判断以下关于线性方程组解的理论的说法是否正确并说明理由:
        
    第四组(一些经典的判断题)
        
        \item 方程组 $AX=b$ 有唯一解等价于方程组 $AX=0$ 只有零解.
        
        \item 设 $A$ 是 $m \times n$ 矩阵,$B$ 是 $n \times s$ 矩阵,若 $AB=O$,则 $B$ 的列向量为方程组 $AX=0$ 的解.
        
        \item 设 $A$ 是 $n$ 阶非零矩阵,则存在非零矩阵 $B$,使得 $AB=O$ 等价于 $r(A)<n$.
        
        \item 方程组 $AX=0$ 的解为 $BX=0$ 的解,则 $r(A)\ge r(B)$.
        
        \item 方程组 $AX=0$ 与 $BX=0$ 为同解方程组等价于 $r(A)=r(B)$.
    \end{enumerate}
    
    \item 设$A$为四阶矩阵,$r(A)<4$,且$A_{21}\neq 0$,求方程组$AX=0$的通解.
    \item 设$A=(\alpha_1,\alpha_2,\alpha_3,\alpha_4)$为四阶矩阵,方程组$AX=0$的通解为$X=k(1,0,-4,0)^\mathrm{T}$,求
    $A^*X=0$的基础解系.
    \item 设$A$为$n$阶实矩阵,$W=\{\beta\in\mathbf{R}^n\ |\ \alpha^\mathrm{T}A\beta=0,\ \forall \alpha\in\mathbf{R}^n\}$,证明:
    \begin{enumerate}[label=(\arabic*)]
        \item $\dim W+r(A)=n$;
        \item $W$为$\mathbf{R}^n$的子空间.
    \end{enumerate}

    \item 已知4级方阵$A=(\alpha_1,\alpha_2,\alpha_3,\alpha_4)$的列向量$\alpha_1,\alpha_2,\alpha_4$线性无关,
    且$\alpha_1=2\alpha_2-\alpha_3$,若$\beta=\alpha_1-\alpha_2+3\alpha_4$,求方程组$AX=\beta$的通解.
    \item 设四元非齐次线性方程组的系数矩阵的秩为3,已知$\eta_1,\eta_2,\eta_3$是它的三个
    解向量,且$\eta_1=(2,3,4,5)^\mathrm{T},\eta_2+\eta_3=(1,2,3,4)^\mathrm{T}$,求该方程组的通解.
    \item 设$\beta_1,\beta_2,\beta_3$是$n$元非齐次线性方程组$AX=b$的三个线性无关的解,
    且$r(A)=n-2$,求:
    \begin{enumerate}[label=(\arabic*)]
        \item 导出组$AX=0$的一个基础解系;
        \item $AX=2b$的一般解.
    \end{enumerate}

    \item 已知$A$是一个$s\times n$矩阵,证明:线性方程组$AX=b$对任意列向量$b_{s\times 1}$都有解的充要条件是$A$行满秩.
    \item 设$A,B$分别是$m \times n$和$n \times s$矩阵,且$r(B)=n$,证明:若$AB=O$,则$A=O$.
    \item 设$A \in \mathbf{F}^{m \times n},B \in \mathbf{F}^{(n-m) \times n}(m<n)$,
	$V_1,V_2$分别为齐次线性方程组$AX=0$和$BX=0$的解空间,证明:$\mathbf{F}^n=V_1\oplus V_2$的解的
	充要条件是$\begin{pmatrix}
		A \\ B
	\end{pmatrix}X$=0只有零解.
	\item 齐次线性方程组
    \[\begin{cases}
		x_2+ax_3+bx_4=0 \\ -x_1+cx_3+dx_4=0 \\ ax_1+cx_2-ex_4=0 \\ bx_1+dx_2+ex_3=0
	\end{cases}\]
	的一般解以$x_3,x_4$作为自由未知量.
    \begin{enumerate}[label=(\arabic*)]
        \item 求$a,b,c,d,e$满足的的条件;
        \item 求齐次线性方程组的基础解系.
    \end{enumerate}
\end{enumerate}

\centerline{\heiti B组}
\begin{enumerate}
    \item 设$A$为$m \times n$矩阵,$r(A)=m$,$B$是$m$阶可逆矩阵,已知$A$的行空间$R(A^\mathrm{T})$是方程组$CX=0$的解空间,证明:
    $BA$的行向量也是$CX=0$的一个基础解系.
    \item 设$A$是$n$阶矩阵,且$A_{11}\neq 0$,证明:方程组$AX=b$($b$为非零向量)有无穷多解的充要条件为$A^*b=0$.
    \item 若$n$阶矩阵$A$的各行、各列元素之和都为0,证明:$|A|$的所有元素的代数余子式都相等.
    \item 已知 $\alpha_1,\ \alpha_2,\ \dots,\ \alpha_s$ 是齐次线性方程组 $AX=0$ 的一组基础解系,向量组

    \[\beta_1=t_1\alpha_1+t_2\alpha_2,\ \beta_2=t_1\alpha_2+t_2\alpha_3,\ \dots,\ \beta_{s-1}=t_1\alpha_{s-1}+t_2\alpha_s\]
    
    试问当实数 $t_1,\ t_2$ 满足何条件时,$AX=0$ 有基础解系包含向量 $\beta_1,\ \beta_2,\ \dots,\ \beta_{s-1}$,并写出该基础解系中的其余向量.
    \item (注:本题一般形式在教材第六章补充题1)已知线性方程组$$\begin{cases}
        a_{11}x_1+a_{12}x_2+\cdots+a_{1,2n}x_{2n}=0 \\
        a_{21}x_1+a_{22}x_2+\cdots+a_{2,2n}x_{2n}=0 \\
        \cdots \\
        a_{n1}x_1+a_{n2}x_2+\cdots+a_{n,2n}x_{2n}=0 \\
    \end{cases}$$
    的一个基础解系为$(b_{11},b_{12},\cdots,b_{1,2n})^\mathrm{T},\ (b_{21},b_{22},\cdots,b_{2,2n})^\mathrm{T},\ (b_{n1},b_{n2},\cdots,b_{n,2n})^\mathrm{T}$,
    求解线性方程组$$\begin{cases}
        b_{11}x_1+b_{12}x_2+\cdots+b_{1,2n}x_{2n}=0 \\
        b_{21}x_1+b_{22}x_2+\cdots+b_{2,2n}x_{2n}=0 \\
        \cdots \\
        b_{n1}x_1+b_{n2}x_2+\cdots+b_{n,2n}x_{2n}=0 \\
    \end{cases}.$$
    \item 设$A,B\in \mathbf{F}^{n\times n}$,且$r(A)=r$,$r(B)=s$,$r\begin{pmatrix}
        A \\ B
    \end{pmatrix}=k$.
    \begin{enumerate}[label=(\arabic*)]
        \item 证明:满足$AX=O$的$n$阶方阵$X$全体构成$\mathbf{F}^{n\times n}$的子空间,并求其维数;
        \item 令满足$AX=O$的$n$阶方阵$X$全体构成的子空间为$V_1$,满足$BX=O$的$n$阶方阵$X$全体构成的子空间为$V_2$,
        求$V_1+V_2$的维数.
    \end{enumerate}
    \item 设$A$是元素全为1的$n$阶方阵.
    \begin{enumerate}
        \item 求行列式$|aE+bA|$,其中$a,b$为实常数;
        \item 已知$1<r(aE+bA)<n$,试确定$a,b$满足的条件,并求下列线性子空间的维数:
        \[W=\{x\ |\ (aE+bA)x=0,\ x\in\mathbf(R)\}.\]
    \end{enumerate}
    \item 已知线性方程组
    \[\begin{cases}
        a_{11}x_1+a_{12}x_2+\cdots+a_{1n}x_n=b_1 \\
        a_{21}x_1+a_{22}x_2+\cdots+a_{2n}x_n=b_2 \\
        \cdots \\
        a_{n1}x_1+a_{n2}x_2+\cdots+a_{nn}x_n=b_n \\
    \end{cases}\]
    的系数矩阵与
    \[\begin{pmatrix}
        a_{11} & a_{12} & \cdots & a_{1n} & b_1 \\
        a_{21} & a_{22} & \cdots & a_{2n} & b_2 \\
        \cdots \\
        a_{n1} & a_{n2} & \cdots & a_{nn} & b_n \\
        b_1 & b_2 & \cdots & b_n & 0
    \end{pmatrix},\]
    秩相等,求证:上述线性方程组有解.
    \item 设$A=(a_{ij})_{m\times n}$,$b$和$X$为$m$元列向量,$Y$为$n$元列向量,证明:
    \begin{enumerate}
        \item 若$AY=b$有解,则$A^\mathrm{T}X=0$的任一组解都满足$b^\mathrm{T}X=0$;
        \item 方程组$AY=b$有解的充要条件是方程组$\begin{pmatrix}
            A^\mathrm{T} \\ b^\mathrm{T}
        \end{pmatrix}X=\begin{pmatrix}
            0 \\ 1
        \end{pmatrix}$无解(其中0是$n$元零向量).
    \end{enumerate}
    \item 判断:设 $A$ 是复数域上 $m \times n$ 阶矩阵,则矩阵秩 $r\left(A^T A\right)=r(A)$.
	\item 证明:对于$m \times n$实矩阵$A$,方程$A^\mathrm{T}AX = A^\mathrm{T}b$总是有解,且$A$为方阵时,$A^\mathrm{T}AX = 0$和$AX=0$同解.
	当$r(A)=n$时求其解,并证明$A(A^\mathrm{T}A)^{-1}A^\mathrm{T}$是幂等的对称矩阵.
	\item 设$A,B,C$为$n$阶实方阵,且$BAA^\mathrm{T}=CAA^\mathrm{T}$,证明:$BA=CA$.
	\item 设$A$为数域$\mathbf{F}$上的$n$阶方阵,又设线性空间$\mathbf{F^n}$的两个
	子空间为$W_1=\{X\in\mathbf{F}^n\ |\ AX=0\}$,$W_2=\{X\in\mathbf{F}^n\ |\ (A-E)X=0\}$.
	证明:$A^2=A \iff \mathbf{F}^n=W_1\oplus W_2$.
	\item $n$阶方阵$A$,$B$满足$AB=BA$,证明:$r(AB)+r(A+B)\le r(A)+r(B)$.
	\item 请按序证明以下结论:
	\begin{enumerate}[label=(\arabic*)]
        \item $A,B$分别是$s \times m,m \times n$矩阵,则$ABX=0$与$BX=0$同解的充要条件是$r(AB)=r(B)$;
        \item $A,B$分别是$s \times m,m \times n$矩阵,且$r(AB)=r(B)$,则对任意的$n \times t$矩阵都有$r(ABC)=r(BC)$;
        \item 设$A$是$n$阶方阵,则存在正整数$k$使得$r(A^k)=r(A^{k+1})=r(A^{k+2})=\cdots$,且对任意正整数$m$,有$r(A^n)=r(A^{n+m})$.
    \end{enumerate}
    \item 如果齐次线性方程组$$\begin{cases}
		x_1+x_2+bx_3-x_4+x_5=0 \\
		2x_1+3x_2+x_3+x_4-2x_5=0 \\
		x_2+ax_3+3x_4-4x_5=0 \\
		-3x_1-3x_2-3bx_3+bx_4+(a+2)x_5=0
	\end{cases}$$
	的解空间是3维的,试求$a,b$的值与解空间的基,解空间可能为2维吗?
	\item 设$W_1,W_2$分别为$n$元齐次线性方程组$AX=0$和$BX=0$的解空间,试构造两个$n$元
	齐次线性方程组,使它们的解空间分别为$W_1 \cap W_2$和$W_1+W_2$.
	\item 已知方程组$$\begin{cases}
		x_1+x_2+ax_3+x_4=1 \\ -x_1+x_2-x_3+bx_4=2 \\ 2x_1+x_2+x_3+x_4=c
	\end{cases}$$与$$\begin{cases}
		x_1+x_4=-1 \\ x_2-2x_4=d \\ x_3+x_4=e
	\end{cases}$$同解,求$a,b,c,d,e$.
	\item 设有两个非齐次线性方程组(1)和(2),它们的通解分别是$X=\gamma+t_1\eta_1+t_2\eta_2=\delta+k_1\xi_1+k_2\xi_2$.
	其中$\gamma=(5,-3,0,0)^\mathrm{T},\eta_1=(-6,5,1,0)^\mathrm{T},\eta_2=(-5,4,0,1)^\mathrm{T},
	\delta=(-11,3,0,0)^\mathrm{T},\xi_1=(8,-1,1,0)^\mathrm{T},\xi_2=(10,-2,0,1)^\mathrm{T}$,求这两个方程组的公共解.
	\item 若方程组
    \[\begin{cases}
		x_1+x_2+x_3=0 \\ x_1+2x_2+ax_3=0 \\ x_1+4x_2+a^2x_3=0
	\end{cases}\]与
    \[x_1+2x_2+x_3=a-1\]
    有公共解,求$a$的值及所有公共解.
\end{enumerate}

\centerline{\heiti C组}
\begin{enumerate}
    \item 用方程组的理论证明:一个$n$次多项式不可能有多于$n$个不同的根.
    \item 相容(即有解)的线性方程组$AX=b$在怎样的条件下,其解中第$k$个未知量$x_k$都是同一个值?
    你给的条件是否是充分必要的?
    \item 已知$A$是$n$阶对称矩阵,$\beta$为$n$元非零列向量,$B=\begin{pmatrix}
		A & \beta \\ \beta^\mathrm{T} & 0
	\end{pmatrix}$,证明:
    \begin{enumerate}[label=(\arabic*)]
        \item 若$r(A)=n$,则$B$可逆的充要条件是$\beta^\mathrm{T}A^{-1}\beta \neq 0$;
        \item 若$r(A)=r$,则$r(B)=r$的充要条件是方程组$\begin{cases}
            AX=\beta \\ \beta^\mathrm{T}X=0
        \end{cases}$有解;
        \item 若$r(A)=n-1$,则$B$可逆的充要条件是$AX=\beta$无解.
    \end{enumerate}
    \item 讨论$b_1,b_2,\cdots,b_n(n\ge 2)$满足什么条件时,下列方程组$$\begin{cases}
		x_1+x_2=b_1 \\ x_2+x_3=b_2 \\ \cdots \\ x_{n-1}+x_n=b_{n-1} \\ x_n+x_1=b_n
	\end{cases}$$有解,并求解.
	\item 已知$A$是$s \times n$矩阵,$B$是$m \times n$矩阵,$X,a,b$分别是$n,s,m$元列向量,证明:
	\begin{enumerate}
        \item 齐次线性方程组$AX=0$和$BX=0$同解的充要条件是$A$与$B$行向量等价(列向量不一定);
        \item 齐次线性方程组$AX=0$和$BX=0$解空间分别为$V_1,V_2$,证明:$V_1 \subseteq V_2$的充要条件是存在$m \times s$
        矩阵$C$使得$B=CA$;
        \item 线性方程组$AX=a$的解都是$BX=b$的解的充要条件是增广矩阵$(B,b)$的每个行向量都可以被$(A,a)$的行向量线性表示;
        \item 线性方程组$AX=a$与$BX=b$同解的充要条件是$(A,a)$与$(B,b)$行向量等价.
    \end{enumerate}
\end{enumerate}