\chapter{史海拾遗}

通过前面十余讲的讨论,我们已经将线性代数的一个核心问题——有关线性方程组的
解的一般理论完成,其意已尽.``历史是一面镜子,它照亮现实,也照亮未来.''
我们很有必要抓住这一时机来完整讨论有关于线性代数的历史,循着历代数学家
的脚步重新审视所学的内容,再次感受其中逻辑的自然与顺畅.很多时候,知道一件事情
为什么、怎么来的更为可贵.另一方面,我们也将在史海中搜寻整个数学大厦中与
线性代数紧密相连的部分,开始我们后一阶段更多``未竟之美''的讨论.

提示:本讲中可能出现大量未学习的内容,事实上很多是后续会学习的,也有部分是非常前沿
的介绍.读者可以留个印象,日后再回过头来看,一定会感觉无比亲切.

\section{起点:初等代数}
\subsection{初等代数简介}
代数的英语为 algebra ,源于阿拉伯语单字``al-jabr''(本义为``重聚''),
出自《代数学》(阿拉伯语:al-Kitāb al-muḫtaṣar fī ḥisāb al-ğabr wa-l-muqābala)
这本书的书名上,意指移项和合并同类项之计算的摘要,其为波斯回教数学家花拉子米
于820年所著.

事实上,初见代数一词我们脑海中便会浮现出小学、初中阶段老师反复强调的``用字母表示数''
的思想,``一元一次方程''、``合并同类项''、``因式分解''等熟悉的词汇也会出现在眼前.
事实也是如此,初等代数的由来正是用字母表示数后,得到了一系列方程和多项式的有趣问题.

亚历山大港的丢番图(Dióphantos ho Alexandreús,
公元200-284),是罗马时代的数学家.大部分有关丢番图生平的信息来源于5世纪时希腊人
梅特罗多勒斯(Metrodorus)在其文集中收录的一篇具有数学谜题性质的《丢番图墓志铭》:
\begin{quote}
    \kaishu
    坟中安葬着丢番图.

    多么令人惊讶,它忠实地记录了所经历的道路.

    上帝给予的童年占六分之一,

    又过十二分之一,两颊长胡,

    再过七分之一,点燃起结婚的蜡烛.

    五年之后天赐贵子,

    可怜迟到的宁馨儿,享年仅及其父之半,便进入冰冷的墓.

    悲伤只有用数论的研究去弥补,

    又过四年,他也走完了人生的旅途.
\end{quote}
据此列一元一次方程可知,丢番图享寿84岁,于33岁时成婚,38岁时生子,80岁时丧子.
丢番图作著的丛书《算术》(Arithmetica)处理求解代数方程组的问题,但其中有不少已经遗失,
但他的研究在数论中占有重要地位,如丢番图方程、丢番图几何、丢番图逼近等都是数学里的重要领域.
我们简要展开丢番图方程的讨论:
\begin{definition}
    形如
    \[a_1x_1^{b_1}+a_2x_2^{b_2}+\ldots+a_nx_n^{b_n}=c,\enspace a_i,b_i,c(i=1,2,\ldots,n)\in\mathbf{Z}\]
    的方程称为丢番图方程(或不定方程).
\end{definition}

简而言之,丢番图方程就是未知数只能使用整数的整数系数多项式等式,虽然定义看起来十分简单,学过
初等数论的同学应当有些熟悉(事实上在之后的讨论中我们可以看到初等代数与初等数论之间是紧密联系的),
但可以说这一方程在数学史上留下了浓墨重彩的一笔.
后来当法国数学家费马研究《算术》一书时,对其中某个方程颇感兴趣并认为其无解,说他对此“已找到一个绝妙的证明”,
但却没有记录下来,直到三个世纪后才出现完整的证明,我们这里简要介绍这一定理,即费马大定理:
\begin{theorem}
    丢番图方程$x^n+y^n-z^n=0$在$n>2$时无正整数解.
\end{theorem}

自费马提出猜想的三百余年以来,无数数学家为证明费马大定理而费尽心血,直至1995年,英国数学家安德鲁·怀尔斯
(Andrew Wiles)最终给出了证明.这一证明使用了代数数论、代数几何等大量现代数学工具.如果读者有兴趣自行搜索
费马大定理证明的历史,我们可以看到对这一看似简单定理的证明的不懈追求推动了现代数学的发展,可谓是意义重大.
或许在数学史中这些中间过程不过是短短的一行文字描述,但这就是人类探索真理的历程的缩影.

或许阅读这一讲义的很多同学都来源于计算机相关专业,我们这里还可以简要介绍丢番图方程与理论计算机
的关联.1900年,希尔伯特提出丢番图问题的可解答性为他在巴黎的国际数学家大会演说中所提出的23个
重要数学问题的第十题.这个问题是对于任意多个未知数的整系数不定方程,要求给出一个可行的算法,
使得借助于它通过有限次运算,可以判定该方程有无整数解.

第十问题的解决是众人集体的智慧结晶.其中美国数学家马丁·戴维斯(Martin Davis)、希拉里·普特南
(Hilary Putnam)和朱莉娅·罗宾逊(Julia Robinson)做出了突出的贡献.而最终的结果,是由俄国数学家
尤里·马季亚谢维奇(Yuri Matiyasevich)于1970年所完成的:不可能存在一个算法能够判定任何丢番图方程是否有解.
这一问题涉及到``可计算性''的问题,相关的讨论读者在学习计算理论后将有进一步的了解.想必读者听闻过
罗素悖论或理发师悖论,或者是图灵停机问题,这些都与可计算性密不可分.实际上,``可计算性''与人类逻辑
与知识的边界密切相关——这显然是个异常宏大的主题,留待后人不断探索.

\subsection{西方初等代数发展史简述}
我们回到对于初等代数历史的讨论——刚刚我们显然有些跳脱了,但这些讨论对于了解数学之美也是必要的.在古代西方,
还有几个重要的时间节点值得提及:
\begin{enumerate}
    \item 公元前1800年左右,旧巴比伦斯特拉斯堡泥板书中记述其寻找著二次椭圆方程的解法;
    \item 公元前1600年左右,普林顿322号泥板书中记述了以巴比伦楔形文字写成的勾股数列表;
    \item 公元前800年左右,印度数学家包德哈亚那在其著作包德哈尔那绳法经中以代数方法找到了勾股数,
    给出了一些二次方程的几何解法,且找出了两组丢番图方程组的正整数解;
    \item 公元前300年左右,在几何原本的第二卷里,欧几里德给出了有正实数根之二次方程的解法,使用尺规作图的几何方法;
    \item 公元前100年左右,写于古印度的巴赫沙里手稿中使用了以字母和其他符号写成的代数标记法,且包含有
    三次与四次方程,多达五个未知道的线性方程之代数解,二次方程的一般代数公式,以及不定二次方程与方程组的解法.
\end{enumerate}

此为丢番图之前的初等代数发展重要节点,蕴含着古人朴素的智慧.丢番图之后,499年,印度数学家阿耶波多在其所著之阿耶波多书里以和
现代相同的方法求得了线性方程的自然数解,描述不定线性方程的一般整数解,给出不定线性方程组的整数解,而描述了微分方程;
628年,印度数学家婆罗摩笈多在其所著之梵天斯普塔释哈塔中,介绍了用来解不定二次方程的宇宙方法,且给出了解线性方程和二次方程的规则.
他发现二次方程有两个根,包括负数和无理数根.

此后便迎来一个更为重要的时间节点.820年,代数(algebra)一词出现,其描述于波斯数学家花拉子米所著之完成和平衡计算法概要中
对于线性方程与二次方程系统性的求解方法.花拉子米常被认为是“代数之父”,其大多数的成果简化后会被收录在书籍之中,
且成为现在代数所用的许多方法之一.990年左右,波斯阿尔卡拉吉在其所著之al-Fakhri中更进一步地以扩展花拉子米的方法论来发展代数,
加入了未知数的整数次方及整数开方.他将代数的几何运算以现代的算术运算代替,且定义了单项式并给出任两个单项式相乘的规则.

此后,初等代数的发展逐步向着现代符号体系与研究方法发展.逐渐演化为了两个方向的问题的讨论:
\begin{enumerate}
    \item 未知数更多的一次方程组的解;
    \item 未知数次数更高的高次方程的解.
\end{enumerate}

前者与我们的主角:线性代数相关,而后者则引发了另一个学科——抽象代数的开端.
初等代数学逐步解决了2、3、4次方程求解问题,这些方程的解都可用系数的四则运算与根式运算给出,
即可用根式解这些方程,此时初等代数也因此而达到顶峰.但当时的数学家们继续探索 
5次与5次以上方程的解也试图用根式解出这些方程,经过200余年却无重要进展,
直到19世纪抽象代数的发展才有了转机,后续我们也将介绍这其中的故事.

\subsection{中国初等代数发展史简述}
在本节的最后,我们将视角转向东方,总结古代中国人在初等代数学中作出的贡献.相信读者都十分熟悉
这一问题:今有雉兔同笼,上有三十五头,下有九十四足,问雉兔各几何?这是《孙子算经》(不晚于473年)
中提出的著名的鸡兔同笼问题.在《孙子算经》中还提出了读者在初等数论中就已十分熟悉的``中国剩余定理'',
直至现代的密码学研究也无法离开这一重要定理.

实际上,早在《孙子算经》出现前500年左右(公元前100年左右),中国古代数学名著《九章算术》中便处理了
代数方程的问题.其中的``方程章''是世界上最早的系统研究代数方程的专门论著.它在世界数学历史上最早创立了
多元一次方程组的筹式表示方法,以及它的多种求解方法.《九章算术》把这些线性方程组的解法称为“方程术”,
其实质相当于现今的高斯消元法(早于高斯约1900年).

除去线性方程组的贡献,在高次方程方面,中国古代也有相当丰富的成果.625年左右,中国数学家王孝通在
《缉古算经》中找出了三次方程的数值解;1247年,南宋数学家秦九韶在《数书九章》中用秦九韶算法解一元高次方程.
1248年,金朝数学家李冶的《测圆海镜》利用天元术将大量几何问题化为一元多项式方程,是一部几何代数化的代表作.
1300年左右,中国数学家朱世杰处理了多项式代数,发明四元术解答了多达四个未知数的多项式方程组,发明
非线性多元方程的消元法,将相关多项式进行乘法、加法和减法运算,逐步消元,将多元非线性方程组化为
单个未知数的高次多项式方程;并以数值解出了288个四次、五次、六次、七次、八次、九次、十次、十一次、十二次和
十四次多项式方程.

\section{演化:线性代数的产生与发展}
如前所述,初等代数经过数个世纪的发展逐渐演化为了两个大的方向:未知数更多的一次方程组
和未知数次数更高的高次方程.在这两个方向上的发展,使得代数学发展到高等代数的阶段,
上面两个方向简而言之就是现在大家熟悉的线性方程组理论(线性代数)和多项式理论
(以致后来的抽象代数).本节我们主要讨论前者,后者我们将在下一节中讨论.

\subsection{行列式与Cramer法则的引入}
在这一部分,我们首先将重点介绍线性方程组理论的开山鼻祖——莱布尼茨.莱布尼茨(Gottfried Wilhelm Leibniz,1646-1716),
德国自然科学家、数学家、物理学家、历史学家和哲学家,和牛顿同为微积分的创建人.他博览群书,涉猎百科,对丰富人类的科学
知识宝库做出了不可磨灭的贡献,行列式与线性方程组理论是他留给人类的财富中很小但很重要的一部分.

莱布尼茨的第一个大的贡献便是引入了新符号.莱布尼兹首先创立了采用两个记号的双标码记法,他在方程中使用系数
10,11,12;20,21,22;30,31,32,因为两个数字各有所指,所以相当于现代数学中方程系数符号的下标,即相当于
$a_{10},a_{11},\ldots$的下标.莱布尼兹在1693年给洛必达的一封信中给出了一个方程组:
\[\begin{cases}
    10x+11y+12z=0\\
    20x+21y+22z=0\\
    30x+31y+32z=0
\end{cases}\]

根据他给出的消元方法,他得到了
\[10\cdot 21\cdot 32+11\cdot 22\cdot 30+12\cdot 20\cdot 31-12\cdot 21\cdot 30-11\cdot 20\cdot 32-10\cdot 22\cdot 31=0,\]
这与现在我们熟知的方程组系数行列式等于0是完全一致的.基于此莱布尼茨也给出了行列式的最原始的定义(虽然形式上有差别)
以及最原始的Cramer法则,因此可以称为这一方向理论的鼻祖.

然而,莱布尼茨的很多工作都是后来才被人们发现的,所以他的方法对后来其他数学家提出的规则几乎没有影响.事实上,
同时代的日本数学家关孝和在其著作《解伏题元法》中也提出了行列式的概念与算法.而在Cramer法则上,麦克劳林
和Cramer的工作更早被人们认识到.

麦克劳林(Maclaurin,1698-1746)是18世纪英国最具有影响的数学家之一.他自幼聪慧勤奋,11岁便步入大学校门,17岁就以有关引力研究的论文
获硕士学位,19岁受聘为阿伯丁马里沙尔学院数学教授,21岁当选为英国皇家学会会员.麦克劳林最为读者熟知的贡献想必是麦克劳林级数展开式,
实际上他还有几何学等方面其他贡献.线性代数方面,在他1748年的遗著《代数论著》(A Treatise of Algebra)中,麦克劳林最先开创了
用行列式的方法来求解含2个、3个和4个未知量的联立线性方程组.遗憾的是,麦克劳林没能进一步给出一个明确的法则来确定符号.虽然,书中
的记法不太好,符号变化的规则又比较模糊,但它确实就是我们今天所使用的Cramer法则.

事实上,现在我们所熟知的Cramer法则是由瑞士数学家加布里埃尔·克莱姆(Gabriel Cramer,1704-1752)在1750年的著作《线性代数分析导言》
(Introduction à l'analyse des lignes courbes algébriques)中给出的.为了确定经过5个点的一般二次曲线的系数,他引入了这一著名的法则,
并且因其符号上更为简洁明了的优越性而被人们所接受.事实上,克莱姆最著名的工作是在1750年发表关于代数曲线方面的权威之作.他最早证明一个
第$n$度的曲线是由$n(n + 3)/2$个点来决定的.

\subsection{线性方程组与行列式的进一步研究}
在前人工作的基础上,关于线性方程组以及行列式的理论有了更快的发展.裴蜀(E.Bezout,1730.1783),法国数学家.曾在海军学校和皇家炮兵学校
任教,主要从事代数方程理论的研究并取得一系列的成果.1764年,裴蜀发表论文提出了行列式中项的构成规则和符号的形成规则.他给出了
行列式的一个循环构造规律,同时用不同于莱布尼兹、克莱姆的方法,给出了项的构成规则和符号确定规则.他所作的成就对后来行列式理论的奠基和发展
起着非常重要的作用.同时,裴蜀在该文中证明了含$n$个未知量的$n$个齐次线性方程组有非零解的条件是其“结式"(系数行列式)等于零,跳出了前人对于
求解方程组计算问题的讨论,转向对一般理论的讨论.

在行列式的发展史上,第一个对行列式理论做出连贯的逻辑的阐述,即把行列式理论与线性方程组求解相分离的人,是法国数学家范德蒙(A-T.Vandermonde,1735-1796)
——他不仅把行列式应用于解线性方程组,而且对行列式理论本身进行了开创性研究.
范德蒙自幼在父亲的指导下学习音乐,但对数学有浓厚的兴趣,后来终于成为法兰西科学院院士.他给出了用二阶子式和它们的余子式来展开行列式的法则,这跳出了
前人从线性方程组角度研究行列式的范畴,因此就对行列式本身这一点来说,他是这门理论的奠基人.当然,范德蒙还有一个读者十分熟知的工作,便是计算了范德蒙行列式,
这一行列式对于后续的研究有非常重要的地位.

除此之外,范德蒙的工作也得到了进一步的推广.1772年,拉普拉斯在一篇论文中证明了范德蒙提出的一些规则,
并推广了他的展开行列式的方法,便有了大家熟知的按多行(多列)展开的拉普拉斯定理.
1779年,裴蜀(正是前面所介绍的,实际上这里介绍的数学家很多都有工作的交织)发表了一篇《代数方程的一般理论》
的文章,这篇论文给出了解决非齐次线性方程组的方法,这个方法是他在克莱姆、范德蒙和拉普拉斯行列式理论基础上的总结.除此之外,
裴蜀在论文中还有其他很多关于行列式理论的发现:他改进了拉普拉斯展开式的另一个改进形式;得出了行列式的两行或两
列相同则结果为零的结论;并结合线性方程的消元法得出了著名的``裴蜀定理''等.

接下来对行列式理论做了可谓``大一统''工作的是著名数学家柯西——是的,又是他,一个和欧拉、高斯一样无处不在的数学家.
1812年,柯西率先使用了双下标的方式表示方程组系数(即$a_{11}$这样的有两个数字组成的下标),有趣的是柯西当年还没有
使用双竖线的方式表示行列式,而是采用$S(\pm a_{11}a_{22}\ldots a_{nn})$的形式.现在为人熟知的双竖线的表示形式是
后文将要介绍的矩阵论创始人凯莱率先使用的.

1815年,柯西发表了一篇关于行列式理论的基础性文章.在这篇文章中它不仅用这个名字代替了几个旧的术语,而且给出了系统的一般行列式乘法定
理,证明了新组的行列式是原来两个组的行列式的乘积.在这篇论文中,柯西第一次论述了包括一个给定的矩阵的伴随矩阵的思
想,以及通过展开任何行或者列来计算行列式的步骤.在柯西的行列式的工作中,还涉及到对称矩阵以及相似变换等问题.在柯西1826年的
《微积分在几何中的应用教程》中,讨论了后续学习中将要介绍的一些二次型理论,以及实对称矩阵特征值均为实数(后续会讲解)等重要结论.
除此之外,柯西在相似行列式的研究中,证明了大家熟知的相似变换有相同的特征值的结论.由此可见,柯西这一数学天才对于后世的影响是无比深远的,
从记号层面的革新,到行列式展开、行列式乘法等理论的大一统,以及现在大家熟知的结论的证明,都能看出柯西贡献的突出与伟大.

继柯西之后在行列式理论方面最高产的人就是德国数学家雅可比(J.Jacobi,1804-1851),他引进了函数行列式,即``雅可比行列式''
(读者学习多元微积分时会十分熟悉这一名词),指出函数行列式在多重积分的变量替换中的作用,给出了函数行列式的导数公式.
雅可比的著名论文《论行列式的形成和性质》标志着行列式系统理论的建成.事实上,行列式在数学分析、几何学、线性方程组理论、二次型理论
等多方面的应用,促使行列式理论自身在19世纪也得到了很大发展.整个19世纪都有行列式的新结果.除了一般行列式的大量定理之外,
还有许多有关特殊行列式的其他定理都相继得到.

\subsection{矩阵理论的发展}
随着线性方程组和行列式理论的建立和发展,在行列式基础之上的矩阵理论发展非常迅速.``j矩阵''这个词是由西尔维斯特首先使用的,
他是为了将数字的矩形阵列区别于行列式而发明了这个术语.而实际上,矩阵这个课题在诞生之前就已经发展的很好了.从行列式的大量工作中明显的表现出来,
不管行列式的值是否与问题有关,方阵本身都可以研究和使用,矩阵的许多基本性质也是在行列式的发展中建立起来的.在逻辑上,
矩阵的概念应先于行列式的概念,然而在历史上次序正好相反.

虽然矩阵一词是西尔维斯特率先发明的,但英国数学家凯莱(A.Cayley,1821-1895)一般被公认为是矩阵论的创立者,因为在西尔维斯特创用矩阵术语以前,
凯莱对于矩阵的有关概念及其性质就有所研究.1843年,凯莱即己研究三阶以上的高阶矩阵的行列式理论(On the theory of
determinants),L.Gegenbauer、M-Lecat、L.H.Rice等在这个领域又进行了扩展.1846年,凯莱定义了转置矩阵以及对称矩阵,与现代的定义完全一致.
在1855—1858年间,凯莱在矩阵方面做了许多开创性的工作.1855年,凯莱注意到在线性方程组中使用矩阵是非常芳便的,因而引进矩阵以简化记号,这就有了
现在我们使用的阶梯矩阵等术语以及$AX=b$的记号.

1858年,凯莱发表了重要文章《矩阵论的研究报告》(A memoir on the theory ofmatrices).在该研究报告中,凯莱系统地阐述了矩阵的理论体系,如
矩阵概念的引入、相关概念和运算的定义,使得矩阵从零散的知识发展为系统完善的理论体系.凯莱定义了矩阵加法和数乘运算,并且从变换的复合引入了
矩阵乘法的运算法则,也给出了一些特殊矩阵例如零矩阵、单位矩阵等,同时也说明了两个矩阵相乘不符合交换律,但也着重强调了矩阵乘法是可结合的.
除此之外,凯莱也引入了逆矩阵的概念.凯莱在文章中采用单个的符号表示矩阵,证明了矩阵$A$可逆时,方程$AX=b$的解可以写为$X=A^{-1}b$,并且也给出了矩阵可逆时
\[A^{-1}=\frac{1}{|A|}A^*.\]
凯莱还利用一般的代数运算和矩阵运算的相似性得出了矩阵的一些结论.例如当行列式为零时矩阵不可逆,零矩阵不可逆,两个非零矩阵乘积可以为零矩阵等结论.
除此之外,凯莱在文章中采用单个的符号表示矩阵,推出了方阵的特征多项式的形式,并说明了特征多项式的根就是特征值的重要结论.除此之外,凯莱也证明了我们
后面要详细介绍的``哈密顿-凯莱''定理的一部分,这被称为``矩阵理论中最著名的理论之一''.

凯莱第一个把矩阵作为独立的概念提出来,并作为独立的理论加以研究.可以说,《矩阵论的研究报告》的公开发表,标志着矩阵理论作为一个独立数学分支的
诞生.

\subsection{线性代数的应用:解析几何的发展}


\subsection{线性空间与线性映射的角度}


\section{进阶:(线性)代数的进一步发展}
\subsection{抽象代数}


\subsection{泛函分析}


\section{未来:从线性代数出发能望到多远}


\section*{附:本讲义未竟专题概览}


\section*{参考资料}
\begin{enumerate}
    \item \href{https://zh.wikipedia.org/wiki/%E4%BB%A3%E6%95%B0}{维基百科:代数}
    \item \href{https://zhuanlan.zhihu.com/p/574858845}{知乎:代数发展史}
    \item 
\end{enumerate}
\vspace{2ex}
\centerline{\heiti \Large 内容总结}

\vspace{2ex}

\centerline{\heiti \Large 习题}
\vspace{2ex}
{\kaishu }
\begin{flushright}
    \kaishu

\end{flushright}
\centerline{\heiti A组}
\begin{enumerate}
    \item 
\end{enumerate}
\centerline{\heiti B组}
\begin{enumerate}
    \item
\end{enumerate}
\centerline{\heiti C组}
\begin{enumerate}
    \item
\end{enumerate}
