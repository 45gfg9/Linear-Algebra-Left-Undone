\chapter{预备知识}

线性代数作为大学的第一门数学课,预修要求并不高. 我们默认读者具有基本的高中数学知识,因此关于集合、映射以及向量的基本知识我们不在此赘述. 这一讲我们将从基本代数结构开始,以便后续线性空间的引入,然后我们将介绍本书中常见的概念——等价类和最常用的算法之一——高斯消元法.

\section{基本代数结构}

我们选择从基本代数结构谈起,因为在以往的实践中我们深切地体会到直接引入线性空间的跳跃. 因此我们希望从更具象的例子开始,首先引入``代数结构''这一基本概念,然后在下一节中自然地引出线性空间的定义.

我们首先考察一个简单的例子:实数集$\mathbf{R}$,它是一个集合. 在初中我们便知道,在$\mathbf{R}$上我们可以定义加法和乘法两种运算. 本质而言,运算是一种映射(或者更通俗而言,函数):

\begin{center}
    \begin{tabular}{rrcl}
        $+\enspace\colon$      & $\mathbf{R}\times\mathbf{R}$ & $\to$     & $\mathbf{R}$ \\
                               & $(a,b)$                      & $\mapsto$ & $a+b$        \\
        $\times\enspace\colon$ & $\mathbf{R}\times\mathbf{R}$ & $\to$     & $\mathbf{R}$ \\
                               & $(a,b)$                      & $\mapsto$ & $a\times b$
    \end{tabular}
\end{center}

上面的定义中出现了一个新的记号,即两个集合之间出现了乘号,这实际上是集合的笛卡尔积运算,定义如下:

\begin{definition}{笛卡尔积}{} \index{dikaerji@笛卡尔积 (Cartesian product)}
    设$A$和$B$是两个非空集合,我们把集合
    \[A\times B=\{(a,b) \mid a\in A, b\in B\}\]
    称为集合$A$和$B$的\term{笛卡尔积}.
\end{definition}

\begin{definition}{$n$ 元笛卡尔积}{}
    设 $A_i~(1\leqslant i\leqslant n)$ 是一族集合,我们把集合
    \[
    \prod_{i=1}^n A_i = \{(a_1, a_2, \cdots, a_n) \mid a_i\in A_i\}
    \]
    称为集族 $\{A_i\}$ 的 \term{$n$ 元笛卡尔积}. 特别地,当所有的 $A_i$ 都为集合 $A$ 时,记
    \[
    A^n = \prod_{i=1}^n A = \{(a_1, a_2, \cdots, a_n) \mid a_i \in A\}
    \]
    为集合 $A$ 的 $n$ 次\term{笛卡尔幂}. 为了和后文的矩阵记号保持一致,严格来说 $\mathbf{R}^n$ 中的符号应当是沿竖向书写的,或者横向书写时添加转置符号,例如
    \[
    \begin{pmatrix}
        1 \\ 2 \\ 3
    \end{pmatrix} = (1, 2, 3)^{\mathrm{T}} \in \mathbf{R}^3
    \]
    但是本书中为了简便性起见,书写时有时会省去右上角的转置符号,因此可能表达行向量或者列向量的含义,请读者在阅读时注意甄别.
\end{definition}

因此我们很容易理解$\mathbf{R}\times\mathbf{R}$作为集合长什么样,它的元素是形如$(a,b)$的有序对,其中$a,b\in\mathbf{R}$. 事实上,我们可以将$\mathbf{R}\times\mathbf{R}$看作平面上的点集,其中的元素$(a,b)$对应于平面上的一个点,这一点的横坐标为$a$,纵坐标为$b$. 类似地,空间中的点可以用三个实数来表示,也就是形如 $(x, y, z)$ 的有序对,其中 $x,y,z\in\mathbf{R}$. 也就是说,$\mathbf{R}^3$ 表示的是空间中的点集,在这种情况下我们有时会用``空间''来代指集合,而用``点''来代指集合的元素.

我们回到运算的映射表示,我们发现$+$和$\times$都以实数的有序对作为函数的自变量,函数值也是一个实数. 或许读者看到这里还是对运算的定义有些许迷茫,但如果我们回忆映射的基本定义$f\colon A\to B$表示给$A$中的任意元素$a$指派一个$B$中的元素$f(a)$,并将加法乘法写成$+(2,3)=5$,$\times(2,3)=6$,想必就会恍然大悟:$+$和$\times$实际上就是函数名,函数做的事情就是输入两个自变量然后进行加法/乘法运算得到结果,并把这个结果指派给自变量作为函数值.

在上述讨论中,我们所做的事情很简单,就是给定一个集合,然后在这一集合的元素之间定义运算. 实际上这就是代数系统的定义:
\begin{definition}{代数系统}{} \index{daishuxitong@代数系统 (algebraic system)}
    一般地,我们把一个非空集合$X$和与$X$相关的若干代数运算$f_1,\ldots,f_k$组成的系统称为\term{代数系统}(简称代数系),记作$\langle X \colon f_1,\ldots,f_k\rangle$.
\end{definition}

特别注意的是,代数系统上定义的运算必须保证封闭性,也就是运算后的结果必须仍然在集合$X$中. 这事实上早已由映射的方式对运算的定义保证了.

不难理解,代数系统其中蕴含的性质与其中定义的运算具有的性质是关联很大的. 例如从经典分析学的角度实数实际上是是这样一些东西的组合 $\langle\mathbf{R}\colon 0, 1, +, *, \leqslant\rangle$,只是我们在没有歧义时将其简记为了 $\mathbf{R}$. 接下来我们以实数域为例,介绍在代数学中关心的几个运算性质. 我们首先讨论实数域上的加法运算,以下性质对于任意$a,b,c\in\mathbf{R}$都成立:

\begin{enumerate}
    \item 结合律:$(a+b)+c=a+(b+c)$;

    \item 单位元:存在一个元素$0$,使得$a+0=0+a=a$;

    \item 逆元:对于任意$a$,存在一个元素$-a$,使得$a+(-a)=(-a)+a=0$(0为单位元);

    \item 交换律:$a+b=b+a$.
\end{enumerate}

对于乘法运算(可记为$\cdot$或$\times$),单位元一般记为$1$(更一般的可以记为$e$),逆元记为$a^{-1}$. 事实上,我们可以给出更多的例子:
\begin{example}{}{Abel 群}
    \begin{enumerate}
        \item 代数系统$\langle \mathbf{R}\setminus\{0\}\colon\circ\rangle$定义的一般乘法运算

        \item 代数系统$\langle \mathbf{R}^2\colon+\rangle$定义的平面向量的加法
    \end{enumerate}
    均满足上述四条运算性质.
\end{example}

事实上,我们可以对上面的定义做进一步的抽象. 我们可以忽略集合中元素的意义差异(元素可以表示实数,也可以在上述例子中表示平面向量等几何对象),同时也可以忽略运算定义的差异,只关心运算作用于集合元素的性质. 对于一般的代数系统$\langle G\colon\circ\rangle$,我们有如下定义:
\begin{definition}{群}{群}\index{qun@群 (group)}
    若运算$\circ$满足结合律,则称代数系统$\langle G\colon\circ\rangle$为\term{半群}\index{qun!banqun@半群 (semigroup)};若在半群基础上存在单位元,则称之为\term{含幺半群}\index{qun!hanyaobanqun@含幺半群 (monoid)};若在含幺半群基础上每个元素存在逆元,则称之为\term{群};若在群的基础上运算还满足交换律,则称之为\term{Abel 群},也称\term{交换群}\index{qun!abel@Abel 群 (Abelian group), 交换群 (commutative group)}.
\end{definition}

\autoref{def:群} 给出了我们本节第一个要讨论的代数结构——群的定义. 简而言之,代数结构就是在集合上定义具有某些特定性质的运算后得到的一类代数系统. 事实上,教材中42--44页给出了大量抽象的例子有助于同学们理解上述一系列群的定义,并且我们在后续学习矩阵的时候也会遇到一些群结构,相信这些实例能使读者体会到``在集合上定义运算''的方式的多样与抽象.

为方便书写,对于\autoref{def:群} 定义的群$\langle G\colon\circ\rangle$,在不引起混淆的情况下我们可以简写为群$G$. 除此之外,我们还需要指出以下两点:
\begin{theorem}{}{群的单位元逆元唯一}
    \begin{enumerate}
        \item 群的单位元唯一;

        \item 群中每个元素的逆元唯一.
    \end{enumerate}
\end{theorem}

\begin{proof}
    \begin{enumerate}
        \item 设$e_1$和$e_2$都是群$G$的单位元,则
              \[e_1=e_1\circ e_2=e_2.\]

        \item 设$b$和$c$都是$a$的逆元,则
              \[b=b\circ e=b\circ(a\circ c)=(b\circ a)\circ c=e\circ c=c.\]
    \end{enumerate}
\end{proof}

其中第一点的证明直接使用了单位元的性质,第二点的证明则使用了结合律和逆元的性质. 这里关于唯一性的证明是非常重要的:我们只需假设要证明唯一的东西有两个,然后说明这两个必然相等即可. 这一思想在之后证明矩阵的逆唯一等问题时也会用到,因此此处特别给出证明强调.

事实上,在很多集合上我们不仅可以定义一种运算,也可以定义两种甚至更多运算,在代数结构中我们仅讨论最多两种运算的情况. 事实上,我们最开始的实数集合定义加法和乘法的例子便可以引入一个新的代数结构——域:
\begin{definition}{域}{} \index{yu@域 (field)}
    我们称代数系统$\langle F\colon+,\circ\rangle$为一个\term{域},如果
    \begin{enumerate}
        \item $\langle F\colon+\rangle$是交换群,其单位元记作0;

        \item $\langle F\setminus\{0\}\colon\circ\rangle$是交换群;

        \item 运算$\circ$对$+$满足左、右分配律,即
              \begin{gather*}
                  a\circ(b+c)=a\circ b+a\circ c \\
                  (b+c)\circ a=b\circ a+c\circ a
              \end{gather*}
    \end{enumerate}
\end{definition}

显然,实数域$\mathbf{R}$上定义一般的实数加法和乘法后构成一个域. 实际上我们熟悉的例如有理数、实数等集合关于一般的加法和乘法运算都构成域,因此我们会经常使用``有理数域''、``实数域''等说法. 我们称数集对数的加法和乘法构成的域为数域,注意此处运算的定义必须是数学分析中定义的数的加法和乘法,不能是自定义的运算.
\begin{theorem}{}{}
    关于数域,我们有如下两个结论:
    \begin{enumerate}
        \item 数集$F$对数的加法和乘法构成数域的充要条件为:$F$包含0,1且对数的加、减、乘、除(除数不为0)运算封闭;

        \item 任何数域都包含有理数域$\mathbf{Q}$,即$\mathbf{Q}$是最小的数域.
    \end{enumerate}
\end{theorem}

上述定理的证明可见教材46页. 事实上,如果加法和乘法的定义不是数的加法和乘法,我们可以定义除了数域之外的域,我们将在本讲介绍完等价类的概念后给出这样的例子.

当然,还有一种代数结构对于$\circ$运算的要求有所降低,但也有广泛的应用,这就是环:
\begin{definition}{环}{} \index{huan@环 (ring)}
    我们称代数系统$\langle R\colon+,\circ\rangle$为一个\term{环},如果
    \begin{enumerate}
        \item $\langle R\colon+\rangle$是交换群,其单位元记作0;

        \item $\langle R\colon\circ\rangle$是幺半群;

        \item 运算$\circ$对$+$满足左、右分配律,即
              \begin{gather*}
                  a\circ(b+c)=a\circ b+a\circ c \\
                  (b+c)\circ a=b\circ a+c\circ a
              \end{gather*}
    \end{enumerate}

    若进一步每个非$0$($+$运算单位元)元素关于$\circ$都有逆元,则称之为\term{除环}\index{huan!chu@除环 (division ring)}. 另外,若上述定义中$\circ$运算满足交换律,则称为\term{交换环}\index{huan!jiaohuan@交换环 (commutative ring)},结合上述除环和交换环两个定义,我们可以发现,交换除环即为域.
\end{definition}

\begin{example}{}{}
    利用定义验证下述关于代数系统的结论:
    \begin{enumerate}
        \item 整数集$\mathbf{Z}$对整数的加法和乘法构成一个交换环,但不是域;

        \item 设$C[a,b]$是闭区间$[a,b]$上的连续函数的集合;它对函数的加法和乘法构成一个环;

        \item 设$\mathbf{Q}(\sqrt{2})=\{a+b\sqrt{2} \mid a,b\in\mathbf{Q}\}$,则$\mathbf{Q}(\sqrt{2})$是一个数域.
    \end{enumerate}
\end{example}

我想大部分读者都会对抽象出代数结构的原因表示不解,如果这个问题无法解答,我想在下一章直接引入抽象的线性空间更会引发同学们对于``学了这个有什么用''的怀疑. 我们可以举一些不那么贴切但具象的例子来说明这其中的意义. 读者高中阶段想必大都经受过解析几何的摧残,大家在拿到题目时总会首先观察到题目属于``定点''、``定值''或是``极值''等问题,大家将自动与自己做题的经验或技巧匹配用于解答这几类问题. 同理,在研究一个特定的代数系统(例如定义了加法和乘法的实数域)的性质时,我们可以首先将其归类为群、环或是域等,然后我们只需要利用群环域各自的性质来研究这个代数系统的性质,而不需要再去研究这个代数系统的具体定义. 在这一过程中我们找到了一个模型,即将一个孤立的问题转化为了对一个更广泛的问题的研究,正如将解决上千道解析几何问题转化为研究几种作为模型的题目的解法. 这一``寻找模型''的思想在将来的学习生活中我们将经常遇见,在实际中例如投资股票时我们可以将投资转化为提高投资组合的期望收益而尽力降低方差(风险)的求取极值的问题,在理论中,例如在计算理论的学习中我们会将各种各样不同的计算机架构抽象成图灵机模型,这在可计算性的研究中是最基础的模型. 对于这类抽象问题感兴趣的同学不妨可以选择数学科学学院的抽象代数等课程,或是阅读本讲义的``后继''教程\href{https://frightenedfoxcn.github.io/notes/series/alg-for-cs/}{《写给计算机系学生的代数》}作进一步的了解. 事实上,对于对理论感兴趣的同学,抽象代数将是必不可少的基础课程,它将是密码学、量子计算、计算理论以及编程语言理论等诸多领域的必要基础.

当然,这段描述因为涉及的知识容量较大,大概无法说服每一个读者. 但我们会在学习线性空间、线性映射的过程中不断重复这些思想,直到读者具备的知识容量足够时,一定能领会其中的奥妙.

\section{复数域的引入}

本书前半段讨论的框架是实数域、复数域都适用的,当然为了简化,我们的例子大都来源于实数. 从多项式一讲开始,我们便会开始强调实数域和复数域结论的不同,因此我们有必要在此引入复数域,复数域有很多种等价的定义方法,此处我们选择使用二维向量的运算定义.

直观来看,实数位于数轴上,复数则分布在二维平面上,因此我们可以先考虑平面点集$\mathbf{R}^2$,并在其上定义加法和乘法运算使其成为一个域. 我们回顾高中学习的平面向量知识,我们记$\vec{e}_1=(1,0)$,$\vec{e}_2=(0,1)$,则$\mathbf{R}^2$上的任一向量$\vec{u}=(x,y)$可写为$x\vec{e}_1+y\vec{e}_2$. 此外,我们仍沿袭高中对向量长度的定义,即$\lvert\vec{u}\rvert=\sqrt{x^2+y^2}$.

在\autoref{ex:Abel 群} 中我们已经验证了$\mathbf{R}^2$上的向量加法满足Abel群的条件,因此我们只需要定义$\mathbf{R}^2$上的乘法使得代数系统$\langle\mathbf{R}^2\setminus\{(0,0)\}\colon\circ\rangle$也为Abel群. 这一乘法的构造需要满足一些自然的条件,同时也能实现构成Abel群的要求. 事实上,我们有如下定理:
\begin{theorem}{}{复数乘法构造}
    平面点集$\mathbf{R}^2$上存在唯一的乘法$\circ$,满足
    \begin{enumerate}
        \item (单位元) $\vec{u}\circ\vec{e}_1=\vec{e}_1\circ\vec{u}=\vec{u},\enspace\forall\vec{u}\in\mathbf{R}^2$;

        \item (长度可乘性) $\lvert\vec{u}\circ\vec{v}\rvert=\lvert\vec{u}\rvert\lvert\vec{v}\rvert$.
    \end{enumerate}
    此乘法满足交换律,且使得$\langle\mathbf{R}^2\colon+,\circ\rangle$成为域.
\end{theorem}

上述定理中第一个条件是非常自然的,因为在二维平面上,$\{(x,0) \mid x\in\mathbf{R}\}$实际上就是实数轴,因此$\vec{e}_1=(1,0)$相当于实数1,因此作为乘法单位元是非常自然的. 第二条长度可乘则看起来没那么自然,但在接下来的证明中我们将会了解到其意义.

\begin{proof}
    对任意向量$\vec{u}=(a,b)=a\vec{e}_1+b\vec{e}_2,\enspace \vec{v}=(c,d)=c\vec{e}_1+d\vec{e}_2$,我们利用乘法的第一条性质有
    \[\vec{u}\circ\vec{v}=ac\vec{e}_1+(ad+bc)\vec{e}_2+bd\vec{e}_2\circ\vec{e}_2.\]
    由此可见$\vec{u}\circ\vec{v}=\vec{v}\circ\vec{u}$,因此乘法满足交换律. 同时可知,要定义乘法,关键是定义$\vec{e}_2\circ\vec{e}_2$的值.

    记$\vec{e}_2\circ\vec{e}_2=(x,y)$,由长度可乘性知$x^2+y^2=1$,另一方面
    \[(\vec{e}_1+\vec{e}_2)\circ(\vec{e}_1-\vec{e}_2)=\vec{e}_1-\vec{e}_2\circ\vec{e}_2=(1-x,-y).\]
    由$|\vec{e}_1+\vec{e}_2|=|\vec{e}_1-\vec{e}_2|=\sqrt{2}$以及长度可乘性可得
    \[4=|(\vec{e}_1+\vec{e}_2)\circ(\vec{e}_1-\vec{e}_2)|^2=(1-x)^2+y^2.\]
    由此求出$x=-1,\enspace y=0$. 这说明
    \[\vec{e}_2\circ\vec{e}_2=-\vec{e}_1.\]
    由此得乘法的定义$\vec{u}\circ\vec{v}=(ac-bd)\vec{e}_1+(ad+bc)\vec{e}_2$,即
    \[(a,b)\circ(c,d)=(ac-bd,ad+bc).\]
    可验证,此乘法以$\vec{e}_1$为单位元,等式$(ac-bd)^2+(ad+bc)^2=(a^2+b^2)(c^2+d^2)$表明乘法满足长度可乘性. 上述证明亦表明乘法唯一(只能这么构造$\vec{e}_2\circ\vec{e}_2$).

    接下来我们很容易验证$\langle\mathbf{R}^2\colon+,\circ\rangle$满足域的定义,我们留作习题供读者自行验证.
\end{proof}

在\autoref{thm:复数乘法构造} 赋予的乘法下,$\langle\mathbf{R}^2\colon+,\circ\rangle$称为复数域$\mathbf{C}$. 我们自然地将$\vec{e}_1$合理简记为1,同时$\vec{e}_2$简记为$\i$,因为此时$(a,b)$即为$a+b\i$,并且利用$\vec{e}_2^2=-\vec{e}_1$可知$\i^2=-1$,这与我们熟知的虚数单位的定义是统一的. 这一代数表示引入的相关概念,如实部、虚部、纯虚数,以及复数四则运算法则在高中阶段大家都已熟知,在此不再赘述.

非零复数$z=x+y\i$也可写为极坐标的形式,即$z=|z|(\cos\theta+\i\sin\theta)$,其中$|z|=\sqrt{x^2+y^2}$为复数的平面表示的模长,$\theta\in\mathbf{R}$为连接原点与$z$的有向线段与$x$轴正方向的夹角(在相差$2\pi$整数倍的意义下唯一). 我们称$\theta$为复数$z$的辐角. 关于复数的模长我们有经典的三角不等式:
\begin{theorem}{}{}
    设$z,w\in\mathbf{C}$,则有$|z+w|\leqslant|z|+|w|$.
\end{theorem}

这一定理的几何意义是非常显然的,我们将$z$和$w$放在平面直角坐标系中观察就可以明白这就是经典三角不等式的复数版本. 等号成立的条件也显而易见,即$z$和$w$要么至少一个为0,要么都非零且$z$和$w$位于从原点出发的同一条射线上. 按照我们给出的定义,这个式子是不言自明的,我们需要证明的反而是一些大家在高中时就已经熟知的结果,例如:

\begin{theorem}{}{}
    设$z = (a, b) \in \mathbf{C}$,称 $\overline{z} = (a, -b)$ 为 $z$ 的共轭复数,则 $\overline{z} z = |z|^2$
\end{theorem}

严格的证明如下:

\begin{proof}
    首先验证 $z$ 和 $\overline{z}$ 具备相同的长度,然后由长度可乘性得到.
\end{proof}

这样定义的复数域的所有性质都和高中所学是完全一致的,此后我们默认读者具有足够的基础知识,不在此赘述.

\section{等价关系}

我们时常需要讨论集合中元素之间的关系. 例如直线间的平行、垂直、相交,或是数之间的大于、等于、小于关系.``关系''在我们的讲义中将会多次出现,因此我们很有必要在此形式化定义这一概念,并强调其中一类特定的关系——等价关系.

我们首先从(二元)关系这一概念入手. 实际上,这里的二元关系和日常生活中的关系是紧密相连的,例如在父子关系中如果将全人类作为谈论的背景集合,那么$(\text{小头爸爸}, \text{大头儿子})$这一有序二元组是符合这一关系的,但$(\text{章鱼哥}, \text{海绵宝宝})$显然不符合. 因此我们可以将父子关系看作笛卡尔积集合$\text{人类}\times\text{人类}$的子集. 更一般化的,集合$A$中的关系可以由$A\times A$的子集
\[\{(a,b) \mid a,b\in A, \enspace a\,R\,b\}\]
来刻画,其中$R$是这个关系本身(实质上是两个元素之间的某种性质),例如之前讨论的父子关系,或是数学中的大于、小于或同余等. 事实上,反过来,由$A\times A$的子集可以确定一个关系,例如我把全世界所有的父子组合放在这个集合中,那么这个集合就定义了人类中的父子关系.
\begin{example}{}{}
    以下是一些关系的例子:
    \begin{enumerate}
        \item 设$A=\mathbf{R}$,则$A\times A$的子集
              \[\{(a,b)\in A\times A \mid a^2+b^2=1\}\]
              定义了一个关系$R$,即
              \[a\,R\,b \iff a^2+b^2=1.\]

        \item 设$A=\{1,2,3\}$,则$A\times A$的子集
              \[\{(1,1),(1,2),(1,3),(2,2),(2,3),(3,3)\}\]
              定义了一个关系$R$,即
              \[a\,R\,b \iff a\leqslant b.\]

        \item 设$A$为任意数集,$A\to A$ 的函数$f$也是一种关系,集合$A\times A$的子集
              \[B=\{(a,b)\in A\times A \mid b=f(a)\}\]
              刻画了这一关系. 换言之,函数是一种特殊的关系,它要求$\forall a\in A$有且仅有一个元素$b\in A$使得$(a,b)\in B$,其中$B$为上述定义的$A\times A$的子集,这个子集就是我们熟知的函数图像.

        \item 设$A=\mathbf{Z}$,关系$R$满足$a\,R\,b\iff a\equiv b \bmod n$,即模$n$同余,则$A\times A$的子集
              \[\{(a,b)\in A\times A \mid a\equiv b \bmod n\}\]
              可以刻画这一关系.
    \end{enumerate}
\end{example}

接下来我们要讨论一种特别的关系,即等价关系. 它对关系$R$有一定的规定:
\begin{definition}{}{等价关系}
    集合$A$中关系若满足以下条件:
    \begin{itemize}
        \item (自反性) $\forall a\in A, \enspace a\,R\,a$;

        \item (对称性) 若$a\,R\,b$,则$b\,R\,a$;

        \item (传递性) 若$a\,R\,b$,$b\,R\,c$,则$a\,R\,c$,
    \end{itemize}
    则称$R$为$A$的一个等价关系. 进一步地,若$R$是集合$A$的一个等价关系且$a,b\in A$,若$a\,R\,b$,则称$a$,$b$关于$R$是等价的,并把$A$中所有与$a$等价的元素集合
    \[\overline{a}=\{b\in A \mid b\,R\,a\}\]
    称为$a$所在的等价类,$a$称为这个等价类的代表元素,并记$\{\overline{a}\}$为所有等价类为元素构成的集族.
\end{definition}

我们可能需要一个例子来理解这些概念. 我们不难证明,初等数论中的同余关系是一种等价关系,以模3同余为例,我们取整体集合为正整数集合,对于3,它的等价类就是所有和3模3同余的元素集合,即所有3的倍数. 同理,对于1,它所在的等价类就是模3余1的全体正整数,2所在的等价类是全体模3余2的正整数. 除此之外,我们还发现一个特点,即这三个等价类将原集合分成了三个无交集的子集
\begin{gather*}
    \overline{0}=\{3k\mid k\in\mathbf{Z}\} \\
    \overline{1}=\{3k+1\mid k\in\mathbf{Z}\} \\
    \overline{2}=\{3k+2\mid k\in\mathbf{Z}\}
\end{gather*}
且它们的并集就是原集合,即这三个等价类构成了原集合的一个\term{分划}\index{fenhua@分划 (partition)}(即分为并为原集合且互不相交的子集). 这一结论对所有等价类都成立,是很直观的结论:
\begin{theorem}{}{等价类的性质}
    设$R$是集合$A$的等价关系,则由所有不同的等价类构成的子集族$\{\overline{a}\}$是$A$的分划. 反之,我们也可以基于分划在$A$中定义等价关系.
\end{theorem}

证明这一定理需要一个引理:
\begin{lemma}{}{}
    设$R$是集合$A$的等价关系,$a,b\in A$,则$\overline{a}=\overline{b}\iff a\,R\,b$.
\end{lemma}
这一引理说明$a$和$b$等价当且仅当它们等价类相同,或者说在同一个等价类中,相信根据等价类的定义这是很显然的结论.

这一引理还有一个重要的推论:
\begin{corollary}{}{等价类的性质}
    设$R$是集合$A$的等价关系,$a,b\in A$,则下面二者必成立其一:
    \begin{enumerate}
        \item $\overline{a}\cap\overline{b}=\varnothing$;

        \item $\overline{a}=\overline{b}$.
    \end{enumerate}
\end{corollary}
即等价类要么相等要么不相交,这一结论也是非常自然的,且由这一结论我们很容易证明\autoref{thm:等价类的性质}. 如果对这些定理的证明细节感兴趣的读者可以参看教材第5页的定理1.1和1.2.

进一步此我们可以定义商集的概念:
\begin{definition}{商集}{} \index{shangji@商集 (quotient set)}
    设$R$是集合$A$的等价关系,以关于$R$的等价类为元素的集合(实际上是集合构成的集合,又称集族)$\{\overline{a}\}$称为$A$对$R$的\term{商集},记为$A/R$. 由
    \[\pi(a) = \overline{a}, \enspace \forall a\in A\]
    定义的$A$到$A/R$上的映射$\pi$称为$A$到$A/R$上的自然映射.
\end{definition}
我们可以看到,自然映射$\pi$将$A$中的元素$a$映到自己所在的等价类$\overline{a}$. 基于上述定义,我们可以完成在基本代数结构一节中遗留的一个问题:我们能否定义非数域的域?答案是肯定的,如果同学们对密码学感兴趣的应当听闻过有限域这一概念,接下来我们将通过简单的例子来说明这一概念.

\begin{example}{}{有限域}
    设$\mathbf{Z}_n$是$\mathbf{Z}$关于模$n$同余关系$R$的商集(也称$\mathbf{Z}$的模$n$剩余类),即
    \[\mathbf{Z}_n=\mathbf{Z}/R=\{\overline{0},\overline{1},\ldots,\overline{n-1}\}.\]
    即$\mathbf{Z}_n$中的元素是$n$个集合,其中第$i$个集合是全体模$n$余$i-1\enspace(i=1,2,\ldots,n)$的整数构成的集合.

    在$\mathbf{Z}_n$上我们仍然希望它可以继承$\mathbf{Z}$上的加法和乘法运算,因为这是整数作为环结构最重要的两种运算(减法就是加上相反数,除法无法在整数集合定义,因为会带来非整数的有理数). 我们首先定义对加法运算的继承:
    \[\overline{a}\oplus\overline{b}=\overline{a+b}.\]
    它的含义是$a$对应的等价类与$b$对应的等价类的和就是$a$和$b$的和对应的等价类,因此这样的定义是自然地``继承''了$\mathbf{Z}$上的加法运算. 举个例子,当$n=3$时,$\overline{1}+\overline{2}=\overline{1+2}=\overline{3}=\overline{0}$. 要注意的是,这里$a$和$b$并不一定要在$0$到$n-1$之间,因为事实上$\overline{a}=\overline{kn+a}\enspace(k\in\mathbf{Z})$. 我们只需对$a$,$b$以及$a+b$对$n$取模就可以将它们控制在$0$到$n-1$之间且表示的是同一个运算表达式(因为本质上只是我们选取了同一个等价类的不同代表元素进行计算,例如$n=3$时,$\overline{1}\oplus\overline{2}=\overline{4}\oplus\overline{8}=\overline{0}$),这被称为是等价关系与加法的相容性,我们在这个例子之后将详细讨论.

    接下来我们需要定义乘法$\circ$,同样是一个自然的定义,即$\overline{a}\circ\overline{b}=\overline{ab}$. 读者不妨自行验证它与等价关系的相容性. 我们很容易验证$\forall n\in\mathbf{Z}$且$n\geqslant 2$,$\langle \mathbf{Z}_n\colon\oplus,\circ\rangle$构成一个含幺交换环. 教材43页例8和45页例3中有详细的证明,因为较为显然此处从略. 我们要讨论的是何时$\langle \mathbf{Z}_n\colon\oplus,\circ\rangle$构成域,由此我们便构造了一个非数域的域,并且元素个数是有限的.

    我们这里可以给出结论:$\langle \mathbf{Z}_n\colon\oplus,\circ\rangle$是域当且仅当$n$是素数. 这一结论的证明需要一些数论的知识,我们放在习题中供感兴趣的同学证明. 如果一个域中只有有限个元素,这样的 $\mathbf{Z}_n$ 称为有限域. 可以证明对于任何一个域,要么自然数集是其一个子集,此时我们称域的特征为 $0$,例如之前介绍的数域(有理数域、实数域、复数域等),要么它包含了某个 $\mathbf{Z}_p$,这时我们称域的特征为 $p$. 本书中没有特别说明时均假设域的特征为 $0$,即我们总是可以找到任意多有限个的不同整数.
\end{example}

然而,这里我们会遇到一个很常见的问题,即上面定义的加法和乘法,真的是``合理''的吗?具体来说,令$R$表示上例中模$n$同余的等价关系,则
\begin{enumerate}
    \item 若$a\,R\,b$,$c\,R\,d$,则是否有$\overline{a}\oplus\overline{c}=\overline{b}\oplus\overline{d}$;
    \item 若$a\,R\,b$,$c\,R\,d$,则是否有$\overline{a}\circ\overline{c}=\overline{b}\circ\overline{d}$?
\end{enumerate}
如果不是,那么$\mathbf{Z}_n$上定义的加法和乘法将不再是一个本讲开头所说的映射. 我们以加法为例展开讨论,实际上乘法同理. 因为$a\,R\,b$表明$\overline{a}=\overline{b}$,$c\,R\,d$表明$\overline{c}=\overline{d}$,因此如果$\mathbf{Z}_n$上的加法是一个映射$\oplus(x,y)\colon\mathbf{Z}_n\to \mathbf{Z}_n$,那么必须有$\oplus(\overline{a},\overline{c})=\oplus(\overline{b},\overline{d})$,否则两个相同的元素的加法会得到不同的元素,也就是说加法可以把同一组元素映到多个值,那么显然不再是一个映射,乘法也是同理. 当然,很幸运的是,上面的定义保证了这一``合理性'',我们事实上在上例中已经给出了这一合理性的例子:$n=3$时,$\overline{1}=\overline{4},\enspace\overline{2}=\overline{8}$,并且$\overline{1}\oplus\overline{2}=\overline{0}=\overline{4}\oplus\overline{8}$,这里加法仍然满足映射的条件. 当然我们可以给出更严谨的一般性证明,在此之前我们先给这一``合理''的性质一个更规范的定义:
\begin{definition}{}{}
    设$A$是一个集合,$R$是$A$上的一个等价关系,若$A$上有二元运算$+$,且我们在$A/R$上以如下自然的方式继承了$A$上的$\oplus$运算:
    \[\overline{a}\oplus\overline{b}=\overline{a+b},\enspace\forall a,b\in A,\]
    则我们称$\oplus$与$R$\term{相容(compatible)}当且仅当$a\,R\,b$,$c\,R\,d$时有$\overline{a}\oplus\overline{c}=\overline{b}\oplus\overline{d}$,或者说等价的,$(a+b)R(c+d)$.
\end{definition}

在这种情况下,我们也称$\oplus$在$A/R$上是\term{良定义(well-defined)}的,但良定义这一词用途较广,一般指这一定义在当前语境下是否可接受,不一定用在商集相关的场合.

这一定义有两个要点,其一是把原集合上的运算自然地继承到商集上,其二是这一继承还能使得运算是一个映射. 事实上,相容性(良定义性)是非常重要的,所谓良定义,即这个东西就其自身不产生矛盾,这在形而上学家的论述中最常出现的例子是``圆的方''这个语汇. 良定义的问题在商集中特别典型,因为这是这个系列中最容易出现从某个集合``诱导''(即自然地投射到商结构)定义出新东西的情形,它也有相当大的可能性(尽管本讲义中显然避免了这种情况)是不良定义的. 在给出规范定义后,我们可以证明如下结论:

\begin{theorem}{}{}
    模$n$同余关系$R$与加法$\oplus$和乘法$\circ$相容.
\end{theorem}
\begin{proof}
    我们只证明加法的相容性,乘法的相容性留作练习. 设$a\,R\,b$,$c\,R\,d$,则可以设$b=a+kn$,$d=c+ln$,其中$k,l\in\mathbf{Z}$,则$\overline{b}\oplus\overline{d}=\overline{b+d}=\overline{a+c+(k+l)n}=\overline{a+c}=\overline{a}\oplus\overline{c}$,故$\oplus$与$R$相容.
\end{proof}

\section{高斯消元法}

高斯消元法是一种求解线性方程组的方法,我们先从一个简单的例子开始:

\begin{example}{}{高斯消元法引入}
    求解线性方程组
    \begin{numcases}{}
        x_1+3x_2+x_3 & =2 \\
        3x_1+4x_2+2x_3 & =9 \\
        -x_1-5x_2+4x_3 & =10
    \end{numcases}

    将方程$(1)$分别乘以$-3$和$1$加到方程$(2)$,$(3)$上消去两个方程中的$(1)$,得

    \begin{numcases}{}
        -5x_2-x_3=3 \\
        -2x_2+5x_3=12
    \end{numcases}

    再将方程$(5)$乘以$-\dfrac{5}{2}$加到方程$(4)$上消去$x_2$可知$x_3=2$,依次代回$(4)$,$(1)$可知$x_1=3$,$x_2=-1$,$x_3=2$.
\end{example}

可以看出,求解线性方程组无非是通过加、减、乘等尽可能地减少每个方程中未知数的个数,即消元.在上面的例子中,通过几次消元后,我们得到了一个更简单的方程组

\begin{numcases}{}
    x_1+3x_2+x_3=2 \\
    -5x_2-x_3=3 \\
    x_3=2
\end{numcases}

回代本质上是为了得到一个更简单的方程组

\begin{numcases}{}
    x_1=3 \\
    x_2=-1 \\
    x_3=2
\end{numcases}.

下面我们将这种方法一般化.
一般的,对于一个由$m$个方程组成的$n$元(即变量数为$n$)线性方程组
\[ \begin{cases} \begin{aligned}
            a_{11}x_1+a_{12}x_2+\cdots+a_{1n}x_n & = b_1           \\
            a_{21}x_1+a_{22}x_2+\cdots+a_{2n}x_n & = b_2           \\
                                                 & \vdotswithin{=} \\
            a_{m1}x_1+a_{m2}x_2+\cdots+a_{mn}x_n & = b_m
        \end{aligned} \end{cases} \]
将其系数按照如下的方式排列,称其为一个矩阵
\[\begin{pmatrix}
        a_{11} & a_{12} & \cdots & a_{1n} \\
        a_{21} & a_{22} & \cdots & a_{2n} \\
        \vdots & \vdots & \ddots & \vdots \\
        a_{m1} & a_{m2} & \cdots & a_{mn}
    \end{pmatrix}\]
且记$\vec{b}=(b_1,b_2,\ldots,b_m)^\mathrm{T}$,若$\vec{b}=\vec{0}$则称此方程为齐次线性方程组,否则为非齐次线性方程组. 再将$n$个未知量记为$n$元列向量$X=(x_1,x_2,\ldots,x_n)^\mathrm{T}$,我们便可以把方程组简记为$AX=\vec{b}$.

令$\vec{\beta}_i=(a_{1i},a_{2i},\ldots,a_{mi})^\mathrm{T}$,即方程组系数矩阵的某一列,则方程组还可以记为$x_1\vec{\beta}_1+x_2\vec{\beta}_2+\cdots+x_n\vec{\beta}_n=\vec{b}$\phantomsection\label{线性方程的向量表示},这一形式将在之后多次见到.

在以上的记号下,我们定义几个概念:

增广矩阵:$(A,\vec{b})$,即左$n$列为系数矩阵,最右列为列向量$\vec{b}$的$b+1$列矩阵.

阶梯矩阵:系数全零行在最下方,并且非零行中,在下方的行的第一个非零元素一定在上方行的右侧(每行第一个非零元素称主元素).

简化阶梯矩阵:阶梯矩阵中每个主元素所在列的其余元素全为$0$.

有了这些定义,我们可以将解线性方程组的过程转化为矩阵的初等行变换. 高斯消元法的一般步骤如下:
\begin{center}
    线性方程组$\overset{1}{\longrightarrow}$增广矩阵$\overset{2}{\longrightarrow}$阶梯矩阵$\overset{3}{\longrightarrow}$(行)简化阶梯矩阵$\overset{4}{\longrightarrow}$解
\end{center}

\begin{enumerate}[label=步骤\arabic*~]
    \item 只需要将线性方程组转化为$(A, \vec{b})$的形式,得到增广矩阵;

    \item 通过初等行变换后,得到阶梯矩阵.

    \item 将主元素化1后将主元素所在列的其他元素均通过初等行变换化为0即可,即化为简化阶梯矩阵.

    \item \label{item:1:解方程组}
          我们分三种情况讨论:
          \begin{enumerate}
              \item 有唯一解:没有全零行,最后一个主元素的行号与系数矩阵的列数相等,且行简化阶梯矩阵对角线上全为1,其余元素均为0,此时可以直接写出解.前面的例子就属于这种情况.

              \item 无解:出现矛盾方程,即系数为0的行的行末元素不为0,此时直接写无解即可;

              \item 有无穷解:非上述情况. 此时设出自由未知量将其令为$k_1,k_2,\ldots$,然后代入增广矩阵对应的方程组即可. 注意选取自由未知量时,选取没有主元素出现的列对应的未知量会与标准答案更贴近,当然选择其他作为自由未知量也可以.例如如果最后得到的简化阶梯矩阵是
              \[\begin{pmatrix}
                1 & 1 & 0 & 2 \\
                0 & 0 & 1 & -1
             \end{pmatrix}\]
              则解为$\vec{X}=(x_1,x_2,x_3)^\mathrm{T}=(2,0,-1)^\mathrm{T}+k(1,-1,0)^\mathrm{T}$.
          \end{enumerate}
\end{enumerate}

高斯消元法是线性代数中最常用的算法之一,是之后解决大量问题所需要掌握的基本方法,同时也是考试中一定会考察的内容,无论是单独一个大题考察,还是嵌入在其它问题中.
注意考试中单独考察解方程时,时间充足时建议将过程写完整,标明初等行变换的具体步骤,并且至少写出阶梯矩阵和行简化阶梯矩阵. 除此之外,需要保证计算中尽量减少错误,时间充足可以解完方程后将答案代入进行检查.

需要强调的是,不要认为本节内容很简单就放过了,实际上如果长期不计算高斯消元法很容易陷入眼高手低的窘境,因此希望各位同学熟悉高斯消元法的基本步骤并熟练应用.



从高斯消元法开始,我们正式进入线性代数的学习. 实际上,上述 \ref*{item:1:解方程组} 中关于方程组解的情况的讨论我们是浮于表面,是基于算法最后得到的矩阵的形式进行的讨论,但事实上,这背后蕴含着更深刻的意义. 我们将会在接下来的十余个章节中讲述线性代数中的核心概念,并在\nameref{chap:朝花夕拾}中回过头来重新审视线性方程组解的问题. 相信在那时,经历十余章各式抽象概念和运算技巧的洗礼后再来回味这一问题的你,定有``守得云开见月明''之感,对线性代数的理解也会更深一层.

\begin{summary}

    本讲为了后续章节讲述方便引入了一些基本概念和算法. 尽管这是一门面向理工科应用的数学课,但我们仍然希望以最自然的方式引入概念,而非填鸭式地轰炸,因此我们首先从大家最熟悉的实数集合开始,讨论在集合上定义运算的方法:我们逐步加强条件,引入了三种基本的代数结构——群、环和域,并且给出了一些例子,并简单讨论了定义代数系统的意义. 事实上,下一讲开始要介绍的线性空间也是一种特殊的代数结构,因此首先引入代数结构对于我们自然展开接下来的讨论有很大的帮助,不至于让读者觉得非常突兀.

    接下来我们也从域的定义入手,构造了$\mathbf{R}^2$上的乘法运算使其构成了一个域,并且我们发现这里的定义与高中学习的复数乘法是完全一致的. 之后我们引入了等价关系的概念,这一概念在后续的讲义中将会多次出现,其重要意义就是将一个集合划分成了几个等价的区域. 最后我们讨论了高斯消元法的一般步骤,这是我们接下来解决线性空间中各类问题绕不开的算法.

\end{summary}

\begin{exercise}
    \exquote[浙江大学数学科学学院教授吴志祥]{我这门课很简单,只有简单的加减乘除四则运算,甚至除法都不太需要.}

    \begin{exgroup}
        \item 完善\autoref{thm:复数乘法构造} 中的证明,即证明$\mathbf{R}^2$在平面向量加法和如\autoref*{thm:复数乘法构造} 定义的乘法下构成一个域.

        \item 完成教材48页第13题.

        \item 求齐次线性方程组$\begin{cases}
                x_1+x_2+x_3+4x_4-3x_5=0   \\
                2x_1+x_2+3x_3+5x_4-5x_5=0 \\
                x_1-x_2+3x_3-2x_4-x_5=0   \\
                3x_1+x_2+5x_3+6x_4-7x_5=0
            \end{cases}$的通解.

        \begin{answer}
            \begin{align*}
                A ={} &
                \begin{pmatrix}
                    1 & 1  & 1 & 4  & -3 \\
                    2 & 1  & 3 & 5  & -5 \\
                    1 & -1 & 3 & -2 & -1 \\
                    3 & 1  & 5 & 6  & -7
                \end{pmatrix}
                \rightarrow
                \begin{pmatrix}
                    1 & 1  & 1 & 4  & -3 \\
                    0 & -1 & 1 & -3 & 1  \\
                    0 & -2 & 2 & -6 & 2  \\
                    0 & -2 & 2 & -6 & 2
                \end{pmatrix} \rightarrow \\
                      &
                \begin{pmatrix}
                    1 & 1 & 1  & 4 & -3 \\
                    0 & 1 & -1 & 3 & -1 \\
                    0 & 0 & 0  & 0 & 0  \\
                    0 & 0 & 0  & 0 & 0
                \end{pmatrix}
                \rightarrow
                \begin{pmatrix}
                    1 & 0 & 2  & 1 & -2 \\
                    0 & 1 & -1 & 3 & -1 \\
                    0 & 0 & 0  & 0 & 0  \\
                    0 & 0 & 0  & 0 & 0
                \end{pmatrix}.
            \end{align*}
            令 $ x_3 = k_1, x_4 = k_2, x_5 = k_3$,有 $x_1 = -2k_1 - k_2 + 2k_3,\enspace\allowbreak x_2 = k_1 - 3k_2 + k_3 $,则
            \[ X = (x_1, x_2, x_3, x_4, x_5)^\mathrm{T} = k_1 \begin{pmatrix} -2 \\ 1 \\ 1 \\ 0 \\ 0 \end{pmatrix} + k_2 \begin{pmatrix} -1 \\ -3 \\ 0 \\ 1 \\ 0 \end{pmatrix} + k_3 \begin{pmatrix} 2 \\ 1 \\ 0 \\ 0 \\ 1 \end{pmatrix} \qquad k_1, k_2, k_3 \in \mathbf{R} \]
        \end{answer}

        \item 求非齐次线性方程组$\begin{cases}
                x_1-x_2+2x_3-2x_4+3x_5=1     \\
                2x_1-x_2+5x_3-9x_4+8x_5=-1   \\
                3x_1-2x_2+7x_3-11x_4+11x_5=0 \\
                x_1-x_2+x_3-x_4+3x_5=3
            \end{cases}$的通解.

        \begin{answer}
            \begin{align*}
                \bar{A} ={} &
                \begin{pmatrix}
                    1 & -1 & 2 & -2  & 3  & 1  \\
                    2 & -1 & 5 & -9  & 8  & -1 \\
                    3 & -2 & 7 & -11 & 11 & 0  \\
                    1 & -1 & 1 & -1  & 3  & 3
                \end{pmatrix}
                \rightarrow
                \begin{pmatrix}
                    1 & -1 & 2  & -2 & 3 & 1  \\
                    0 & 1  & 1  & -5 & 2 & -3 \\
                    0 & 1  & 1  & -5 & 2 & -3 \\
                    0 & 0  & -1 & 1  & 0 & 2
                \end{pmatrix} \rightarrow \\
                            &
                \begin{pmatrix}
                    1 & -1 & 2  & -2 & 3 & 1  \\
                    0 & 1  & 1  & -5 & 2 & -3 \\
                    0 & 0  & -1 & 1  & 0 & 2  \\
                    0 & 0  & 0  & 0  & 0 & 0
                \end{pmatrix}
                \rightarrow
                \begin{pmatrix}
                    1 & 0 & 0 & -4 & 5 & 4  \\
                    0 & 1 & 0 & -4 & 2 & -1 \\
                    0 & 0 & 1 & -1 & 0 & -2 \\
                    0 & 0 & 0 & 0  & 0 & 0
                \end{pmatrix}.
            \end{align*}
            令 $ x_4 = k_1, x_5 = k_2 $,有 $x_1 = 4k_1 - 5k_2 + 4,\enspace\allowbreak x_2 = 4k_1 - 2k_2 - 1,\enspace\allowbreak x_3 = k_1 -2 $,则
            \[ X = (x_1, x_2, x_3, x_4, x_5)^\mathrm{T} = k_1 \begin{pmatrix} 4 \\ 4 \\ 1 \\ 1 \\ 0 \end{pmatrix} + k_2 \begin{pmatrix} -5 \\ -2 \\ 0 \\ 0 \\ 1 \end{pmatrix} + \begin{pmatrix} 4 \\ -1 \\ -2 \\ 0 \\ 0 \end{pmatrix} \qquad k_1, k_2 \in \mathbf{R} \]
        \end{answer}

        \item 求解线性方程组$\begin{cases}
                x_1+x_2+x_3=1   \\
                x_1+2x_2-5x_3=2 \\
                2x_1+3x_2-4x_3=5
            \end{cases}$.

        \begin{answer}
            见教材P33例3. 无解.
        \end{answer}
    \end{exgroup}

    \begin{exgroup}
        \item 设$A$是一个Abel群,$A$的运算是加法. 在$A$中定义乘法运算为$ab=0,\enspace\forall a,b\in A$. 证明:$A$为一个环,而且它的加法单位元与乘法单位元相同(我们称这种环为\term{零环}\index{huan!ling@零环 (zero ring)}).

        \item 证明:若集合$A$上的二元关系$R$满足
        \begin{enumerate}
            \item $a\,R\,a,\enspace\forall a\in A$;

            \item $\forall a,b,c\in A$,若$a\,R\,b$且$a\,R\,c$,则$b\,R\,c$.
        \end{enumerate}
        则$R$为$A$上的等价关系.
    \end{exgroup}

    \begin{exgroup}
        \item 证明:\autoref{ex:有限域} 中定义的$\langle \mathbf{Z}_n\colon\oplus,\circ\rangle$是域当且仅当$n$是素数.(提示:无论$n$是否为素数,$n\in\mathbf{Z}$且$n\geqslant 2$时$\langle \mathbf{Z}_n\colon\oplus,\circ\rangle$为交换环,因此是否为素数将决定这一结构中每个元素是否有逆元. 在初等数论中,我们熟知的裴蜀定理可以解决这一问题.)

        \item 本讲我们构造了$\mathbf{R}^2$上的乘法,从而定义了复数域的乘法运算. 本题希望探讨的是:$\mathbf{R}^3$无法构造出乘法使其成为一个域. 在高中的学习中我们知道,$\mathbf{R}^3$空间向量的一组基底为$\{\vec{e}_1=(1,0,0),\vec{e}_2=(0,1,0),\vec{e}_3=(0,0,1)\}$. 证明:$\mathbf{R}^3$没有乘法同时满足以下性质:
        \begin{enumerate}
            \item (单位元) $\forall \vec{u}\in\mathbf{R}^3,\enspace\vec{e}_1\cdot \vec{u}=\vec{u}\cdot \vec{e}_1$;

            \item (交换性) $\forall \vec{u},\vec{v}\in\mathbf{R}^3,\enspace\vec{u}\cdot \vec{v}=\vec{v}\cdot \vec{u}$;

            \item (长度可乘性) $\forall \vec{u},\vec{v}\in\mathbf{R}^3,\enspace|\vec{u}\cdot\vec{v}|=|\vec{u}||\vec{v}|$.
        \end{enumerate}
        按照如下思路给出详细证明过程:采用反证法. 假设乘法存在,则
        \begin{enumerate}
            \item 通过计算$(\vec{e}_1+\vec{e}_2)\cdot(\vec{e}_1-\vec{e}_2)$,$(\vec{e}_1+\vec{e}_3)\cdot(\vec{e}_1-\vec{e}_3)$,证明\[\vec{e}_2\cdot\vec{e}_2=\vec{e}_3\cdot\vec{e}_3=-\vec{e}_1.\]

            \item 证明$(\vec{e}_2+\vec{e}_3)\cdot(\vec{e}_2-\vec{e}_3)=0$得出矛盾.
        \end{enumerate}

        \item 尝试构造一个 4 元的域 $\mathbf{F}_4$.(提示:我们在上面的第一题中已经给出了 2 元的域 $\mathbf{Z}_2$,而且我们也知道 $\mathbf{Z}_4$ 不构成一个域,因此,我们考虑怎么把两个二元的域拼起来,可以尝试笛卡尔积的方式,并在其中定义二元运算来满足所需的性质. 进一步的两个事实是,对于所有的素数 $p$,都可以用类似的方法构造 $p^n$ 元的域;以及不是任意两个域的笛卡尔积都是域,这个例子已经被上面的第二题所说明.)
    \end{exgroup}
\end{exercise}
