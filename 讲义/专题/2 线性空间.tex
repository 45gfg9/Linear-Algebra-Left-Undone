\chapter{线性空间}

本讲我们将开始回答上一讲中最后留下的问题,即线性方程组有唯一解、无穷解或无解的本质原因.
这段旅程或许有些漫长,中间会有很多的铺垫,我们将从其中最为基础的概念——线性空间出发进行探讨.

回忆高斯消元法,方程组中每一行或一列都可以视为向量. 我们可以先看下面这个例子:
\begin{example}\label{ex:2:线性空间引入}
    考虑如下两个方程组
    \begin{multicols}{2}
        \begin{enumerate}
        \item $\begin{cases}
                x_1+x_2+x_3=0 \\ x_1+2x_2+3x_3=0 \\ 2x_1+3x_2+4x_3=0
            \end{cases}$
        \item $\begin{cases}
                x_1+x_2+x_3=0 \\ x_1+2x_2+3x_3=0 \\ x_1+3x_2+4x_3=0
            \end{cases}$
        \end{enumerate}
    \end{multicols}
    不难解得,第一个方程组有无穷解,第二个方程组有唯一解. 从高斯消元法的过程来看,
    第一个方程组的简化阶梯矩阵出现了全零行,其原因是显而易见的:因为方程组第一行
    和第二行相加正好是第三行,因此可以直接消去第三行,即三行的系数矩阵的三个行向量
    \[\alpha_1=(1,1,1),\enspace\alpha_2=(1,2,3),\enspace\alpha_3=(2,3,4)\]
    满足$\alpha_1+\alpha_2=\alpha_3$. 而第二个方程组系数矩阵行向量间没有类似的
    可互相消去的关系.
\end{example}

从上面这一例子中我们可以看出,方程组的解与系数矩阵的行向量之间的关系密切相关.
因此我们会有一个很自然的想法,即我们需要研究向量之间的关联. 受第一讲基本代数结构的启发,
我们应当自然地想到我们需要引入一个代数结构,从而使得我们可以统一地研究向量间的关联,
这一代数结构便是线性空间.

\section{线性空间的定义}
线性空间是我们接触的第一个核心概念,作为一种代数结构,它需要在非空集合$V$上定义运算. 其中第一个
运算是我们熟知的加法$+$. 在线性空间的定义中,我们要求$\langle V:+\rangle$构成Abel群,
即其中元素满足如下运算律:
\begin{enumerate}
    \item 结合律:$a+(b+c)=(a+b)+c,\enspace\forall a,b,c\in V$;

    \item 加法单位元:$\exists 0 \in V$,使得$\forall\alpha\in V$ 有 $a+0=0+a$;

    \item 逆元:$\forall\alpha\in V,\enspace \exists b\in V$,有$a+b=b+a=0$,记$b=-a$;

    \item 交换律:$\forall\alpha,\enspace \beta\in V,\enspace \alpha+\beta=\beta+\alpha$.
\end{enumerate}

第二种运算和之前学习的其他代数结构不同,我们需要首先引入一个数域$\mathbf{F}$,接下来在
$\mathbf{F}\times V$上定义取值于$V$的数乘运算,即$\mathbf{F}\times V$中的每个元素
$(\lambda,\vec{\alpha})\mapsto \lambda\vec{\alpha}\in V$,并且数乘运算满足以下性质:
$\forall \alpha,\beta \in V,\enspace\forall \lambda,\mu\in\mathbf{F}$以及$\mathbf{F}$
上的乘法单位元1,有
\begin{enumerate}
    \item $1\cdot \alpha=\alpha$;

    \item $\lambda(\mu\alpha)=(\lambda\mu)\alpha$;

    \item $(\lambda+\mu)\alpha=\lambda\alpha+\mu\alpha$;

    \item $\lambda(\alpha+\beta)=\lambda\alpha+\lambda\beta$.
\end{enumerate}

事实上这与我们高中学习的平面向量的数乘是非常类似的. 基于此,我们完整定义了一个线性空间,我们一般称集合$V$
关于上述两种运算在域$\mathbf{F}$上构成一个线性空间,简称为$V$在域$\mathbf{F}$上的线性空间,记作
$V(\mathbf{F})$. 如果$\mathbf{F}$是实(复)数域,则称$V$为实(复)数域上的线性空间.
关于线性空间的定义,我们还有如下说明:
\begin{enumerate}
    \item 由于加法运算构成Abel群,因此加法零元和逆元是唯一的,我们可以定义减法运算为加上一个元素的逆,即
    $\alpha-\beta=\alpha+(-\beta)$;
    \item 综合上述性质我们有方程$\lambda\beta+\lambda_1\alpha_1+\lambda_2\alpha_2+\cdots+\lambda_r\alpha_r=0$
    在$\lambda\neq 0$时的解为$\beta=-\lambda^{-1}\lambda_1\alpha_1-\lambda^{-1}\lambda_2\alpha_2-\cdots-\lambda^{-1}\lambda_r\alpha_r$;
    \item 线性空间还有一个重要的概念是运算封闭,即线性空间中的元素进行加法或数乘运算后,得到的元素仍然是属于线性空间的.
    这一点是定义要求的,加法封闭是 Abel 群的要求,数乘注意前述定义中数乘运算``取值于$V$''的要求;
    \item 或许同学们会疑惑为什么线性空间会要求上述这8条性质(加法、数乘各4条)而不能增减其中几条,对这一问题感兴趣的同学可以学习
    抽象代数中的模论,那时这一定义会显得非常自然. 这门课只要求你记忆这8条性质,并请务必牢记于心,考试可能要求你验证线性空间.
    记忆难度也并不大,Abel 群四条性质都有名称标注,数乘运算也是易于记忆的结合律和分配律加单位元性质;
    \item 特别注意线性空间定义在非空集合上,事实上根据加法构成Abel群的要求,最小的线性空间也必须至少包含加法单位元(可以记为$V=\{0\}$).
\end{enumerate}

\begin{example}
    三种非常常见的线性空间,希望读者能熟知其性质:
    \begin{enumerate}
        \item (多项式)$\mathbf{F}[x]_{n+1}=\{a_0+a_1x+\cdots+a_nx^n \mid a_i\in\mathbf{F}\}$关于多项式的加法和数乘构成线性空间,但
        \[\mathbf{F}[x]_{n+1}=\{a_0+a_1x+\cdots+a_nx^n \mid a_i\in\mathbf{F},a_i\neq 0\}\]
        不构成线性空间.

        注:书上常将多项式记为$\mathbf{F}[x]_{n+1}$,表示次数不超过$n$的多项式的集合.

        注意常见记号:$(k_1p_1+k_2p_2)(x)=k_1p_x(x)+k_2p_2(x)$.
        \item (复数与实数)可以验证:全体复数构成的集合是数域$\mathbf{C}/\mathbf{R}$上的线性空间. 此处一定注意复数$\mathbf{C}$在此处同时出现在集合和数域中.

        注意:这一例子表明,同一集合可以在不同数域上构成不同的线性空间,在下一讲接触维数的定义后,我们也将知道二者的
        维数是不一样的(见\autoref{ex:3:不同数域的维数}).

        当然,不同的集合也可以在同一个数域上构成不同的线性空间,例如$\mathbf{C(R)}$和$\mathbf{R(R)}$.
    \end{enumerate}
\end{example}

在上例以及习题中我们可以看到很多特殊的线性空间,它们集合中的元素不一定是数或向量,运算也不一定是熟知的
数的运算和向量的数乘,对这些空间我们需要学会熟练判断,从而加深对``在集合上定义运算''的理解.

\section{线性子空间}
我们首先介绍线性子空间的定义:
\begin{definition}
    设$W$是线性空间$V(\mathbf{F})$的非空子集,如果$W$对$V$中的运算也构成域$\mathbf{F}$
    上的线性空间,则称$W$是$V$的\keyterm{线性子空间}[linear subspace](简称\keyterm*{子空间}[subspace]).
\end{definition}

请一定注意定义中的非空子集,建议验证子空间时先验证非空. 接下来自然的问题便是,什么时候$V$的子集$W$对$V$中的运算
也构成域$\mathbf{F}$上的线性空间?事实上这一条件是惊人地简单与美观的:
\begin{theorem}\label{thm:2:子空间判别}
    线性空间$V(\mathbf{F})$的非空子集$W$为$V$的子空间的充分必要条件是$W$对于$V(\mathbf{F})$的线性运算封闭.
\end{theorem}

综上表明只要子空间非空且其中的元素满足对原空间的加法和数乘运算封闭即可构成原空间的子空间. 这一定理的证明也非常
简单,必要性显然(构成线性空间必须满足运算封闭),充分性我们只需要作如下思考:
\begin{enumerate}
    \item 结合律、分配律运算律是一定不变的,例如我们回顾加法结合律的定义$a+(b+c)=(a+b)+c,\enspace\forall a,b,c\in V$,
    由于这一性质对于任意$V$中元素成立,则若$a,b,c\in W\subset V$也必有这一性质成立(更通俗而言就是子集$W$中的
    元素也是$V$中的,因此必然受$V$中运算性质的限制);
    \item 我们根据上面的原则对8条性质一一验证,发现加法单位元和逆元仍不能保证存在,因为这不仅与运算法则相关,更与
    集合中元素的存在相关——取子集可能使得加法单位元和逆元被拿掉. 但在定理要求的数乘封闭性下这是不可能的:由于$\mathbf{F}$
    是数域,因此所有有理数都是其子集,因此$0,-1\in\mathbf{F}$. $\forall \vec{\alpha}\in V$,我们由于数乘封闭性
    可知,$0\cdot\vec{\alpha}=\vec{0}\in W$,$(-1)\cdot\vec{\alpha}=-\vec{\alpha}\in W$,因此$W$中也有加法单位元和逆元.
\end{enumerate}

证明具体书写见教材62-63页. 下面我们来看两个常见的例子体会子空间的判别方法:
\begin{example}\label{ex:2:常见子空间}
    回答下列关于子空间的判定问题:
    \begin{enumerate}[label=(\arabic*)]
        \item 说明$\mathbf{R}[x]_2$是$\mathbf{R}[x]_3$的子空间;

        \item 判断$W_1=\left\{(x,y,z) \mid \dfrac{x}{3}=\dfrac{y}{2}=z\right\},\enspace
            W_2=\{(x,y,z) \mid x+y+z=1,\enspace x-y+z=1\}$是否为$\mathbf{R}^3$的子空间;

        \item (线性方程组的解)试说明齐次线性方程组$AX=0$的解集是线性空间$\mathbf{F}^n$的一个子空间,
        但非齐次线性方程组的解不再构成线性空间(因为加法运算不封闭,具体见教材P62的2.2节开头的例子以及P86习题3(3).
    \end{enumerate}
\end{example}

上例中第二小问表明过原点的直线/平面构成三维空间的子空间,不过原点的无法保持线性性. 事实上 (2) 和 (3)
在表述同一个问题,(2) 从几何角度描述了 (3) 中齐次/非齐次线性方程组的解集. 事实上,在定义了子空间后,
如果一个线性空间的子集也构成线性空间,我们就可以对其进行同样的研究. 这一想法在我们后续的内容中十分重要,
现在需要大家先熟知子空间的定义和判别.

最后我们需要注意一个名词的定义. 线性空间有两个子空间称为平凡子空间,即仅含零元的子集$\{0\}$和其自身$V$.
而其它子空间称为非平凡子空间.

\section{线性表示 \quad 线性扩张}
在高中平面向量的学习中我们知道,两个单位向量$(0,1)$和$(1,0)$可以表示出整个平面的所有向量,高中我们也称
这样的向量为平面向量的基底. 接下来我们将二维平面扩展至任意线性空间,同样讨论有关于``表示''、``基底''的问题.

我们首先来看线性组合和线性表示的概念:
\begin{definition}
    设$V(\mathbf{F})$是一个线性空间,$\alpha_i\in V,\enspace\lambda_i\in \mathbf{F}\enspace(i=1,2,\ldots,m)$,
    则向量$\alpha=\lambda_1\alpha_1+\lambda_2\alpha_2+\cdots+\lambda_m\alpha_m$
    称为向量组$\alpha_1,\alpha_2,\ldots,\alpha_m$在域$\mathbf{F}$的线性组合,或说$\alpha$
    在域$\mathbf{F}$上可用向量组$\alpha_1,\alpha_2,\ldots,\alpha_m$线性表示.
\end{definition}
这和我们高中所学的用向量的基底表示其他向量是完全一致的. 基于此,我们给出线性扩张的定义:
\begin{definition}
    设$S$是线性空间$V(\mathbf{F})$的非空子集,我们称
    \[ \spa(S)=\{\lambda_1\alpha_1+\cdots+\lambda_k\alpha_k \mid \lambda_1,\ldots,\lambda_k\in\mathbf{F},\enspace\alpha_1,\ldots,\alpha_k\in S,\enspace k\in\mathbf{N_+}\} \]
    为$S$的\keyterm{线性扩张}[linear span],即$S$中所有有限子集在域$\mathbf{F}$上的一切线性组合组成的$V(\mathbf{F})$的子集.
\end{definition}
注意,$\spa$参考的是《线性代数应该这样学》的记号,《大学数学——代数与几何》中使用$L$表示线性扩张. 考虑到本讲义记号统一性,我们采用
更加常用并且不会与之后其它定义的记号冲突的$\spa$.

下面的定理告诉我们可以通过线性扩张构造子空间:
\begin{theorem}\label{thm:2:线性扩张构造子空间}
    线性空间$V(\mathbf{F})$的非空子集$S$的线性扩张$\spa(S)$是$V$中包含$S$的最小子空间.
\end{theorem}
仍然利用平面向量进行直观的理解,平面(也显然在平面向量加法和数乘下构成线性空间)$\mathbf{R}^2$可以由向量
$(1,0)$和$(0,1)$扩张而成.由这一定理的结果我们可以将一个向量组的线性扩张称为向量组的张成空间.这一定理的证明思想
非常重要,因此在此给出:

\begin{proof}
    \begin{enumerate}
        \item 首先我们证明$\spa(S)$是$V$的子空间.
        \begin{enumerate}[label=(\arabic*)]
            \item $\spa(S)$非空:由于$S$非空,且$S\subset\spa(S)$显然成立:取$\lambda=1,\enspace\forall s\in S,\enspace
            \lambda s=s\in\spa(S)$. 因此$\spa(S)$非空;
            \item 设$\vec{\alpha},\vec{\beta}\in\spa(S)$,则存在$\lambda_1,\ldots,\lambda_k\in\mathbf{F},\enspace
            \alpha_1,\ldots,\alpha_k\in S,\enspace\mu_1,\ldots,\mu_l\in\mathbf{F},\enspace\beta_1,\ldots,\beta_l\in S$,使得
            \begin{gather*}
                \vec{\alpha}=\lambda_1\alpha_1+\cdots+\lambda_k\alpha_k \\
                \vec{\beta}=\mu_1\beta_1+\cdots+\mu_l\beta_l
            \end{gather*}
            因此我们可以得到$\spa(S)$
            \begin{enumerate}
                \item 关于加法封闭:$\vec{\alpha}+\vec{\beta}=\lambda_1\alpha_1+\cdots+\lambda_k\alpha_k+\mu_1\beta_1+\cdots+\mu_l\beta_l\in\spa(S)$;
                \item 关于数乘封闭:$\lambda\vec{\alpha}=\lambda\lambda_1\alpha_1+\cdots+\lambda\lambda_k\alpha_k\in\spa(S)$
                (数域关于乘法运算封闭,故$\lambda\lambda_i\in\mathbf{F}\enspace(i=1,\ldots,k)$).
            \end{enumerate}
        \end{enumerate}
        综上,$\spa(S)$是$V$的子空间;
        \item 接下来我们证明$\spa(S)$是包含$S$的最小子空间. 设$W$是$V$的任一子空间,我们只需证明$\spa(S)\subset W$.

        事实上,类似于前面$S\subset\spa(S)$的证明我们有$S\subset W$,故$S$中元素都在$W$中. 且由\autoref{thm:2:子空间判别}
        可知子空间中元素一定关于加法、数乘封闭,因此$\forall {\alpha}=\lambda_1\alpha_1+\cdots+\lambda_k\alpha_k\in\spa(S)$.
        由于$\alpha_1,\ldots,\alpha_k\in S\subset W,\enspace\lambda_1,\ldots,\lambda_k\in\mathbf{F}$,因此$\alpha\in W$,从而
        $\spa(S)\subset W$,由此得证.
    \end{enumerate}
\end{proof}

上述证明的重要性在于,我们在这一个证明中练习了子集的证明方法、子空间的充要条件以及对于``最小''问题证明的一般方法.
希望读者能掌握其中的每一个思想与技巧.

最后我们再说明有限维线性空间和无限维线性空间的定义,本课程研究的内容都在有限维线性空间,如果少数时间拓展至无限维空间
我们会给出说明:
\begin{definition}
    $V(\mathbf{F})$称为有限维线性空间,如果$V$中存在一个有限子集$S$使得$\spa(S)=V$,反之称为无限维线性空间.
\end{definition}
\begin{example}
    证明:$\mathbf{R}[x]_3$是有限维线性空间,$\mathbf{R}[x]$是无限维线性空间.
\end{example}
\begin{proof}
    \begin{enumerate}
        \item 显然$\mathbf{R}[x]_3$的有限子集$S=\{1,x,x^2\}$可以张成$\mathbf{R}[x]_3$,因此$\mathbf{R}[x]_3$
        是有限维线性空间;
        \item 对于$\mathbf{R}[x]$,我们只需证明其任意有限子集都无法张成其本身. 我们取其任意有限子集,则其中多项式元素的
        次数一定有最大值,我们记为$m$,那么$z^{m+1}$以及更高次数的无法被表示,因此$\mathbf{R}[x]$是无限维线性空间.
    \end{enumerate}
\end{proof}

\vspace{2ex}
\centerline{\heiti \Large 内容总结}

\vspace{2ex}

\centerline{\heiti \Large 习题}
\vspace{2ex}
{\kaishu 1520年以来,全世界只有85个机构存活至今,其中50家是大学. 大学依靠梦想、希望生存下去——这就是大学的历史.}
\begin{flushright}
    \kaishu
    ——美国哥伦比亚大学校长L·C·柏林格
\end{flushright}
\centerline{\heiti A组}
\begin{enumerate}
    \item 检验下列集合对指定的加法和数乘运算是否构成实数域上的线性空间.
    \begin{enumerate}[label=(\arabic*)]
        \item 有理数集$\mathbf{Q}$对普通的数的加法和乘法;

        \item 集合$\mathbf{R}^2$对通常的向量加法和如下定义的数量乘法:$\lambda\cdot(x,y)=(\lambda x,y)$;

        \item $\mathbf{R}_+^n$(即$n$元正实数向量)对如下定义的加法和数乘运算:
        \begin{gather*}
            (a_1,\ldots,a_n)+(b_1,\ldots,b_n)=(a_1b_1,\ldots,a_nb_n) \\
            \lambda\cdot(a_1,\ldots,a_n)=(a_1^\lambda,\ldots,a_n^\lambda)
        \end{gather*}

        \item 请继续完成教材P86第二章习题第1题第(9)-(11)问关于函数的加法数乘定义线性空间的问题.
    \end{enumerate}
    \item 请完成教材P86-87第二章习题第3题. 第(5)问平常问题较多,实际上就是要判断满足一定条件的
          多项式是否构成子空间.
\end{enumerate}
\centerline{\heiti B组}
\begin{enumerate}
    \item 设$V$是一个线性空间,$W$是$V$的子集,证明:$W$是$V$的子空间$\iff \spa(W)=W$.
    \item 设$\mathbf{R}_+$是所有正实数组成的集合,加法和数乘定义如下:
    \[ \forall a,b \in \mathbf{R}_+,\enspace k\in \mathbf{R}\colon\enspace a\oplus b = ab,\enspace k\odot a = a^k \]
    则 $\mathbf{R}_+$关于这一加法和数乘构成一个实线性空间. 求$\mathbf{R}_+$的一组基;
\end{enumerate}
\centerline{\heiti C组}
\begin{enumerate}
    \item 设$\mathbf{E}$是域$\mathbf{F}$的一个子域.
    \begin{enumerate}[label=(\arabic*)]
        \item 证明:$\mathbf{F}$关于自身的加法和乘法构成一个$\mathbf{E}$上的向量空间,并举一例;

        \item 举例说明:$\mathbf{E}\enspace(\mathbf{E}\neq \mathbf{F})$不是$\mathbf{F}$上的线性空间;

        \item 证明:若$V$是$\mathbf{F}$上的一个线性空间,则$V$也是$\mathbf{E}$上的一个线性空间.
    \end{enumerate}
\end{enumerate}
