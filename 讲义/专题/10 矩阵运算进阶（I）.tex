\chapter{矩阵运算进阶(I)}

关于矩阵运算进阶的两讲我们将讲述线性代数这门课中关于矩阵计算问题的一些进阶内容. 需要注意的是,本节除了一些基本概念在之后讨论矩阵的秩时还有关键作用之外,大都是技巧性的内容,基本都已经脱离了之前描述的抽象空间和映射. 但我们仍需承认这些内容的重要性,因为运算技巧将会使得未来的很多性质的研究更为便捷. 同时我们也应当感到欣慰,我们终于从抽象空间一步一步如同搭积木一般走到了具象的运算,并且这是建立在了充分理解了矩阵背后的内在含义的基础上的.

本节我们首先介绍有关初等矩阵、分块矩阵和矩阵方程的相关内容. 本节是之后讨论矩阵的秩的重要基础,并且涉及一些很重要的算法和技巧,需要引起重视.

\section{初等矩阵}

接下来一节我们将介绍一种特别的矩阵——初等矩阵. 其中要介绍的初等矩阵和可逆矩阵的关联将成为我们求解矩阵的逆的一个重要手段,也是后面大量内容的讨论基础,因此我们在此展开叙述.

\subsection{基本概念与性质}

\begin{definition}
    将单位矩阵$E$做一次初等变换得到的矩阵称为初等矩阵,与三种初等行、列变换对应的三类初等矩阵为:
    \begin{enumerate}
        \item 将单位矩阵第$i$行(或列)乘$c$,得到初等倍乘矩阵$E_i(c)$;

        \item 将单位矩阵第$i$行乘$c$加到第$j$行,或将第$j$列乘$c$加到第$i$列,得到初等倍加矩阵$E_{ij}(c)$;

        \item 将单位矩阵第$i,j$行(或列)对换,得到初等对换矩阵$E_{ij}$.
    \end{enumerate}
\end{definition}
教材136页上方给出了这三类矩阵具体的形状. 事实上,初等矩阵的定义就将我们在高斯消元法中使用的初等变换的三种形式对应到了矩阵的形式上. 我们很容易通过计算验证,简而言之,对单位矩阵$E$做了一次初等变换后得到的矩阵$P$,乘以其他任何矩阵$A$的效果就是对$A$做了和对$E$做的同样的初等变换.

当我们对矩阵左乘一个初等矩阵时,相当于对矩阵做了对应的初等行变换;右乘一个初等矩阵时,相当于对矩阵做了对应的初等列变换. 所以在高斯消元法中,假设系数矩阵为$A$,简化阶梯矩阵为$U$,我们做的初等行变换分别为$P_1,P_2,\ldots,P_k$,则有
\[P_kP_{k-1}\cdots P_2P_1A=U.\]

大家非常关心为什么初等矩阵左乘代表行变换,右乘代表列变换. 事实上读者只需要回顾矩阵乘法一小节中说明的如下性质:``矩阵$A$与$B$相乘,乘积的每一列都是矩阵$A$各列的线性组合,每一行都是矩阵$B$各行的线性组合''即可. 左乘的时候情况对应于$A$是初等矩阵,它乘在$B$的左边,那么乘积$AB$的每一行都是$B$的各行的线性组合,即$A$左乘$B$后相对于对$B$进行了行变换. 右乘的情况对应于$B$为初等矩阵,结果$AB$的每一列都是$A$的各列的线性组合,即$B$右乘$A$后相当于对$A$进行了列变换.

接下来我们还有几个细节需要讨论:
\begin{enumerate}
    \item 倍加变化请注意$i$和$j$在行列变换的情况下的不同,行变换是第$i$行乘$c$加到第$j$行,列变换是第$j$列乘$c$加到第$i$列;

    \item 注意三类矩阵不是三个矩阵,例如倍乘矩阵乘以哪一行/哪一列,以及乘以多少都是不唯一的;

    \item 三种初等矩阵都是可逆的,且$E_i^{-1}(c)=E_i\left(\dfrac{1}{c}\right)$,$E_{ij}^{-1}(c)=E_{ij}(-c)$,$E_{ij}^{-1}=E_{ij}$. 原因非常简单,只需要记住这三类矩阵在单位矩阵基础上做了什么,需要反过来作用什么来抵消就可以理解;

    \item 三种初等矩阵的转置:$E_i^\mathrm{T}(c)=E_i(c)$,$E_{ij}^\mathrm{T}(c)=E_{ji}(c)$,$E_{ij}^\mathrm{T}=E_{ij}$,因此初等矩阵转置前后分别对应于同样的行列变换操作. 例如倍乘行变换表示对第$i$行乘以$c$,转置后如果视为列变换则表示对第$i$列乘以$c$,行列操作一致,对换也是如此. 而倍加$E_{ij}(c)$在行变换表示将第$i$行乘$c$加到第$j$行,转置后如果视为列变换,$E_{ji}(c)$表示将第$i$列乘$c$加到第$j$列,这两者在行列操作上保持了一致性.

          总结而言就是$P$为初等矩阵,则$PA$和$AP^\mathrm{T}$表示同一个变换,只是把行/列两个字变了一下.
\end{enumerate}

关于初等矩阵我们有一个相当重要的定理,这一定理在之后很多讨论和解题中都扮演关键角色:
\begin{theorem}\label{thm:10:可逆与初等变换}
    任意可逆矩阵都可以被表示为若干个初等矩阵的乘积.
\end{theorem}

\begin{proof}
    回顾可逆矩阵一节中逆矩阵的求解方法的讨论,当$A$可逆时,线性方程组$AX=b$仅有唯一解,因此其简化阶梯矩阵必然满足行列数相同且对角线上全为1,事实上这就是单位矩阵$E$.

    假设从$A$到$E$所做的初等行变换为$P_1,P_2,\ldots,P_k$,则有
    \[P_kP_{k-1}\cdots P_2P_1A=E.\]
    又初等矩阵可逆且逆矩阵也为初等矩阵,由矩阵乘积的逆的公式有$(P_kP_{k-1}\cdots P_2P_1)^{-1}=P_1^{-1}P_2^{-1}\cdots P_{k-1}^{-1}P_k^{-1}$,因此$A=P_1^{-1}P_2^{-1}\cdots P_{k-1}^{-1}P_k^{-1}$,即$A$可以表示为若干个初等矩阵的乘积.
\end{proof}

\subsection{逆矩阵的求解(基本方法II)}

本节我们将基于上述对初等矩阵的讨论给出逆矩阵求解中另一种基本且更常用的方法.

我们首先给出一个引理:
\begin{lemma}
    设$A$为$n$阶可逆矩阵,如果对$A$和$n$阶单位矩阵$E$做相同的初等行变换,即$P_1,P_2,\ldots,P_k$后$A$变为$E$时,$E$变为$A^{-1}$.
\end{lemma}

\begin{proof}
    由题意有$P_kP_{k-1}\cdots P_2P_1A=E$,即$P_kP_{k-1}\cdots P_2P_1=A^{-1}$,因此对$E$做相同的初等行变换有$P_kP_{k-1}\cdots P_2P_1E=A^{-1}E=A^{-1}$.
\end{proof}

我们可以将上述过程写成$(A,E)\xrightarrow{\text{初等行变换}}(E,A^{-1})$. 事实上初等列变换也有类似过程:
\[\begin{pmatrix}
        A \\ E
    \end{pmatrix}\xrightarrow{\text{初等列变换}}\begin{pmatrix}
        E \\ A^{-1}
    \end{pmatrix}\]
原因是对$A$做列变换$P_1,P_2,\ldots,P_k$后,$A$变为$E$,这一过程可以写为$AP_1P_2\cdots P_k=E$,因此$P_1P_2\cdots P_k=A^{-1}$,因此对$E$做相同的列变换有$EP_1P_2\cdots P_k=EA^{-1}=A^{-1}$.

注意,上面我们在行变换时将$A$和$E$放在一行是为了方便我们实际操作的时候,我们可以对$A$和$E$同时做行变换,列变换放在一列表示的原因同理,只是为了实际操作的时候更容易看清,在后面的例子中我们可以仔细体会到这一点.

注意,基于初等变换的方法是非常重要的,我们很多时候使用的方法就是初等行变换(列变换也可以用但一般更习惯行变换). 我们将通过下面这个例子详细介绍这种方法的计算过程:
\begin{example}
    用上述方法求矩阵$A=\begin{pmatrix}0 & 2 & -1 \\ 1 & 1 & 2 \\ -1 & -1 & -1\end{pmatrix}$的逆矩阵.
\end{example}

\begin{solution}
    见教材138页例.
\end{solution}

\subsection{线性映射与初等变换}

我们来看下面这个例子:
\begin{example}
    设$\sigma:V\to W$为线性映射,取$V$和$W$的基$\alpha_1,\alpha_2,\ldots,\alpha_n$和$\beta_1,\beta_2,\ldots,\beta_m$,且$\sigma$在这两组基下的矩阵表示为$A=(a_{ij})_{m\times n}$.
    \begin{enumerate}
        \item 若将$\alpha_i$换为$c\alpha_i$($c$为常数),求$\sigma$在新基下的矩阵表示;

        \item 若将$\beta_i$换为$c\beta_i$($c$为常数),求$\sigma$在新基下的矩阵表示;

        \item 若将$\alpha_i$换为$\alpha_i+k\alpha_j$($k$为常数),求$\sigma$在新基下的矩阵表示;

        \item 若将$\beta_i$换为$\beta_i+k\beta_j$($k$为常数),求$\sigma$在新基下的矩阵表示;

        \item 若将$\alpha_i$和$\alpha_j$对换,求$\sigma$在新基下的矩阵表示;

        \item 若将$\beta_i$和$\beta_j$对换,求$\sigma$在新基下的矩阵表示.
    \end{enumerate}
\end{example}

\begin{solution}
    根据线性映射矩阵表示的定义,即写出
    \[\sigma(\alpha_i)=a_{1i}\beta_1+a_{2i}\beta_2+\cdots+a_{mi}\beta_m,\enspace i=1,2,\ldots,n,\]
    我们很容易得出下面的结果:
    \begin{enumerate}
        \item 矩阵表示$A_1$就是$A$的第$i$列数乘$c$;

        \item 矩阵表示$A_2$就是$A$的第$i$行数乘$1/c$;

        \item 矩阵表示$A_3$就是$A$的第$i$列加上第$j$列数乘$k$;

        \item 矩阵表示$A_4$就是$A$的第$j$行减去第$i$行数乘$k$;

        \item 矩阵表示$A_5$就是$A$的第$i$列和第$j$列对换;

        \item 矩阵表示$A_6$就是$A$的第$i$行和第$j$行对换.
    \end{enumerate}
\end{solution}

这一例子给了我们一个很大的启示:我们可以将对于基向量组的上述操作也视为一种初等变换,那么对基的初等变换的作用效果就是表示矩阵也做了初等变换. 其中对出发空间的基做变换相当于对矩阵的列做变换;对目标空间的基做变换相当于对矩阵的行做变换.

我们回顾$\sigma(a)=b$和$AX=b$之间的关联,其中$A$是$\sigma$在出发空间和到达空间基$B_1$和$B_2$下的表示矩阵. 我们说解$X$是$b$在$\sigma$下的原像在基$B_1$下的坐标. 我们回顾高斯消元法,高斯消元法都在对矩阵进行初等行变换,事实上根据上面例题这对应于对$B_2$中的向量进行初等变换得到$B_2'$,但没有影响$B_1$中的向量.

因此,对$A$做初等变换得到简化阶梯矩阵$U$后,$U$实际上是$\sigma$在出发空间基$B_1$和到达空间基$B_2'$下的表示矩阵. 因此$UX=b$仍对应于$\sigma(a)=b$,解$X$仍然是$b$在$\sigma$下的原像在基$B_1$下的坐标,因为$B_1$在行变换过程中完全没有变化!这就给高斯消元法使用初等行变换一个合理的解释——从线性映射的角度来看,虽然矩阵在行变换后发生了变化,方程从$AX=b$变为$UX=b$,但解不会改变. 但如果做列变换则改变了$B_1$得到$B_1'$,这时解$X$将是$b$在$\sigma$下的原像在基$B_1'$下的坐标,因此会发生改变,所以高斯消元法没有采用列变换.

除此之外,在后续学习中我们会介绍矩阵的三种标准形,每种标准形都是矩阵在某一特定的基下的矩阵表示. 事实上我们可以首先写出矩阵在任意一组基下的矩阵表示,然后通过对矩阵做初等变换得到标准形,同时对基做``初等变换''即可得到标准形对应的基. 日后我们提到矩阵标准形时再进一步讨论这一点,现在可以只留一个印象.

\section{分块矩阵}

矩阵分块在矩阵计算中是非常核心的一种手段,这可以使得我们将大矩阵分为更容易处理的小矩阵,结合并行计算等工具能大大加速矩阵计算. 除此之外,基于分块矩阵的初等变换也是研究矩阵求逆、矩阵的秩以及矩阵分解等多个问题的重要工具.

\subsection{运算性质}

\begin{definition}
    一般地,对于$m \times n$矩阵$A$,如果在行的方向分成$s$块,在列的方向分成$t$块,就得到$A$的一个$s \times t$\term{分块矩阵}\index{fenkuaijuzhen@分块矩阵 (block matrix)},记作$A=(A_{kl})_{s \times t}$,其中$A_{kl}\enspace(k=1,\ldots,s,\enspace l=1,\ldots,t)$称为$A$的子块.
\end{definition}
实际上上述表示方法就是将一般矩阵表示$A=(a_{ij})_{m \times n}$中的$a_{ij}$替换为了小块矩阵,字母含义并无变化,内层代表索引,外层代表总行列数(只是分块矩阵是块索引和块数). 我们接下来考察分块矩阵的运算性质.
\begin{enumerate}
    \item 分块矩阵的加法:设分块矩阵$A=(A_{kl})_{s \times t},\enspace B=(B_{kl})_{s \times t}$. 如果$A$与$B$对应的子块$A_{kl}$和$B_{kl}$都是同型矩阵,则
          \[A+B=(A_{kl}+B_{kl})_{s \times t}\]
          由此我们看到分块矩阵加法要求小块形状和行列分块数都一致,实际上回顾一般矩阵加法要求矩阵完全同型即可理解这一要求.

    \item 分块矩阵的数乘:设分块矩阵$A=(A_{kl})_{s \times t}$,$\lambda$是一个数,则
          \[\lambda A=(\lambda A_{kl})_{s \times t}\]
          实际上数乘最好理解,因为如此计算的效果相当于一般矩阵数乘的效果,即给每个元素都乘以一个常数$\lambda$.

    \item 分块矩阵的乘法:设$A=(a_{ij})_{m \times n},\enspace B=(b_{ij})_{n \times p}$,如果把$A,B$分别分块为$r \times s$和$s \times t$分块矩阵,且$A$的列分块法与$B$的行分块法相同(注意这些条件始终保证可乘性成立),则
          \[AB=\begin{pmatrix}
                  A_{11} & A_{12} & \cdots & A_{1s} \\
                  A_{21} & A_{22} & \cdots & A_{2s} \\
                  \vdots & \vdots & \ddots & \vdots \\
                  A_{r1} & A_{r2} & \cdots & A_{rs}
              \end{pmatrix}\begin{pmatrix}
                  B_{11} & B_{12} & \cdots & B_{1t} \\
                  B_{21} & B_{22} & \cdots & B_{2t} \\
                  \vdots & \vdots & \ddots & \vdots \\
                  B_{s1} & B_{s2} & \cdots & B_{st}
              \end{pmatrix}=C=(C_{kl})_{r \times t}\]
          其中$C$是$r \times t$分块矩阵,且$C_{kl}$与一般矩阵计算类似,即为$A$第$k$行块$B$的$l$列块对应元素相乘后相加,即
          \[C_{kl}=A_{k1}B_{1l}+A_{k2}B_{2l}+\cdots+A_{ks}B_{sl},\enspace k=1,\ldots,r,\enspace l=1,\ldots,t\]

    \item 分块矩阵的转置:大、小矩阵都要转置,这是分块矩阵与普通矩阵的一大性质差异;即$s \times t$分块矩阵$A=(A_{kl})_{s \times t}$转置后$A^\mathrm{T}=(B_{lk})_{t \times s}$为$t \times s$分块矩阵,且$B_{lk}=A_{kl}^\mathrm{T}$. 例如$\begin{pmatrix}
                  A_{11} & A_{12} \\ A_{21} & A_{22}
              \end{pmatrix}^\mathrm{T}=\begin{pmatrix}
                  A_{11}^\mathrm{T} & A_{21}^\mathrm{T} \\ A_{12}^\mathrm{T} & A_{22}^\mathrm{T}
              \end{pmatrix}$.
\end{enumerate}

补充以下注意事项:
\begin{enumerate}
    \item 常见的行列分块方法:将矩阵按行/列分块,注意$A(\beta_1,\ldots,\beta_n)=(A\beta_1,\ldots,A\beta_n)$成立,但当$A$在右侧时并不可乘,因为$\beta$是列向量,只有当$A$为行向量时才能使$\beta A$乘法是有意义的. 事实上按行分块也有对称的结论,即写成
          \[A=\begin{pmatrix}
                  A_1 \\ \vdots \\ A_s
              \end{pmatrix}\]
          时,我们有
          \[AB=\begin{pmatrix}
                  A_1B \\ \vdots \\ A_sB
              \end{pmatrix}.\]

    \item 分块矩阵求逆通常有两种方法,其一直接使用设未知数的方式完成,我们下面将给出例子,当然也可以利用后续介绍的分块矩阵初等变换进行解决:
          \begin{example}
              设$n$阶矩阵$A$分块为$A=\begin{pmatrix}
                      B & O \\ C & D
                  \end{pmatrix}$,其中$B,D$分别为$k$阶、$m$阶矩阵,证明:$A$可逆的充分必要条件为$B,D$可逆,并求$A^{-1}$.
          \end{example}

          \begin{solution}
              本题即教材147页例5,关于可逆的充要条件可以参考教材,我们这里只强调一下如何求解$A^{-1}$. 这里我们使用的方法就是直接设出$A^{-1}$的形式,然后验证即可. 之后我们还会学习一种基于分块矩阵初等变换的进阶方法(事实上考试如果考察的话基本是本题的解法,分块矩阵初等变换是在教材中是小字部分). 设$A^{-1}=\begin{pmatrix}
                      X & Y \\ Z & T
                  \end{pmatrix}$,其中$X,T$分别为$k,m$阶矩阵,那么我们有
              \[\begin{pmatrix}
                      B & O \\ C & D
                  \end{pmatrix}\begin{pmatrix}
                      X & Y \\ Z & T
                  \end{pmatrix}=\begin{pmatrix}
                      BX & BY \\ CX+DZ & CY+DT
                  \end{pmatrix}=\begin{pmatrix}
                      E_k & O \\ O & E_m
                  \end{pmatrix},\]
              又由题意$A$可逆有$B,D$可逆,因此$BX=E_k$可得$X=B^{-1}$,$BY=O$可得$Y=O$,$CY+DT=DT=E_m$可得$T=D^{-1}$,$CX+DZ=CB^{-1}+DZ=O$可得$Z=-D^{-1}CB^{-1}$,因此
              \[A^{-1}=\begin{pmatrix}
                      B^{-1} & O \\ -D^{-1}CB^{-1} & D^{-1}
                  \end{pmatrix}.\]
          \end{solution}

    \item 分析分块矩阵与普通矩阵的运算性质的异同:
          \begin{enumerate}
              \item 分块矩阵转置需要注意大矩阵小分块都要转置;

              \item 分块矩阵每一块不一定是数,而是矩阵,因此小分块中出现$^{-1}$表示小分块求逆,但如果是一般矩阵就是矩阵元素直接求倒数即可;

              \item 分块矩阵加法乘法一定要保证块大小对应,否则不可加、不可乘;

              \item 其他很多性质都是将单个元素推广为一块,例如满足可加、可乘后的加法、乘法计算.
          \end{enumerate}
\end{enumerate}

\begin{example}
    设\[A=\begin{pmatrix}
            1 & 2 & 0  & 0  & 0  \\
            2 & 5 & 0  & 0  & 0  \\
            0 & 0 & -2 & 1  & 0  \\
            0 & 0 & 0  & -2 & 1  \\
            0 & 0 & 0  & 0  & -2
        \end{pmatrix},\enspace B=\begin{pmatrix}
            1  & 0 & 1 & 0 \\
            -1 & 2 & 3 & 0 \\
            1  & 2 & 0 & 4 \\
            0  & 1 & 2 & 4 \\
            0  & 0 & 1 & 4
        \end{pmatrix}\]
    利用分块矩阵的方法,求$A^2,\enspace AB,\enspace A^\mathrm{T},\enspace A^{-1}$.
\end{example}

\begin{solution}
    将$A$和$B$分别分块为
    \[A=\begin{pmatrix}
            A_1 & O \\ O & A_2
        \end{pmatrix},\enspace B=\begin{pmatrix}
            B_1 & B_2 \\ B_3 & B_4
        \end{pmatrix},\]
    其中$A_1=\begin{pmatrix}
            1 & 2 \\ 2 & 5
        \end{pmatrix},\enspace A_2=\begin{pmatrix}
            -2 & 1 & 0 \\ 0 & -2 & 1 \\ 0 & 0 & -2
        \end{pmatrix},\enspace B_1=\begin{pmatrix}
            1 & 0 \\ -1 & 2
        \end{pmatrix},\enspace B_2=\begin{pmatrix}
            1 & 0 \\ 3 & 0
        \end{pmatrix},\enspace B_3=\begin{pmatrix}
            1 & 2 \\ 0 & 1 \\ 0 & 0
        \end{pmatrix},\enspace B_4=\begin{pmatrix}
            0 & 4 \\ 2 & 4 \\ 1 & 4
        \end{pmatrix}$. 因此$A^2=\begin{pmatrix}
            A_1^2 & O \\ O & A_2^2
        \end{pmatrix},\enspace AB=\begin{pmatrix}
            A_1B_1 & A_1B_2 \\ A_2B_3 & A_2B_4
        \end{pmatrix},\enspace A^\mathrm{T}=\begin{pmatrix}
            A_1^\mathrm{T} & O \\ O & A_2^\mathrm{T}
        \end{pmatrix},\enspace A^{-1}=\begin{pmatrix}
            A_1^{-1} & O \\ O & A_2^{-1}
        \end{pmatrix}$. 上面的具体展开计算略过,我们这里只需要体会分块矩阵的运算性质即可.
\end{solution}

在这个例子中我们可以得到一个很关键的经验:分块对角矩阵求逆实际上就是对每一个分块求逆.

\subsection{分块矩阵初等变换(打洞法)}

分块矩阵的初等变换实际上可以视为一般矩阵初等变换的推广,实际上也有三种相应的推广形式:
\begin{enumerate}
    \item 交换分块矩阵两行(列)(实际上对应于交换原矩阵若干行/列);

    \item 对分块的某一行(列),左(右)乘一个可逆矩阵(对应于普通矩阵初等变换就是对普通矩阵的一行乘以非零数);

    \item 将分块矩阵中的某一行(列),左(右)乘矩阵后加到另一行(对应于普通矩阵初等变换就是将一行乘以非零数加到另一行).
\end{enumerate}
特别注意上面对行变换是要左乘矩阵,对列变换要右乘矩阵.

回顾一般矩阵的初等矩阵,就是对单位矩阵做了一次初等变换得到的. 在分块初等矩阵中,实际应用时我们一般只会出现$2\times 2$的情况,因此对应的三种矩阵为对单位矩阵
\[\begin{pmatrix}
        E & O \\ O & E
    \end{pmatrix}\]做了三种初等变换得到的矩阵,即:
\begin{enumerate}
    \item 交换分块矩阵的两行(列),对应的矩阵均为:
          \[\begin{pmatrix}
                  O & E \\ E & O
              \end{pmatrix}\]
          该矩阵左(右)乘以分块矩阵相当于对分块矩阵交换两行(列);

    \item 倍乘矩阵:
          \[\begin{pmatrix}
                  P & O \\ O & E
              \end{pmatrix},\enspace\begin{pmatrix}
                  E & O \\ O & P
              \end{pmatrix}\]
          其中$P$为可逆矩阵,左(右)乘以上述第一个分块矩阵相当于对分块矩阵的第一行(列)乘以$P$,第二个矩阵则对应第二行(列)的倍乘;

    \item 倍加矩阵:
          \[\begin{pmatrix}
                  E & O \\ P & E
              \end{pmatrix},\enspace\begin{pmatrix}
                  E & P \\ O & E
              \end{pmatrix}\]
          \begin{enumerate}
              \item 左乘第一个矩阵相当于对分块矩阵的第一行乘以$P$后加到第二行,右乘第一个矩阵相当于对分块矩阵的第二列乘以$P$后加到第一列;

              \item 右乘第一个矩阵相当于对分块矩阵的第二列乘以$P$后加到第一列,左乘第一个矩阵相当于对分块矩阵的第一列乘以$P$后加到第二列.
          \end{enumerate}
\end{enumerate}
上述分块矩阵的确可以做到相应的初等变换,我们是很容易验证的,只需要将它们乘以一个任意的$2\times 2$分块矩阵$\begin{pmatrix}
        A & B \\ C & D
    \end{pmatrix}$即可. 事实上我们并不需要特别记忆,因为这和之前普通的初等变换并无本质区别,只需要注意左乘右乘即可. 除此之外,上述矩阵的逆矩阵也是同理可得的,只需要思考什么样的逆变换能变回单位矩阵即可,此处不再赘述,后面会有例题进行练习.

分块矩阵初等行变换的一个重要的应用就是``打洞法'',常用于分块矩阵求逆的运算,在之后行列式的一些技巧性处理中也很常见. 例如:
\begin{enumerate}
    \item 当$A$可逆时,我们可以通过初等行变换消去$C$:
          \[ \begin{pmatrix}
                  E & O \\ -CA^{-1} & E
              \end{pmatrix}\begin{pmatrix}
                  A & B \\ C & D
              \end{pmatrix}=\begin{pmatrix}
                  A & B \\ O & D-CA^{-1}B
              \end{pmatrix} \]
          可以继续做列变换消去$B$:
          \[ \begin{pmatrix}
                  A & B \\ O & D-CA^{-1}B
              \end{pmatrix}\begin{pmatrix}
                  E & -A^{-1}B \\ O & E
              \end{pmatrix}=\begin{pmatrix}
                  A & O \\ O & D-CA^{-1}B
              \end{pmatrix} \]
          这里读者可能第一次接触这样的写法,笔者还是在此进行以下耐心的解释. 第一步我们希望消去$C$,对于分块矩阵而言,由于已知$A$可逆,如果我们采用行变换,我们就给第一行左乘$-CA^{-1}$使$A$变为$-C$然后加到第二行,这样第二行的$C$就会变为$O$. 这里的思考是很直接的,然后我们就可以根据我们之前想要做的行变换写出初等矩阵左乘在原矩阵上即可. 特别注意这里有两层左乘:第一层是小块内要左乘$-CA^{-1}$,如果这里思考成了右乘就会错写为乘以$-A^{-1}C$才能使$A$变为$-C$,第二层是初等矩阵$\begin{pmatrix}
                  E & O \\ -CA^{-1} & E
              \end{pmatrix}$要左乘原分块矩阵. 第二步的思考是同理的,只是我们使用了列变换,需要注意右乘. 我们每一次的变换都希望将整个分块变为零矩阵,事实上这就像是在矩阵上挖了个洞填0,因此这一方法又被称为打洞法.

    \item 特别地,对于对称矩阵$\begin{pmatrix}A & B \\ B^\mathrm{T} & D\end{pmatrix}$,其中$A$和$D$也是对称方阵,则$A$可逆时,可以通过下述变换(称为合同变换)消除$B$和$B^\mathrm{T}$,即
          \[ \begin{pmatrix}
                  E & -A^{-1}B \\ O & E
              \end{pmatrix}^\mathrm{T}\begin{pmatrix}
                  A & B \\ B^\mathrm{T} & D
              \end{pmatrix}\begin{pmatrix}
                  E & -A^{-1}B \\ O & E
              \end{pmatrix}=\begin{pmatrix}
                  A & O \\ O & D-B^\mathrm{T}A^{-1}B
              \end{pmatrix} \]
\end{enumerate}
利用初等变换得到分块对角矩阵后,求取逆矩阵就很容易了,因为分块对角矩阵求逆矩阵就是对每个小对角块求逆. 所以利用打洞法解决分块矩阵求逆首先要利用分块矩阵初等变换进行对角化,然后如果得到了$PAQ=\varLambda$,其中$P$和$Q$为分块初等矩阵,$\varLambda$为分块对角矩阵,利用分块对角矩阵的逆和分块初等矩阵的逆都是容易计算的特点计算$Q^{-1}A^{-1}P^{-1}=\varLambda^{-1}$,即可得到$A^{-1}=Q\varLambda^{-1}P$.
\begin{example}
    当$D$可逆时,仿照上面的步骤对角化分块矩阵$P=\begin{pmatrix}A & B \\ C & D\end{pmatrix}$并求逆矩阵.
\end{example}

\begin{solution}
    $D$可逆时,我们可以通过分块矩阵初等变换用$D$将$C$消去:
    \[ PQ=\begin{pmatrix}
            A & B \\ C & D
        \end{pmatrix}\begin{pmatrix}
            E & O \\ -D^{-1}C & E
        \end{pmatrix}=\begin{pmatrix}
            A-BD^{-1}C & B \\ O & D
        \end{pmatrix},\]
    进一步地,将$B$消去:
    \[ RPQ=\begin{pmatrix}
            E & -BD^{-1} \\ O & E
        \end{pmatrix}\begin{pmatrix}
            A-BD^{-1}C & B \\ O & D
        \end{pmatrix}=\begin{pmatrix}
            A-BD^{-1}C & O \\ O & D
        \end{pmatrix}=\varLambda.\]
    因此
    \begin{align*}
        P^{-1} & =Q\varLambda^{-1}R                                                                   \\
               & =\begin{pmatrix}
                      E & O \\ -D^{-1}C & E
                  \end{pmatrix}
        \begin{pmatrix}
            (A-BD^{-1}C)^{-1} & O \\ O & D^{-1}
        \end{pmatrix}
        \begin{pmatrix}
            E & -BD^{-1} \\ O & E
        \end{pmatrix}                                                                         \\
               & =\begin{pmatrix}
                      (A-BD^{-1}C)^{-1} & -(A-BD^{-1}C)^{-1}BD^{-1} \\ -D^{-1}C(A-BD^{-1}C)^{-1} & D^{-1}
                  \end{pmatrix}.
    \end{align*}
    当然推导过程中发现这一矩阵有逆矩阵还要一个条件:$A-BD^{-1}C$可逆.
\end{solution}

\subsection{分块矩阵与数学归纳法}

分块矩阵经常运用在数学归纳法中,我们在之后的课程中也会经常用到这样的思想来证明一些结论,这一思想基于以下内容:

对于$\begin{pmatrix}
        A_1 & \alpha \\ \beta & a_{nn}
    \end{pmatrix}$,假设$A_1$可逆,我们有
\[\begin{pmatrix}
        E_{n-1} & O \\ -\beta A_1^{-1} & 1
    \end{pmatrix}\begin{pmatrix}
        A_1 & \alpha \\ \beta & a_{nn}
    \end{pmatrix}=\begin{pmatrix}
        A_1 & \alpha \\ O & a_{nn}-\beta A_1^{-1}\alpha
    \end{pmatrix}\]
我们通过一个例子来体会如何利用上式结合数学归纳法得到一些矩阵分析中的结论:
\begin{example}
    若$n$阶矩阵$A$的各阶左上角子块矩阵都可逆,则存在主对角元全为1的下三角矩阵$L$和上三角矩阵$U$,使得$A=LU$($L$-$U$分解).
\end{example}

\begin{solution}
    由于主对角元全为1的下三角矩阵可逆,其逆矩阵也是主对角元全为1的下三角矩阵,因此只要证明存在主对角元全为1的下三角矩阵$S$使得$SA=U$,我们可以利用数学归纳法来证明这一结论.

    当$n=1$时,结论显然成立. 假设命题对$n-1$阶矩阵成立. 对$n$阶矩阵$A$,将$A$分块为
    \[A=\begin{pmatrix}
            A_1 & \alpha_1 \\ \alpha_2 & a_{nn}
        \end{pmatrix},\]
    其中$A_1$为满足命题条件的$n-1$阶矩阵,根据归纳假设,存在$n-1$阶主对角元全为1的下三角矩阵$S_1$和上三角矩阵$U_1$使得$S_1A_1=U_1$. 根据我们前面的讨论,我们可以对$A$做分块矩阵初等变换将其化为上三角块矩阵,即
    \[PA=\begin{pmatrix}
            E_{n-1} & O \\ -\alpha_2A_1^{-1} & 1
        \end{pmatrix}A=\begin{pmatrix}
            A_1 & \alpha \\ O & a_{nn}-\alpha_2A_1^{-1}\alpha_1
        \end{pmatrix},\]
    再取$Q=\begin{pmatrix}
            S_1 & O \\ O & 1
        \end{pmatrix}$,即得
    \[QPA=\begin{pmatrix}
            U_1 & S_1\alpha_1 \\ O & a_{nn}-\alpha_2A_1^{-1}\alpha_1
        \end{pmatrix}=U.\]
    其中$U$为上三角矩阵,$S=QP$是主对角元全为1的下三角矩阵,故存在$L=S^{-1}$和$U$使得$A=LU$.
\end{solution}

\section{矩阵方程}

本节我们将讨论矩阵方程这一概念,即矩阵作为未知量的方程.
\begin{enumerate}
    \item 设$A,B,C,X$为矩阵,且$A,B$可逆,考虑以下情形:
          \begin{enumerate}
              \item $AX=B \implies X=A^{-1}B, \enspace XA=B \implies X=BA^{-1}$;

              \item $AXB=C \implies X=A^{-1}CB^{-1}$;
          \end{enumerate}

    \item 考虑以下情形:$AX=B$但$A$不可逆($X$不一定是列向量),根据矩阵乘法的性质``$A$和$X$乘积的第$k$列等于$A$乘以$X$的第$k$列''直接对$B$逐列解方程即可;

    \item 考虑以下求解方式的合理性:
          \begin{enumerate}
              \item 若求$A^{-1}$,只需对$(A,E)$只做初等行变换,可以得到$(E,A^{-1})$;

              \item 若求$A^{-1}B$,只需对$(A,B)$只做初等行变换,可以得到$(E,A^{-1}B)$;

              \item 若求$BA^{-1}$,只需对$\begin{pmatrix}
                            A \\ B
                        \end{pmatrix}$只做初等列变换,可以得到$\begin{pmatrix}
                            E \\ BA^{-1}
                        \end{pmatrix}$;

              \item 对$\begin{pmatrix}
                            A & E_n \\ E_n & O
                        \end{pmatrix}$的前$n$行与$n$列做相同的行列变换,可以得到$\begin{pmatrix}
                            P^\mathrm{T}AP & P^\mathrm{T} \\ P & O
                        \end{pmatrix}$.
          \end{enumerate}
          前三点的证明与利用初等变换求解矩阵的逆的方法引理证明一致,此处不再赘述. 第四点需要用到分块矩阵,我们简要说明:

          \begin{proof}
              事实上,对前$n$行做初等行变换需要的初等矩阵就是对单位矩阵前$n$行做了一次初等行变换后得到的矩阵,我们可以记为
              \[Q=\begin{pmatrix}
                      P & O \\ O & E
                  \end{pmatrix}\]
              我们应当将其左乘原矩阵从而相对于对原矩阵做了初等行变换. 回顾前述初等矩阵转置前后分别对应于相同的行列变换,我们对原矩阵右乘$Q^{\mathrm{T}}=\begin{pmatrix}
                      P^\mathrm{T} & O \\ O & E
                  \end{pmatrix}$(回顾分块矩阵转置)就相当于对原矩阵做了和初等行变换相对应的同样的初等列变换,因此我们有
              \[\begin{pmatrix}
                      P^\mathrm{T} & O \\ O & E
                  \end{pmatrix}\begin{pmatrix}
                      A & E_n \\ E_n & O
                  \end{pmatrix}\begin{pmatrix}
                      P & O \\ O & E
                  \end{pmatrix}=\begin{pmatrix}
                      P^\mathrm{T}AP & P^\mathrm{T} \\ P & O
                  \end{pmatrix}\]
              这与上面第四点的叙述是一致的.
          \end{proof}
\end{enumerate}

\begin{example}
    设$A=\begin{pmatrix}1 & 0 & 0 \\ 1 & 1 & 0 \\ 1 & 1 & 1\end{pmatrix},\ B=\begin{pmatrix}0 & 1 & 1 \\ 1 & 0 & 1 \\ 1 & 1 & 0\end{pmatrix}$,求矩阵$X$满足:
    \[AXA+BXB=AXB+BXA+A(A-B)\]
\end{example}

\begin{solution}
    由$AXA+BXB=AXB+BXA+A(A-B)$得$(A-B)X(A-B)=A(A-B)$,又$A-B=\begin{pmatrix}
            1 & -1 & -1 \\ 0 & 1 & -1 \\ 0 & 0 & 1
        \end{pmatrix}$,很容易判断$A-B$可逆(可以参考\autoref{thm:11:可逆等价条件}),因此$(A-B)X=A$,即$X=(A-B)^{-1}A$.

    我们直接对$(A-B,A)$做初等行变换,有
    \[\begin{pmatrix}
            1 & -1 & -1 & 1 & 0 & 0 \\
            0 & 1  & -1 & 1 & 1 & 0 \\
            0 & 0  & 1  & 1 & 1 & 1
        \end{pmatrix}\xrightarrow{\text{初等行变换}}\begin{pmatrix}
            1 & 0 & 0 & 4 & 3 & 2 \\
            0 & 1 & 0 & 2 & 2 & 1 \\
            0 & 0 & 1 & 1 & 1 & 1
        \end{pmatrix},\]
    因此$X=\begin{pmatrix}
            4 & 3 & 2 \\ 2 & 2 & 1 \\ 1 & 1 & 1
        \end{pmatrix}$.
\end{solution}

事实上如果记不住简便方法也不一定要$A-B$和$A$同时做初等变换,我们也可以直接求出$A-B$的逆矩阵.

\vspace{2ex}
\centerline{\heiti \Large 内容总结}

事实上,这一讲是本讲义第一次接触技巧性较强的内容,我们的重点从之前对于定理的证明以及用例子巩固定理的应用转变为了对于一些技巧性的处理. 我们首先介绍了三类初等变换(请务必注意正文中强调的几个定义的细节),证明了任意可逆矩阵都可以被表示成若干初等矩阵的乘积,也基于此引出了第二种求解逆矩阵的方法——初等变换法,这也是我们未来最常用的方法. 除此之外,我们也联系了线性映射矩阵表示和初等变换,研究了高斯消元法背后的合理性.

接下来我们介绍了分块矩阵的基本运算性质,比较了分块矩阵和一般矩阵运算的异同(特别是乘法和转置),介绍了两种分块矩阵求逆的方法:其一直接设出逆矩阵,其二利用所谓打洞法(分块矩阵初等变换),其中打洞法是一个很重要的技巧,虽然教材列为小字部分,考试中一般不考察,但拥有这一技巧对我们未来证明很多结论都有很大的帮助. 最后我们介绍了矩阵方程的求解方法,本质而言介绍了几种基于初等变换的求解方法,读者理解其内涵即可.

\vspace{2ex}
\centerline{\heiti \Large 习题}

\vspace{2ex}
{\kaishu 龙生龙,凤生凤,华罗庚的学生会打洞.}
\begin{flushright}
    \kaishu
    ——线性代数教学俗语
\end{flushright}

\centerline{\heiti A组}
\begin{enumerate}
    \item 设$A$为三阶矩阵,将$A$的第二列加到第一列得到矩阵$B$,再对调$B$的2、3行得到单位矩阵. 令$P_1=\begin{pmatrix}1 & 0 & 0 \\ 1 & 1 & 0 \\ 0 & 0 & 1\end{pmatrix}\enspace
              P_2=\begin{pmatrix}1 & 0 & 0 \\ 0 & 0 & 1 \\ 0 & 1 & 0\end{pmatrix}$,试用$P_1$和$P_2$表示$A$.

    \item 设$A$为可逆矩阵,将$A$的第$i$行和第$j$行对调得到矩阵$B$,证明矩阵$B$可逆并求$AB^{-1}$.

    \item 设$A$为三阶可逆矩阵,且$P^{-1}AP=\begin{pmatrix}1 & 0 & 0 \\ 0 & 1 & 0 \\ 0 & 0 & 2\end{pmatrix}$,其中$P=(\alpha_1,\alpha_2,\alpha_3)$,令$Q=(\alpha_1+\alpha_2,\alpha_2,\alpha_3)$,求$Q^{-1}AQ$.

    \item 求下列矩阵的可逆的条件与逆矩阵:$\begin{pmatrix}
                  A & B \\ O & D
              \end{pmatrix},\enspace \begin{pmatrix}
                  O & B \\ C & D
              \end{pmatrix},\enspace \begin{pmatrix}
                  O & B \\ C & O
              \end{pmatrix}$.
\end{enumerate}

\centerline{\heiti B组}
\begin{enumerate}
    \item 已知$\mathbf{R}^3$的基$B_1=\{\alpha_1,\alpha_2,\alpha_3\}$变为基$B_2=\{\xi_1,\xi_2,\xi_3\}$的变换矩阵为$A=(a_{ij})_{3 \times 3}$,求:
          \begin{enumerate}
              \item 基$B_3=\{\alpha_2,\alpha_1,\alpha_3\}$变为基$B_2$的变换矩阵;

              \item 基$B_4=\{-\alpha_1,\alpha_2,\alpha_3\}$变为基$B_2$的变换矩阵;

              \item 基$B_4$变为基$B_5=\{\xi_3,\xi_2,-\xi_1\}$的变换矩阵;

              \item 基$B_4$变为基$B_6=\{\xi_1+\xi_2,\xi_2+\xi_3,\xi_3+\xi_1\}$的变换矩阵.
          \end{enumerate}

    \item 设$P=\begin{pmatrix}
                  1 & 1 & 0 \\ 0 & 1 & 0 \\ 0 & 0 & 0
              \end{pmatrix}$,$Q=\begin{pmatrix}
                  0 & 0 \\ 1 & 0
              \end{pmatrix}$,定义$\mathbf{R}^{3\times 2}$上映射$\sigma(A)=PAQ$.
          \begin{enumerate}
              \item 验证$\sigma$是线性映射;

              \item 求$\ker\sigma$和$\im \sigma$;

              \item 求$\mathbf{R}^{3\times 2}$的两组基,使得$\sigma$关于这两组基的表示矩阵是对角矩阵.
          \end{enumerate}

    \item 设$n$阶矩阵$A$分块为$A=\begin{pmatrix}
                  B & O \\ C & D
              \end{pmatrix}$,其中$B,D$分别为$k$阶、$m$阶矩阵,证明:$A$可逆的充分必要条件为$B,D$可逆,并求$A^{-1}$.
\end{enumerate}

\centerline{\heiti C组}
\begin{enumerate}
    \item 若$n$阶矩阵$A$的各阶左上角子块矩阵都可逆,则存在$n$阶下三角矩阵$B$,使得$BA$为上三角矩阵.

    \item 设$A$是数域$\mathbf{F}$上的$n$阶可逆矩阵,把$A$和$A^{-1}$做如下分块:
          \[A=\begin{pmatrix}
                  A_{11} & A_{12} \\ A_{21} & A_{22}
              \end{pmatrix},\enspace A^{-1}=\begin{pmatrix}
                  B_{11} & B_{12} \\ B_{21} & B_{22}
              \end{pmatrix}\]
          其中$A_{11}$是$l \times k$矩阵,$B_{11}$是$k \times l$矩阵,$l$,$k$是小于$n$的正整数. 用$W$表示$A_{12}X=0$的解空间,$U$表示$B_{12}Y=0$的解空间,其中$X$和$Y$为列向量,证明$W$与$U$同构.
\end{enumerate}
