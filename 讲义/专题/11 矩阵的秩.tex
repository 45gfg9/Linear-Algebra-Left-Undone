\chapter{矩阵的秩}

事实上,在前述介绍中我们已经有了充足的关于线性空间、线性映射以及矩阵的相关背景知识.回顾之前讨论的
\autoref{exp:2:线性空间引入}中的例子以及线性映射核空间与齐次线性方程组的解的关联,我们是时候彻底
揭开矩阵与抽象的线性空间和线性映射之间那层若隐若现的薄膜,而我们的工具正是本节讨论的核心——矩阵的秩.
相信在理解了本节内容后,我们可以说是一只手已经触碰到线性方程组解的一般理论了.

\section{矩阵的秩}
我们首先给出矩阵的三个秩的定义:
\begin{definition}
    设$A$是线性映射$\sigma$对应的矩阵,我们把$\sigma$的秩也称为矩阵$A$的秩,
    记为$r(A)$.我们将矩阵$A$的所有行向量组成的秩称为$A$的\keyterm*{行秩}[row rank],
    所有列向量组成的向量组的秩称为$A$的\keyterm*{列秩}[column rank].
\end{definition}
对于以上定义的三个秩,我们有定理如下,这一定理无论是证明还是结果都非常关键:
\begin{theorem}
    任意矩阵的秩 = 行秩 = 列秩.
\end{theorem}
这一定理的证明,矩阵的秩 = 列秩的部分根据线性映射的相关概念是显然的,行秩的部分
教材中有较为繁琐的证明,在本讲义下面的内容会有更加形象的解释.

实际上,这一定理有两个重要的直接推论,一是将求矩阵的秩的问题转化为求矩阵行/列极大线性无关向量组的问题,
第二是矩阵的秩等于其转置的秩.

除此之外,我们还应当强调以下结论,在我们后续研究线性方程组解的相关性质时是常用的:
\begin{theorem}
    线性映射是单射当且仅当其矩阵表示为列满秩矩阵,线性映射是满射当且仅当其矩阵表示为行满秩矩阵.
\end{theorem}
这一结论的证明比较基本,希望大家能透过这一个结论看到列满秩矩阵与行满秩矩阵更本质的特征.

可能很多同学对于行秩、列秩相等以及转置的几何意义很感兴趣.实际上我们有两种获得转置矩阵的
方式,第一种来源于我们之前讨论的对偶空间上的线性映射对应的矩阵,这种方式可能不够直观.
另一种获得的方法基于内积,感兴趣的同学可以了解矩阵的伴随(不是行列式中的伴随矩阵).

我们可以研究矩阵及其转置的关系,我们可以用一个图形来表示:

\begin{figure}[h]
    \centering
    \small
    \begin{tikzpicture}
        \tikzset{->-/.style={decoration={
            markings,
            mark=at position .6 with {\arrow{stealth}}},postaction={decorate}}}

        \draw[rotate=45] (0,6) rectangle (-3,3) rectangle (-5,0)
            (-3,3) rectangle(-3.35,3.35)
            coordinate (xr) at (-2,4)
            coordinate (xn) at (-4,2)
            coordinate (x) at (-2,2)
            coordinate (0n) at (-3,3)
            node at (-1,5) {行空间}
            node at (-4,1) {$A$的核空间}
            node at (-1.5,6.5) {$\dim r$}
            node at (-4,4) {$\mathbf{R}^n$}
            node at (-6,3) {$\dim n-r$};

        \draw[rotate=30] (6,2) rectangle (3.5,-2) rectangle (0,-4)
            (3.5,-2) rectangle (3.85,-2.35)
            coordinate (b) at (4.5,0.5)
            coordinate (0m) at (3.5,-2)
            node at (5,1.5) {列空间}
            node at (2,-3) {$A^{\mathrm{T}}$的核空间}
            node at (7,0) {$\dim r$}
            node at (5,-3) {$\mathbf{R}^m$}
            node at (4,-4.5) {$\dim m-r$};

        \foreach \point in {xr, x, xn, 0n, b, 0m}
            \fill[black] (\point) circle (1pt);

        \node [left] at (xr) {$x_r$};
        \node [below right] at (x) {$x=x_r+x_n$};
        \node [left] at (xn) {$x_n$};
        \node [right] at (0n) {0};
        \node [right] at (b) {$b$};

        \draw[->-,very thick] (xr) -- node[above,sloped] {$Ax_r = b$} (b);
        \draw[->-,very thick] (x) -- node[below,sloped] {$Ax = b$} (b);
        \draw[->-,very thick] (xn) -- node[below,sloped] {$Ax_n = 0$} (0m);

        \draw[dashed,thick] (xr) -- (x) -- (xn);

    \end{tikzpicture}
\end{figure}

我们观察到以下几点:
\begin{enumerate}
    \item 矩阵的行空间与解空间(零空间)互为正交补(直观理解两个空间就是互相垂直且互为补空间),这一点应当是在正交的内容中有所提及的;
    \item 矩阵的列空间与其转置矩阵的零空间互为正交补,这一点实际与上一条等价.
\end{enumerate}

接下来我们来看行秩(列秩比较显然,此处不再详细展开).我们首先得到解空间($N(A)$)的维数,这可以直接
根据维数公式得到:$\dim N(A) =n-r(A)$,根据正交补的性质,我们的可以得到行秩即为
$n-(n-r(A))=r(A)$.于是我们得到了一个基于正交补的行秩解释.

我们需要熟练掌握可逆矩阵的几个等价条件:
\begin{theorem}
    设$A \in \mathbf{M}_n{\mathbf{F}}$,则下列命题等价:
    \begin{enumerate}
        \item $A$可逆;

        \item $r(A)=n$;

        \item $A$的$n$个行(列)向量线性无关;

        \item 齐次线性方程组$AX=0$只有零解;

        \item $|A|\neq 0$.
    \end{enumerate}
\end{theorem}

\section{三个重要的定理}
这一讲我们将在引入逆矩阵的基础上,讨论三个容易混淆但各有十分重要内涵的定理.我们首先用一个例子来引入第一个定理:
\begin{example}
    
\end{example}

\begin{theorem} \label{thm:6:线性映射对向量坐标的影响}
    \textbf{\heiti 线性映射对向量坐标的影响}

    设$\sigma \in \mathcal{L}(V_1,V_2)$关于$V_1$和$V_2$的基$B_1$和基$B_2$的矩阵为$A=(a_{ij})_{m \times n}$,
    且$\alpha$与$\sigma(\alpha)$在基$B_1$和基$B_2$下的坐标分别为$X$和$Y$,则$Y=AX$.
\end{theorem}
上述即教材定理4.1,这一定理给出一个向量经过线性映射之后,其坐标的变化. 我们可以用下图表示:

\begin{figure}[htbp]
    \centering
    \begin{tikzpicture}[>=Stealth]
        \node (V) at (0,0) {$V$};
        \node (W) at (3,0) {$W$};
        \node (Fn) at (0,-3) {$\mathbf{F}^n$};
        \node (Fm) at (3,-3) {$\mathbf{F}^m$};
        \draw[->,thick] (V) -- node[below]{表示矩阵:$A$} (W);
        \draw[<->,thick] (V) -- node[right]{同构} (Fn);
        \draw[<->,thick] (W) -- node[left]{同构} (Fm);
        \draw[->,thick] (Fn) -- node[above]{$\sigma(\alpha)=A\alpha$} (Fm);
    \end{tikzpicture}
\end{figure}

图中我们可以看出通过坐标映射后得到的新映射即为\autoref{thm:6:线性映射对向量坐标的影响} 描述的映射.

在描述下一定理之前,我们首先介绍过渡矩阵(变换矩阵)的概念.
\begin{definition}
    设$B_1=\{\alpha_1,\alpha_2,\ldots,\alpha_n\}$与$B_2=\{\beta_1,\beta_2,\ldots,\beta_n\}$是线性空间
    $V(\mathbf{F})$的任意两组基,$B_2$中每个基向量被基$B_1$表示为
    \[ \left\{
    \begin{array}{rcl}
        \beta_1&=&a_{11}\alpha_1+a_{21}\alpha_2+\cdots+a_{n1}\alpha_n \\
        \beta_2&=&a_{12}\alpha_1+a_{22}\alpha_2+\cdots+a_{n2}\alpha_n \\
        &\vdots& \\
        \beta_n&=&a_{1n}\alpha_1+a_{2n}\alpha_2+\cdots+a_{nn}\alpha_n
    \end{array}
    \right. \]
    将上式用矩阵表示为
    \[(\beta_1,\beta_2,\cdots,\beta_n)=(\alpha_1,\alpha_2,\cdots,\alpha_n)\begin{pmatrix}
        a_{11} & a_{12} & \cdots & a_{1n} \\
        a_{21} & a_{22} & \cdots & a_{2n} \\
        \vdots & \vdots & \ddots & \vdots \\
        a_{n1} & a_{n2} & \cdots & a_{nn}
    \end{pmatrix}\]
    我们将这一矩阵称为即$B_1$变为基$B_2$的变换矩阵(或过渡矩阵).
\end{definition}
简单而言就是将$B_2$中的向量在$B_1$下的坐标按列排列.需要特别注意定义的是$B_1$变为基$B_2$还是反过来,
这两种情况下得到的矩阵互逆.注意过渡矩阵一定是基与基之间的表示矩阵,并且过渡矩阵一定可逆.
\begin{theorem}
    \textbf{\heiti 基的选择对向量坐标的影响}

    设线性空间$V$的两组基为$B_1$和$B_2$,且基$B_1$到$B_2$的变换矩阵(过渡矩阵)为$A$,如果
    $\xi \in V(\mathbf{F})$,且在$B_1$和$B_2$下的坐标分别为$X$和$Y$,则$Y=A^{-1}X$.
\end{theorem}
上述即教材定理4.10,描述同一个向量在不同基下坐标之间的关系.事实上,这与本节同构关系紧密,因为
同构意味着两个线性空间结构一致,故同构映射可以保持向量组的线性关系不变.在同构关系下,
线性组合对应线性组合,线性无关对应线性无关,线性相关对应线性相关.我们有如下定理:

\begin{theorem}
    设$(\alpha_1,\alpha_2,\ldots,\alpha_n)$是线性无关的向量组,且
    \[(\beta_1,\beta_2,\ldots,\beta_s)=(\alpha_1,\alpha_2,\ldots,\alpha_n)A\]
    则向量组$(\beta_1,\beta_2,\ldots,\beta_s)$的秩等于矩阵$A$的秩.
\end{theorem}
定理的证明需要用到坐标映射是同构映射这一事实,我们不难发现等式左侧向量组与$A$的列向量组是等价的.
事实上我们也可以由此发现,过渡矩阵一定是可逆矩阵.

\begin{theorem}
    已知$\beta_i=a_{1i}\alpha_1+a_{2i}\alpha_2+\cdots+a_{ni}\alpha_n\enspace(i=1,2,\ldots,n)$,
    且$A=(a_{ij})$可逆,则$\alpha_1,\alpha_2,\ldots,\alpha_n$与$\beta_1,\beta_2,\ldots,\beta_n$
    等价.
\end{theorem}
实际上这一定理与上一定理的思想都是类似的,我们可以看一个例题练习一下:
\begin{example}
    已知$\beta_1=\alpha_2+\alpha_3,\enspace\beta_2=\alpha_1+\alpha_3,\enspace\beta_3=\alpha_1+\alpha_2$,
    证明$\alpha_1,\alpha_2,\alpha_3$与$\beta_1,\beta_2,\beta_3$等价.
\end{example}
\begin{theorem}
    \textbf{\heiti 基的选择对映射矩阵的影响}

    设线性变换$\sigma \in \mathcal{L}(V,V)$,$B_1=\{\alpha_1,\ldots,\alpha_n\}$和$B_2=\{\beta_1,\ldots,\beta_n\}$
    是线性空间的$V(\mathbf{F})$的两组基,基$B_1$变为基$B_2$的变换矩阵为$C$,如果$\sigma$在基$B_1$下的矩阵为$A$,
    则$\sigma$关于基$B_2$所对应的矩阵为$C^{-1}AC$.
\end{theorem}
上述即教材定理7.4,研究同一个映射在不同基下表示矩阵之间的关系.实际上我们将在下一专题初等矩阵一节进一步讨论.
这一定理的证明需要用到我们之前描述的两种线性映射矩阵表示的统一性.

\section{相抵标准形}
此处我们需要首先回顾一个基本定理:
\begin{theorem}
    初等变换不改变矩阵的秩(包括行变换和列变换).
\end{theorem}
由这一定理我们可以推导出相抵标准形:
\begin{theorem}
    若$r(A_{m \times n})=r$,则存在可逆矩阵$P$和$Q$,使得
    \[PAQ=\begin{pmatrix}
        E_r & 0 \\ 0 & 0
    \end{pmatrix}=U_r\]
    其中$E_r$表示$r$阶单位矩阵.
\end{theorem}
这一定理证明直接使用定理4以及可逆矩阵可以拆分为初等矩阵的乘积即可.
其中$U_r$称为相抵标准形.我们称两个矩阵相抵即两个矩阵可以通过一系列
初等变换可以互相转化.由此我们得到关于矩阵相抵的两个等价命题:

1. 矩阵$A$与$B$相抵$\iff$存在可逆矩阵$P$和$Q$使得$PAQ=B$;

2. 矩阵$A$与$B$相抵$\iff r(A)=r(B)$.

\begin{example}
    设$A=\begin{pmatrix}
        1 & 0 & 2 & -4 \\ 2 & 1 & 3 & -6 \\ -1 & -1 & -1 & 2
    \end{pmatrix}$. 求
    \begin{enumerate}
        \item $A$的秩$r$和相抵标准形;

        \item 3 阶可逆矩阵$P$和 4 阶可逆矩阵$Q$使得$PAQ=\begin{pmatrix}
            E_r & 0 \\ 0 & 0
        \end{pmatrix}$.
    \end{enumerate}
\end{example}

关于相抵标准形,我们需要在此补充一个常用的技术,即相抵标准形的分解:

我们对$s \times n$矩阵$\begin{pmatrix}
    E_r & O \\ O & O
\end{pmatrix}$有一种很重要的分解:
\[\begin{pmatrix}
    E_r & O \\ O & O
\end{pmatrix}=\begin{pmatrix}
    E_r \\ O
\end{pmatrix}\begin{pmatrix}
    E_r & O
\end{pmatrix}\]
由此我们可以知道任意一个非零矩阵都可以被分解成一个列满秩矩阵和一个
行满秩矩阵的乘积:

\[A=P\begin{pmatrix}
    E_r & O \\ O & O
\end{pmatrix}Q=P\begin{pmatrix}
    E_r \\ O
\end{pmatrix}\begin{pmatrix}
    E_r & O
\end{pmatrix}Q\]
记$P_1=P\begin{pmatrix}
    E_r \\ O
\end{pmatrix}$,$Q_1=\begin{pmatrix}
    E_r & O
\end{pmatrix}Q$,则$A=P_1Q_1$,且$P_1$和$Q_1$分别为列满秩、行满秩矩阵.

我们可以利用相抵标准形解决很多问题,例如下一节中部分秩不等式的证明:
\begin{example}
    \begin{enumerate}
        \item $r\begin{pmatrix}
            A & O \\ O & B
        \end{pmatrix}=r(A)+r(B)$.

        \item $r\begin{pmatrix}
            A & D \\ O & B
        \end{pmatrix}\geqslant r(A)+r(B),\enspace r\begin{pmatrix}
            A & O \\ C & B
        \end{pmatrix}\geqslant r(A)+r(B)$.
    \end{enumerate}
\end{example}

\section{秩不等式}
本节的内容实际上部分内容有一定的技巧性,对于荣誉课程来说还是以理解为主(所以
其实本节中提到的很多内容都只是介绍性的,而非要求大家熟练掌握,但是遇见了要有
一些基本的思路而不能完全不理解),可能下面列出定理的时候显得比较繁冗,但是实
际上我们更重视其中的理解而非硬套结论.

我们首先给出一些常见的秩相关的不等式或等式,这些式子希望各位同学能够理解其含义,
而非机械记忆套用.下面这些等式/不等式的证明方式非常多,实际上可以利用之前所说化为
相抵标准形的方法,也可以利用线性相关性的方法,也可以回到线性映射进行考量.总之
解决的方法非常多,希望各位同学能熟练推导理解.
\begin{enumerate}
    \item $r(A)=r(PA)=r(AQ)=r(PAQ)$,其中$P$、$Q$可逆
    \item $|r(A)-r(B)|\leqslant r(A\pm B) \leqslant r(A)+r(B)$
    \item $r(AB) \leqslant \min\{r(A),\ r(B)\}$
    \item $r(A)=r(A^\mathrm{T})=r(AA^\mathrm{T})=r(A^\mathrm{T}A)$(注意第二个等号需要实矩阵作为前提条件)
    \item $A \in \mathbf{F}^{s \times n}$,$B \in \mathbf{F}^{n \times m}$,
    则$r(AB) \geqslant r(A)+r(B)-n$.(可以视为结论6的推论,特例$AB=O$时有$r(A)+r(B)\leqslant n$)
    \item $r(ABC) \geqslant r(AB)+r(BC)-r(B)$.(还可以考虑$A,B,C$相等的特殊情况的结果)
\end{enumerate}

分块矩阵的相关公式在上一小节的例题中已经书写过,此处不再重复.

一般而言,解决较为复杂的秩的问题时,我们可以采用如下方法:
\begin{enumerate}
    \item 利用(分块)矩阵初等变换;

    \item 利用线性方程组解的一般理论(将在专题五讲解);

    \item 利用向量组线性相关性;

    \item 利用已知的矩阵秩的等式和不等式.实际上等式很多时候基于可逆矩阵变换或者两个不等号夹逼.
\end{enumerate}

相关方法的应用都在本节最后的习题中有所体现,当然首要的任务是掌握上述基本的秩不等式的证明,
很多也利用了上面的思想,并且解法不唯一.

\vspace{2ex}
\centerline{\heiti \Large 内容总结}

\vspace{2ex}

\centerline{\heiti \Large 习题}
\vspace{2ex}
{\kaishu }
\begin{flushright}
    \kaishu

\end{flushright}
\centerline{\heiti A组}
\begin{enumerate}
    \item
\end{enumerate}
\centerline{\heiti B组}
\begin{enumerate}
    \item 设$W$是$n$维线性空间$V$的一个非平凡子空间,$W$中取一组基$\delta_1,\cdots,\delta_m$,按如下两种方式将其扩充为
	$V$的一组基:
	$$B_1=\{\delta_1,\cdots,\delta_m,\alpha_{m+1},\cdots,\alpha_n\};$$
	$$B_2=\{\delta_1,\cdots,\delta_m,\beta_{m+1},\cdots,\beta_n\}.$$
	设基$B_1$到$B_2$的过渡矩阵为$P$,求商空间$V/W$的基$\alpha_{m+1}+W,\cdots,\alpha_n+W$到
	$\beta_{m+1}+W,\cdots,\beta_n+W$的过渡矩阵.
\end{enumerate}
\centerline{\heiti C组}
\begin{enumerate}
    \item
\end{enumerate}
