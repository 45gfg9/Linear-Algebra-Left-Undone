\chapter{相似标准形:复数域上的尝试与理论}

\section{对角矩阵}

在介绍完相似的概念和基本性质后,我们便可以讨论如何寻找一组基使得线性变换的矩阵表示是一个很简单的矩阵,因为其它不够简单的矩阵只是线性变换在其他基下的表示,因此其他矩阵都与这一很简单的矩阵相似,因此称这一很简单的矩阵为相似标准形.

我们心中最简单的矩阵除了全零的矩阵外可能就是只有对角线非零的矩阵了. 这样的矩阵我们称为对角矩阵,本节我们将探讨什么情况下线性变换可以找到一组基使得表示矩阵为对角矩阵,以及如何求解出对角矩阵以及对应的基.

\subsection{对角化问题的一般解法}

首先我们需要说明可对角化的定义,事实上线性变换和矩阵都有相应的定义:
\begin{definition}
    \begin{enumerate}
        \item 线性变换可对角化:设$\sigma\in\mathcal{L}(V)$,如果存在$V$的一组基使得$\sigma$在这组基下的矩阵是对角矩阵,则称$\sigma$可对角化;

        \item 矩阵可对角化:设$A\in\mathbf{F}^{n\times n}$,如果存在可逆矩阵$P$使得$P^{-1}AP$是对角矩阵,则称$A$可对角化(等价于$A$相似于对角矩阵).
    \end{enumerate}
\end{definition}

事实上这两者之间的关联是密不可分的. 因为如果线性变换$\sigma$可对角化,则它的任意一组基$B_1$下的表示矩阵$A$都是可对角化的. 原因在于$\sigma$在某组基$B_2$下的矩阵是对角矩阵$\varLambda$,因此$A$和$\varLambda$是$\sigma$在两组基下的表示矩阵,因此$A$相似于对角矩阵,故$A$可对角化. 反之亦然,若$\sigma$的任意一组基下的表示矩阵$A$可对角化,这说明$A$相似于对角矩阵,因此$\sigma$在某一组基下的矩阵是对角矩阵,因此$\sigma$可对角化.

在本小节中,我们先探讨如下题型:我们假定线性映射$\sigma$和矩阵$A$可对角化,那么如何解出这一对角矩阵,以及解出
\begin{enumerate}
    \item $\sigma$在何组基下的矩阵是对角矩阵;

    \item 什么样的矩阵$P$使得$P^{-1}AP$是对角矩阵.
\end{enumerate}

我们将分别进行讨论. 我们先讨论矩阵的情况. 将$P^{-1}AP=\varLambda$变形为$AP=P\varLambda$,并将矩阵$P$按列分块为$P=(X_1,X_2,\ldots,X_n)$,则有
\[A(X_1,X_2,\ldots,X_n)=(X_1,X_2,\ldots,X_n)\diag(\lambda_1,\lambda_2,\ldots,\lambda_n),\]
利用分块矩阵乘法我们有$AX_j=\lambda_jX_j\enspace(X_j\neq 0,\enspace j=1,2,\ldots,n)$. 由于$P$是可逆矩阵,因此其列向量必然构成$\mathbf{F}^n$的一组基,即$A$相似于对角矩阵当且仅当$A$有$n$个线性无关的特征向量,并且这些特征向量按列排列就是我们要求解的$P$.

对于线性变换我们也可以做类似的分析. 事实上,若$\sigma$可对角化,我们可以简要做以下分析. 设$\sigma$在$V$的一组基$\alpha_1,\ldots,\alpha_n$下的矩阵为对角矩阵$\varLambda=\diag(\lambda_1,\ldots,\lambda_n)$,由线性映射矩阵表示的定义,这等价于
\[\sigma(\alpha_i)=\lambda_i\alpha_i\enspace(i=1,\ldots,n),\]
即$\alpha_1,\ldots,\alpha_n$是$\sigma$的$n$个线性无关的特征向量,因此$\sigma$可对角化当且仅当$\sigma$有$n$个线性无关的特征向量,并且这$n$个特征向量就是使得$\sigma$矩阵表示为对角矩阵的那组基.

然而,如果我们直接求解线性变换的对角问题有一个很大的问题,即我们实际上没有程序化的方法求解特征值和特征向量. 如果给定的映射形式比较简单,可能可以试出结果,但很多时候我们很难解出特征值和特征向量. 但回顾矩阵$A$的特征值只需求解$|\lambda E-A|=0$,特征向量只需要在得到$\lambda$后解线性方程组$AX=\lambda X$即可. 但我们可以回顾上一讲中推导的线性变换和矩阵特征值、特征向量的关联,我们知道线性变换$\sigma$与其在任意一组基$B$下的矩阵$A$有相同的特征值,且$A$的特征向量是$\sigma$特征向量在$B$下的坐标. 因此我们可以``曲线救国'',得到求解对角化问题的一般流程如下:

\begin{enumerate}
    \item 先任意写出$\sigma$在一组基$B$下的矩阵$A$,当然为了计算方便一般选取自然基;

    \item 利用特征多项式$f(\lambda)=|\lambda E-A|=0$求出$A$的所有不同特征值;

    \item 解线性方程组$AX=\lambda X$(实际上就是方程组$(\lambda E-A)X=0$,其中$\lambda$是上一步求出的特征值)求出$A$在不同特征值下的线性无关特征向量;

    \item 第三步中求得的所有向量就是$\sigma$的特征向量在基$B$下的坐标,根据前面的讨论,$\sigma$的特征向量也就是使得$\sigma$的矩阵表示为对角矩阵的那组基.

    \item 当然,如果题目中直接给出求$P$使得$P^{-1}AP$为对角矩阵,那么我们只需进行2、3两步,并将3中得到的向量按列排列成矩阵$P$即可.
\end{enumerate}

下面我们来看几个例子练习一下上面的求解过程:
\begin{example}
    求矩阵
    \[A=\begin{pmatrix}
            0  & -1 & 1 \\
            -1 & 0  & 1 \\
            1  & 1  & 0
        \end{pmatrix}\]
    的所有特征值,对应的特征子空间,以及与 $A$ 相似的一个对角矩阵.
\end{example}

\begin{solution}
    对于求解矩阵的对角化问题,首先求出其特征多项式(具体步骤不展开,实际上就是三阶行列式的计算,可以使用按行(列)展开、公式法或者初等变换化为三角矩阵等方法)$f(\lambda)=|\lambda E-A|=(\lambda-1)^2(\lambda+2)$,令$f(\lambda)=0$,解得特征值为 $\lambda_1=\lambda_2=1,\lambda_3=-2$.

    接下来求解特征向量和特征子空间,即求解$(E-A)x=0$和解$(-2E-A)x=0$,得到特征值1对应的特征子空间为$\spa((-1,1,0)^{\mathrm{T}},(1,0,1)^{\mathrm{T}})$,特征值-2对应的特征子空间为$\spa((-1,-1,1)^{\mathrm{T}})$.

    与$A$相似的对角矩阵实际上就是特征值排列在对角线上的结果,即 $\diag(1,1,-2)$.
\end{solution}

\begin{example}
    设 $T$ 是次数小于等于 2 的实多项式线性空间 $V$ 上的变换,对任意 $f(x) \in V$,定义
    \[T(f(x))=\frac{\mathrm{d}((x-2)f(x))}{\mathrm{d}x}\]
    证明 $T$ 是 $V$ 上的线性变换,且$T$可对角化.
\end{example}

\begin{proof}
    首先证明这是线性变换. 首先验证线性性,对于任意$f(x),g(x)\in V$,$a,b\in\mathbf{R}$,我们有
    \begin{align*}
        T(af(x)+bg(x)) & =\frac{\mathrm{d}((x-2)(af(x)+bg(x)))}{\mathrm{d}x}                                    \\
                       & =\frac{\mathrm{d}(axf(x)-2af(x)+bxg(x)-2bg(x))}{\mathrm{d}x}                           \\
                       & =a\frac{\mathrm{d}((x-2)f(x))}{\mathrm{d}x}+b\frac{\mathrm{d}((x-2)g(x))}{\mathrm{d}x} \\
                       & =aT(f(x))+bT(g(x)).
    \end{align*}
    然后说明这是$V$上的线性变换,即该映射的到达空间是$V$,即$T(f(x))\in V$, 因为$f(x)$是次数小于等于2的实多项式,设$f(x)=ax^2+bx+c$,则
    \begin{align*}
        T(f(x)) & =\frac{\mathrm{d}((x-2)(ax^2+bx+c))}{\mathrm{d}x}          \\
                & =\frac{\mathrm{d}(ax^3+(b-2a)x^2+(c-2b)x-2c)}{\mathrm{d}x} \\
                & =3ax^2+2(b-2a)x+(c-2b)\in V.
    \end{align*}
    因此$T$是$V$上的线性变换.

    下面我们来判断$T$是否可对角化. 线性变换的可对角化问题第一步要转化为任意一组基下的矩阵,然后判断矩阵是否可对角化,因此我们先任意选取一组基,为方便我们选取自然基$\{1,x,x^2\}$,然后求出$T$在这组基下的矩阵$A=\begin{pmatrix}
            1 & -2 & 0 \\ 0 & 2 & -4 \\ 0 & 0 & 3
        \end{pmatrix}$,然后求出其特征多项式$f(\lambda)=|\lambda E-A|=(\lambda-1)(\lambda-2)(\lambda-3)$,令$f(\lambda)=0$,解得特征值为 $\lambda_1=1,\lambda_2=2,\lambda_3=3$. 即该3阶矩阵有3个不同的特征值,因此由\autoref{cor:19:可对角化必要条件} 可知$A$可对角化,即$T$可对角化.
\end{proof}

除此之外,我们还可以利用对角化求解矩阵的幂的问题. 若一个矩阵$A$可对角化,即存在可逆矩阵$P$使得$A=P^{-1}\varLambda P$(其中$\varLambda$为对角矩阵),在这种形式下$A$的幂是很好求的,因为$A^k=P^{-1}\varLambda^kP$,$\varLambda$为对角矩阵,因此其幂是好求的. 我们来看一个例子
\begin{example}
    已知$A=\begin{pmatrix}
            0 & \dfrac{1}{2}  & \dfrac{1}{2} \\[2ex]
            1 & -\dfrac{1}{2} & \dfrac{1}{2} \\[2ex]
            1 & -\dfrac{1}{2} & \dfrac{1}{2}
        \end{pmatrix}$,求$A^n$.
\end{example}

\begin{solution}
    首先求出$A$的特征多项式$f(\lambda)=|\lambda E-A|=\lambda(\lambda-1)(\lambda+1)$,令$f(\lambda)=0$,解得特征值为 $\lambda_1=0,\lambda_2=1,\lambda_3=-1$.

    接下来求解特征向量和特征子空间,实际上就是求解$(0E-A)x=0,(-E-A)x=0,(E-A)x=0$,得到特征向量为
    \[\eta_1=\begin{pmatrix}
            1 \\ 1 \\ -1
        \end{pmatrix},\enspace \eta_2=\begin{pmatrix}
            1 \\ 1 \\ 1
        \end{pmatrix},\enspace \eta_3=\begin{pmatrix}
            1 \\ -1 \\ -1
        \end{pmatrix}.\]
    所以记$P=(\eta_1,\eta_2,\eta_3)$,则$A=P\diag(0,1,-1)P^{-1}$,因此
    \[A^n=P\diag(0^n,1^n,(-1)^n)P^{-1},\]
    进一步计算得到
    \[A^n=\frac{1}{2}\begin{pmatrix}
            1+(-1)^n     & (-1)^{n+1} & 1 \\
            1+(-1)^{n+1} & (-1)^n     & 1 \\
            1+(-1)^{n+1} & (-1)^n     & 1
        \end{pmatrix}.\]
\end{solution}

\subsection{可对角化的条件}

在我们推导求解对角化问题的一般流程时,我们提到了一个条件:矩阵$A$可对角化当且仅当$A$有$n$个线性无关的特征向量,线性变换$\sigma$可对角化当且仅当$\sigma$有$n$个线性无关的特征向量. 这一条件是非常重要的,基于这一观察我们可以扩展到下面这一定理:
\begin{theorem}
    设$V$是数域$\mathbf{F}$上的$n$维线性空间,$\sigma$是$V$上的线性变换,$\lambda_1,\lambda_2,\ldots,\lambda_s\in\mathbf{F}$是$\sigma$的所有互异特征值,则以下条件等价:
    \begin{enumerate}
        \item \label{item:19:可对角化条件:1}
              $\sigma$可对角化;

        \item \label{item:19:可对角化条件:2}
              $\sigma$有$n$个线性无关的特征向量,它们构成$V$的一组基;

        \item \label{item:19:可对角化条件:3}
              $V$有在$\sigma$下不变的一维子空间$U_1,\ldots,U_n$,使得$V=U_1\oplus\cdots\oplus U_n$.

        \item \label{item:19:可对角化条件:4}
              $V=V_{\lambda_1}\oplus V_{\lambda_2}\oplus\cdots\oplus V_{\lambda_s}$;

        \item \label{item:19:可对角化条件:5}
              $n=\dim V_{\lambda_1}+\dim V_{\lambda_2}+\cdots+\dim V_{\lambda_s}$;

        \item \label{item:19:可对角化条件:6}
              $\sigma$每个特征值的代数重数等于几何重数.
    \end{enumerate}
\end{theorem}
实际上对于矩阵我们有对应的定理(除了矩阵没有不变子空间外,其余将线性变换替换为矩阵即可),此处不再赘述. 下面我们简要证明这一定理:

\begin{proof}
    \begin{itemize}
        \item[\ref*{item:19:可对角化条件:1}$\implies$\ref*{item:19:可对角化条件:2}] 根据我们在推导求解线性变换对角矩阵流程时的分析,这一结论是成立的;

        \item[\ref*{item:19:可对角化条件:2}$\implies$\ref*{item:19:可对角化条件:3}] 由于$\sigma$有$n$个线性无关的特征向量,记为$\alpha_1,\ldots,\alpha_n$,则令$U_i=\spa(\alpha_i)$,则$U_i$是$\sigma$的不变子空间(因为$\sigma(\alpha_i)=\lambda\alpha_i\in U_i,\enspace\forall\alpha_i\in U_i$),且$V=U_1\oplus\cdots\oplus U_n$;

        \item[\ref*{item:19:可对角化条件:3}$\implies$\ref*{item:19:可对角化条件:4}] 我们将这些$U_i$中包含的向量按属于哪个$\lambda_i$的特征向量进行分类,然后每一类内的$U_i$进行直和即可得到特征子空间. 根据\autoref{thm:4:多空间直和} 可知结论成立;

        \item[\ref*{item:19:可对角化条件:4}$\implies$\ref*{item:19:可对角化条件:5}] 根据\hyperref[thm:4:直和等价命题]{直和的维数公式}显然成立;

        \item[\ref*{item:19:可对角化条件:5}$\implies$\ref*{item:19:可对角化条件:6}] 设$\lambda_1,\ldots,\lambda_s$的代数重数为$r_1,\ldots,r_s$,则$n=r_1+\cdots+r_s$,又根据\autoref{thm:18:代数重数与几何重数},$\dim V_{\lambda_i}\leqslant r_i,\enspace i=1,\ldots,s$,因此由$n=\dim V_{\lambda_1}+\cdots+\dim V_{\lambda_s}$可知必须有$\dim V_{\lambda_i}=r_i,\enspace i=1,\ldots,s$,即每个特征值的代数重数等于几何重数;

        \item[\ref*{item:19:可对角化条件:6}$\implies$\ref*{item:19:可对角化条件:1}] 由于每个特征值的代数重数等于几何重数,因此特征子空间维数之和为$n$,故存在$n$个线性无关的特征向量,根据我们在推导求解线性变换对角矩阵流程时的分析,这表明$\sigma$可对角化.
    \end{itemize}
\end{proof}

我们有一个显然的推论如下:
\begin{corollary}\label{cor:19:可对角化必要条件}
    若$n$维空间上的线性变换$\sigma$有$n$个不同的特征值,则$\sigma$可对角化. 反之,$\sigma$可对角化不一定有$n$个特征值.
\end{corollary}

\begin{proof}
    若$n$维空间上的线性变换$\sigma$有$n$个不同的特征值$\lambda_1,\ldots,\lambda_n$,设$v_1,\ldots,v_n$分别是$\lambda_1,\ldots,\lambda_n$对应的特征向量,由于$v_1,\ldots,v_n$线性无关,因此$v_1,\ldots,v_n$构成$V$的一组基,故$\sigma$可对角化.

    反之,我们只需要举出最简单的反例,例如$\sigma=I_V$,即$V$上的恒等映射,它的特征值只有一个,即$\lambda=1$,但它可对角化(因为在任意一组基下的矩阵都是单位矩阵).
\end{proof}

实际上由特征值的性质,我们容易知道数域$\mathbf{F}$上矩阵$A$可对角化,对于数域$\mathbf{F}$上任意多项式$f(x)$,$f(A)$也可对角化,且$A$可逆时,$A^{-1}$和$A^*$也可对角化.

\begin{proof}
    根据我们对特征值性质的讨论,若$A$有特征值$\lambda$,则$f(A)$对应的特征值为$f(\lambda)$,且$A$可逆时,$A^{-1}$和$A^*$有特征值$\lambda^{-1}$和$|A|\lambda^{-1}$,并且相应的特征向量保持不变,因此特征子空间都不变,一定也能保证特征子空间直和为$V$,因此它们都可对角化.
\end{proof}

接下来我们将给出一些基本的例子来运用上面的定理:
\begin{example}
    线性变换 $T : \mathbf{R}^3 \to \mathbf{R}^3$ 的定义是:
    \[T(x_1,x_2,x_3)=(4x_1+x_3,2x_1+3x_2+2x_3,x_1+4x_3)\]
    \begin{enumerate}
        \item 求出$T$的特征多项式及特征值;

        \item 判断$T$是否可对角化,并给出理由.
    \end{enumerate}
\end{example}

\begin{solution}
    \begin{enumerate}
        \item 我们知道,线性变换与其在任意一组基下的矩阵有相同的特征值和特征多项式,因此我们可以任意选取一组基(为方便我们取$\mathbf{R}^3$的自然基,然后求出$T$在这组基下的矩阵,即
              \[A=\begin{pmatrix}
                      4 & 0 & 1 \\
                      2 & 3 & 2 \\
                      1 & 0 & 4
                  \end{pmatrix}\]
              故$T$的特征多项式为$f(\lambda)=|A-\lambda E|=\begin{vmatrix}
                      4-\lambda & 0         & 1         \\
                      2         & 3-\lambda & 2         \\
                      1         & 0         & 4-\lambda
                  \end{vmatrix}=-(\lambda-3)^2(\lambda-5)$,$T$的特征值为$f(\lambda)=0$的解,即为$\lambda_1=\lambda_2=3,\lambda_3=5$.

        \item 对于$\lambda=3$,我们求解$(T-3I)\alpha$得到特征向量(其中令$\alpha=(x_1,x_2,x_3)$),解得$\alpha_1=(1,0,-1)^\mathrm{T},\alpha_2=(0,1,0)^\mathrm{T}$,因此$T$在$\lambda=3$时的特征子空间为$V_3=\spa(\alpha_1,\alpha_2)$,维数为2,因此$\lambda=3$时代数重数与几何重数相等. 而$\lambda=5$是一重特征值,因此也必有代数重数与几何重数相等(因为一定有特征向量,那么特征子空间维数至少为1,再结合代数重数大于等于几何重数可知,特征子空间维数恰为1). 因此$T$可对角化.
    \end{enumerate}
\end{solution}

\begin{example} \label{ex:16:若当块不可对角化}
    证明$r$阶上三角矩阵$(r>1)$
    \[J_0=\begin{pmatrix}
            \lambda_0 & 1         &        &           \\
                      & \lambda_0 & \ddots &           \\
                      &           & \ddots & 1         \\
                      &           &        & \lambda_0
        \end{pmatrix}\]
    不与对角阵相似.
\end{example}

\begin{solution}
    首先求出特征多项式为$f(\lambda)=|\lambda E-J_0|=(\lambda-\lambda_0)^r$,因此$J_0$只有一个特征值$\lambda_0$,且代数重数为$r$.

    接下来求几何重数,即$J_0X=\lambda_0X$的解空间维数,即$(\lambda_0 E-J_0)X=0O$的解空间维数,事实上由于$r(\lambda_0 E-J_0)=r-1$,因此解空间维数为$r-(r-1)=1$,即几何重数为$1<r$,因此不可对角化.
\end{solution}

事实上,上例中的矩阵我们称之为若当块矩阵,我们未来将会有完整第一章来介绍这一类型矩阵,因为它在我们的讨论中具有非常重要的地位——不可对角化的线性变换能得到的最简单的矩阵表示就是由多个若当块矩阵构成的,因此得到这一矩阵形式将是我们未来讨论的一大目标.

\begin{example}\label{ex:19:秩1矩阵可对角化}
    设$\alpha$和$\beta$是$\mathbf{R}^n\enspace (n>1)$中两个列向量,$A=\alpha\beta^\mathrm{T}\neq O$.
    \begin{enumerate}
        \item 求$A$的特征值;

        \item 证明:$\alpha^\mathrm{T}\beta=0\iff A$不可对角化.
    \end{enumerate}
\end{example}

\begin{solution}
    \begin{enumerate}
        \item 我们知道,$r(A)\leqslant\min{\{r(\alpha),r(\beta)\}}=1$,并且$A\neq O$因此$r(A)>0$,故$A$的秩为1. 而$n>1$,因此$A$一定不可逆,故0一定是$A$的特征值,且对应的特征子空间维数为$AX=0$的解空间维数,即为$n-1$.

              由此我们知道$A$最多有两个特征值,因为0的代数重数(即作为$n$次特征多项式的零点次数)必然大于等于其几何重数$n-1$,当期代数重数为$n-1$时可能还有一个一重特征值. 我们利用特征值之和等于$A$的迹来找出可能的第二个特征值. 我们设$\alpha=(a_1,a_2,\ldots,a_n)^\mathrm{T},\beta=(b_1,b_2,\ldots,b_n)^\mathrm{T}$,则$A=\alpha\beta^\mathrm{T}=\begin{pmatrix}
                      a_1b_1 & a_1b_2 & \cdots & a_1b_n \\
                      a_2b_1 & a_2b_2 & \cdots & a_2b_n \\
                      \vdots & \vdots & \ddots & \vdots \\
                      a_nb_1 & a_nb_2 & \cdots & a_nb_n
                  \end{pmatrix}$,因此$A$的迹为$\sum\limits_{i=1}^na_ib_i=\alpha^\mathrm{T}\beta=\sum\limits_{i=1}^n\lambda_i$,其中$\lambda_i$为$A$的特征值. 若$\alpha^\mathrm{T}\beta\neq 0$,则$\lambda_i=0,\enspace i=1,\ldots,n-1$,$\lambda_n=\alpha^\mathrm{T}\beta$. 若$\alpha^\mathrm{T}\beta=0$,则$A$的所有特征值均为0.

        \item 由上一小问可知,若$\alpha^\mathrm{T}\beta=0$即$A$的全部特征值为0,因此只有一个$n-1$维的特征子空间,故特征子空间直和不等于$V$,故不可对角化.

              反之,若$A$不可对角化,我们用反证法. 假设$\alpha^\mathrm{T}\beta\neq 0$,则$A$有两个特征值,一个为0,一个为$\alpha^\mathrm{T}\beta$,因此$A$有两个特征子空间,一个是0对应的$n-1$维特征子空间,一个是$\alpha^\mathrm{T}\beta$对应的一维特征子空间,因此$V$可分解为两个特征子空间的直和,与$A$不可对角化矛盾,因此$\alpha^\mathrm{T}\beta=0$.
    \end{enumerate}
\end{solution}

本例非常经典,请读者务必掌握本例的结论和解决方法. 事实上这一例题的结论与这一论述是等价的:秩为1的矩阵$A$可对角化的充要条件是$A$的迹不为0.

下面我们来看一些非常经典的可对角化问题,希望读者能够熟知:
\begin{example}\label{ex:19:可对角化经典例题}
    解决以下关于可对角化的基本问题:
    \begin{enumerate}
        \item 设$A$为$n$阶矩阵,且$A^2=2A$. 证明:$A$可对角化,并求出与之相似的对角矩阵(注:本题结论可推广到任意的$A^2=kA$);

        \item 设$A$为$n$阶非零矩阵,且$A^m=O\enspace(m\in\mathbf{N}_+,\enspace m>1)$. 证明:$A$不可对角化;

        \item 设$A$为二阶矩阵,非零向量$\alpha$不是$A$的特征向量,且$A^2\alpha-3A\alpha+2\alpha=0$. 证明:$\alpha$和$A\alpha$线性无关且$A$可对角化并求与$A$相似的对角矩阵.
    \end{enumerate}
\end{example}

\begin{proof}
    \begin{enumerate}
        \item 由题意$A^2-2A=O$,因此$A$的特征值$\lambda$满足\autoref{ex:18:特征值的性质} 第三问中提到的形式,因此$A$的特征值就是方程$\lambda^2-2\lambda=0$的解,即$\lambda_1=0,\lambda_2=2$.

              接下来我们需要说明0和2对应的特征子空间维数之和为$n$,即$\dim V_0+\dim V_2=n$,其中$V_0$和$V_2$分别为0和2对应的特征子空间. 事实上,由$A^2=2A$可知$A(A-2E)=O$,由\autoref{ex:15:线性方程组理论与秩不等式} 知$r(A)+r(A-2E)\leqslant n$,又根据秩不等式$r(A)+r(B)\geqslant r(A+B)$,因此$r(A)+r(A-2E)=r(A)+r(2E-A)\geqslant r(A+(2E-A))=r(2E)=n$. 综上可知,$r(A)+r(A-2E)=n$.

              实际上,$V_0$就是$AX=0$的解空间,$V_2$就是$(A-2E)X=0$的解空间,因此$\dim V_0=n-r(A),\dim V_2=n-r(A-2E)$,因此由$r(A)+r(A-2E)=n$可知$\dim V_0+\dim V_2=2n-n=n$,即0和2对应的特征子空间维数之和为$n$,因此$A$可对角化.

              由于可对角化矩阵代数重数等于几何重数,因此特征值0对应的代数重数为$n-r(A)$,特征值2对应的代数重数为$r(A)$,因此我们可以得到与$A$相似的对角矩阵为$\diag(0,\ldots,0,2,\ldots,2)$,其中0的个数为$n-r(A)$,2的个数为$r(A)$.

        \item 设$\lambda$是$A$的特征值,由题意$\lambda^m=0$,即$\lambda=0$,因此$A$的所有特征值都为0. 但$r(A)>0$(因为$A$不是零矩阵),因此0对应的特征子空间维数为$n-r(A)<n$,因此$A$不可对角化.

        \item 反证法,假设$\alpha$和$A\alpha$线性相关,则存在不全为零的常数$k_1,k_2$使得$k_1\alpha+k_2A\alpha=0$. 显然$k_2\neq 0$,因为假设$k_2=0$,则$k_1\alpha=0$,由于$\alpha\neq 0$,故$k_1=0$,这与$k_1,k_2$不全为0矛盾. 因此我们有$A\alpha=-\dfrac{k_1}{k_2}\alpha$,即$\alpha$是$A$的特征向量,这与题设矛盾,因此$\alpha$和$A\alpha$线性无关.

              由题意,$A^2\alpha-3A\alpha+2\alpha=0$,即$(A^2-3A+2E)\alpha=0$,又$\alpha\neq 0$,因此$A^2-3A+2E$不可逆,从而$|A^2-3A+2E|=|E-A||2E-A|=0$,故$|E-A|=0$或$|2E-A|=0$.

              若$|E-A|\neq 0$,则$E-A$可逆,因此$(A^2-3A+2E)\alpha=(E-A)((2E-A)\alpha)=0$可知$(2E-A)\alpha=0$,即$A\alpha=2\alpha$,故$\alpha$为$A$的特征向量,这与条件矛盾,因此$|E-A|=0$. 同理,$|2E-A|=0$,因此$A$有两个特征值1和2,又$A$是2阶矩阵,因此由\autoref{cor:19:可对角化必要条件} 可知$A$一定可对角化,且对角矩阵为$\begin{pmatrix}
                      1 & 0 \\
                      0 & 2
                  \end{pmatrix}$.
    \end{enumerate}
\end{proof}

最后需要说明一点,如果一个矩阵可对角化,则有$P^{-1}AP=\varLambda$,其中$\varLambda$是对角矩阵,$P$就是求解对角化问题过程中用到的由特征向量组成的矩阵. 那么我们可以将$A$表示为$A=P\varLambda P^{-1}$,这就是所谓特征值分解. 实际上之前相抵的讨论中我们也提到了类似的分解的技巧,它可以帮助我们解决很多问题. 事实上,之后学习的若当标准形、相合等都有类似的表示思想,在解决一些问题时是重要的.
\begin{example}
    设三阶矩阵$A$的特征值为$\lambda_1=-2,\lambda_2=1,\lambda_3=2$,对应的特征向量分别为$\alpha_1=(1,1,0)^\mathrm{T},\alpha_2=(1,0,1)^\mathrm{T},\alpha_3=(1,1,1)^\mathrm{T}$,求矩阵$A$.
\end{example}

\begin{solution}
    根据特征值分解可知,$A=P\varLambda P^{-1}$,其中$P=(\alpha_1,\alpha_2,\alpha_3),\varLambda=\begin{pmatrix}
            -2 & 0 & 0 \\
            0  & 1 & 0 \\
            0  & 0 & 2
        \end{pmatrix}$,因此
    \[A=P\varLambda P^{-1}=\begin{pmatrix}
            1 & 1 & 1 \\
            1 & 0 & 1 \\
            0 & 1 & 1
        \end{pmatrix}\begin{pmatrix}
            -2 & 0 & 0 \\
            0  & 1 & 0 \\
            0  & 0 & 2
        \end{pmatrix}\begin{pmatrix}
            1  & 0  & -1 \\
            1  & -1 & 0  \\
            -1 & 1  & 1
        \end{pmatrix}=\begin{pmatrix}
            -3 & 1 & 4 \\
            -4 & 2 & 4 \\
            -1 & 1 & 2
        \end{pmatrix}.\]
\end{solution}

\begin{example}
    设$A$相似于对角矩阵,$\lambda_0$是$A$的特征值,$X_0$是$A$对应于$\lambda_0$的特征向量,证明:
    \begin{enumerate}
        \item $r(A-\lambda_0 E)^2=r(A-\lambda_0 E)$;

        \item 不存在$Y$使得$(A-\lambda_0 E)Y=X_0$.
    \end{enumerate}
\end{example}

\begin{proof}
    \begin{enumerate}
        \item 由已知,设$A$的$n$个特征值为$\lambda_0,\lambda_1,\ldots,\lambda_{n-1}$,故$A$相似于对角矩阵$\varLambda=\diag(\lambda_0,\lambda_1,\ldots,\lambda_{n-1})$,设$P^{-1}AP=\varLambda$,因此
              \[A-\lambda_0 E=P\varLambda P^{-1}-\lambda_0 PP^{-1}=P\diag(0,\lambda_1-\lambda_0,\ldots,\lambda_{n-1}-\lambda_0)P^{-1},\]
              从而
              \[(A-\lambda_0 E)^2=P\diag(0,(\lambda_1-\lambda_0)^2,\ldots,(\lambda_{n-1}-\lambda_0)^2)P^{-1},\]
              由于$\lambda_i-\lambda_0=0$的充要条件是$(\lambda_i-\lambda_0)^2=0$,所以
              \begin{align*}
                  r((A-\lambda_0 E)^2) & =r(P\diag(0,(\lambda_1-\lambda_0)^2,\ldots,(\lambda_{n-1}-\lambda_0)^2)P^{-1}) \\
                                       & =r(\diag(0,(\lambda_1-\lambda_0)^2,\ldots,(\lambda_{n-1}-\lambda_0)^2))        \\
                                       & =r(\diag(0,\lambda_1-\lambda_0,\ldots,\lambda_{n-1}-\lambda_0))                \\
                                       & =r(P\diag(0,\lambda_1-\lambda_0,\ldots,\lambda_{n-1}-\lambda_0)P^{-1})         \\
                                       & =r(A-\lambda_0 E).
              \end{align*}
              故命题得证.

        \item 反证法,假设存在$Y$使得$(A-\lambda_0 E)Y=X_0$,则$(A-\lambda_0 E)^2Y=(A-\lambda_0 E)X_0=0$(因为$X_0$是$A$对应于$\lambda_0$的特征向量).

              由于$r((A-\lambda_0 E)^2)=r(A-\lambda_0 E)$,因此$(A-\lambda_0 E)^2X=0$与$(A-\lambda_0 E)X=0$的解空间维数相同,又$(A-\lambda_0 E)X=0$的解显然一定也是$(A-\lambda_0 E)^2X=0$的解,因此实际上两方程组同解(回顾$U$和$W$都是$V$的非零子空间,$U\subseteq W$,且$\dim U=\dim W$,则$U=W$).

              由于$(A-\lambda_0 E)^2Y=0$,因此也有$(A-\lambda_0 E)Y=0$,但已知$(A-\lambda_0 E)Y=X_0\neq 0$,矛盾,因此不存在$Y$使得$(A-\lambda_0 E)Y=X_0$.
    \end{enumerate}
\end{proof}

事实上,在一般的教材中还会专门探讨实对称矩阵的对角化问题. 这一问题涉及到后续要讲解的正交概念,因此我们会在内积空间上的线性变换中通过谱定理讨论这一问题. 届时我们将讨论在内积空间中的线性变换满足什么条件时一定可以对角化.

\subsection{幂等矩阵}

本节我们专门讨论一个常见的特殊矩阵:幂等矩阵. 若$n$阶方阵$A$满足$A^2=A$,则$A$称为幂等矩阵. 幂等矩阵具有如下基本性质:
\begin{enumerate}
    \item $A$是幂等矩阵等价于$r(A)+r(A-E)=n$;

    \item $A$为幂等矩阵则一定可对角化,特征值为0和1,其中1的重数等于$r(A)$;

    \item $A$是幂等矩阵时,$r(A)=\tr(A)$;

    \item 所有秩为1迹也为1的矩阵均为幂等矩阵.
\end{enumerate}

\begin{proof}
    \begin{enumerate}
        \item 参考\autoref{ex:15:线性方程组理论与秩不等式} 第四点的证明即可.

        \item 与\autoref{ex:19:可对角化经典例题} 第一问中解法类似可得$A$为幂等矩阵则一定可对角化,且特征值为0和1. 因为$A$可对角化,故0和1的代数重数等于几何重数(统称重数),且二者重数之和为$n$. 由于0的重数(从几何重数的角度看)就等于$AX=0$的解空间维数,即等于$n-r(A)$,因此1的重数等于$n-(n-r(A))=r(A)$.

        \item 事实上$\tr(A)$等于$A$的所有特征值之和. 事实上由上面的结论,幂等矩阵$A$特征值由$n-r(A)$个0和$r(A)$个1组成,即与幂等矩阵相似的对角矩阵为$\varLambda=\diag(0,\ldots,0,1,\ldots,1)$,其中有$r(A)$个1. 又因为相似矩阵有相同的特征值,因此对于任意的幂等矩阵$A$都有$\tr(A)=\tr(\varLambda)=r(A)$.

        \item 根据我们在相抵标准形中讨论的分解可知,所有秩为1的矩阵都可分解为一个列向量乘以行向量的形式,即$A=\alpha\beta^\mathrm{T}$,其中$\alpha,\beta$都是列向量. 并且同\autoref{ex:19:秩1矩阵可对角化} 中的讨论,$A$的迹即为$\alpha^\mathrm{T}\beta=\beta^\mathrm{T}\alpha=1$,因此
              \[A^2=(\alpha\beta^\mathrm{T})(\alpha\beta^\mathrm{T})=\alpha(\beta^\mathrm{T}\alpha)\beta^\mathrm{T}=\alpha\beta^\mathrm{T}=A,\]
              因此$A$是幂等矩阵.
    \end{enumerate}
\end{proof}

实际上,幂等矩阵还有很多其他的性质,我们可以回到映射的角度去理解这一矩阵,讨论其与投影变换的等价性,这一点我们将在后续内积空间讲解投影变换时中给出详细说明. 下面是一个技巧性较强的题目,读者可以在此题中体会``幂等''这一性质的特点:
\begin{example}
    设$A$,$B$为两个$n$阶幂等矩阵,证明:
    \begin{enumerate}
        \item $A+B$为幂等矩阵当且仅当$AB=BA=O$;

        \item $A-B$为幂等矩阵当且仅当$AB=BA=B$;

        \item 若$AB=BA$,则$AB$为幂等矩阵. 反之,若$AB$为幂等矩阵,是否必有$AB=BA$;

        \item 若$E-A-B$可逆,则$r(A)=r(B)$.
    \end{enumerate}
\end{example}

\begin{proof}
    \begin{enumerate}
        \item 必要性:由于$A+B$幂等,所以
              \[A+B=(A+B)^2=A^2+AB+BA+B^2=A+B+AB+BA,\]
              因此$AB+BA=O$,即$AB=-BA$. 又由于$A$和$B$均幂等,从而也有
              \[AB=A^2B=A(AB)=A(-BA)=-(AB)A=(BA)A=BA^2=BA.\]
              于是有$AB=BA=O$.

              充分性:由于$AB=BA=O$,因此
              \[(A+B)^2=A^2+AB+BA+B^2=A+B.\]
              因此$A+B$幂等.

        \item 必要性:由于$A-B$幂等,所以
              \[A-B=(A-B)^2=A^2-AB-BA+B^2=A+B-AB-BA,\]
              因此$AB+BA=2B$,即$AB=2B-BA$,从而也有
              \[AB=AB^2=(2B-BA)B=2B^2-B(AB)=2B^2-B(2B-BA)=B^2A=BA.\]
              于是有$AB=BA=B$.

              充分性:由于$AB=BA=B$,因此
              \[(A-B)^2=A^2-AB-BA+B^2=A-B.\]
              因此$A-B$幂等.

        \item 由于$AB=BA$,因此
              \[(AB)^2=ABAB=A(BA)B=A^2B^2=AB,\]
              因此$AB$幂等. 反之,取
              \[A=\begin{pmatrix}
                      1 & 0 & 1 \\ 0 & 0 & 0 \\ 0 & 0 & 0
                  \end{pmatrix},\enspace B=\begin{pmatrix}
                      1 & 1 & 0 \\ 0 & 0 & 0 \\ 0 & 0 & 0
                  \end{pmatrix},\]
              可以验证$AB$幂等,但$AB=B\neq A=BA$.

        \item 由于$E-A-B$可逆,因此
              \[r(A)=r(A(E-A-B))=r(A-A^2-AB)=r(AB)\leqslant r(B);\]
              \[r(B)=r(B(E-A-B))=r(B-B^2-BA)=r(BA)\leqslant r(A).\]
              因此$r(A)=r(B)$.
    \end{enumerate}
\end{proof}

\section{上三角矩阵}

我们从上三角矩阵出发,首先因为之前的讨论中上三角矩阵的一些优良性质大家已经非常熟悉,并且事实上,复向量空间中的所有线性变换都可以在某组基下得到上三角矩阵的矩阵表示. 我们有如下定理:
\begin{theorem}\label{thm:17:上三角矩阵存在}
    设$V$是有限维复向量空间,$\sigma\in \mathcal{L}(V)$,则
    \begin{enumerate}
        \item $\sigma$关于$V$的某组基有上三角矩阵,记为$A$;

        \item $\sigma$可逆的充要条件是$A$的主对角元均不为0;

        \item $\sigma$的特征值恰为$A$的主对角元.
    \end{enumerate}
\end{theorem}

\begin{proof}
    \begin{enumerate}
        \item 根据相似的定义,这一结论与``任意$n$阶复矩阵一定相似于一个上三角矩阵''是等同的,我们可以使用分块矩阵结合数学归纳法的方法进行证明. $n=1$时结论显然,因为任意一阶矩阵本身就是上三角矩阵. 现假设$n-1$阶复矩阵都可以相似于上三角矩阵,设$A$为$n$阶复矩阵,我们任取$A$的一个复特征值$\lambda_1$,设$\alpha_1$为$A$对应于$\lambda_1$的特征向量. 我们把$\alpha_1$扩充为$\mathbf{C}^n$的一组基,记为$\alpha_1,\alpha_2,\ldots,\alpha_n$,记$P_1=(\alpha_1,\alpha_2,\ldots,\alpha_n)$,则$P_1$可逆,且
              \[P_1^{-1}AP_1=\begin{pmatrix}
                      \lambda_1 & \alpha' \\ 0 & A_{n-1}
                  \end{pmatrix},\]

              我们对$n-1$阶矩阵$A_{n-1}$应用归纳假设,因此存在可逆矩阵$P_2$使得$P_2^{-1}A_{n-1}P_2$为上三角矩阵,取$P=P_1\begin{pmatrix}
                      1 & 0 \\ 0 & P_2
                  \end{pmatrix}$,直接有
              \[P^{-1}AP=\begin{pmatrix}
                      \lambda_1 & \alpha'P_1 \\ 0 & P_2^{-1}A_{n-1}P_2
                  \end{pmatrix}\]
              为上三角矩阵,因此$n$阶复矩阵一定相似于上三角矩阵.

        \item 这一结论在矩阵计算进阶一讲中已经讨论,此处不再赘述;

        \item 利用线性映射的特征值等于表示矩阵的特征值,且表示矩阵的特征值就是$|\lambda E-A|=0$的零点即可得到这一结论.
    \end{enumerate}
\end{proof}

除此之外,在矩阵计算进阶中我们也提到上三角矩阵相乘结果中对角线上元素是原矩阵对角线上对应元素相乘的结果,其逆的对角线上元素是原矩阵对角线对应元素的逆. 以上性质表明,上三角矩阵是所有线性变换都可以在某组基下得到的且有良好性质的矩阵类型. 还有一个需要说明的性质是,一个元素出现在上三角矩阵对角线上的次数恰好等于它作为特征值的重数,我们放在习题中供读者验证.

事实上,我们在这里介绍这一证明的主要目的在于,这一证明给了我们一个求解线性变换(或矩阵)上三角化的方法. 与对角化类似,线性变换的上三角化依赖于矩阵上三角化,因此我们这里只讨论矩阵的情况. 根据上面的证明,我们只需任意挑选$n$阶矩阵的一个特征值和一个对应的特征向量,然后问题就转化为求解$n-1$阶矩阵的上三角化问题,那么我们继续求出$n-1$阶矩阵的一个特征值和一个对应的特征向量,依次类推直到一阶的情况. 我们给出下面的例子供读者运用这一方法:
\begin{example}
    设$\sigma\in\mathbf{C}^3$定义为$\sigma(x,y,z)=(2x+y,5y+3z,8z)$,求$\mathbf{C}^3$的一组基使得$\sigma$在这组基下的矩阵为上三角矩阵.
\end{example}

\begin{solution}

\end{solution}

事实上,我们研究上三角矩阵(以及后续分块对角矩阵)的思路与研究对角矩阵是类似的,我们会讨论计算方法以及相关的等价条件. 下面我们给出关于对角矩阵的一些等价条件:
\begin{theorem}\label{thm:17:上三角矩阵等价条件}
    设$\sigma\in \mathcal{L}(V)$,且$v_1,v_2,\ldots,v_n$是$V$的基,则以下条件等价:
    \begin{enumerate}
        \item $\sigma$关于$v_1,v_2,\ldots,v_n$的矩阵是上三角的;

        \item 对每个$j=1,\ldots,n$有$\sigma(v_j)\in\spa(v_1,\ldots,v_j)$;

        \item 对每个$j=1,\ldots,n$有$\spa(v_1,\ldots,v_j)$在$\sigma$下不变.
    \end{enumerate}
\end{theorem}

\begin{proof}
    \begin{enumerate}
        \item

        \item

        \item
    \end{enumerate}
\end{proof}

这一定理给出了上三角矩阵的几个充要条件,基于这些充要条件我们可以有如下进一步的理解:
\begin{enumerate}
    \item 我们可以给出\autoref{thm:17:上三角矩阵存在} 的另外两种证明,即《线性代数应该这样学》
          5.27给出的两种证明. 事实上5.27的证明也使用了数学归纳法,但是之前分块矩阵的归纳是针对矩阵的阶数,这里是针对线性变换所在线性空间的维数. 我们这给出证明,以便深刻体会数学归纳法的思想:

          \begin{proof}
              \begin{enumerate}
                  \item

                  \item
              \end{enumerate}
          \end{proof}

    \item 比较对角矩阵的充要条件中要求存在一维不变子空间的分解,这里的等价条件表明上三角矩阵要求线性变换存在任意维数的不变子空间即可(对角矩阵的条件实际上蕴含这一条件,因为存在一维不变子空间直和分解,我们任意通过直和运算合并这些一维不变子空间就可以得到任意维数的不变子空间,因此对角矩阵的要求更强,上三角矩阵不需要这些任意维数的不变子空间能拆解成一维不变子空间的直和).

          事实上,因为任意复向量空间上的线性变换都存在一组基使得矩阵表示为上三角矩阵,因此任意复向量空间上的线性变换都存在任意维数的不变子空间. 因此下面这一例子的结论是显然的:
          \begin{example}
              设$V$是$n$维复向量空间. $\sigma\in \mathcal{L}(V)$,证明:对任意的正整数$r\enspace(1\leqslant r\leqslant n)$,$\sigma$有$r$维不变子空间.
          \end{example}
          \begin{proof}

          \end{proof}
\end{enumerate}

接下来我们要讨论一个特别的问题,即线性变换/矩阵可交换的性质. 我们有如下定理:
\begin{theorem}
    设$V$为$n$维复向量空间,$\sigma,\tau\in \mathcal{L}(V)$,$\sigma\tau=\tau\sigma$,则
    \begin{enumerate}
        \item $\sigma$的每个特征子空间都是$\tau$的不变子空间;

        \item $\sigma,\tau$有公共的特征向量.
    \end{enumerate}
\end{theorem}
将这一定理的线性变换改为矩阵实际上是等价的.

\begin{proof}
    \begin{enumerate}
        \item

        \item
    \end{enumerate}
\end{proof}

接下来我们希望应用这上述定理解决下面的问题:
\begin{example}
    设$V$为$n$维复向量空间,$\sigma,\tau\in \mathcal{L}(V)$,$\sigma\tau=\tau\sigma$,证明:
    \begin{enumerate}
        \item 若$\sigma$有$s$个不同的特征值,则$\sigma,\tau$至少有$s$个公共且线性无关的特征向量;

        \item 存在$V$的一组基,使得$\sigma$和$\tau$在这组基下的矩阵均为上三角矩阵.
    \end{enumerate}
\end{example}

\begin{proof}
    \begin{enumerate}
        \item

        \item
    \end{enumerate}
\end{proof}

这一例子的结论告诉我们:线性变换可交换对应于同时上三角化. 例中 2 的结论如果换为矩阵表述应当是:设$A,B$是复数域上的两个$n$阶矩阵,且$AB=BA$,则存在可逆矩阵$P$使得$P^{-1}AP$和$P^{-1}BP$同时为上三角矩阵.

\section{核空间的性质 \quad 幂零矩阵}

这一节我们将为后续讨论分块对角矩阵做准备,同时幂零矩阵一节中也将讨论这一特殊矩阵的很多特别的、有趣的性质.

\subsection{核空间的性质}

事实上,根据可对角化的等价条件,我们知道一个线性变换不可对角化实际上是因为它没有足够多的线性无关的特征向量,也即特征子空间直和后比原空间略小. 事实上,我们知道$\sigma$在特征值$\lambda$下的特征子空间实际上就是$\ker(\lambda I-\sigma)$. 我们回顾\autoref{thm:6:核空间性质}:
\begin{theorem}\label{thm:17:核空间性质}
    设$\sigma\in \mathcal{L}(V)$,则有
    \begin{enumerate}
        \item $\{0\}=\ker \sigma^0\subseteq\ker \sigma^1\subseteq\cdots\subseteq\ker \sigma^k\subseteq\ker \sigma^{k+1}\subseteq\cdots$;

        \item 设$m$是非负整数使得$\ker \sigma^m=\ker \sigma^{m+1}$,则
              \[\ker \sigma^m=\ker \sigma^{m+1}=\ker \sigma^{m+2}=\ker \sigma^{m+3}=\cdots\]

        \item 令$n=\dim V$,则$\ker \sigma^n=\ker \sigma^{n+1}=\ker \sigma^{n+1}=\cdots$.
    \end{enumerate}
\end{theorem}

我们发现,如果我们提高线性变换的幂次,那么我们可以获得更大的核空间,这样扩张后的核空间的直和是否可以张成整个原空间呢?我们可以提前给出答案:可以,具体的说明我们在讨论完下面的幂零矩阵之后将会详细展开.

\subsection{幂零矩阵}

基于上面核空间的讨论,并为了方便后面小节的研究,我们将讲解幂零线性变换与幂零矩阵的相关准备知识.
\begin{definition}[幂零] \index{miling@幂零 (nilpotent)}
    \begin{enumerate}
        \item 一个线性变换称为\term{幂零}的,如果它的某个幂等于零映射(即将所有向量都映射到0的映射);

        \item 一个矩阵$A$称为幂零的,如果存在正整数$m$使得$A^m=O$.
    \end{enumerate}
\end{definition}

根据线性映射矩阵表示很容易知道,幂零线性变换在任意一组基下的矩阵表示都是幂零矩阵. 我们接下来首先讨论幂零线性变换的一些基本性质:
\begin{theorem} \label{thm:17:幂零线性变换性质}
    设线性变换$N\in \mathcal{L}(V)$是幂零的,则
    \begin{enumerate}
        \item \label{item:17:幂零线性变换性质:1}
              $N$的所有特征值均为0(等价定义);

        \item \label{item:17:幂零线性变换性质:2}
              $N^{\dim V}$=0;

        \item \label{item:17:幂零线性变换性质:3}
              $V$有一组基使得$N$关于这组基的矩阵对角线和对角线下方元素均为0(等价定义);

        \item \label{item:17:幂零线性变换性质:4}
              $N\pm I$可逆.
    \end{enumerate}
\end{theorem}

\begin{proof}
    \begin{enumerate}
        \item 这一结论我们将在下一讲中介绍 \hyperref[thm:21:HC]{Hamilton-Cayley 定理}后给出证明;

        \item

        \item

        \item
    \end{enumerate}
\end{proof}

事实上 \ref*{item:17:幂零线性变换性质:1},\ref*{item:17:幂零线性变换性质:2},\ref*{item:17:幂零线性变换性质:4} 都有相应的矩阵的结论,我们将线性变换替换为它的矩阵表示即可,此处不再赘述. 而第三点则解释了我们在求矩阵的幂时将一些矩阵分解为一个矩阵加一个对角线上全为0的矩阵的合理性,因为后者一定是幂零的. 接下来我们通过几个例子进一步讨论、运用幂零矩阵、幂零线性变换的性质:
\begin{example}
    证明:$A$为幂零矩阵$\iff \forall k \in \mathbf{N}_+,\enspace\tr(A^k)=0$.
\end{example}

\begin{proof}

\end{proof}

\begin{example}
    若$A,B$为两个$n$阶矩阵且满足$AB-BA=A$,证明:
    \begin{enumerate}
        \item $A$不可逆;

        \item $A$是幂零矩阵.
    \end{enumerate}
\end{example}

\begin{proof}
    \begin{enumerate}
        \item

        \item
    \end{enumerate}
\end{proof}

\section{分块对角矩阵}

\subsection{广义特征子空间与分块对角矩阵}

上一节中我们已经讨论了不可对角化线性变换获得简化矩阵的一般思想,即试图利用核空间增长的性质扩张特征子空间,使得扩张后的特征子空间(称为广义特征子空间)的直和为原空间. 下面我们给出严谨定义:
\begin{definition}
    设$\sigma\in \mathcal{L}(V)$,$\lambda\in\mathbf{F}$是$\sigma$的特征值,若向量$v\neq 0$且存在正整数$j$使得$(\sigma-\lambda I)^jv=0$,则称$v$为$\sigma$对应于$\lambda$的\term{广义特征向量}\index{tezhengxiangliang!guangyi@广义 (generalized eigenvector)}. $\sigma$对应于$\lambda$的全体广义特征向量与0向量构成的集合称为$\sigma$相应于$\lambda$的\term{广义特征子空间}\index{tezhengzikongjian!guangyi@广义 (generalized eigenspace)},记为$G(\lambda,\sigma)$.
\end{definition}
注意我们不定义广义特征值,因为若$\lambda$原先不是特征值,因此$\sigma-\lambda I$可逆,可逆映射复合仍可逆,故对于任意的$j$,$(\sigma-\lambda I)^j$仍可逆,即特征值是不会随着线性变换幂次增加而增加的.

实际上,根据\autoref{thm:17:核空间性质},我们有$G(\lambda,\sigma)=\ker (\sigma-\lambda I)^{\dim V}$. 需要补充说明的是,此处引入两个概念称为代数重数(或称重数)和几何重数,其中$\lambda$的代数重数定义为广义特征子空间的维数,几何重数定义为特征子空间的维数. 实际上在不变子空间一讲中我们有类似的定义,我们将在下一讲中讲解它们的关联.

我们接下来的目标转向我们的主线,即证明任意线性变换的广义特征子空间的和为直和且和为原空间. 下面这一定理读者可以回顾特征值、特征向量的性质以及可对角化的等价条件,我们会发现这些定理具有很大的相似性,因此记忆难度并不大:
\begin{theorem} \label{thm:17:广义特征性质}
    设$V$是复数域上的有限维线性空间,$\sigma\in \mathcal{L}(V)$. 用$\lambda_1,\ldots,\lambda_m$表示$\sigma$的所有互异特征值.
    \begin{enumerate}[label=(\arabic*)]
        \item $\sigma$对应于不同特征值的广义特征向量线性无关;

        \item \label{item:17:广义特征性质:2}
              $\sigma$不同特征值对应的广义特征子空间的和为直和,且$V=G(\lambda_1,\sigma)\oplus\cdots\oplus
                  G(\lambda_m,\sigma)$;

        \item $V$有一个由$\sigma$的广义特征向量组成的基;

        \item 每个$G(\lambda_i,\sigma)$在$\sigma$下都是不变的;

        \item \label{item:17:广义特征性质:5}
              每个$(\sigma-\lambda_j I)\vert_{G(\lambda_j,\sigma)}$都是幂零的.
    \end{enumerate}
\end{theorem}

\begin{proof}
    \begin{enumerate}
        \item

        \item

        \item

        \item

        \item
    \end{enumerate}
\end{proof}

上述定理更重要的结果在于它我们可以得到任何复向量空间上的线性变换都有如下的分块对角矩阵的标准形:
\begin{theorem} \label{thm:17:分块对角矩阵}
    设$V$是复向量空间,$\sigma\in \mathcal{L}(V)$. 设$\lambda_1,\ldots,\lambda_m$是$\sigma$的所有互不相同的特征值,重数分别为$d_1,\ldots,d_m$,则$V$有一组基使得$\sigma$关于这组基的有分块对角矩阵
    \[\begin{pmatrix}
            A_1 &  & O \\  & \ddots &  \\ O &  & A_m
        \end{pmatrix}\]
    其中每个$A_j$都是如下所示的$d_j\times d_j$上三角矩阵
    \[A_j=\begin{pmatrix}
            \lambda_j &  & * \\  & \ddots &  \\ O &  & \lambda_j
        \end{pmatrix}\]
\end{theorem}

\begin{proof}

\end{proof}

由此我们得到了一个相比于上三角矩阵更为简单,并且所有线性变换都可以获得的标准形. 在介绍完其存在性后,我们按照惯例需要讨论如何将这一标准形求解出来. 事实上,根据上述定理的证明,我们发现每个对角块都是从一个广义特征子空间得来的,因此我们只需求出各个广义特征子空间的基,然后写出对应的矩阵即可. 如果得到的对角块不是上三角矩阵,我们可以使用在上三角矩阵求法中讲解的方法进行调整. 我们来看一个例子:
\begin{example}
    设$\sigma\in \mathcal{L}(\mathbf{C}^3)$定义为
    \[\sigma(z_1,z_2,z_3)=(6z_1+3z_2+4z_3,6z_2+2z_3,7z_3),\]求一组基使其有分块对角矩阵并写出对应的分块对角矩阵.
\end{example}

\begin{solution}

\end{solution}

事实上,读者会发现虽然整体思路是很简单的,但是中间求解广义特征子空间的过程还是存在一定的困难. 因为当$\dim V$较大时,$G(\lambda,\sigma)=\ker (\sigma-\lambda I)^{\dim V}$的求解需要反复计算幂次,是很困难的,但事实上根据核空间停止增长的性质可以知道,我们只需要不断提升矩阵的幂次,直到得到的广义特征子空间不再发生改变就能够停止计算.

\begin{example}
    设$\sigma,\tau\in \mathcal{L}(V)$可逆,证明:$\sigma$和$\tau^{-1}\sigma\tau$有相同的特征值,且重数也相同.
\end{example}

\begin{proof}

\end{proof}

\subsection{平方根问题}

在进入下一个话题前,我们先简单介绍线性变换平方根的概念,这一概念在之后内积空间线性变换会进一步说明.
\begin{definition}
    我们称线性变换$\sigma\in \mathcal{L}(V)$的平方根是满足$\tau^2=\sigma$的线性变换$\tau\in \mathcal{L}(V)$.
\end{definition}
在复向量空间中,我们有如下两个结论:
\begin{theorem} \label{thm:20:幂零平方根}
    设$V$是复向量空间.
    \begin{enumerate}
        \item 设$N\in \mathcal{L}(V)$幂零,则$(I+N)$有平方根;

        \item \label{item:20:幂零平方根:2}
              若$\sigma\in \mathcal{L}(V)$可逆,则$\sigma$有平方根.
    \end{enumerate}
\end{theorem}

\begin{proof}
    \begin{enumerate}
        \item

        \item
    \end{enumerate}
\end{proof}

我们发现,这一定理的证明思路基于$\sqrt{1+x}$的泰勒展开,我们不是第一次看到使用泰勒展开的情况,在求解矩阵的逆的进阶方法中,求逆的分式思想中也使用了 $\vphantom{\cfrac{1}{1-x}}\dfrac{1}{1-x}$的泰勒展开,足以体现一些数学直觉对于我们解决一些问题的重要性.

\begin{example}
    定义$N\in \mathcal{L}(\mathbf{F}^5)$为
    \[N(x_1,x_2,x_3,x_4,x_5)=(2x_2,3x_3,-x_4,4x_5,0)\]
    求$(I+N)$的一个平方根.
\end{example}

\begin{solution}

\end{solution}

最后,在开始习题内容前,我们需要讲解一类特殊的题型,即举例或举反例的问题. 一般而言,我们有如下两种思路:
\begin{enumerate}
    \item 考虑几何意义:例如旋转矩阵,特征值的几何意义等
          \begin{example}
              找出有限维实向量空间的一个线性变换$\sigma$,使得0是$\sigma$仅有的特征值但$\sigma$不是幂零线性变换.
          \end{example}
          \begin{example}
              找出一个$\sigma\in L(\mathbf{R}^2)$使得$\sigma^4=-I$.
          \end{example}

    \item 考虑简单的情况:例如考虑2阶、3阶的简单线性变换/矩阵
          \begin{example}
              证明或给出反例:$V$上的幂零线性变换的集合是$L(V)$的子空间.
          \end{example}
          很多时候一些反例很难构想就选择记住这一构造思想即可. 一些反例可能基于一些简单的结论,但如果未思考到位可能很难构造.
\end{enumerate}

\vspace{2ex}
\centerline{\heiti \Large 内容总结}

本讲我们介绍了相似的定义与性质,并介绍了第一个且是最简单的相似标准形——对角矩阵. 我们介绍了线性变换和矩阵可对角化的定义以及二者的统一性,介绍了如何求解线性变换/矩阵的对角化问题. 我们也探讨了线性变换/矩阵可对角化的几个充分必要条件,并通过大量的例题运用了这些条件. 最后我们介绍了幂等矩阵这一特殊矩阵,它有很多值得探讨的性质,并且很适合于作为本讲的一个运用.

从下一讲开始,我们将要介绍当线性变换/矩阵不可对角化时,它们可以有哪些退而求其次的也比较简单的相似标准形.

\vspace{2ex}
\centerline{\heiti \Large 习题}

\vspace{2ex}
{\kaishu 事类相推,各有攸归,故枝条虽分而同本干知,发其一端而已.}
\begin{flushright}
    \kaishu
    ——刘徽,《九章算术注·原序》
\end{flushright}

\centerline{\heiti A组}
\begin{enumerate}
    \item 请举例:存在两个矩阵相抵但不相似.

    \item 求矩阵
          \[A = \begin{pmatrix}
                  0  & -1 & 1 \\
                  -1 & 0  & 1 \\
                  1  & 1  & 0
              \end{pmatrix}\]
          的所有特征值,对应的特征子空间,以及与 $A$ 相似的一个对角矩阵.

    \item 设$A=\begin{pmatrix}
                  a & b \\ c & d
              \end{pmatrix}$为二阶实矩阵.
          \begin{enumerate}
              \item 若$|A|<0$,问:$A$与对角矩阵是否相似;

              \item 若$ad-bc=1$,$|a+d|>2$,问:$A$是否可对角化.
          \end{enumerate}

    \item 设$A=\begin{pmatrix}
                  1 & 1 & a \\
                  1 & a & 1 \\
                  a & 1 & 1
              \end{pmatrix}$,$\beta=\begin{pmatrix}
                  1 \\ 1 \\ -2
              \end{pmatrix}$,方程组$AX=\beta$有解但不唯一.
          \begin{enumerate}
              \item 求$a$的值;

              \item 求可逆矩阵$P$使得$P^{-1}AP$为对角矩阵.
          \end{enumerate}

    \item 设$A$为三阶矩阵,$\alpha_1,\alpha_2,\alpha_3$线性无关,且$A\alpha_1=\alpha_1,A\alpha_2=\alpha_1+\alpha_2-2\alpha_3,A\alpha_3=\alpha_1-2\alpha_2+\alpha_3$,求$A$的特征值.

    \item 设三阶实对称矩阵$A$的各行元素之和为3,向量$\alpha_1=(-1,2,-1)^\mathrm{T}$,$\alpha_2=(0,-1,1)^\mathrm{T}$是方程组$AX=0$的两个解,求矩阵$P$使得$P^{-1}AP$为对角矩阵.
\end{enumerate}

\centerline{\heiti B组}
\begin{enumerate}
    \item 设$a\neq b$,且$(aE-A)(bE-A)=O$. 证明:$A$可对角化(特例:对合矩阵);

    \item 证明:满足$A^2=E$且特征值只有1的矩阵只能是$E$,特征值只有$-1$的矩阵只能是$-E$.

    \item 设$A$为三阶矩阵,$A^2=A$且$r(A)=r$,求$|A-2E|$.

    \item 设$T\in \mathcal{L}(\mathbf{C}^3)$使得6和7是$T$的特征值,且$T$不可对角化. 证明:存在$(x,y,z)\in\mathbf{C}^3$使得$T(x,y,z)=(17+8x,\sqrt{5}+8y,2\pi+8z)$.

    \item 证明:两个可对角化的同阶矩阵特征值相同(包括重数)等价于它们相似. 对于不可对角化的矩阵来说,这一结论还成立吗?

    \item 设$A=\begin{pmatrix}
                  2 & 2 & 0 \\ 8 & 2 & a \\ 0 & 0 & 6
              \end{pmatrix}$相似于对角矩阵,求常数$a$,并求可逆矩阵$P$使得$P^{-1}AP$为对角矩阵.

    \item 设$A=(a_{ij})_{n\times n}$是上三角矩阵.
          \begin{enumerate}
              \item 求$A$的全部特征值;

              \item 若$A$主对角元互不相等,证明:$A$与对角阵相似;

              \item 若$n$个主对角元相等且$A$不为对角矩阵,证明:$A$不与对角阵相似.
          \end{enumerate}

    \item 已知$\mathbf{R}^3$的一个线性变换
          \[\sigma(x_1,x_2,x_3)=(2x_1-2x_2,-2x_1+x_2-2x_3,-2x_2).\]
          \begin{enumerate}
              \item 求$\sigma$关于自然基$\{e_1,e_2,e_3\}$所对应的矩阵$A$;

              \item 求$\sigma$关于基$\{(1,1,1),(0,1,1),(0,0,1)\}$所对应的矩阵$B$;

              \item 求矩阵$C_1$,使$C_1^{-1}BC_1=A$.
          \end{enumerate}

    \item 设$A=\begin{pmatrix}
                  0 & 0 & 1 \\ 0 & 0 & 0 \\ 1 & 0 & 0
              \end{pmatrix},B=\begin{pmatrix}
                  1 & 0 & 0 \\ 0 & 1 & 2 \\ 0 & -1 & -2
              \end{pmatrix}$,证明:$A\sim B$,并求可逆矩阵$P$使得$P^{-1}AP=B$.

    \item 已知$A=\begin{pmatrix}
                  2 & 0 & 0 \\ 0 & 0 & 1 \\ 0 & 1 & x
              \end{pmatrix}$与$B=\begin{pmatrix}
                  2 & 0 & 0 \\ 0 & y & 0 \\ 0 & 0 & -1
              \end{pmatrix}$相似.
          \begin{enumerate}
              \item 求$x$和$y$;

              \item 求一个可逆矩阵$P$,使$P^{-1}AP$为对角矩阵.
          \end{enumerate}

    \item 设$A=\begin{pmatrix}
                  1 & 2 & 0 & 0  & 0 \\ 4 & 3 & 0 & 0 & 0 \\ 0 & 0 & 1 & -3 & 3 \\ 0 & 0 & 3 & -5 & 3 \\
                  0 & 0 & 6 & -6 & 4
              \end{pmatrix}$,求$A$的特征值. 若$A$可对角化,求可逆矩阵$P$,使$P^{-1}AP$为对角矩阵.

    \item 设三阶矩阵$A$的特征值为$\lambda_1=1,\lambda_2=2,\lambda_3=3$,它们对应的特征向量为$\xi_1=(1,1,1)^\mathrm{T}, \xi_2=(1,2,4)^\mathrm{T},\xi_3=(1,3,9)^\mathrm{T}$,又$\beta=(1,1,3)^\mathrm{T}$,计算$A^n\beta$.

    \item 设$A=\begin{pmatrix}
                  3 & 4 & 0 & 0 \\ 4 & -3 & 0 & 0 \\ 0 & 0 & 2 & 4 \\ 0 & 0 & 0 & 2
              \end{pmatrix}$,求$A^n(n\in\mathbf{N}_+)$.

    \item 已知三阶矩阵$A$和三元列向量$X$,使得向量组$X,AX,A^2X$线性无关,且满足
          \[A^3X=3AX-2A^2X.\]
          \begin{enumerate}
              \item 记$P=(X,AX,A^2X)$,求三阶矩阵$B$使得$A=PBP^{-1}$;

              \item 计算行列式$|A+E|$.
          \end{enumerate}
\end{enumerate}

\centerline{\heiti C组}
\begin{enumerate}
    \item 设$V$是有限维复向量空间,$\sigma\in \mathcal{L}(V)$. 证明:$\sigma$可对角化当且仅当对每个$\lambda\in\mathbf{C}$有$V=\ker(\sigma-\lambda I)\oplus\im(\sigma-\lambda I)$.

    \item 设$B=\alpha\alpha^\mathrm{T}$,其中$\alpha=(a_1,\ldots,a_n)^\mathrm{T}\neq 0\enspace(a_i\in\mathbf{R},\enspace i=1,2,\ldots,n)$.
          \begin{enumerate}
              \item 证明:$B^k=tB$,其中$k$为正整数,$t$为常数,并求$t$;

              \item 求可逆阵$P$使得$P^{-1}BP$为对角矩阵,并写出该对角矩阵.
          \end{enumerate}

    \item (秩为1的矩阵)设$n$阶矩阵$A$的元素均为1.
          \begin{enumerate}
              \item 求$A$的特征值,并求矩阵$P$使得$P^{-1}AP$为对角矩阵;

              \item 若$f(x)$是$x$的$m$次多项式,且常数项为0,证明:存在$k\in\mathbf{R}$使得$f(A)=kA$,并求出$k$;

              \item 设$B$是$n$阶实对称矩阵,每行元素之和都为$b$,若$b$是$f(\lambda)=|\lambda E-B|$的单根,求$B$属于$b$的特征向量;当$f(\lambda)=(\lambda-b)g(\lambda)$时(其中$f(B)=0$),证明:$g(B)=kA$,其中$k$为常数,$A$为元素全部为1的$n$阶矩阵.
          \end{enumerate}

    \item 设$A,B$为$n$阶矩阵,且$A+B=AB$,求证:
          \begin{enumerate}
              \item $A,B$的特征向量是公共的;

              \item $A$相似于对角矩阵当且仅当$B$相似于对角矩阵;

              \item $r(A)=r(B)$.
          \end{enumerate}

    \item 设$A,B\in \mathbf{M}_n(\mathbf{R})$,证明$A$与$B$在$\mathbf{R}$上相似当且仅当在$\mathbf{C}$上相似.

          {\kaishu 注:实际上相似这一性质与数域无关,本题是这一结论的特例.}

    \item \begin{enumerate}
              \item 设$A,B\in \mathbf{M}_n(\mathbf{F})$,$A$有$n$个不同的特征值,证明:$AB=BA$当且仅当$A$的特征向量也是$B$的特征向量;

              \item 若$A,B$均可对角化,且$AB=BA$,则对角化的过渡矩阵可以相同.
          \end{enumerate}

    \item 设$A,B\in \mathbf{M}_n(\mathbf{F})$,$A$有$n$个不同的特征值,且$AB=BA$. 证明:存在次数小于等于$n-1$的多项式$f(x)$使得$B=f(A)$.

    \item 设$T\in \mathcal{L}(V)$,$\lambda\in\mathbf{F}$. 证明:对$V$的每个使得$T$有上三角矩阵的基,$\lambda$出现在$T$的矩阵的对角线上的次数等于$\lambda$作为$T$的特征值的重数.
\end{enumerate}
