\chapter{二次型}

\section{二次型的定义}

\begin{definition}[二次型] \index{ercixing@二次型 (quadratic form)}
    $n$个元$x_1,x_2,\ldots,x_n$的二次齐次多项式
    \begin{align*}
        f(x_1,x_2,\ldots,x_n) & = \sum_{i=1}^{n}a_{ii}x_i^2+\sum\limits_{1\leqslant i<j\leqslant n}2a_{ij}x_ix_j    \\
                              & = a_{11}x_1^2+a_{22}x_2^2+\cdots+a_{nn}x_n^2                                        \\
                              & \quad +2a_{12}x_1x_2+\cdots+2a_{1n}x_1x_n+2a_{23}x_2x_3+\cdots+2a_{n-1,n}x_{n-1}x_n
    \end{align*}
    称为数域$\mathbf{F}$上的$n$元二次型(简称\term{二次型}).
\end{definition}
本学期研究的主要是实二次型. 若令$a_{ij}=a_{ji}\enspace(1\leqslant i<j\leqslant n)$,则二次型可表示为
\[f(x_1,x_2,\ldots,x_n)=\sum_{i=1}^{n}\sum_{j=1}^{n}a_{ij}x_ix_j=X^\mathrm{T}AX\]
其中$X=(x_1,x_2,\ldots,x_n)^\mathrm{T}\in\mathbf{R}^n$,$A=(a_{ij})_{n\times n}$为实对称矩阵,并称对称矩阵$A$为二次型$f(x_1,x_2,\ldots,x_n)$的矩阵.

注意,二次型实际上是一个$\mathbf{R}^n\to\mathbf{R}$的函数,所以本质上代入$x_1,\ldots,x_n$后就是一个实数,写成矩阵形式我们也可以发现矩阵相乘结果为$1\times 1$矩阵,即一个实数,因此不必把二次型想得过于复杂.

同时需要注意,二次型对应矩阵一定是对称矩阵. 实际上一个形如$f(x_1,x_2,\ldots,x_n)=\displaystyle\sum_{i=1}^{n}\displaystyle\sum_{j=1}^{n}a_{ij}x_ix_j$的函数可以对应的矩阵是很多的,但我们要求$a_{ij}=a_{ji}$才能得到二次型对应的矩阵.
\begin{example}
    已知二次型
    \[f(X)=(x_1,x_2,x_3,x_4)\begin{pmatrix}
            1 & 2 & 3 & -4 \\ 3 & 2 & 1 & 4 \\ -4 & 3 & -7 & 2 \\ 0 & -6 & 8 & 4
        \end{pmatrix}\begin{pmatrix}
            x_1 \\ x_2 \\ x_3 \\ x_4
        \end{pmatrix}\]
    写出二次型$f(X)$的矩阵.
\end{example}

\begin{example}
    回答以下问题:
    \begin{enumerate}
        \item 已知$A$是一个$n$阶矩阵,则$A$为反对称矩阵的充要条件是对任意$n$元列向量$X$都有$X^\mathrm{T}AX=0$;

        \item 若二次型$f(x_1,x_2,\ldots,x_n)=X^\mathrm{T}AX$对任意$n$元列向量$X$都有$f(x_1,x_2,\ldots,x_n)=0$,证明:$A=O$;

        \item 设二次型$f(x_1,x_2,\ldots,x_n)=X^\mathrm{T}AX,\enspace g(x_1,x_2,\ldots,x_n)=X^\mathrm{T}BX$.\\
              证明:若$f(x_1,x_2,\ldots,x_n)=g(x_1,x_2,\ldots,x_n)$,则$A=B$.
    \end{enumerate}
\end{example}

\section{矩阵相合的定义与性质}

\begin{definition}
    我们称$n$阶矩阵$A$相合于$B$(记作$A\simeq B$),如果存在可逆矩阵$C$使得$B=C^\mathrm{T}AC$.
\end{definition}
矩阵相合(合同)有如下基本性质:
\begin{enumerate}
    \item 合同是等价关系;合同不同于相似,是与数域有关的;合同要求$C$必须可逆,因此是一种特殊的相抵;

    \item $A\simeq B$一般不能得到$A^m\simeq B^m$(但是$A,B$为实对称矩阵时可以),但如果可逆,我们有$A^{-1}\simeq B^{-1}$,同时如果$A_1\simeq A_2,B_1\simeq B_2$,则有$\begin{pmatrix}
                  A_1 & O \\ O & B_1
              \end{pmatrix}\simeq\begin{pmatrix}
                  A_2 & O \\ O & B_2
              \end{pmatrix}$;

    \item $A\simeq B$表明$A$可以每次做相同的初等行列变换得到$B$,反之亦然. 这实际上就是初等变换法求相合标准形的基本原理,详见教材260页小字部分,感兴趣同学可以了解,一般不会要求使用这一方法.
\end{enumerate}

\begin{example}
    设$A\simeq B$,$C\simeq D$,且它们都是$n$阶实对称矩阵,问:$A+C\simeq B+D$ 是否成立.
\end{example}

\begin{example}
    判断:矩阵相似是否一定合同?矩阵合同是否一定相似?对于实对称矩阵上述论断又是否正确呢?正确请说明理由,不正确请举出反例.
\end{example}
实际上,教材中引入合同与二次型使用了双线性函数这一概念,实际上与双线性函数的度量矩阵有关,感兴趣的同学可以了解,但这部分属于小字,考试一般不做考查要求.

% TODO 小字介绍

% 小字介绍

\section{二次型标准形的定义与求解}

实际上二次型可以视为一个空间曲线/曲面方程,我们希望这些方程化为标准形式,有助于我们讨论一些问题. 由于实二次型对应矩阵为实对称矩阵,实对称矩阵一定可以相似对角化,故有下面的定理:
\begin{theorem}
    任意二次型$f(X)=X^\mathrm{T}AX$总可以通过可逆的线性变换$X=PY$(其中$P$可逆)化为标准形,即$f(X)=X^\mathrm{T}AX\xlongequal{X=PY}Y^\mathrm{T}(P^\mathrm{T}AP)Y=d_1y_1^2+d_2y_2^2+\cdots+d_ny_n^2$.
\end{theorem}
一般而言,我们有三种方法求解二次型标准形,分别为正交变换法,配方法和初等变换法. 正交变换法由于涉及正交因此不作要求,初等变换法之前已经提及并且较为复杂,不推荐优先使用. 因此我们接下来主要使用配方法.

注意,求二次型标准形不应使用之前求相似标准形的一般方法,因为只有正交矩阵才能保证$P^{-1}=P^\mathrm{T}$,一般矩阵无法保证. 当然实际上求得的对角矩阵都是由特征值按重数排列而成的,只是矩阵$P$不合要求,应当做 Schmidt 正交化.

配方法的思想非常简单,就是利用配方消除混合乘积项,将二次型表示成几个平方和的形式,最后通过坐标变换$X=CY$(又称仿射变换,其中$C$可逆)化标准形.
\begin{example}
    用配方法把三元二次型
    \[f(x_1,x_2,x_3)=2x_1^2+3x_2^2+x_3^2+4x_1x_2-4x_1x_3-8x_2x_3\]
    化为标准形,并求所用的坐标变换$X=CY$即变换矩阵$C$.
\end{example}
配方法是合理的,因为$X=CY$,其中$C$可逆,则$X^\mathrm{T}AX=Y^\mathrm{T}(C^\mathrm{T}AC)Y$,配方法使得$C^\mathrm{T}AC$为对角矩阵,因此可以得到相合标准形. 但是这种方法不能用来求相似对角化,原因仍然是$C^{-1}=C^\mathrm{T}$需要$C$为正交矩阵,但坐标变换矩阵不一定满足. 所以一定要区分好求解相似、相合标准形使用的方法,不能因为题目经常给的是实对称矩阵而混淆,只有正交变换法是通用的,因为正交矩阵满足$P^{-1}=P^\mathrm{T}$使得相似、相合的定义统一.

注意:有的同学可能知道正交变换法的具体操作流程,如果能保证计算正确且题目不强制配方法时可以使用,但是历年考试经常出现部分题目求解特征值时三次方程解不出的情况,此时一定要立刻醒悟,转向配方法解决问题.

\section{相合规范形 \quad 惯性定理}

事实上,一个二次型通过正交变换标准化得到的对角矩阵对角线上元素为特征值按重数排列的结果,但是使用配方法、初等变换法则不一定,甚至配方方式或者初等变换顺序不同都会产生不同的对角矩阵,因此相合标准形不唯一. 但我们知道,相抵标准形唯一,相似标准形不考虑排列组合因素也是唯一的,因此我们也需要统一相合标准形.

我们不难发现,任一对角矩阵一定相合于$\diag(1,\ldots,1,-1,\ldots,-1,0,\ldots,0)$(我们很容易写出对应的可逆变换矩阵),我们称这一相合标准形为相合规范形,其中$+1$的个数称为矩阵的正惯性指数,$-1$的个数称为矩阵的负惯性指数. 并且由于变换矩阵可逆,根据相抵标准形的结论,我们有原矩阵$A$的秩$r(A)$等于这一对角矩阵的秩,于是也等于正负惯性指数之和. 显然,$A$可逆时,其相合规范形主对角元没有0.

但我们没有说明一个矩阵的相合规范形是否唯一,实际上这就是下面惯性定理的结果:
\begin{theorem}[惯性定理]
    实对称矩阵的相合规范形唯一.
\end{theorem}
这一定理有很多等价表述,例如实对称矩阵正、负惯性指数唯一,或者实对称矩阵相合标准形中对角线上正、负、零的个数唯一. 或者实对称矩阵特征值中正、负、零的个数唯一等. 这一定理的证明方法比较经典,最关键的一步在于代入数值导出矛盾. 代入的方法是在两种表达的正负号分界线前后分别置0,使得两种表达形式一个大于0,一个小于等于0.
\begin{example}
    解答如下问题:
    \begin{enumerate}
        \item 设$n$元二次型$f(x_1,x_2,\ldots,x_n)=l_1^2+\cdots+l_p^2-l_{p+1}^2-\cdots-l_{p+q}^2$,其中$l_i\enspace (i=1,2,\ldots,p+q)$是关于$x_1,x_2,\ldots,x_n$的一次齐次式. 证明:$f(x_1,x_2,\ldots,x_n)$的正惯性指数$\leqslant p$,负惯性指数$\leqslant q$;

        \item 已知$A$为$m$阶实对称矩阵,$C$为$m\times n$实矩阵,证明:$C^\mathrm{T}AC$的正负惯性指数分别小于等于$A$的正负惯性指数.
    \end{enumerate}
\end{example}

\begin{example}
    确定二次型$f(x_1,x_2,\ldots,x_{10})=x_1x_2+x_3x_4+x_5x_6+x_7x_8+x_9x_{10}$的秩以及正、负惯性指数.
\end{example}

惯性定理的``惯性''二字与物理中的惯性有关,实际上透露着某种不变性. 根据惯性定理,我们有如下结论:
\begin{enumerate}
    \item 我们可以按相合关系对全体$n$阶实对称矩阵分类,因为实对称矩阵相合意味着规范形唯一,我们可以按照$+1$、$-1$、0个数的不同划分为$\vphantom{\cfrac{n+1}{2}}\dfrac{(n+1)(n+2)}{2}$个等价类(相抵、相似也是等价关系,可以思考划分等价类的方式与个数);

    \item 实数域上两个实对称矩阵相合的充要条件是它们有相同的正负惯性指数,两个对角矩阵相合的充要条件是对角线上正、负、零个数相同.
\end{enumerate}
注:复数域上两个对称矩阵相合的充要条件是它们的秩相同(可以思考其证明),例如$E_n$和$-E_n$在复数域上相合,但实数域上不相合.
\begin{example}
    设$A=\begin{pmatrix}
            1 & 2 & 0 \\ 2 & 1 & 0 \\ 0 & 0 & 3
        \end{pmatrix},\enspace B=\begin{pmatrix}
            -2 & 0 & 0 \\ 0 & 2 & 1 \\ 0 & 1 & 2
        \end{pmatrix}$,判断$A$与$B$是否相合.
\end{example}

\section{标准形的应用}

我们在本学期讨论了三种标准形,即相抵标准形,相似标准形和相合标准形,实际上它们之间的关系我们已经讨论,即相似一定相抵,相合一定相抵,但相似和相合互相没有包含关系. 本节我们考虑一些基于矩阵分解的问题,利用之前所学的相抵标准形、相似标准形、相合标准形的分解解决一些问题. 本节内容可以选择性掌握.

首先看一个关于幂等矩阵的例题,需要用到相抵标准形、相似标准形的分解:
\begin{example}
    解答以下两个问题:
    \begin{enumerate}
        \item 证明:任意一个方阵都可以分解成一个可逆矩阵和一个幂等矩阵的乘积;

        \item 已知$A$是一个秩为$r$的$n$级非零矩阵,证明:$A$为幂等矩阵的充要条件是存在列满秩的$n\times r$矩阵$B$和行满秩的$r\times n$矩阵$C$使得$A=BC$且$CB=E_r$.
    \end{enumerate}
\end{example}
下面是一个利用相合标准形进行分解的例子:
\begin{example}
    (与正交有关)证明:每个秩为$r$的$n\enspace(r<n)$阶实对称矩阵均可表示为$n-r$个秩为$n-1$的实对称矩阵的乘积.
\end{example}

\vspace{2ex}
\centerline{\heiti \Large 内容总结}

\vspace{2ex}
\centerline{\heiti \Large 习题}

\vspace{2ex}
{\kaishu }
\begin{flushright}
    \kaishu

\end{flushright}

\centerline{\heiti A组}
\begin{enumerate}
    \item
\end{enumerate}

\centerline{\heiti B组}
\begin{enumerate}
    \item
\end{enumerate}

\centerline{\heiti C组}
\begin{enumerate}
    \item
\end{enumerate}
