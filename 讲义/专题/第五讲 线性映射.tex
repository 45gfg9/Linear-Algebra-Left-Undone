\chapter{线性映射}

在前几讲的学习中,我们从开始的令八条运算性质出发,利用这些线性运算
的特点导出线性扩张与子空间的关联,然后经过线性相关性的讨论最终得到线性空间的本质结构实际上就是可以由基经过一系列线性
运算扩张而来,因此我们对线性空间的研究很多时候只需要研究其基和维数即可,由此我们对线性空间的研究和描述
就可以转为研究基和维数,——这是线性空间的基本结构属性.当然我们最后也讨论了线性空间之间的运算.从本讲开始我们将
研究不同线性空间之间的关联,我们的手段是定义两个线性空间之间的线性映射,由此发掘出比较不同线性空间之间最本质的
差别是什么,使我们的抽象更深一层,然后在抽象的制高点将抽象转化为具象,讨论矩阵这一对线性映射的``有形''描述和
线性映射本身的联系,为后文详细讨论矩阵作铺垫.

\section{线性映射的定义}
\subsection{线性映射的定义}
\begin{definition}
    从线性空间$V_1(\mathbf{F})$到$V_2(\mathbf{F})$的一个映射$\sigma$是线性的,
    如果$\forall \alpha,\beta \in V_1$和$\forall \lambda,\mu \in \mathbf{F}$都有
    \begin{equation}\label{eq:5:线性映射}
        \sigma(\lambda\alpha+\mu\beta)=\lambda\sigma(\alpha)+\mu\sigma(\beta).
    \end{equation}

    从线性空间$V$到自身的线性映射$\sigma$也叫作$V$上的\keyterm{线性变换}[linear transformation],
    从线性空间$V(\mathbf{F})$到域$\mathbf{F}$的线性映射$f$叫作$V$上的线性泛函(或称线性函数,线性形式).

    为方便称呼,我们称对于$V_1(\mathbf{F})$到$V_2(\mathbf{F})$的线性映射$\sigma$,$V_1(\mathbf{F})$
    是其出发空间,$V_2(\mathbf{F})$是其到达空间,也可简记为$\sigma: V_1\to V_2$.
\end{definition}
实际上,上述定义式\eqref{eq:5:线性映射}可以分拆为以下二式:
\begin{equation} \tag{加性} % TODO tag 左对齐
    \sigma(\alpha+\beta)=\sigma(\alpha)+\sigma(\beta)
\end{equation}
\begin{equation} \tag{齐次性}
    \sigma(\lambda\alpha)=\lambda\sigma(\alpha)
\end{equation}

根据定义,我们容易知道熟悉的过原点的一次函数是线性映射,而不过原点的一次函数不代表线性映射.
这似乎与平常的称呼不同,因为一次函数我们经常都称它们为``线性的'',这里我们必须强调的是,至少在线性代数的框架下,
我们研究的``线性''性质是包含加性和齐次性两条要求的,事实上不过原点的一次函数我们可以视作非齐次线性方程,
这里的``非齐次''的含义便很清晰了.

另一方面,如果我们不将一次函数视为映射,而将视为平面点集(不过原点的一条直线),我们可以回顾线性子空间中的描述,
我们说过原点的直线构成平面的线性子空间,但不过原点的直线不是,我们判断的依据是不过原点的直线内的两点关于
加法和数乘不封闭——仔细一想,是不是与线性映射定义中不满足加性、齐次性是同样的道理呢?

事实上,``线性性''在数学中是一个非常基本的性质,下一小节的例子中我们将认识到这一点.

本小节最后我们讨论线性映射的两个重要的性质:
\begin{theorem}\label{thm:5:线性映射零元性质}
    设$\sigma$是线性空间$V_1(\mathbf{F})$到$V_2(\mathbf{F})$的线性映射,则
    $\sigma(0_1)=0_2$.
\end{theorem}
注意定理中$0_1$为出发空间$V_1$中的零元,$0_2$代表到达空间$V_2$中的零元.这只是为了区分两个空间零元而引入的记号,
实际上下标也可以省略,直接写为0也可以.

\begin{proof}
    根据线性性,$\sigma(0+0)=\sigma(0)+\sigma(0)$,两边同时减去$\sigma(0)$可知$\sigma(0)=0$.
\end{proof}

\begin{theorem}\label{thm:5:线性映射保相关性}
    设$\sigma$是线性空间$V_1(\mathbf{F})$到$V_2(\mathbf{F})$的线性映射,如果$V_1$中向量
    $\alpha_1,\alpha_2,\cdots,\alpha_n$线性相关,则
    $\sigma(\alpha_1),\sigma(\alpha_2),\cdots,\sigma(\alpha_n)$也线性相关.

    反之,若$\sigma(\alpha_1),\sigma(\alpha_2),\cdots,\sigma(\alpha_n)$线性无关,则
    $\alpha_1,\alpha_2,\cdots,\alpha_n$必线性无关.
\end{theorem}
这一性质表明线性映射保持线性相关性.定理中两个描述互为逆否命题,因此我们可以只证明前者.

\begin{proof}
    设$\alpha_1,\alpha_2,\cdots,\alpha_n$线性相关,则存在不全为0的数$\lambda_1,\lambda_2,\cdots,\lambda_n$使得
    $\lambda_1\alpha_1+\lambda_2\alpha_2+\cdots+\lambda_n\alpha_n=0$,于是
    \[\sigma(\lambda_1\alpha_1+\lambda_2\alpha_2+\cdots+\lambda_n\alpha_n)=\lambda_1\sigma(\alpha_1)+\lambda_2\sigma(\alpha_2)+\cdots+\lambda_n\sigma(\alpha_n)=0\]
    因此存在不全为零的数$\lambda_1,\lambda_2,\cdots,\lambda_n$使得$\lambda_1\sigma(\alpha_1)+\lambda_2\sigma(\alpha_2)+\cdots+\lambda_n\sigma(\alpha_n)=0$,
    因此$\sigma(\alpha_1),\sigma(\alpha_2),\cdots,\sigma(\alpha_n)$线性相关.
\end{proof}

需要注意的是,线性映射可能将线性无关的向量组映射为线性相关的向量组,例如
\begin{example}
    设$\sigma$是线性空间$\mathbf{R}^2$到$\mathbf{R}^2$的线性映射,定义$\sigma(x,y)=(x+y,x+y)$,
    则$\sigma$将线性无关的向量组$(1,0),(0,1)$映射为线性相关的向量组$(1,1),(1,1)$.
\end{example}

\subsection{线性映射举例}
我们首先来看以下几个在数学中非常基本的概念,它们都是线性映射的例子:
\begin{example}
    数学分析与概率论中的线性映射:
    \begin{enumerate}
        \item (极限)$\lim\limits_{n\to +\infty}(\lambda a_n+\mu b_n)=\lambda\lim\limits_{n\to +\infty}a_n+\mu\lim\limits_{n\to +\infty}b_n$;
        \item (求导)$(\lambda f(x)+\mu g(x))'=\lambda f'(x)+\mu g'(x)$;
        \item (积分)$\int_a^b(\lambda f(x)+\mu g(x))\mathrm{d}x=\lambda\int_a^bf(x)\mathrm{d}x+\mu\int_a^bg(x)\mathrm{d}x$;
        \item (数学期望)$E(\lambda X+\mu Y)=\lambda E(X)+\mu E(Y)$.
    \end{enumerate}
\end{example}

有的读者可能会有另外的疑惑:上面的例子为什么能称其为线性映射?它们是从线性空间到线性空间的映射吗?事实上,上面的例子中
到达空间都是数域——这是符合定义的,出发空间对于极限而言就是任意有极限的数列构成的线性空间,对于求导、求积分而言就是任意可导、可积的函数
构成的线性空间,对于数学期望而言就是期望存在的随机变量构成的线性空间.读者可以自行验证这些的确构成线性空间,此处不再赘述.

相信看到这里,我们便能逐渐理解线性性在数学中的基础地位.很多时候一些初看有些抽象的概念,当我们将其与学过的知识联系时,
便会真切地体会到一种相通的美感.事实上很多时候学习过程就是如此,当我们知识储量不断上升的时候,我们会不断发现很多宝贵的思想
跨越学科,凝聚着人类智慧的结晶,这种感觉是非常美妙的.更重要的是,一旦我们知道它们是线性映射之后,我们便可以用之后我们
将要讨论的所有线性映射相关的性质来研究它们,这便是一个抽象的概念给我们带来的力量.

接下来希望读者阅读教材3.1节例1-9了解其它常见的线性映射,特别是其中的几何意义(虽然不会直接考察,但是对理解有帮助).
其中例1、7、8、9希望读者当做练习,熟悉线性映射定义的验证,这在考试中也是常见的.例4、5中的放缩与错切是常见的几何变换,例2中旋转变换在之后有很多的
应用场景,其几何意义能帮助我们理解很多内容.例3镜面变换在内积空间中会详细介绍,例6投影变换将在幂等矩阵中我们会再次提及.
\begin{example}
    写出下列映射的出发空间和到达空间,并判断其是否为线性映射:
    \begin{enumerate}[label=(\arabic*)]
        \item $\sigma(x_1,x_2)=(x_1-x_2,x_1,x_1+x_2)$;
        \item $\sigma(x_1,x_2)=(x_1x_2,x_1+x_2)$;
        \item $\sigma(p(x))=p(x+1)-p(x),\ \forall p(x) \in \mathbf{R}[x]_n$;
        \item $\sigma(p(x))=p(a),\ \forall p(x)$,其中$a$为常数;
        \item $\sigma(\xi)=2\xi+\xi_0$,其中$\xi$是线性空间$V$中的一个固定向量.
    \end{enumerate}
\end{example}

\subsection{线性映射的基本运算}
我们在之前的学习中已经了解,连续函数关于函数的加法数乘运算可以构成线性空间,事实上线性映射可以视为特殊的函数,因此我们希望在本节讨论
怎样的运算定义能使其构成线性空间,除此之外也介绍线性映射乘法(即复合)和逆运算.

我们需要首先说明一个记号,我们把线性空间$V_1(\mathbf{F})$到$V_2(\mathbf{F})$的所有线性映射组成的集合记作$\mathcal{L}(V_1,V_2)$
(类似于将定义在$[a,b]$上取值于实数集的连续函数全体记为$C[a,b]$).
我们希望在该集合上定义线性空间,于是需要定义其中元素(线性映射)的加法和数乘运算:
\begin{definition}
    设$\sigma,\tau\in \mathcal{L}(V_1,V_2)$,规定$\sigma$与$\tau$之和及$\lambda$与
    $\sigma$的数乘$\lambda\sigma$分别为
    \begin{gather*}
        (\sigma+\tau)(\alpha)=\sigma(\alpha)+\tau(\alpha),\enspace\forall\alpha\in V_1 \\
        (\lambda\sigma)(\alpha)=\lambda(\sigma(\alpha)),\enspace\forall\alpha\in V_1
    \end{gather*}
\end{definition}
\begin{theorem}
    $\mathcal{L}(V_1,V_2)$与上述定义的线性映射加法和数乘构成域$\mathbf{F}$上的线性空间.
\end{theorem}
下面讨论线性映射的复合.设$\sigma \in \mathcal{L}(V_1,V_2),\enspace\tau \in \mathcal{L}(V_2,V_3)$,则$\tau\sigma$是$\mathcal{L}(V_1,V_3)$
中的元素,且$\tau\sigma(\alpha)=\tau(\sigma(\alpha)),\enspace\forall \alpha \in V_1$.
\begin{theorem}
    上述定义的映射$\tau\sigma$是线性映射.
\end{theorem}
注意:在上述定义中一定注意$\sigma$和$\tau$的顺序,我们需要先使用$\sigma$将$V_1$中的元素映射到
$V_2$,然后再用外层的$\tau$映射到$V_3$.

下面定义逆映射.如果可逆映射$\sigma:V_1 \to V_2$的逆映射为$\sigma^{-1}$,则$\sigma^{-1}\sigma=I_{V_1}$且
$\sigma\sigma^{-1}=I_{V_2}$.其中$I_{V}$的含义为$V$上的恒等映射,即$I_V(\alpha)=\alpha,\enspace \forall \alpha \in V$.
\begin{theorem}
    上述定义的逆映射$\sigma^{-1}$为线性映射.
\end{theorem}

上述三个定理的证明是非常基本的,详细的证明在教材103-104页,读者可以先自行尝试,如果不会证明
则说明对于线性空间和线性映射的定义熟悉程度仍需提高,因为这里的证明都只需要机械地套用定义.

\section{线性映射的像与核}
我们在之前的讨论中已经了解了线性映射的定义与运算,接下来我们需要关心的问题是:定义出的线性映射能将出发空间完整映射到到达空间吗,
还是到达空间中有些向量无法被映到?线性映射是否可以是单射?单射的充要条件又是什么?这与我们研究一般的映射的思路是类似的.
因此我们希望在本节讨论线性映射的像和核.
\begin{definition}
    设$\sigma$是线性空间$V_1(\mathbf{F})$到$V_2(\mathbf{F})$的线性映射. $V_1$的所有元素
    在$\sigma$下的像组成的集合
    \[\sigma(V_1)=\{\beta \mid \beta=\sigma(\alpha),\enspace \alpha \in V_1\}\]
    称为$\sigma$的\keyterm{像}[image](或\keyterm{值域}[range]),记作$\im \sigma$,或记作 $\operatorname{range} \sigma$.

    $V_2$的零元$0_2$在$\sigma$下的完全原像
    \[\sigma^{-1}(0_2)=\{\alpha \mid \sigma(\alpha)=0_2,\enspace \alpha \in V_!\}\]
    称为$\sigma$的\keyterm{核}[kernel](或\keyterm{零空间}[null space]),记作$\ker \sigma$,或记作 $\operatorname{null} \sigma$.
\end{definition}

关于像与核的定义,我们需要强调以下几点:
\begin{enumerate}
    \item 实际上,像空间的定义就类似于函数的值域,核空间可以视为到达空间中0的原像集合,因此理解起来是很简单的;
    \item 注意线性映射的像和核分别是$V_2$和$V_1$的子空间.同样地,若$W_1$和$W_2$分别是$V_1$和$V_2$的
    子空间,则$\sigma(W_1)$和$\sigma^{-1}(W_2)$也分别是$V_2$和$V_1$的子空间.前者证明见教材102页,后者我们作为习题留给读者,
    实际上都非常简单,只是为读者熟悉定义而在此处提及.
\end{enumerate}

% TODO 符号体系说明

接下来我们要讨论如何计算线性映射的像与核,这在考试中非常常见,请务必牢记,无论线性映射有多么复杂多么抽象,基本的方法都是:
\begin{enumerate}
    \item 设出发空间的一组基为$B=\{\alpha_1,\alpha_2,\cdots,\alpha_n\}$,则像空间
    \[\im \sigma=\sigma(V_1)=\spa(\sigma(\alpha_1),\sigma(\alpha_2),\cdots,\sigma(\alpha_n)).\]
    即线性映射在出发空间一组基下的像的线性扩张,解答时写出极大线性无关组然后扩张即可;

    当然读者可能质疑其合理性,因为与定义不完全一致.我们可以证明这一方法是合理的,即线性映射在出发空间一组基下像的线性扩张就是
    其像空间.
    
    \begin{proof}
        首先我们知道$\sigma(V_1)$包含$\sigma(\alpha_1),\sigma(\alpha_2),\cdots,\sigma(\alpha_n)$,并且是$V_2$的子空间,
        又根据\autoref{thm:2:线性扩张构造子空间},$\sigma(V_1)$是包含$\sigma(\alpha_1),\sigma(\alpha_2),\cdots,\sigma(\alpha_n)$
        的最小子空间,因此我们可以得到$\spa(\sigma(\alpha_1),\sigma(\alpha_2),\cdots,\sigma(\alpha_n))\subset\sigma(V_1)$.

        接下来证明另一半包含,根据线性扩张定义可知只需证$V_1$中任意元素的像都可以被$\sigma(\alpha_1),\sigma(\alpha_2),\cdots,\sigma(\alpha_n)$线性表示.
        任取$\alpha\in V_1$,则$\alpha$可由$V_1$一组基$\{\alpha_1,\alpha_2,\cdots,\alpha_n\}$线性表示为
        $\alpha=\lambda_1\alpha_1+\lambda_2\alpha_2+\cdots+\lambda_n\alpha_n$,于是,
        \[\sigma(\alpha)=\sigma(\lambda_1\alpha_1+\lambda_2\alpha_2+\cdots+\lambda_n\alpha_n)=\lambda_1\sigma(\alpha_1)+\lambda_2\sigma(\alpha_2)+\cdots+\lambda_n\sigma(\alpha_n)\]
        即$\sigma(\alpha)$可由$\sigma(\alpha_1),\sigma(\alpha_2),\cdots,\sigma(\alpha_n)$线性表示,
        即出发空间任意向量在$\sigma$下的像都可以由$\sigma(\alpha_1),\sigma(\alpha_2),\cdots,\sigma(\alpha_n)$线性表示,
        因此$\sigma(V_1)\subset\spa(\sigma(\alpha_1),\sigma(\alpha_2),\cdots,\sigma(\alpha_n))$.
    \end{proof}

    \item 核空间可以直接利用定义令$\sigma(\alpha)=0$,利用解线性方程组得到解集即为结果,注意也许表示为线性扩张的形式.
\end{enumerate}
\begin{example}
    已知$\mathbf{R}^3$到$\mathbf{R}^2$的映射$\sigma$为$\sigma(x_1,x_2,x_3)=(x_1+x_2,x_2-x_3)$,
    求$\sigma$的像和核.
\end{example}
\begin{solution}
\begin{enumerate}
    \item 首先求像空间,取出发空间$\mathbf{R}^3$的一组基$B=\{(1,0,0),(0,1,0),(0,0,1)\}$,则
    $\im \sigma=\sigma(\mathbf{R}^3)=\spa(\sigma(1,0,0),\sigma(0,1,0),\sigma(0,0,1))=\spa((1,0),(1,1),(0,-1))$.
    根据求解极大线性无关组的方法(或者这么简单的情况瞪眼法也可以)得到像空间$\im \sigma=\spa((1,0),(0,-1))=\mathbf{R}^2$.
    \item 接下来求解核空间,设$\sigma(\alpha)=0$,其中$\alpha=(x_1,x_2,x_3)$,即$\sigma(x_1,x_2,x_3)=(x_1+x_2,x_2-x_3)=(0,0)$,
    解得解向量为$k(-1,1,1)(k\in\mathbf{R})$,写成线性扩张的形式为$\spa(-1,1,1)$.
\end{enumerate}
\end{solution}

事实上教材102页例3给出了另一种求像空间的方法,但是为了防止读者混淆这一方法和之后线性映射矩阵表示的方法,
希望读者能按照笔者介绍的方法求解.

事实上,研究一般映射我们会很在意映射是否为单射或满射.是否为满射通过我们介绍的求解像空间的方法是很好判断的,但单射似乎并不能直接利用像空间
或者核空间直接判断,但我们只需稍作转化就可以发现单射和核空间有着密不可分的联系:
\begin{theorem}
    线性映射$\sigma$是单射当且仅当$\ker \sigma=\{0\}$.
\end{theorem}
这个定理告诉我们,线性映射是单射和0的逆象只有0是等价的.这一结论也是非常强的,因为我们知道一般的函数是不满足这一结论的,例如$f(x)=x^2$,
虽然只有$f(0)=0$,但在$\mathbf{R}$上显然不是单射.这一定理证明非常简单,希望读者掌握:

\begin{proof}
    首先我们证明$\sigma$是单射时$\ker\sigma=\{0\}$.事实上$\sigma$是单射意味着任意到达空间中的元素的逆象唯一,又线性映射必须满足$\sigma(0)=0$,
    则0的逆象唯一为0是显然的.

    接下来我们证明$\ker \sigma=\{0\}$时$\sigma$是单射.事实上,$\sigma(\alpha)=\sigma(\beta)$
    等价于$\sigma(\alpha)-\sigma(\beta)=0$,即$\sigma(\alpha-\beta)=0$,由于$\ker \sigma=\{0\}$,因此$\alpha-\beta=0$,
    即$\alpha=\beta$,因此$\sigma$是单射.
\end{proof}

\section{线性映射的确定}
我们知道,两个函数相等当且仅当它们的定义域相等且对于任意定义域内的元素,它们的函数值相等.线性映射则有更好的性质,
即有限维空间上的线性映射可以被基上的像唯一确定,即
\begin{theorem}\label{thm:5:线性映射唯一确定}
    已知线性映射$\sigma,\tau\in \mathcal{L}(V_1,V_2)$,且有$V_1$的基$B=\{\alpha_1,\alpha_2,\ldots,\alpha_n\}$.
    若$\sigma(\alpha_i)=\tau(\alpha_i),\enspace\forall \alpha_i \in B$,则有$\sigma=\tau$.
\end{theorem}
即映射在一组基上的像确定了,则映射是唯一的.证明非常简单:

\begin{proof}
    实际上我们只需证明$\sigma(\alpha)=\tau(\alpha),\enspace\forall \alpha \in V_1$即可,因为这就是一般映射相等的条件.
    事实上,任取$\alpha \in V_1$,则$\alpha$可由$B$线性表示为$\alpha=\lambda_1\alpha_1+\lambda_2\alpha_2+\ldots+\lambda_n\alpha_n$,
    于是
    \begin{gather*}
        \sigma(\alpha)=\sigma(\lambda_1\alpha_1+\lambda_2\alpha_2+\ldots+\lambda_n\alpha_n)=\lambda_1\sigma(\alpha_1)+\lambda_2\sigma(\alpha_2)+\ldots+\lambda_n\sigma(\alpha_n) \\
        \tau(\alpha)=\tau(\lambda_1\alpha_1+\lambda_2\alpha_2+\ldots+\lambda_n\alpha_n)=\lambda_1\tau(\alpha_1)+\lambda_2\tau(\alpha_2)+\ldots+\lambda_n\tau(\alpha_n)
    \end{gather*}
    由于$\sigma(\alpha_i)=\tau(\alpha_i),\enspace\forall \alpha_i \in B$,因此$\sigma(\alpha)=\tau(\alpha)$,即$\sigma=\tau$.
\end{proof}

事实上这与之前证明求解像空间的方法合理性(即证线性映射在一组基上的像的线性扩张就是线性映射的像空间)是完全相通的.
这里也蕴含了一个数学的基本想法.我们发现线性映射比一般的函数要求更高,因为它要求了两个运算性质,我们说这里构成线性映射的条件
比构成一般函数的条件``更强''.更强的要求必然带来更美妙的结果,例如线性映射可被基上的像唯一确定,而一般函数则不存在这样的性质.

\begin{theorem}\label{thm:5:线性映射构造}
    设$B=\{\alpha_1,\alpha_2,\ldots,\alpha_n\}$是$V_1$的基,$S=\{\beta_1,\beta_2,\ldots,\beta_n\}$
    是$V_2$中任意$n$个向量,则存在唯一的$\sigma\in \mathcal{L}(V_1,V_2)$使得$\sigma(\alpha_i)=\beta_i,\enspace i=1,2,\ldots,n$.
\end{theorem}
这一定理即教材107-108页定理3.6,证明也是简单的,只需先定义出这个映射,$\forall \alpha \in V_1$,则$\alpha$可由$B$线性表示为
$\alpha=\lambda_1\alpha_1+\lambda_2\alpha_2+\ldots+\lambda_n\alpha_n$,于是定义
\[\sigma(\alpha)=\sigma(\lambda_1\alpha_1+\lambda_2\alpha_2+\ldots+\lambda_n\alpha_n)=\lambda_1\beta_1+\lambda_2\beta_2+\ldots+\lambda_n\beta_n\]
即可满足条件,因为我们可以验证这是线性映射(见教材108页),并且唯一性在\autoref{thm:5:线性映射唯一确定}中已经说明.
实际上对于初学而言难点在于定义,实际上证明后会发现这一定义太自然了,就是向着线性性定义的,因此很多构造不需要太复杂的想法,自然的、满足要求的是最好的.

最后我们讨论一个初学时容易困惑的问题,如下例所示:
\begin{example}\label{ex:5:线性映射判断1}
    是否存在$\mathbf{R}^2$到$\mathbf{R}^3$的线性映射$\sigma$使得$\sigma(1,0)=(1,0,0)$,
	$\sigma(0,1)=(0,1,0)$,$\sigma(1,1)=(0,0,1)$.
\end{example}

初学时感到困难是因为不能熟练应用线性映射的各类性质,找不到映射定义也不敢下结论不存在,或者发现必要条件都满足了却不敢构造.
我们这里给出几个解决策略:
\begin{enumerate}
    \item 根据\autoref{thm:5:线性映射零元性质}可知,如果我们发现题目给定的条件无法满足将出发空间零元映射至到达空间零元则一定不是线性映射;
    \item 根据\autoref{thm:5:线性映射保相关性}可知,如果我们发现映射将线性相关的向量组映射到了线性无关向量组,则一定不是线性映射;
    \item 一定不存在从低维线性空间到高维线性空间的满射,原因是简单的:我们取低维出发空间的一组基$B_\{\alpha_1,\alpha_2,\ldots,\alpha_m\}$,
    则它们的像的线性扩张$\spa(\sigma(\alpha_1),\sigma(\alpha_2),\ldots,\sigma(\alpha_m))$就是像空间.我们取高维到达空间的一组基
    $B_2=\{\beta_1,\beta_2,\ldots,\beta_n\}$,则由于维数更高有$n>m$.由于$\sigma$是满射,因此
    $\sigma(\alpha_1),\sigma(\alpha_2),\ldots,\sigma(\alpha_m)$可以线性表示出$B_2$中任意向量,
    根据\autoref{thm:3:线性表示}可知(这是长的向量可以被短的向量线性表示),向量组$B_2$线性相关,但我们知道这是一组基,因此矛盾!

    这一性质在下一讲介绍了线性映射基本定理后会有更简单的证明,但此处的证明也是很重要的,体现了\autoref{thm:3:线性表示}作为源泉定理的重要性.

    \item 如果题目给定的映射不违反上述线性映射的必要条件,那我们可以按照\autoref{thm:5:线性映射构造}构造出相应的映射.
\end{enumerate}

根据上面的描述,我们发现\autoref{ex:5:线性映射判断1}中的映射定义违反了明显违反了不能是从低维到高维满射的条件.实际上,$\sigma$
也将线性相关的向量组映射到了线性无关的向量组,并且根据定义,$\sigma(0)=\sigma((1,0)+(0,1)-(1,1))=(1,1,-1)$,因此所有的必要条件
都被违反了,因此这一映射一定不是线性映射.事实上这一例子也表明很多时候三个必要条件可能是同时违反的,因为它们都是由基本的线性映射和线性相关性质
推导而来,并非完全独立的判据.

我们还需强调的是,如果\autoref{thm:5:线性映射构造}前提成立,即题目给我们的是一组基下的像,则一定不会违反上述三个必要条件.对于条件1,
给定一组基$\alpha_1,\ldots,\alpha_n$,我们要凑出$\sigma(0)$只能通过$\sigma(0)=\sigma(0\alpha_1+\ldots+0\alpha_n)=0\sigma(\alpha_1)+\ldots+0\sigma(\alpha_n)=0$,
因此不可能违反第一条.对于2,我们给定的是基,因此不存在将线性相关向量组映射到线性无关向量组的情况.对于3,如果题目给出的是低维到高维的映射,
由于我们只给出了低维出发空间的基下的像,这些像不可能张成整个高维到达空间(原理和$n-1$个向量无法张成$n$维空间一致),因此也不可能违反3,
因此\autoref{thm:5:线性映射构造}并不与我们的必要条件相矛盾,相反,如果题目给出的是一组基下的像,我们就可以毫无顾虑地说映射一定存在.

最后我们再看一个例子来练习我们上面的策略:
\begin{example}\label{ex:5:线性映射判断2}
    是否存在$\mathbf{R}^3$到$\mathbf{R}^2$的线性映射$\sigma$使得$\sigma(1,-1,1)=(1,0)$,
	$\sigma(1,1,1)=(0,1)$.
\end{example}
\begin{solution}
    事实上这里的定义完全不违反上述的必要条件,因此我们考虑证明这一线性映射存在.根据\autoref{thm:5:线性映射构造},我们考虑
    构造出$\sigma$在一组基(我们取自然基)下的像,然后我们就可以根据\autoref{thm:5:线性映射构造}知道这一映射一定存在.

    实际上,根据提给条件我们有
    \begin{gather*}
        \sigma(1,-1,1)=\sigma(e_1-e_2+e_3)=\sigma(e_1)-\sigma(e_2)+\sigma(e_3)=(1,0) \\
        \sigma(1,1,1)=\sigma(e_1+e_2+e_3)=\sigma(e_1)+\sigma(e_2)+\sigma(e_3)=(0,1)
    \end{gather*}
    我们希望解出$\sigma(e_1),\sigma(e_2),\sigma(e_3)$,这样就可以直接根据\autoref{thm:5:线性映射构造}构造出这一线性映射.
    但这一方程组只有2个方程却有三个未知量.事实上我们可以任意定义$\sigma(e_3)=(0,0)$,然后解方程组得到
    \[\sigma(e_1)=\cfrac{1}{2}(1,1),\enspace \sigma(e_2)=\cfrac{1}{2}(-1,1)\]
    又有$\sigma(e_3)=(0,0)$,根据\autoref{thm:5:线性映射构造},满足题目条件的线性映射存在.
\end{solution}
需要注意的是题目中$\sigma(e_3)$不一定要定义为$(0,0)$,这样只是为了计算方便,事实上定义成任何值都可以得到$\sigma$在
一组基下的像,从而根据\autoref{thm:5:线性映射构造}得到这一线性映射存在.如果题目要求我们写出映射也并不复杂,根据我们在
\autoref{thm:5:线性映射构造}中的构造方法,我们可以写出$\forall\alpha=(x,y,z)=xe_1+ye_2+ze_3$,
\[\sigma(\alpha)=\sigma(xe_1+ye_2+ze_3)=x\sigma(e_1)+y\sigma(e_2)+z\sigma(e_3)=\cfrac{1}{2}(x-y,x+y)\]
符合题目条件.事实上根据$\sigma(e_3)$定义的不唯一,我们可以得到不同的线性映射,这里只是给出一种可能的解.

\section{线性映射的矩阵表示}
\subsection{一个最基本的定义}
\begin{definition}
    设$B_1=\{\varepsilon_1,\varepsilon_2,\ldots,\varepsilon_n\}$是$V_1(F)$的基,$B_2=\{\alpha_1,\alpha_2,\cdots,\alpha_m\}$是$V_2(F)$的基.
    则线性映射$\sigma \in \mathcal{L}(V_1,V_2)$被它作用于基$B_1$的像
    \[\sigma(B_1)=\{\sigma(\varepsilon_1),\sigma(\varepsilon_2),\ldots,\sigma(\varepsilon_n)\}\]
    所唯一确定,而$\sigma(B_1)$是$V_2$的一个子集,于是
    \[ \left\{
     \begin{array}{rcl}
        \sigma(\varepsilon_1)&=&a_{11}\alpha_1+a_{21}\alpha_2+\ldots+a_{m1}\alpha_m \\
        \sigma(\varepsilon_2)&=&a_{12}\alpha_1+a_{22}\alpha_2+\ldots+a_{m2}\alpha_m \\
        &\vdots& \\
        \sigma(\varepsilon_n)&=&a_{1n}\alpha_1+a_{2n}\alpha_2+\ldots+a_{mn}\alpha_m
     \end{array}
    \right. \]

    我们将$\sigma(B_1)=\{\sigma(\varepsilon_1),\sigma(\varepsilon_2),\ldots,\sigma(\varepsilon_n)\}$
    关于基$B_2$的坐标排列成矩阵$\mathbf{M}(\sigma)$,即
    \[\mathbf{M}(\sigma)=\begin{pmatrix}
        a_{11} & a_{12} & \cdots & a_{1n} \\
        a_{21} & a_{22} & \cdots & a_{2n} \\
        \vdots & \vdots & \ddots & \vdots \\
        a_{m1} & a_{m2} & \cdots & a_{mn}
    \end{pmatrix}\]
\end{definition}
更通俗来说,线性映射矩阵表示就是将线性映射在一组基上的像在另一组基下的坐标按列排列的结果.

之后我们会经常看见两种记号,即$(\sigma(\varepsilon_1),\sigma(\varepsilon_2),\ldots,\sigma(\varepsilon_n))$
和$\sigma(\varepsilon_1,\varepsilon_2,\ldots,\varepsilon_n)$.实际上是等价的,等价原因是
$(\sigma(\varepsilon_1),\sigma(\varepsilon_2),\ldots,\sigma(\varepsilon_n))A=(\sigma(\varepsilon_1,\varepsilon_2,\ldots,\varepsilon_n))A=\sigma((\varepsilon_1,\varepsilon_2,\ldots,\varepsilon_n)A)$成立,
这一性质在之后会有运用,证明并不复杂,可以自行尝试或参考我的矩阵辅学授课.

除此之外,我们还应当强调以下结论,在我们后续研究线性方程组解的相关性质时是常用的:
\begin{theorem}
    线性映射是单射当且仅当其矩阵表示为列满秩矩阵,线性映射是满射当且仅当其矩阵表示为行满秩矩阵.
\end{theorem}
这一结论的证明比较基本,希望大家能透过这一个结论看到列满秩矩阵与行满秩矩阵更本质的特征.
\subsection{一组简单的例子}
\begin{example}\label{example:5:矩阵表示1}
    已知$\sigma \in \mathcal{L}(\mathbf{R}^3,\mathbf{R}^3)$且$\sigma(x_1,x_2,x_3)=(x_1+x_2,x_1-x_3, x_2)$
    \begin{enumerate}[label=(\arabic*)]
        \item 求$\sigma$的像空间和核空间;

        \item 求$\sigma$关于$\mathbf{R}^3$自然基的矩阵.
    \end{enumerate}
\end{example}

\begin{example}\label{example:5:矩阵表示2}
    设$A=\begin{pmatrix}1 & 0 & 2 \\ -1 & 2 & 1 \\ 1 & 2 & 5\end{pmatrix}$为两个三维线性空间之间的线性映射$\sigma$对应的矩阵,
    求$\sigma$的像空间和核空间.
\end{example}

\begin{example}\label{example:5:矩阵表示3}
    已知3阶矩阵$A=\begin{pmatrix}
        1 & 0 & 1 \\ 0 & -1 & 0 \\ -1 & 1 & -1
    \end{pmatrix}$. 定义$\mathbf{F}^{3 \times 3}$上的线性变换$\sigma(X)=AX,\enspace X \in \mathbf{F}^{3 \times 3}$.
    求$\sigma$的像和核.
\end{example}
实际上,例题2.4.1和2.4.3都是属于已知映射求像和核的题目,具体方法在像和核一节已经讲述,并且求矩阵表示也是根据上面的定义
即可,都是程式化的.然而例7则有不同,但此题与例2.4.1、2.4.2也有关联.实际上
此类问题像空间就是以矩阵列空间为坐标的向量的线性扩张,核空间是以矩阵零空间的基(即$AX=0$的基础解系)为坐标的向量的线性扩张,
推导见例7解析或我的矩阵辅学,希望各位同学能掌握推导并理解这三个例题之间的关系与区别. % TODO 编号系统 autoref

\vspace{2ex}
\centerline{\heiti \Large 内容总结}

\vspace{2ex}

\centerline{\heiti \Large 习题}
\vspace{2ex}
{\kaishu }
\begin{flushright}
    \kaishu

\end{flushright}
\centerline{\heiti A组}
\begin{enumerate}
    \item 设$\sigma: V_1\to V_2$是线性映射,证明:$\sigma(W_1)$和$\sigma^{-1}(W_2)$分别是$V_2$和$V_1$的子空间
\end{enumerate}
\centerline{\heiti B组}
\begin{enumerate}
    \item
\end{enumerate}
\centerline{\heiti C组}
\begin{enumerate}
    \item
\end{enumerate}
