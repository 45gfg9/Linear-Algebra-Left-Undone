\chapter{内积空间上的算子}

\section{自伴算子和正规算子}

由前面一章,我们成功的给线性空间加上了度量,使其升格成了内积空间,认识了一些新朋友(投影映射),或是更了解了一些老朋友(线性泛函). 但之前学的那些线性映射似乎还没搭上边,那么本章我们就要研究一下它们与内积有关的性质.

\subsection{伴随}

\begin{definition}[伴随] \index{bansui@伴随 (adjoint)}
    设 $ T \in \mathcal{L}(V, W) $, $ T $ 的\term{伴随} $ T^*: W \rightarrow V $满足如下条件: $ \forall v \in V, w \in W, \langle Tv, w \rangle = \langle v, T^*w \rangle$
\end{definition}

这样的一个东西定义出来,在本书的视角下一般先考虑以下问题:是个映射吗?良定义吗?线性吗?好消息是,对伴随而言,这三个问题都是肯定的. 这里就良定义做个解释,线性的验证留给读者.

我们考虑如下的线性泛函 $ \varphi : V \rightarrow F, \enspace \varphi (v) = \langle Tv, w \rangle $,那么利用一下刚学到的 \hyperref[thm:23:Riesz]{Riesz 表示定理},存在唯一的 $ u \in W $,使得 $ \varphi (v) = \langle v, u \rangle $,再结合一下伴随的定义,只需要定义 $ T^*w = u $ 即可.

讲完了定义就轮到了性质,伴随有如下的运算性质.
\begin{enumerate}
    \item $ \forall S, T \in \mathcal{L}(V, W),\enspace (S + T)^* = S^* + T^* $;

    \item $ \forall \lambda \in \mathbf{F},\enspace T \in \mathcal{L}(V, W),\enspace (\lambda T)^* = \overline{\lambda} T^* $;

    \item $ \forall T \in \mathcal{L}(V, W),\enspace (T^*)^* = T $;

    \item 对 $ V $ 上的恒等算子 $ I $ 有 $ I^* = I $;

    \item $ \forall T \in \mathcal{L}(V, W),\enspace S \in \mathcal{L}(W, U),\enspace (ST)^* = T^*S^* $.
\end{enumerate}

接着再研究一下它的核空间和像空间.

设 $ T \in \mathcal{L}(V, W) $. 则

\begin{enumerate}
    \item $ \ker T^* = (\im T)^{\perp} $;

    \item $ \im T^* = (\ker T)^{\perp} $;

    \item $ \ker T = (\im T^*)^{\perp} $;

    \item $ \im T = (\ker T^*)^{\perp} $.
\end{enumerate}

以上性质均不难证明,大家可以自己试试,顺带回顾一下内积的运算方法和证明线性空间相等的方法.

而关于特征值、不变子空间等等的性质,就先简单看两道例题,有一个最基本的了解.

\begin{example} \label{ex:24:伴随与特征值}
    设 $ T \in \mathcal{L}(V),\enspace \lambda \in \mathbf{F} $. 证明:$ \lambda $ 是 $ T $ 的特征值当且仅当 $ \overline{\lambda} $是 $ T^* $ 的特征值.
\end{example}

\begin{example} \label{ex:24:伴随与不变子空间}
    设 $ T \in \mathcal{L}(V) $ 且 $ U $ 是 $ V $ 的子空间. 证明:$ U $ 在 $ T $ 下不变当且仅当 $ U^{\perp} $ 在 $ T^* $ 下不变.
\end{example}

映射本身研究的差不多了,我们就该看看对应的矩阵有些什么性质了. 不过在此之前,我们要讨论一种新的对矩阵的操作.

\begin{definition}[共轭转置] \index{gongezhuanzhi@共轭转置 (conjugate transpose)}
    $ m \times n $ 矩阵的\term{共轭转置}是先互换行和列,然后对每个元素取复共轭得到的 $ n \times m $ 矩阵.

    即矩阵 $ A = (a_{ij})_{m \times n} $,则 $ A $ 的共轭转置阵 $ \overline{A}^{\mathrm{T}} = (\overline{a_{ji}})_{n \times m} $
\end{definition}

有了这重铺垫,我们就可以好好讨论一下伴随映射对应的矩阵了.

确定一个映射的矩阵都是要取定基的,而在内积空间上,我们取基的时候更喜欢用标准正交基,所以注意,下面这个定理只对标准正交基成立.

\begin{theorem}
    设 $ T \in \mathcal{L}(V, W) $,$ e_1, \ldots , e_n $ 是 $ V $ 的一组标准正交基,$ f_1, \ldots , f_m $ 是 $ W $ 的一组标准正交基,有 $ T(e_1, \ldots , e_n) = (f_1, \ldots , f_m)A $,$ A = (a_{ij})_{m \times n} $,$ T^*(f_1, \ldots , f_m) =(e_1, \ldots , e_n)B $,$ B = (b_{ij})_{n \times m} $,则 $ B $ 是 $ A $ 的共轭转置.
\end{theorem}

\begin{proof}
    首先确定矩阵 $ A $ 的元素. 因为 $ f_1, \ldots , f_m $ 是 $ W $ 的一组标准正交基,所以有
    \[ Te_j = \langle Te_j, f_1 \rangle f_1 + \cdots + \langle Te_j, f_m \rangle f_m,\enspace \forall j = 1, \ldots , n \]
    也就是说,$ a_{ij} = \langle Te_j, f_i \rangle $. 那么同理,对于矩阵 $ B $ 而言,$ b_{ij} = \langle T^*f_j, e_i \rangle $. 所以有
    \[ a_{ij} = \langle Te_j, f_i \rangle = \langle e_j, T^*f_i \rangle = \overline{\langle T^*f_i, e_j \rangle} = \overline{b_{ji}} \]
    所以,矩阵 $ B $ 是 $ A $ 的共轭转置.
\end{proof}

对于一般线性映射的伴随就介绍上面的这些了. 接下来还是看些限制更多、性质更好的线性映射,比如算子.

\subsection{自伴算子}

限制成算子的话,原本的算子与其伴随就被限制在同一块内积空间上了. 很自然的,我们就会开始思考一件事情,如果一个算子和它的伴随相等,那么会发生什么?

\begin{definition}[自伴算子] \index{zibansuanzi@自伴算子 (self-adjoint operator)}
    若算子 $ T \in \mathcal{L}(V) $ 满足 $ T = T^* $,则其被称为\term{自伴算子}.
\end{definition}

写成内积的语言就是 $ \forall v, w \in V,\enspace \langle Tv, w \rangle = \langle v, Tw \rangle $.

容易验证自伴算子对加法和数乘都是封闭的. 而根据上面对伴随的阐述,我们可以做一个类比:伴随在 $ \mathcal{L}(V) $ 上的作用如同复共轭在 $ \mathbf{C} $ 上的作用. 所以自伴算子可以类比为实数. 关于这方面的类比在实内积空间上的算子一章会进行更深入的阐述.

那么,自伴算子有这么好的定义,自然也少不了几条优美的性质.

\begin{theorem}
    自伴算子的特征值都是实数.
\end{theorem}

证明只需要结合特征值和自伴算子的定义就行了. 这条性质的几何意义就是自伴算子对特征向量方向上的向量仅仅是拉伸的作用,而不产生旋转或对称的作用.

以下的两条定理建立在复数域上,也是对复内积空间的结构进行一个初步的了解,以及加深一下算子和数的类比.

\begin{theorem} \label{thm:24:复内积空间}
    设 $ V $ 是复内积空间,$ T \in \mathcal{L}(V) $. 若 $ \forall v \in V,\enspace \langle Tv, v \rangle = 0 $,则 $ T = 0 $.
\end{theorem}

证明利用的正是之前的\autoref{eq:23:内积和范数的性质4}.

\begin{theorem}
    设 $ V $ 是复内积空间,$ T \in \mathcal{L}(V) $. 则 $ T $ 是自伴的当且仅当 $ \forall v \in V, \langle Tv, v \rangle \in \mathbf{R} $
\end{theorem}

证明利用的则是实数减去其共轭等于 0 推出的一系列等价变形. 这一定理也进一步地显示出自伴算子与实数的相似性.

下面这个定理是\autoref{thm:24:复内积空间} 的一般情况.

\begin{theorem}
    若 $ T $ 是 $ V $ 上的自伴算子,$ \forall v \in V,\enspace \langle Tv, v \rangle = 0 $,则 $ T = 0 $.
\end{theorem}

复内积空间上已经处理过了,实内积空间上利用\autoref{eq:23:内积和范数的性质2} 与内积的对称性即可证明. 事实上,在实内积空间上能做到 $ \forall v \in V ,\enspace \langle Tv, v \rangle = 0 $的算子绝大部分都是非自伴的,下面这道例题给出了其满足的性质.

\begin{example}
    设 $ V $ 是实内积空间, $ T \in \mathcal{L}(V),\enspace \forall v \in V,\enspace \langle Tv, v \rangle = 0 $. 证明:$ T^* = -T $.
\end{example}

满足这样性质的算子在实内积空间上叫做反对称算子,如果我们故意将虚轴定义错误(即将 0 包括进去)的话,反对称算子就可以类比为``虚轴''上的数. 自伴算子和反对称算子的交集是 0 算子,就如同实轴与``虚轴''的交点是原点.

\subsection{正规算子}

自伴算子的讨论就先阐述这么多. 之前将算子和数进行了类比,着重关注了他们相似的地方,现在来看看它们的不同之处,而最大的不同应该就是算子对乘法并没有交换律. 所以,如果一个算子与其伴随的乘法是可交换的,它又会有些什么特殊之处呢?

\begin{definition}[正规算子] \index{zhengguisuanzi@正规算子 (normal operator)}
    若算子 $ T \in \mathcal{L}(V) $ 满足 $ TT^* = T^*T $,则其被称为\term{正规算子}.
\end{definition}

很显然,自伴算子其实也是正规算子.

和自伴算子一样,我们来简单研究一下正规算子的性质. 首先是正规算子的一个等价条件.

\begin{theorem} \label{thm:24:正规算子的等价条件}
    算子 $ T \in \mathcal{L}(V) $ 是正规的当且仅当 $ \forall v \in V,\enspace \lVert Tv \rVert = \lVert T^*v \rVert $.
\end{theorem}

这也表明,对于任意一个正规算子 $ T $ ,其核空间和其伴随映射的核空间相等.

接下来的两条性质则是着重关注正规算子的特征向量.

\begin{theorem} \label{thm:24:正规算子的特征向量}
    设 $ T \in \mathcal{L}(V) $ 是正规的,且 $ v \in V $ 是 $ T $ 相应于特征值 $ \lambda $ 的特征向量,则 $ v $ 也是 $ T^* $ 相应于特征值$ \overline{\lambda} $ 的特征向量.
\end{theorem}

这是\autoref{ex:24:伴随与特征值} 在正规算子条件下的加强,它不仅反映了算子与其伴随的特征值在数值上的关系,也反映出了特征空间的关系. 从这里出发,你可以先思考一下正规算子的不变子空间是怎样的,如果有困难的话不妨结合一下\autoref{ex:24:伴随与不变子空间}.

在学特征值时我们就学过,同一映射的属于不同特征值的特征向量是线性无关的. 在正规算子条件下,这一结论也得到了加强,从原先的线性无关变为互相正交.

\begin{theorem} \label{thm:24:正规算子的特征向量正交}
    设 $ T \in \mathcal{L}(V) $ 是正规的,则 $ T $ 的相应于不同特征值的特征向量是正交的.
\end{theorem}

\begin{proof}
    设 $ \alpha, \beta $,是 $ T $ 的不同特征值,$ u, v $ 分别是相应的特征向量,则 $ Tu = \alpha u,\enspace Tv = \beta v $. 由\autoref{thm:24:正规算子的特征向量} 有$ T^*v = \overline{\beta} v $. 从而
    \begin{align*}
        (\alpha - \beta)\langle u, v \rangle
         & = \langle \alpha u, v \rangle - \langle u, \overline{\beta}v \rangle \\
         & = \langle Tu, v \rangle - \langle u, T^*v \rangle                    \\
         & = 0.
    \end{align*}
    而 $ \alpha \neq \beta $,所以 $ \langle u, v \rangle = 0 $,即 $ u, v $ 正交.
\end{proof}

这个定理很有意思,因为它既涉及了可对角化条件中的特征向量,也涉及了内积空间上的正交. 而这两条正是我们寻求在内积空间上算子对应矩阵简化表示的重要条件,将在下一节进行着重阐述.

\section{谱定理}

算子的谱这个概念其实还是挺遥远的,想详细了解的话就得移步到泛函分析了. 不过在有限维的线性空间上,其可以被理解为算子的特征值的集合. 所以说,我们研究的谱定理其实是和特征值相关,而特征值是矩阵对角化的重要元素. 因而,线性代数研究的谱定理实际上是描述了一族符合某种性质而可在内积空间上借助标准正交基进行对角化的算子. 谱定理根据数域不同分为复谱定理和实谱定理,复数域上的处理更简单,条件更弱,我们先说它.

\subsection{复谱定理}

有了\autoref{thm:24:正规算子的特征向量正交} 的铺垫,复谱定理描述的这样一族算子就呼之欲出了.

\begin{theorem}[复谱定理] \label{thm:24:复谱定理} \index{pudingli@谱定理 (spectral theorem)!fu@复谱定理}
    设 $ \mathbf{F} = \mathbf{C} $ 且 $ T \in \mathcal{L}(V) $. 则以下条件等价:
    \begin{enumerate}
        \item \label{item:24:复谱定理:1}
              $ T $ 是正规的.

        \item \label{item:24:复谱定理:2}
              $ V $ 有一个由 $ T $ 的特征向量构成的标准正交基.

        \item \label{item:24:复谱定理:3}
              $ T $ 关于 $ V $ 的某个标准正交基具有对角矩阵.
    \end{enumerate}
\end{theorem}

\ref*{item:24:复谱定理:2} 和 \ref*{item:24:复谱定理:3} 的等价性我们在可对角化的条件中就已经论述过,所以我们只需要证明 \ref*{item:24:复谱定理:1} 和 \ref*{item:24:复谱定理:3} 的等价性就行了.

\begin{proof}
    假设 \ref*{item:24:复谱定理:3} 成立,也就是 $ T $ 关于 $ V $ 的某个标准正交基具有对角矩阵,那么 $ T^* $关于同一组基的矩阵是 $ T $ 的共轭转置,也是对角矩阵. 任意两个对角矩阵是可交换的,所以$ T $ 和 $ T^* $ 是可交换的,所以 $ T $ 是正规的.

    % FIXME
    假设 \ref*{item:24:复谱定理:1} 成立,即 $ T $ 是正规的. 由 \hyperref[thm:23:Schur]{Schur 定理},可知 $ V $ 上存在一组标准正交基 $ (e_1, \ldots , e_n) $ 使得 $ T $ 关于其的矩阵是上三角矩阵,设为 $ A $.
    \[ A = \begin{pmatrix}
             & a_{11} & \cdots & a_{1n} \\
             &        & \ddots & \vdots \\
             & 0      &        & a_{nn} \\
        \end{pmatrix} \]
    接下来的任务就是证明它其实是个对角矩阵.

    我们逐个对向量进行讨论. 先考虑 $ e_1 $ ,从上面的矩阵得到
    \[ \lVert Te_1 \rVert^2 = \lvert a_{11} \rvert^2 \]
    而伴随映射的矩阵是原矩阵的共轭转置,所以
    \[ \lVert T^*e_1 \rVert^2 = \lvert a_{11} \rvert^2 + \lvert a_{12} \rvert^2 + \cdots + \lvert a_{1n} \rvert^2 \]
    由\autoref{thm:24:正规算子的等价条件},我们有 $ \lVert Te_1 \rVert = \lVert T^*e_1 \rVert $,所以 $ a_{1i} = 0,\enspace i = 2, \ldots , n $.

    现在考虑 $ e_2 $,因为证明了 $ a_{12} = 0 $,所以
    \[ \lVert Te_2 \rVert^2 = \lvert a_{22} \rvert^2 \]
    且
    \[ \lVert T^*e_2 \rVert^2 = \lvert a_{22} \rvert^2 + \lvert a_{23} \rvert^2 + \cdots + \lvert a_{2n} \rvert^2 \]
    同理有 $ a_{2i} = 0,\enspace i = 3, \ldots , n $.

    如此反复,最终证得 $ A $ 是对角矩阵.
\end{proof}

借助复谱定理,我们也可以对正规算子和自伴算子的关系做更深入的理解.

\begin{example}
    证明:复内积空间上的正规算子是自伴的当且仅当其所有的特征值都是实的.
\end{example}

\subsection{实谱定理}

实谱定理相对于复谱定理而言复杂了许多,我们需要几个引理先上手,并且从不变子空间的思路去证明它. 不过,虽然实谱定理是针对实内积空间的,但这几条引理在复内积空间上也是适用的.

\begin{lemma} \label{lem:24:实谱定理引理1}
    设 $ T \in \mathcal{L}(V) $ 是自伴的,并设 $ b, c \in \mathbf{R} $使得 $ b^2 < 4c $,则
    \[ T^2 + bT + cI \]
    是可逆的.
\end{lemma}

这个引理长的非常像实系数二次多项式恒大于 0 的定理,也进一步加强了自伴算子和实数的联系.

\begin{proof}
    取 $ V $ 中的一非零向量 $ v $. 则
    \begin{align*}
        \langle(T^2+bT+cI)v,v\rangle & = \langle T^2v,v \rangle + b\langle Tv,v \rangle + c\langle v,v \rangle                                  \\
                                     & = \langle Tv,Tv \rangle + b\langle Tv,v \rangle + c\lVert v \rVert^2                                     \\
                                     & \geqslant \lVert Tv \rVert^2 - \lvert b \rvert \lVert Tv \rVert \lVert v \rVert +  c\lVert v \rVert^2    \\
                                     & = \left(\lVert Tv \rVert - \frac{|b| \lVert v \rVert}{2}\right)^2 + (c - \frac{b^2}{4})\lVert v \rVert^2 \\
                                     & > 0
    \end{align*}
    从而 $ (T^2 + bT + cI)v \neq 0 $,$ T^2 + bT + cI $ 是单射,从而可逆.
\end{proof}

它实际上是为了自伴算子的多项式在实内积空间上的分解做准备的. 也就是下面这个结论证明的预备定理.

\begin{lemma} \label{lem:24:实谱定理引理2}
    设 $ V \neq \{ \vec{0} \} $ 且 $ T \in \mathcal{L}(V) $ 是自伴算子,则$ T $ 恒有特征值.
\end{lemma}

复内积空间上无论算子自伴与否都有特征值,不再赘述,下面针对实内积空间进行证明.

\begin{proof}
    设 $ V $ 是实内积空间,$ n = \mathrm{dim} V $. 取 $ v \in V, v \neq 0 $. 则
    \[ v, Tv, \ldots , T^nv \]
    必是线性相关的. 故存在不全为 0 的实数 $ a_0, \ldots , a_n $ 使得
    \[ \vec{0} = a_0v + a_1Tv + \cdots + a_nT^nv. \]

    以 $ a_0, \ldots , a_n $ 为系数构建一多项式,并将其在实数域上分解成
    \[ a_0 + a_1x + \cdots + a_nx^n  = c(x^2 + b_1x + c_1)\cdots(x^2 + b_Mx + c_M)(x - \lambda_1)\cdots(x - \lambda_m), \]
    其中 $ c $ 是非零实数, $ b_j, c_j \enspace(j = 1, \ldots , M) $ ,$ \lambda_i \enspace(i = 1, \ldots , m) $ 均是实数,且 $ b_j^2 < 4c_j ,\enspace j = 1, \ldots , M $,$ m + M \geqslant 1 $. 上式对 $ \forall x \in \mathbf{R} $ 均成立. 那么我们可以将算子多项式分解如下
    \begin{align*}
        \vec{0} & = a_0v + a_1Tv + \cdots + a_nT^nv                                                      \\
                & = (a_0I + a_1T + \cdots + a_nT^n)v                                                     \\
                & = c(T^2 + b_1T + c_1I)\cdots(T^2 + b_MT + c_MI)(T - \lambda_1I)\cdots(T - \lambda_mI)v
    \end{align*}
    而由\autoref{lem:24:实谱定理引理1} 可知,$ T^2 + b_jT + c_jI, j = 1, \ldots , M $ 均是可逆的. 而 $ c \neq 0 $,所以 $ m > 0 $ 且
    \[ \vec{0} = (T - \lambda_1I)\cdots(T - \lambda_mI)v. \]
    所以 $ \exists i $ 使得 $ T - \lambda_iI $ 不是单射. 所以 $ T $ 必有特征值.
\end{proof}

注意此证明中使用到的 $ v, Tv, \ldots , T^nv $ 构造线性相关,我们在证明复向量空间上的算子均有特征值时也使用到了.

有了特征值的存在,也就有了非平凡不变子空间的存在,进而有性质较好的限制算子的存在.

\begin{lemma} \label{lem:24:实谱定理引理3}
    设 $ T \in \mathcal{L}(V) $ 是自伴的,并设 $ U $ 是 $ V $ 在 $ T $ 下不变的子空间. 则
    \begin{enumerate}
        \item $ U^{\perp} $ 在 $ T $ 下不变;

        \item $ T|_U \in \mathcal{L}(U) $ 是自伴的;

        \item $ T|_{U^{\perp }} \in \mathcal{L}(U^{\perp }) $ 是自伴的.
    \end{enumerate}
\end{lemma}

此处证明结合不变子空间和自伴算子定义即可.

忙活了这么久,接下来就到了最激动人心的时刻了.

\begin{theorem}[实谱定理] \label{thm:24:实谱定理} \index{pudingli!shi@实谱定理}
    设 $ \mathbf{F} = \mathbf{R} $ 且 $ T \in \mathcal{L}(V) $. 则以下条件等价:
    \begin{enumerate}
        \item \label{item:24:实谱定理:1}
              $ T $ 是自伴的.

        \item \label{item:24:实谱定理:2}
              $ V $ 有一个由 $ T $ 的特征向量构成的标准正交基.

        \item \label{item:24:实谱定理:3}
              $ T $ 关于 $ V $ 的某个标准正交基具有对角矩阵.
    \end{enumerate}
\end{theorem}

\begin{proof}
    我们将采取 $\implies$ 1 $\implies$ 2 $\implies$ 3 进行证明.

    \begin{itemize}
        \item[\ref*{item:24:实谱定理:3}$\implies$\ref*{item:24:实谱定理:1}] $ T $ 关于 $ V $ 的某个标准正交基具有对角矩阵,实内积空间上对角矩阵等于其共轭转置,故 $ T^* = T $,$ T $ 是自伴的.

        \item[\ref*{item:24:实谱定理:1}$\implies$\ref*{item:24:实谱定理:2}] 采用数学归纳法.

            $ \mathrm{dim}V = 1 $ 时显然成立.

            设 $ \mathrm{dim}V > 1 $ 且在维数更小的实内积空间上成立. 因为\autoref{lem:24:实谱定理引理2},设 $ T $ 有一个特征向量 $ u $ 且 $ \lVert u \rVert = 1 $,$ U = \spa(u) $,则 $ U $ 是 $ V $ 的一个一维子空间且在 $ T $ 下不变,有\autoref{lem:24:实谱定理引理3},算子 $ T|_{U^{\perp }} \in \mathcal{L}(U^{\perp }) $ 是自伴的.

            由归纳假设,$ U^{\perp } $ 有一个由 $ T|_{U^{\perp }} $ 的特征向量构成的标准正交基. 将 $ u $ 添加进这组基,就得到了 $ V $ 的一组由 $ T $ 的特征向量构成的标准正交基,得证.

        \item[\ref*{item:24:实谱定理:2}$\implies$\ref*{item:24:实谱定理:3}] 这是平凡的.
    \end{itemize}
\end{proof}

下面这道题目则是实谱定理和复谱定理的对照.

\begin{example}
    仿照实谱定理的证明方法,证明复谱定理.
\end{example}

至此,谱定理的证明就完成了.

复谱定理给出了复向量空间上正规算子的完全描述,实谱定理给出了实向量空间上自伴算子的完全描述. 它们是非常强大的定理,因为它们给出的都是充要条件. 所以在处理内积空间上可对角化问题时,它们就是你最好的帮手.

可能很多同学对于行秩、列秩相等以及转置的几何意义很感兴趣. 实际上我们有两种获得转置矩阵的方式,第一种来源于我们之前讨论的对偶空间上的线性映射对应的矩阵,这种方式可能不够直观. 另一种获得的方法基于伴随算子. 接下来我们将说明这些定义的统一性,深刻理解转置的内涵.

我们可以研究矩阵及其转置的关系,我们可以用一个图形来表示:

\begin{figure}[H]
    \centering
    \small
    \begin{tikzpicture}
        \tikzset{->-/.style={decoration={
                markings,
                mark=at position .6 with {\arrow{stealth}}},postaction={decorate}}}

        \draw[thick,rotate=45] (0, 6) rectangle (-3, 3) rectangle (-5, 0)
            (-3, 3) rectangle(-3.35, 3.35)
            coordinate (xr) at (-2, 4)
            coordinate (xn) at (-4, 2)
            coordinate (x) at (-2, 2)
            coordinate (0n) at (-3, 3);
        \node at (-1, 5) {行空间};
        \node at (-4, 1) {$A$的核空间};
        \node at (-1.5, 6.5) {$\dim r$};
        \node at (-4, 4) {$\mathbf{R}^n$};
        \node at (-6, 3) {$\dim n-r$};

        \draw[thick,rotate=30] (6, 2) rectangle (3.5, -2) rectangle (0, -4)
            (3.5, -2) rectangle (3.85, -2.35)
            coordinate (b) at (4.5, 0.5)
            coordinate (0m) at (3.5, -2);
        \node at (5, 1.5) {列空间};
        \node at (2, -3) {$A^{\mathrm{T}}$的核空间};
        \node at (7, 0) {$\dim r$};
        \node at (5, -3) {$\mathbf{R}^m$};
        \node at (4, -4.5) {$\dim m-r$};

        \foreach \point in {xr, x, xn, 0n, b, 0m}
            \fill[black] (\point) circle (1pt);

        \node[left] at (xr) {$x_r$};
        \node[below right] at (x) {$x=x_r+x_n$};
        \node[left] at (xn) {$x_n$};
        \node[right] at (0n) {0};
        \node[right] at (b) {$b$};

        \draw[->-,very thick] (xr) -- node[above,sloped] {$Ax_r = b$} (b);
        \draw[->-,very thick] (x) -- node[below,sloped] {$Ax = b$} (b);
        \draw[->-,very thick] (xn) -- node[below,sloped] {$Ax_n = 0$} (0m);

        \draw[dashed,thick] (xr) -- (x) -- (xn);
    \end{tikzpicture}
\end{figure}

我们观察到以下几点:
\begin{enumerate}
    \item 矩阵的行空间与解空间(零空间)互为正交补(直观理解两个空间就是互相垂直且互为补空间),这一点应当是在正交的内容中有所提及的;

    \item 矩阵的列空间与其转置矩阵的零空间互为正交补,这一点实际与上一条等价.
\end{enumerate}

接下来我们来看行秩(列秩比较显然,此处不再详细展开). 我们首先得到解空间$N(A)$的维数,这可以直接根据维数公式得到:$\dim N(A) = n-r(A)$,根据正交补的性质,我们的可以得到行秩即为$n-(n-r(A))=r(A)$. 于是我们得到了一个基于正交补的行秩解释.

\vspace{2ex}
\centerline{\heiti \Large 内容总结}

本章基于内积引入了一种新的映射:伴随映射,并介绍了其相应的性质. 之后我们将范围缩小到算子上,介绍了两类特殊的算子:自伴算子和正规算子. 再然后便到了本书最重要的定理之一:谱定理. 它有复内积空间和实内积空间两个版本,不仅解答了为什么要介绍自伴算子和正规算子,还给出了对应内积空间上可对角化的充要条件. 希望大家能好好体会复谱定理和实谱定理的证明过程,都是相当精彩且优美的. 最后还有算子和数的类比,这个伏笔我们会在之后的章节回收.

\vspace{2ex}
\centerline{\heiti \Large 习题}

\vspace{2ex}
{\kaishu }
\begin{flushright}
    \kaishu

\end{flushright}

\centerline{\heiti A组}
\begin{enumerate}
    \item
\end{enumerate}

\centerline{\heiti B组}
\begin{enumerate}
    \item \item 设$A$为$n$阶实对称幂等矩阵,$r(A)=r$,求$|A-2E|$.
\end{enumerate}

\centerline{\heiti C组}
\begin{enumerate}
    \item
\end{enumerate}

前面我们对正规算子和自伴算子做了相当充分的工作,从这章开始我们准备对一般的算子做些工作.

\section{正交矩阵和酉矩阵}

本节我们将唤醒一些沉睡的记忆,如果你已经忘了过渡矩阵或矩阵的相似,可以移步到前面的章节再回顾一下. 如果你还在这的话,那么坐稳,我们马上开始.

\subsection{定义}

为了更好地引进正交矩阵和酉矩阵,我们有必要把共轭转置说的更清楚些. 共轭转置有着以下的运算性质,虽然都是看起来很显然的事情,此处还是稍稍赘述一下:

设有矩阵 $ A, B $ 和数 $ \lambda \in \mathbf{C}$,则

\begin{enumerate}
    \item $ (\overline{A + B})^{\mathrm{T}} = \overline{A}^{\mathrm{T}} + \overline{B}^{\mathrm{T}} $;

    \item $ \overline{(AB)}^{\mathrm{T}} = \overline{B}^{\mathrm{T}} \overline{A}^{\mathrm{T}} $;

    \item $ (\overline{\lambda A})^{\mathrm{T}} = \overline{\lambda} \enspace \overline{A}^{\mathrm{T}} $;

    \item $ \overline{\overline{A}^{\mathrm{T}}}^{\mathrm{T}} = A $.
\end{enumerate}

共轭转置说清楚后,便可以由此定义正交矩阵和酉矩阵.

\begin{definition} \index{youjuzhen@酉矩阵 (unitary matrix)} \index{zhengjiaojuzhen@正交矩阵 (orthogonal matrix)}
    在复数域(实数域)上,矩阵 $ A $ 满足 $ \overline{A}^{\mathrm{T}} A = E $( $ {A}^{\mathrm{T}} A = E $ ),则矩阵 $ A $ 被称为\term{酉矩阵}(\term{正交矩阵}).
\end{definition}

而如何刻画正交矩阵和酉矩阵的性质呢?下面的一个定理揭示了其与标准正交基的关系,可以从中窥得一些性质.

\begin{theorem}
    设 $ (e_1, e_2, \ldots , e_n) $ 是复(实)内积空间 $ V $ 上的标准正交基,$ (f_1, f_2, \ldots , f_n) $ 是 $ V $ 上的一组基,从 $ (e_1, e_2, \ldots , e_n) $ 到 $ (f_1, f_2, \ldots , f_n) $ 的过渡矩阵为 $ A $. 则 $ (f_1, f_2, \ldots , f_n) $ 是标准正交基的充要条件是 $ A $ 为酉矩阵(正交矩阵).
\end{theorem}

以下仅针对复内积空间的情况进行证明.

\begin{proof}
    由过渡矩阵的定义,$ (f_1, f_2, \ldots , f_n) $ = $ (e_1, e_2, \ldots , e_n)A $,$ A = (a_{ij})_{n \times n} $.

    由矩阵乘法的运算,可以得到
    \[ f_i = \sum_{j = 1}^{n} a_{ji}e_j , \enspace f_k = \sum_{j = 1}^{n} a_{jk}e_j. \]

    对两者做内积,有
    \[ \langle f_i, f_k \rangle = \left\langle \sum_{j = 1}^{n} a_{ji}e_j, \sum_{j = 1}^{n} a_{jk}e_j \right\rangle = \sum_{j = 1}^{n} a_{ji}\overline{a_{jk}} \]

    注意到 $ a_{ji},\enspace j = 1, \ldots , n $ 是 $ A^{\mathrm{T}} $ 的第 $ i $ 行的元素,$ \overline{a_{jk}},\enspace j = 1, \ldots , n $ 是 $ \overline{A} $ 的第 $ k $ 列的元素.

    定义 $ B = A^{\mathrm{T}}\overline{A} = (b_{ik})_{n \times n} $,则 $ \langle f_i, f_k \rangle = b_{ik} $.

    必要性:如果 $ f_1, f_2, \ldots , f_n $ 是一组标准正交基,则
    \[b_{ik} = \langle f_i, f_k \rangle = \delta_{ik} =
        \begin{cases}
            1 & i = k    \\
            0 & i \neq k
        \end{cases}\]

    由此可知 $ B = E $, $ \overline{B} = \overline{A}^{\mathrm{T}} A = \overline{E} = E $,即 $ A $ 是酉矩阵.

    充分性:将必要性证明推理过程倒写即可.
\end{proof}

如果这条定理中的 $ e_1, e_2, \ldots , e_n $ 取为该空间的自然基,就会有 $ (f_1, f_2, \ldots , f_n) = A $,我们便可以不太严谨地得到如下的这个结论

\begin{theorem}
    矩阵 $ A $ 是酉矩阵(正交矩阵)等价于其列向量构成标准正交基.
\end{theorem}

证明是平凡的,就交给你自己验证了.

那提到了过渡矩阵,我们也就不得不提与之息息相关的一个等价关系——相似了. 相信你已经回忆起来,相似实际上是同一个算子在不同基下的矩阵表示之间的关系,实现这个变化正是依赖于两组基之间的过渡矩阵. 而我们的主线正是依靠基变换实现的,只不过我们现在用的都是标准正交基,在基变换上也要有所升级. 所以,让我们先定义两个特殊一点的相似关系:

\begin{definition}
    \begin{enumerate}
        \item \term{酉相似}:复内积空间上,若 $ B = P^{-1}AP = \overline{P}^{\mathrm{T}}AP $,则称矩阵 $ A $ 与矩阵 $ B $ 酉相似.

        \item \term{正交相似}:实内积空间上,若 $ B = P^{-1}AP = {P}^{\mathrm{T}}AP $,则称矩阵 $ A $ 与矩阵 $ B $ 正交相似.
    \end{enumerate}
\end{definition}

它俩的特殊之处你可能一下子没看出来,不过没关系,我们可以先回到它们对应的算子上去看看.

\subsection{等距同构}

由之前一章,我们知道,算子与其伴随在同一组标准正交基下的矩阵表示是互为共轭对称的,所以设对应的算子是 $ S $,则其应该满足 $ S^*S = I $. 那么这个性质能将我们导向何处呢?

考虑两侧同时作用向量 $ u $,再与 向量 $ v $ 做内积,那么我们得到了如下的式子:
\[ \langle S^*Su, v \rangle = \langle u, v \rangle. \]
再结合伴随的定义,稍微做个变换,就有了下面这个美妙的结果:
\[ \langle Su, Sv \rangle = \langle u, v \rangle. \]
也就是说,这个算子 $ S $ 同时作用在两个向量上的话不改变它们的内积. 更进一步的话,如果取 $ v = u $,我们就能得到最终的结果:
\[ \lVert Su \rVert = \lVert u \rVert \]
算子 $ S $ 保持范数.

\begin{definition}[等距同构] \index{tonggou!dengju@等距同构 (isometric isomorphism)}
    算子 $ S \in \mathcal{L}(V) $ 称为\term{等距同构},如果 $ \forall v \in V $都有 $ \lVert Su \rVert = \lVert u \rVert $.
\end{definition}

注意我们这里虽然使用了共轭对称,但是从伴随的角度上来说复内积空间和实内积空间其实是一样的,也就是说等距同构的概念在这两类空间上是一致的,只不过刻画上会有所差距,之后会有所介绍. 此外,也常称实内积空间上的等距同构为正交算子,复内积空间上的等距同构称为酉算子.

让我们看道简单的例题加深一下对等距同构的印象.
\begin{example} \label{ex:25:等距同构}
    设 $ \lambda_1, \ldots , \lambda_n $ 都是模为 1 的标量,$ e_1, \ldots , e_n $ 是 $ V $ 的标准正交基,$ S \in \mathcal{L}(V) $满足 $ Se_j = \lambda_je_j $,证明 $ S $ 是等距同构.
\end{example}

然后来介绍一下等距同构的等价条件,虽然很多,但大部分都是我们刚才推理过程中已经得到的结果.

\begin{theorem}
    设 $ S \in \mathcal{L}(V) $ ,则以下条件等价:
    \begin{enumerate}
        \item $ S $ 是等距同构;

        \item 对所有 $ u, v \in V $ 均有 $ \langle Su, Sv \rangle = \langle u, v \rangle $;

        \item 对 $ V $ 中的任意标准正交向量组 $ e_1, \ldots , e_m $ 均有 $ Se_1, \ldots , Se_m $ 是标准正交的;

        \item $ V $ 有规范正交基 $ e_1, \ldots ,e_n $ 使得 $ Se_1, \ldots , Se_n $ 是标准正交基;

        \item $ SS^* = S^*S = I $;

        \item $ S^* $ 是等距同构;

        \item $ S $ 是可逆的且 $ S^{-1} = S^* $.
    \end{enumerate}
\end{theorem}

证明大部分都在上面的过程中证明过了,剩下的请大家自行验证.

我们关注第 3 个条件,即等距同构将标准正交组映射成标准正交组. 那么其对应的矩阵,即酉矩阵或正交矩阵也保有这样的性质,在基变换时它们能将标准正交基仍然变换成标准正交基,这正是我们所希望看到的,也就是酉相似(正交相似)的特殊之处.

这样我们就可以用酉相似和正交相似对谱定理进行矩阵语言的刻画.

\begin{theorem}
    \begin{enumerate}
        \item 复方阵 $ A $ 酉相似于对角矩阵的充要条件是 $ A $ 是正规矩阵;

        \item 实方阵 $ A $ 正交相似于对角矩阵的充要条件是 $ A $ 是实对称矩阵.
    \end{enumerate}
\end{theorem}

继续关注第 5 个条件,很容易的就会发现,等距同构其实也是正规算子,那么其相较于正规算子又有什么加强呢?在复内积空间下,结合\autoref{ex:25:等距同构},就会有下面这个优美的等价条件.

\begin{theorem}
    设 $ V $ 是复内积空间,$ S \in \mathcal{L}(V) $. 则以下条件等价:
    \begin{enumerate}
        \item $ S $ 是等距同构;

        \item $ V $ 有一个由 $ S $ 的特征向量构成的标准正交基,相应的特征值的绝对值均为 1.
    \end{enumerate}
\end{theorem}

几何意义其实是相当直观的,所有绝对值为 1 的复数作用在模为 1 的向量上都不会改变其模,而只是进行对称与旋转(事实上旋转也可以表为对称,之后会有所介绍). 这在几何上的运用是相当广泛的.

\section{正定矩阵}

在数学分析课程中,我们常常会讨论多元函数的极值,极值的刻画依赖的正是矩阵是否有定(definite matrix),正定(positive definite)还是负定(negative definite)还是半正定(positive semidefinite)还是半负定(negative semidefinite).

有定可以被解释为无论这个非零向量是怎样的,其经过某种规定运算得到的数的符号是确定的. 正定矩阵自然就是指任何非零向量经某种运算后得到的实数一定是正的. 但这所谓的``某种运算''在不同的数域下仍然有差异.

\begin{definition}[正定矩阵] \index{zhengdingjuzhen@正定矩阵 (positive definite matrix)}
    \begin{enumerate}
        \item 实数域:对 $ n $ 阶实对称矩阵 $ M $,若对于所有非零实系数向量 $ z $,均有$ z^{T}Mz > 0 $,则称矩阵 $ M $ 为正定矩阵;

        \item 复数域:对 $ n $ 阶 Hermite 矩阵 $ M $,若对于所有非零向量 $ z $,$ z^{H}Mz > 0 $,则称矩阵 $ M $ 为正定矩阵.
    \end{enumerate}
\end{definition}

复数域上的定义合理性是由``对于 Hermite 矩阵 $ M $,$ z^{H}Mz $ 必为实数''保证的.

由实数域上的正定矩阵的定义,我们可以发现其与二次型的相关性,我们也可以利用从二次型中所学来判定实正定矩阵.

\begin{theorem}
    设 $ A $ 为 $ n $ 阶实对称矩阵,则以下条件等价:
    \begin{enumerate}
        \item $ A $ 是正定矩阵;

        \item $ A $ 的正惯性指数为 $ n $,即 $ A \simeq E $;

        \item 存在可逆矩阵 $ P $,使得 $ A = P^{T}P $;

        \item $ A $ 的 $ n $ 个特征值 $ \lambda_1, \lambda_2, \ldots, \lambda_n $ 均为正.
    \end{enumerate}
\end{theorem}

对应的复数域版本相信大家也很容易就能够联想得到,只需要将转置变为共轭转置即可.

以下是一些更深层次地判别矩阵是否正定的条件,同时它们也是正定矩阵的一些重要的性质.

\begin{theorem}
    $ A $ 是 $ n $ 阶的 Hermite 矩阵. 以下条件等价.
    \begin{enumerate}
        \item $ A $ 是正定矩阵;

        \item 双线性函数 $ \langle x, y \rangle = x^{H}Ay $定义了一个 $ \mathbf{C}^n $ 上的一个内积. 事实上, $ \mathbf{C}^n $ 所有内积都可视作由某个正定矩阵以此方式得到;

        \item $ A $ 是向量 $ x_1, \ldots , x_n \in \mathbf{C}^k $ 构成的
              Gram 矩阵. 即 $ A = B^{H}B $,其中 $ B $ 未必是方阵,但一定是单的,并且这种分解方式不唯一.

        \item \term{Cholesky 分解}\index{Cholesky@Cholesky 分解 (Cholesky decomposition)}:存在唯一的下三角矩阵 $ L $,其主对角元均为正数,使得 $ A = LL^{H} $

        \item \term{Sylvester 定理}\index{Sylvester@Sylvester 定理 (Sylvester's criterion)}:$ A $ 的所有顺序主子式均为正. (但对于半正定矩阵而言,顺序主子式非负不能推出矩阵半正定)
    \end{enumerate}
\end{theorem}

像 2 就道出了正定矩阵和内积之间的关系,3 和 4 给出了正定矩阵的一些分解方式.

现在让我们跳开去,先去看看算子上的事情,不过出于更实用的原因,我们研究半正定矩阵对应的算子.

\begin{definition}[正算子] \index{zhengsuanzi@正算子 (positive operator)}
    设算子 $ T \in \mathcal{L}(V) $,如果 $ T $ 是自伴的且 $ \forall v \in V $均有 $ \langle Tv, v \rangle \geqslant 0 $.
\end{definition}

如果 $ V $ 是复向量空间,则 $ T $ 自伴的条件可以从定义中去除.

但对于正算子的定义似乎和对于半正定矩阵的定义方向完全不同,前者依托内积,后者则是依托二次型. 不过,若是你还记得我们曾经提到过内积本身是一种正定齐次双线性函数,以及二次型可以通过双线性函数引入,就可以捕获这其中的相关之处. 我们接下来进行推导.

\begin{proof}
    设 $ V $ 的一组标准正交基为 $ e = (e_1, e_2, \ldots, e_n) $,任取向量 $ \alpha \in V $,设其在 $ e $ 下的坐标为 $ x = (x_1, x_2, \ldots, x_n)^{T} $. 设正算子 $ T \in \mathcal{L}(V) $在 $ e $ 下的矩阵为 $ A = (a_{ij})_{n \times n}$,则
    \begin{align*}
        \langle Tv, v \rangle
         & = \langle Tex, ex \rangle = \langle eAx, ex \rangle                                                                                       \\
         & = \left\langle (e_1, e_2, \ldots ,e_n)
        \begin{pmatrix}
            a_{11} & a_{12} & \ldots & a_{1n} \\
            a_{21} & a_{22} & \ldots & a_{2n} \\
            \vdots & \vdots & \ddots & \vdots \\
            a_{n1} & a_{n2} & \ldots & a_{nn}
        \end{pmatrix}
        \begin{pmatrix}
            x_1    \\
            x_2    \\
            \vdots \\
            x_n
        \end{pmatrix} ,(e_1, e_2, \ldots ,e_n)
        \begin{pmatrix}
            x_1    \\
            x_2    \\
            \vdots \\
            x_n
        \end{pmatrix} \right\rangle                                                                                                                  \\
         & = \langle \sum_{i = 1}^{n}e_{i}\sum_{j = 1}^{n}a_{ij}x_{j}, \sum_{i = 1}^{n}x_{i}e_{i} \rangle
        = \overline{x_1}\sum_{j = 1}^{n}a_{1j}x_{j} + \overline{x_2}\sum_{j = 1}^{n}a_{2j}x_{j} + \cdots + \overline{x_n}\sum_{j = 1}^{n}a_{nj}x_{j} \\
         & = (\overline{x_1}, \overline{x_2}, \ldots, \overline{x_n})
        \begin{pmatrix}
            \sum\limits_{j = 1}^{n}a_{1j}x_{j} \\
            \sum\limits_{j = 1}^{n}a_{2j}x_{j} \\
            \vdots                             \\
            \sum\limits_{j = 1}^{n}a_{nj}x_{j}
        \end{pmatrix}
        = x^{H}Ax.
    \end{align*}

    此对应复内积空间的情形,实内积空间的情形也就显然了.
\end{proof}

我们定义出的正算子虽然名为正算子,但我们类比的时候它其实是类似于非负数,非负数很重要的一种运算就是开方运算. 类似的,我们也可以定义算子的平方根.

\begin{definition}[平方根] \index{pingfanggen@平方根 (square root)}
    算子 $ R $ 被称为算子 $ T $ 的平方根,如果 $ R^{2} = T $.
\end{definition}

以下是正算子的刻画.

\begin{theorem}
    设 $ T \in \mathcal{L}(V) $. 则以下条件等价.
    \begin{enumerate}
        \item \label{item:25:正算子刻画:1}
              $ T $ 是正的;

        \item \label{item:25:正算子刻画:2}
              $ T $ 是自伴的且 $ T $ 的所有特征值非负;

        \item \label{item:25:正算子刻画:3}
              $ T $ 有正的平方根;

        \item \label{item:25:正算子刻画:4}
              $ T $ 有自伴的平方根;

        \item \label{item:25:正算子刻画:5}
              存在算子 $ R \in \mathcal{L}(V) $ 使得 $ T = R^{*}R $.
    \end{enumerate}
\end{theorem}

从这里我们进一步加深类比. \ref*{item:25:正算子刻画:3} 就相当于复数非负当且仅当其有非负的平方根,\ref*{item:25:正算子刻画:4} 就相当于复数非负当且仅当其有实的平方根,\ref*{item:25:正算子刻画:5} 就相当于复数 $ z $ 非负当且仅当存在复数 $ w $ 使得 $ z = \overline{w}w $.

每个非负数都有唯一的非负平方根,下面这个定理表明正算子也具有类似的性质.

\begin{theorem}
    $ V $ 上每个正算子都有唯一的正平方根.
\end{theorem}

这个涉及到唯一性的证明. 在我们最初学习线性代数的时候应该就涉及到了,如果线性映射在线性空间的一组基下的对应的像是确定的,则该线性映射是被唯一确定的. 而在内积空间上,我们倾向选用标准正交基,正算子以及其正平方根又是自伴的,所以证明借助了谱定理.

\begin{proof}
    设 $ T \in \mathcal{L}(V) $ 是正的,$ v \in V $ 是 $ T $ 的一个特征向量,则有 $ \lambda \geqslant 0 $ 使得 $ Tv = \lambda v $.

    设 $ R $ 是 $ T $ 的正平方根,我们只需要证明 $ Rv = \sqrt{\lambda} v $,因为这样就代表 $ R $ 在 $ T $ 的特征向量上是唯一确定的,而 $ T $ 是自伴的,$ V $ 上肯定有一组以 $ T $ 的特征向量构成的标准正交基,从而唯一确定 $ R $.

    设 $ V $ 上有一组以 $ R $ 的特征向量构成的标准正交基 $ e_1, \ldots , e_n $. $ R $ 是正算子,所以其特征值均非负,即存在非负数 $\lambda_1, \ldots , \lambda_n $使得对每个 $ j = 1, \ldots , n $ 均有 $ Re_j = \sqrt{\lambda_j}e_j $.

    因为 $ e_1, \ldots , e_n $ 是 $ V $ 的一组标准正交基,所以有 $ a_1, \ldots , a_n \in \mathbf{F} $使得
    \[ v = a_1e_1 + \cdots + a_ne_n. \]

    于是
    \[ Rv = a_1\sqrt{\lambda_1}e_1 + \cdots + a_n\sqrt{\lambda_n}e_n. \]

    从而
    \[ Rv = a_1\lambda_1e_1 + \cdots + a_n\lambda_ne_n. \]

    又 $ R^{2} = T $ 且 $ Tv = \lambda v $,所以有
    \[ a_1\lambda e_1 + \cdots + a_n\lambda e_n = a_1\lambda_1e_1 + \cdots + a_n\lambda_ne_n \]
    上式意味着对 $ j = 1, \ldots, n $ 有 $ a_j(\lambda - \lambda_j) = 0 $. 所以
    \[ v = \sum_{\{j \mid \lambda_j = \lambda \}} a_je_j \]

    所以
    \[ Rv = \sum_{\{j \mid \lambda_j = \lambda \}} a_j\sqrt{\lambda}e_j = \sqrt{\lambda}v. \]

    命题得证.
\end{proof}

我们将正算子 $ T $ 的唯一正平方根记作 $ \sqrt{T} $.

另外,虽然正算子与非负数相似之处很多,但也有差异. 比如正算子是可以有无穷多个平方根的,但非负数最多只能有两个.

\vspace{2ex}
\centerline{\heiti \Large 内容总结}

\vspace{2ex}
\centerline{\heiti \Large 习题}

\vspace{2ex}
{\kaishu }
\begin{flushright}
    \kaishu

\end{flushright}

\centerline{\heiti A组}
\begin{enumerate}
    \item 证明:上三角的酉矩阵必为对角矩阵.

    \item 证明:任一 $ n $ 级可逆复矩阵 $ A $ 一定可以被唯一分解成 $ A = PB $,其中 $ P $ 是 $ n $ 级酉矩阵,$ B $ 是主对角元均为正实数的 $ n $ 级上三角矩阵.
\end{enumerate}

\centerline{\heiti B组}
\begin{enumerate}
    \item 设 $ V $ 是有限维复内积空间,$ S, T \in \mathcal{L}(V) $ 均为正规算子. 证明:若 $ ST = TS $,则
          \begin{enumerate}
              \item $ V $ 上存在一组标准正交基,使得 $ S, T $ 在此基下的矩阵都是对角矩阵.

              \item $ S $ 与 $ T $ 的复合也是正规算子.
          \end{enumerate}
\end{enumerate}

\centerline{\heiti C组}
\begin{enumerate}
    \item
\end{enumerate}
