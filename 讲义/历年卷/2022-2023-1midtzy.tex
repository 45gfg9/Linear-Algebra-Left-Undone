\phantomsection
\section*{2022-2023学年线性代数I(H)期中}
\addcontentsline{toc}{section}{2022-2023学年线性代数I(H)期中(谈之奕老师)}
\label{sec:2022-2023-1midtzy}

\begin{center}
    任课老师:谈之奕\hspace{4em} 考试时长:90分钟
\end{center}
\begin{enumerate}
	\item[一、](15分)参数$\lambda$取何值时,线性方程组
	\[\begin{cases}
        2x_1-4x_2+5x_3+3x_4=1 \\
        3x_1-6x_2+4x_3+2x_4=2 \\
        4x_1-8x_2+3x_3+x_4=\lambda
    \end{cases}\]
    有解?当方程组有解时,求其一般解.
	\item[二、](15分)设$\alpha_1=(1,1,1,2)^\mathrm{T}$,$\alpha_2=(3,4,4,7)^\mathrm{T}$,$\alpha_3=(-2,1,2,-3)^\mathrm{T}$,$\alpha_4=(5,3,4,6)^\mathrm{T}$,$\alpha_5=(4,5,3,13)^\mathrm{T}$,试求向量组$\{\alpha_1,\alpha_2,\alpha_3,\alpha_4,\alpha_5\}$的一组极大线性无关组.
	\item[三、](15分)
    \begin{enumerate}[label=(\arabic*)]
        \item 已知$\mathbf{R}^2$上的线性变换$\sigma(x_1,x_2)=(x_1-x_2,2x_1-2x_2)$,$I$为$\mathbf{R}^2$上的恒等变换,求$\ker\sigma$和$\im(I-\sigma)$;

        \item 设$\tau$为$\mathbf{R}^n$上任一线性变换,$I$为$\mathbf{R}^n$上恒等变换,证明:$\ker\tau\subseteq\im(I-\tau)$. 又$\im\tau\subseteq\ker(I-\tau)$是否一定成立?
    \end{enumerate}
	\item[四、](25分)设$\mathbf{R}[x]_3$是次数小于等于3的实系数多项式的全体和零多项式一起组成的集合关于多项式加法和数乘构成的实数域上的线性空间.
	\begin{enumerate}[label=(\arabic*)]
        \item 证明:$W=\{f(x)\in\mathbf{R}[x]_3\mid f(1)=0\}$是$\mathbf{R}[x]_3$的子空间,并求$\dim W$和$W$的一组基;

        \item 定义从$\mathbf{R}[x]_3$到$\mathbf{R}$的映射$T$如下:对任意$f(x)\in\mathbf{R}[x]_3$,$T(f(x))=f(1)$,证明:$T$是线性映射,并求$\dim\ker T$和$\im T$;

        \item 设$f,g,h\in\mathbf{R}[x]_3$,且$f(1)=g(1)=h(1)=0$,证明:$f,g,h$线性相关.
    \end{enumerate}
	\item[五、](30分)判断下列命题的真伪,若它是真命题,请给出简单的证明;若它是伪命题,给出理由或举反例将它否定.
    \begin{enumerate}[label=(\arabic*)]
        \item 向量$\beta$不能由向量组$\alpha_1,\ldots,\alpha_n$线性表示,则向量组$\alpha_1,\ldots,\alpha_n,\beta$线性无关;

        \item 实数集$\mathbf{R}$对实数的加法与实数的乘法构成任意数域上的线性空间;

        \item 记$\mathbf{R}_k^n$为至多有$k$个分量为非零实数的$n$元向量全体,$\mathbf{R}_k^n$是$\mathbf{R}^n$的子空间;

        \item 若$S$是线性空间$V$的线性相关子集,则$S$的每个向量都是$S$的其他向量的线性组合;

        \item 若线性映射$T:V\to W$的核是$K$,则$\dim V=\dim W+\dim K$;

        \item 线性空间$V$的任何子空间$W$都是某个映射$T:V\to V$的核.
    \end{enumerate}
\end{enumerate}

\clearpage
