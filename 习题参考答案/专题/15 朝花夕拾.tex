\phantomsection
\section*{15 朝花夕拾}
\addcontentsline{toc}{section}{15 朝花夕拾}

\vspace{2ex}

\centerline{\heiti A组}
\begin{enumerate}
    \item \begin{enumerate}
              \item 参考教材定理6.1.

              \item 提示. 考虑$A=\begin{pmatrix}
                            a_{11} & a_{12} & \cdots & a_{1n} \\
                            a_{21} & a_{22} & \cdots & a_{2n} \\
                            \vdots & \vdots & \ddots & \vdots \\
                            a_{m1} & a_{m2} & \cdots & a_{mn}
                        \end{pmatrix}$. 令$\alpha_1=\begin{pmatrix}
                            a_{11} \\
                            a_{21} \\
                            \vdots \\
                            a_{m1}
                        \end{pmatrix},\ldots,\alpha_n=\begin{pmatrix}
                            a_{1n} \\
                            a_{2n} \\
                            \vdots \\
                            a_{mn}
                        \end{pmatrix}$. $AX=0\implies x_1\alpha_1+\cdots+x_n\alpha_n=0$\\
                    只有零解表明$\alpha_1,\ldots,\alpha_n$线性无关,故$A$列满秩,$r(A)=n$.\\
                    有非零解(无穷解)则相反.

              \item 提示同上.

              \item 参考教材定理6.2.

              \item $A(k_1X_1+\cdots+k_sX_s)=k_1AX_1+\cdots+k_sAX_s=0+\cdots+0=0$.

              \item $A(k_1X_1+\cdots+k_sX_s+\eta_0)=k_1AX_1+\cdots+k_sAX_s+A\eta_0=b$

              \item $A(\eta_1-\eta_2)=A\eta_1-A\eta_2=b-b=0$

              \item 令$\bar{X}=k_1X_1+\cdots+k_sX_s$,有
                    \begin{align*}
                        A\bar{X}=b & \iff A(k_1X_1+\cdots+k_sX_s)=b        \\
                                   & \iff k_1AX_1+\cdots+k_sAX_s=b         \\
                                   & \iff (k_1+\cdots+k_s)b=b              \\
                                   & \iff k_1+\cdots+k_2=1\quad (b\neq 0).
                    \end{align*}

              \item 类似上题.

              \item 错误.\\
                    必要性正确. 令$A$为$m\times n$的矩阵,则$AX=b$有唯一解表明$r(A)=n$. 而$r(A)=n\implies AX=0$只有零解成立.\\
                    充分性错误. 注意到$AX=0$只有零解$\implies r(A)=n$. 但$r(A)=m$不代表$AX=b$有唯一解,因为还有无解的可能.\\
                    (齐次线性方程组一定有解,但非齐次线性方程组不一定有解,一定要注意这个差别)\\
                    简单的反例如$A=\begin{pmatrix}
                            1 & 0 \\
                            0 & 1 \\
                            0 & 0
                        \end{pmatrix},b=\begin{pmatrix}
                            0 \\
                            0 \\
                            1
                        \end{pmatrix}$

              \item 正确. 令$B=(\beta_1,\ldots,\beta_s)$. $AB=0\implies A(\beta_1,\ldots,\beta_s)=0$,即$(A\beta_1, \ldots, A\beta_s)=0$,故$B$的列向量为方程组$AX=0$的解. (注:$A(\beta_1,\ldots,\beta_s)=(A\beta_1, \ldots, A\beta_s)$利用分块矩阵性质即可)

              \item 利用上一题的结论易知正确.

              \item 设$A$的解空间为$N(A)$,$B$的解空间为$N(B)$,则由题意有$N(A)\subseteq N(B)$,故$\dim{N(A)}\leqslant \dim{N(B)}$. 而$r(A)+\dim{N(A)}=r(B)+\dim{N(B)}$(参考第1题),故$r(A)\geqslant r(B)$正确.

              \item 错误.\\
                    必要性正确. $AX=0$与$BX=0$为同解方程组可得$N(A)=N(B)\implies\dim{N(A)}=\dim{N(B)}$. 同上一题,可得$r(A)=r(B)$.\\
                    充分性错误. 反例有$A=\begin{pmatrix}
                            1 & 0 & 0 \\
                            0 & 1 & 0
                        \end{pmatrix}, B=\begin{pmatrix}
                            1 & 0 & 0 \\
                            0 & 0 & 1
                        \end{pmatrix}$.\\
                    实际上,只需考虑解空间均为$\mathbf{R}^n$的子空间,$\mathbf{R}^n=\spa(\alpha_1,\ldots,\alpha_n)$. 令$N(A)=\spa(\alpha_i), N(B)=\spa(\alpha+j),i\neq j$即为反例.
          \end{enumerate}

    \item $r(A^*)=
              \begin{cases}
                  1 & r(A)=3 \\
                  0 & r(A)<3
              \end{cases}$,而已知$A_{21}\neq 0$,故$r(A^*)=1$,故$r(A)=3$,故$AX = 0$解空间维数为$4-r(A)=1$. 而$|A|=0$,由行列式展开可知
          \[ a_{i1}A_{21}+a_{i2}A_{22}+a_{i3}A_{23}+a_{i4}A_{24}=0 \qquad i=1,2,3,4 \]
          故解为$k(A_{21},A_{22},A_{23},A_{24})^\mathrm{T},\enspace k\in \mathbf{R}$.

    \item 由题意得$r(A)=3<4$且$\alpha_1-4\alpha_3=0$,或$\alpha_1=4\alpha_3$. 由$r(A)=3$得$r(A^*)=1$,$A^*X=0$的基础解系含3个线性无关的解向量. 由$A*A=|A|E=O$得$\alpha_1,\alpha_2,\alpha_3,\alpha_4$为$A^*X=0$的解,从而$\alpha_1,\alpha_2,\alpha_4$或$\alpha_2,\alpha_3,\alpha_4$为$A^*X=0$的基础解系.

    \item $\forall \alpha \in \mathbf{R}^n, \alpha^\mathrm{T}A\beta =0\implies A\beta =0$.

    \item $r(A)=3\implies \dim{N(A)}=4-r(A)=1$.

          $\beta = x_1\alpha_1+x_2\alpha_2+x_3\alpha_3+x_4\alpha_4\implies$特解为$(1,-1,0,3)^\mathrm{T}$

          $\alpha_1-2\alpha_2+\alpha_3=0\implies$导出组基础解系为$(1,-2,1,0)^\mathrm{T}$.

          故通解为$k(1,-2,1,0)^\mathrm{T}+(1,-1,0,3)^\mathrm{T},\enspace k\in\mathbf{R}$

    \item 设方程组$Ax=b$,由$r(A)=3$知其导出组$Ax=0$的基础解系只含一个解向量.

          而$A\eta_1=b,A(\eta_2+\eta_3)=2b$,故$2\eta_1-(\eta_2+\eta_3)=(3,4,5,6)^\mathrm{T}$为$Ax=0$的基础解系,从而所求通解为
          \[ x = \eta_1+c(3,4,5,6)^\mathrm{T} = (2,3,4,5)^\mathrm{T} + c(3,4,5,6)^\mathrm{T} \]
          其中$c$为任意常数.

    \item \begin{enumerate}
              \item 取$\beta_1-\beta_2,\beta_1-\beta_3$(取法不唯一),有$A(\beta_1-\beta_2)=A(\beta_1-\beta_3)=0$且$\beta_1-\beta_2,\beta_1-\beta_3$线性无关,否则$\beta_1,\beta_2,\beta_3$线性相关.

              \item 取$2\beta_1+k_1(\beta_1+\beta_2)+k_2(\beta_2+\beta_3)$可行(取法不唯一).
          \end{enumerate}

    \item 对于任意的$b_{s\times 1}$,$AX=b$都有解,说明$A$的列向量可以线性表出出任意$s$维向量$b$,从而$r(A)\geqslant s$,而$r(A)\leqslant s$,所以$r(A)=s$.

          反之,如果$r(A)=s$,则$A$有$s$个线性无关的列向量,它们就是$s$维空间$V$的一组基,对于$V$中任意向量$b$,都可以由$A$的列向量线性表出,从而$AX=b$有解.

          当然,也可以直接用秩不等式:$s=r(A)\leqslant r(A,b)\leqslant s$得到$r(A,b)=s$,所以$AX=b$有解.

    \item $AB=0\implies r(A)+r(B)\leqslant n,r(B)=n\implies r(A)\leqslant 0\implies A=O$.

    \item 必要性. 有条件可知$V_1\cap V_2=\{0\}$,$\begin{pmatrix}A\\B\end{pmatrix}\in\mathbf{F}^{n\times n}$,而$\begin{pmatrix}A\\B\end{pmatrix}x=0$只有零解,故$\begin{pmatrix}A\\B\end{pmatrix}$可逆,从而$r(A)=m,r(B)=n-m$,于是
          \begin{align*}
              \dim V_1 & =n-r(A)=n-m,       \\
              \dim V_2 & =n-r(B)=n-(n-m)=m.
          \end{align*}
          又$V_1+V_2$是$\mathbf{F}^n$的子空间,且$\dim(V_1+V_2)=\dim V_1+\dim V_2-\dim(V_1\cap V_2)=\dim\mathbf{F}^n$,故$\mathbf{F}^n=V_1\oplus V_2$.

          充分性. 若$\begin{pmatrix}A\\B\end{pmatrix}x=0$有非零解$x_1$,则$x_1\in V_1\cap\V_2$. 这与$\mathbf{F}^n=V_1\oplus V_2$矛盾.

    \item \begin{enumerate}
              \item 由条件知,方程组系数矩阵的为2,系数矩阵
                    \[ A=\begin{pmatrix}
                            0  & 1 & a & b  \\
                            -1 & 0 & c & d  \\
                            a  & c & 0 & -e \\
                            b  & d & e & 0
                        \end{pmatrix}\rightarrow
                        \begin{pmatrix}
                            0  & 1 & a          & b       \\
                            -1 & 0 & c          & d       \\
                            0  & 0 & 0          & ad-e-bc \\
                            0  & 0 & -(ad-e-bc) & 0
                        \end{pmatrix} \]
                    故$ad-e-bc=0$.

              \item 易求得基础解系为$(c,-a,1,0)^\mathrm{T},(d,-b,0,1)^\mathrm{T}$.
          \end{enumerate}
\end{enumerate}

\centerline{\heiti B组}
\begin{enumerate}
    \item 见教材P210第10题.

    \item 由$r(A)=m$可知,$A$的$m$个行向量($A^\mathrm{T}$的$m$个列向量)线性无关,它们是方程组$CX=0$的一个基础解系. 由$CA^\mathrm{T}=O$和$n-r(C)=m$,得$r(C)=n-m$. 因此,由$CA^\mathrm{T}B^\mathrm{T}=C(BA^\mathrm{T})=O$,可知$(BA)^\mathrm{T}$的$m$个行向量线性无关,它们也是$CX=0$的一个基础解系.

    \item 必要性:设$AX=b$有无穷多解,则$r(A)<n$,从而$|A|=0$,于是$A^*b=A^*AX=|A|X=0$.

          充分性:设$A^*b=0$,即方程组$A^*b=0$有非零解,则$r(A^*)<n$,又$A_{11}\neq 0$,所以$r(A)=n-1$.

          令$A=(\alpha_1,\alpha_2,\ldots,\alpha_n)$,因为$A^*A=|A|E=O$,所以$\alpha_1,\alpha_2,\ldots,\alpha_n$为方程组$A^*X=0$的解,又因为$A_{11}\neq 0$,所以$\alpha_1,\alpha_2,\ldots,\alpha_n$线性无关.

          由$r(A^*)=1$,得方程组$A^*X=0$的基础解系含有$n-1$个线性无关的解向量,所以$\alpha_2,\alpha_3,\ldots,\alpha_n$为方程组$A^*X=0$的一个基础解系.

          因为$A^*b=0$,所以$b$为方程组$A^*X=0$的一个解,从而$b$可由$\alpha_2,\alpha_3,\ldots,\alpha_n$线性表示,$b$也可由$\alpha_1,\alpha_2,\ldots,\alpha_n$线性表示,于是$r(A) = r\begin{pmatrix}A & b\end{pmatrix}=n-1<n$,故方程组$AX=b$有无穷多解.

    \item 由各列的元素之和等于0,得$|A|=0$. 利用教材第6章习题7的结论:

          如果$r(A)<n-1$,则$r(A^*)=0$,$A^*=O$,所以,$A_{ij}=0,\enspace i,j=1,2,\ldots,n$;

          如果$r(A)=n-1$,则$r(A^*)=1$,且$AA^*=O$,于是$A^*$的每一列$(A_{i1},A_{i2},\cdots A_{in})^\mathrm{T},\enspace i=1,2,\ldots,n$都是$AX=0$的解.

          由于$AX=0$的解空间的维数为
          \[ n-r(A)=n-(n-1)=1 \]
          所以,$AX=0$的任意两个解成比例. 又元素全部为1的$n$元向量$e=(1,1,\ldots,1)^\mathrm{T}$满足方程组$AX=0$,因此$A^*$任意一列都与$e$成比例,即
          \[ A_{i1}=A_{i2}=\cdots=A_{in} \qquad i=1,2,\ldots,n \]
          所以,$A^*$的每一列元素(即$|A|$的每一行元素的代数余子式)都相等.

          同理,$(A^\mathrm{T})^*$的每一列元素(即$|A^\mathrm{T}|$的每一行元素,也是$|A|$的每一列元素的代数余子式)都相等,即
          \[ A_{1j}=A_{2j}=\cdots=A_{nj} \qquad j=1,2,\ldots,n \]

    \item 见2019-2020学年线性代数I(H)期末第五题.

    \item 令原方程组$AX=0$,解系组成矩阵$B$. 则新方程组为$B^\mathrm{T}X=0$. 而$B^\mathrm{T}A^\mathrm{T}=(AB)^\mathrm{T}=0$,推测$A^\mathrm{T}$的列向量为基础解系. 而由维数公式,
          \[ r(A)=\dim V-\dim\ker(A) = 2n-n=n. \]
          故$A^\mathrm{T}$列空间维数为$n$,即$r_c(A^\mathrm{T})=n$. 而由题意$r(B)=n$(否则不是基础解系),故
          \[ \dim{\ker(B)}=\dim{V^{\prime}}-r(B)=2n-n=n=r_c(A^\mathrm{T}). \]
          猜想成立.

    \item \begin{enumerate}
              \item 略.

              \item \begin{align*}
                        \dim(V_1+V_2) & =\dim V_1+\dim V_2-\dim(V_1\cap V_2) \\
                                      & =n(n-r)+n(n-s)-n(n-k).
                    \end{align*}
          \end{enumerate}

    \item \begin{enumerate}
              \item 容易计算$|aE=bA|=(a+nb)a^{n-1}$.

              \item 由$1<r(aE+bA)<n$知$|aE+bA|=0$. 故$a\neq 0$,且$a+nb=0$,此时$aE+bA$左上角的$n-1$阶子式
                    \[ \begin{vmatrix}
                            a+b    & b      & \cdots & b      \\
                            b      & a+b    & \cdots & b      \\
                            \vdots & \vdots & \ddots & \vdots \\
                            b      & b      & \cdots & a+b
                        \end{vmatrix}=(a+(n-1)b)a^{n-2}=\frac{a^{n-1}}{n}\neq 0, \]
                    故$\dim{W}=n-r(aE+bA)=n-(n-1)=1$.
          \end{enumerate}

    \item 设系数矩阵为$A$,第二个矩阵为$B$. 由于$(A,b)$为$B$子矩阵,故$r(A)\leqslant r(A,b)\leqslant r(B)$,而$r(A)=r(B)$,故$r(A)=r(A,b)$成立.

    \item 见教材P213第6题.

    \item 假. 取$A=\begin{pmatrix}
                  1-i & 1+i  \\
                  1+i & -1+i
              \end{pmatrix}$,有$A^\mathrm{T}A=0$.

    \item 见教材P214第8题.

    \item 由$XA=0$与$XAA^{\mathrm{T}}=0$同解. 又由条件知$(C-B)AA^\mathrm{T}=0$,故$(C-B)A=0$. 即$CA=BA$.

    \item 由$r(A)+r(E-A)=n \iff A^2=A$, 易证.

    \item 方法一. 用分块矩阵的方法,我们知道
          \[ \begin{pmatrix}
                  A & O \\
                  O & B
              \end{pmatrix}
              \rightarrow
              \begin{pmatrix}
                  A & O \\
                  A & B
              \end{pmatrix}
              \rightarrow
              \begin{pmatrix}
                  A & A   \\
                  A & A+B
              \end{pmatrix}. \]
          结合$AB=BA$,我们知道
          \[ \begin{pmatrix}
                  A & A   \\
                  A & A+B
              \end{pmatrix}
              \underbrace{
                  \begin{pmatrix}
                      A+B & O \\
                      -A  & E
                  \end{pmatrix}
              }_{\text{非广义初等变换,难以想到}}
              =
              \begin{pmatrix}
                  AB & A   \\
                  O  & A+B
              \end{pmatrix}. \]
          于是
          \[ r(A)+r(B)=r
              \begin{pmatrix}
                  A & O \\
                  O & B
              \end{pmatrix}=
              \begin{pmatrix}
                  A & A   \\
                  A & A+B
              \end{pmatrix}\geqslant r
              \begin{pmatrix}
                  AB & A   \\
                  O  & A+B
              \end{pmatrix}\geqslant
              r(AB)+r(A+B). \]
          方法二. 设方程组$AX=0$与$BX=0$的解空间分别是$V_1, V_2$,方程组$ABX=BAX=0$与$(A+B)X=0$的解空间分别为$W_1, W_2$,则$V_1\subseteq W_1, V_2\subseteq W_1$,从而$V_1+V_2\subseteq W_1$,同时$V_1\cap V_2\subseteq W_1$,同时$V_1\cap V_2\subseteq W_2$,利用维数公式就有
          \[ \dim V_1+\dim V_2=\dim(V_1+V_2)+\dim(V_1\cap V_2)\leqslant \dim W_1+\dim W_2. \]
          即
          \[ (n-r(A))+(n-r(B))\leqslant (n-r(AB))+(n-r(A+B)). \]
          化简便知$r(A)+r(B)\geqslant r(AB)+r(A+B)$.

    \item \begin{enumerate}
              \item 必要性:$ABX=0$与$BX=0$同解可知它们基础解系所含向量个数相同,即
                    \[ n-r(AB)=n-r(B)\implies r(AB)=r(B). \]
                    充分性:由必要性,当$r(AB)=r(B)$时,$ABX=0$与$BX=0$的基础解系所含向量个数相同,而$BX=0$的解都是$ABX=0$的解,所以$ABX=0$与$BX=0$同解.

              \item $r(AB)=r(B)$说明$ABX=0$与$BX=0$同解,用$CX$代替$X$就得到$ABCX=0$与$BCX=0$同解,从而有$r(ABC)=r(BC)$.\\
                    \textbf{推论.}设$A$是一个方阵,且存在正整数$k$使得$r(A^{k+1})=r(A^k)$,递推就有
                    \[ r(A^k)=r(A^{k+1})=r(A^{k+2})=\cdots. \]

              \item 当$A$可逆时,结论是显然的. 当$A$不可逆时,有$r(A)\leqslant n-1$,现在考虑$n+1$个矩阵$A,A^2,\ldots,A^{n+1}$,有
                    \[ n-1\geqslant r(A)\geqslant r(A^2)\geqslant\cdots\geqslant r(A^n)\geqslant r(A^{n+1})\geqslant 0. \]
                    这$n+1$个矩阵的秩只能从$0,1,\ldots,n-1$这$n$个数中取,所以必有两个矩阵的秩相同,即存在$m(1\leqslant m\leqslant n)$使得$r(A^m)=r(A^{m+1})$,由上面的推论可得:
                    \[ r(A^m)=r(A^{m+1})=\cdots=r(A^{n})=r(A^{n+1})=\cdots. \]
                    特别的,对于任意的正整数$k$有$r(A^n)=r(A^{n+k})$.
          \end{enumerate}

    \item 增广矩阵$A=\begin{pmatrix}
                  1  & 1  & b   & -1 & 1   \\
                  2  & 3  & 1   & 1  & -2  \\
                  0  & 1  & a   & 3  & -4  \\
                  -3 & -3 & -3b & b  & a+2
              \end{pmatrix}$. 由解空间维数为3可知,$r(A)=2$. 由于$\beta_1=\begin{pmatrix}
                  1 \\
                  2 \\
                  0 \\
                  3
              \end{pmatrix}
              \beta_2=\begin{pmatrix}
                  1 \\
                  3 \\
                  1 \\
                  -3
              \end{pmatrix}$线性无关,则后三列必能被其表示. 对于$\alpha_2=\begin{pmatrix}
                  -1 \\
                  1  \\
                  3  \\
                  b
              \end{pmatrix}=k_1\beta_1+k_2\beta_2\implies k_1=-4,k_2=3\implies b=3$\\
          $\alpha_3=\begin{pmatrix}
                  1  \\
                  -2 \\
                  -4 \\
                  a+2
              \end{pmatrix}=k_1\beta_1+k_2\beta_2\implies k_1=5,k_2=1\implies a=-25$. \\
          而$\alpha_1=\begin{pmatrix}
                  b \\
                  1 \\
                  a \\
                  -3b
              \end{pmatrix}=\begin{pmatrix}
                  3  \\
                  1  \\
                  -5 \\
                  -9
              \end{pmatrix}=k_1\beta_1+k_2\beta_2\implies k_1=8,k_2=-5$成立. 故$a=-5,b=3$. 解空间的基略.\\
          若解空间为2维,则$r(A)=3$,令$A=\begin{pmatrix}
                  \alpha_1 \\
                  \alpha_2 \\
                  \alpha_3 \\
                  \alpha_4
              \end{pmatrix}$,由于$\alpha_1,\alpha_2$必线性无关,故$\alpha_3,\alpha_4$中有且仅有一个可被$\alpha_1,\alpha_2$表示. 若为$\alpha_3$,$\alpha_2-2\alpha_1=(0,1,1-2b,3,-4)=(0,1,4,3,-4)$,则$a=1-2b$. 故$a+2b=1$,且$\alpha_4$不能被表示,故$\alpha_4\neq -3\alpha_1\implies b\neq 3, a\neq -5$. 故$\begin{cases}
                  a+2b=1 \\
                  b\neq 3, a\neq -5
              \end{cases}$即可.

    \item $W_1\cap W_2:\begin{pmatrix}
                  A \\ B
              \end{pmatrix}X=0$. \qquad
          $W_1+ W_2:\begin{cases}
                  AX = 0 \\
                  BX = 0
              \end{cases}$.

    \item 方法一. 令$x_4=t$,则方程组$\begin{cases}
                  x_1+x_4=-1 \\
                  x_2-2x_4=d \\
                  x_3+x_4=e
              \end{cases}$的一般解为$\begin{cases}
                  x_1=-1-t \\
                  x_2=d+2t \\
                  x_3=e-t  \\
                  x_4=t
              \end{cases}$.

          代入方程组$\begin{cases}
                  x_1+x_2+ax_3+x_4=1  \\
                  -x_1+x_2-x_3+bx_4=2 \\
                  2x_1+x_2+x_3+x_4=c
              \end{cases}$可得$\begin{cases}
                  (2-a)t=2-d-ae \\
                  (b+4)t=1-d+e  \\
                  0=c-d-e+2
              \end{cases}$. 由$t$的任意性,可得$a=2,b=-4$. 从而$\begin{cases}
                  0=2-d-2e \\
                  0=1-d+e  \\
                  0=c-d-e+2
              \end{cases}$,解得$d=\dfrac{4}{3},e=\dfrac{1}{3},c=-\dfrac{1}{3}$.

          方法二. 由于
          \[\begin{pmatrix}
                  A & b \\
                  B & d
              \end{pmatrix}=
              \begin{pmatrix}
                  1  & 1 & a  & 1  & 1  \\
                  -1 & 1 & -1 & b  & 2  \\
                  2  & 1 & 1  & 1  & c  \\
                  1  & 0 & 0  & 1  & -1 \\
                  0  & 1 & 0  & -2 & d  \\
                  0  & 0 & 1  & 1  & e
              \end{pmatrix}\rightarrow
              \begin{pmatrix}
                  1 & 0 & 0 & 0     & 0            \\
                  0 & 0 & 0 & b-a+6 & 3-2d+(1-a)e  \\
                  0 & 0 & 0 & 2a-4  & c-2+d(2a-1)e \\
                  0 & 0 & 0 & a-2   & d-2+ae       \\
                  0 & 1 & 0 & 0     & 0            \\
                  0 & 0 & 1 & 0     & 0
              \end{pmatrix}.\]
          易知$r(B)=3$,由$r\begin{pmatrix}
                  A & b \\
                  B & d
              \end{pmatrix}=r(B)$可得$\begin{cases}
                  a-2=0          \\
                  2a-4=0         \\
                  b-a+6=0        \\
                  3-2d+(1-a)e=0  \\
                  c-2+d(2a-1)e=0 \\
                  d-2+ae=0
              \end{cases}$,解得$\begin{cases}
                  a=2          \\
                  b=-4         \\
                  c=-\dfrac 13 \\[1ex]
                  d=\dfrac 43  \\[1ex]
                  e=\dfrac 13
              \end{cases}$.

    \item 注意如果$X$是两个方程组的公共解,这等价于存在$t_1,t_2,k_1,k_2$使得
          \[ X=\gamma+t_1\eta_1+t_2\eta_2=\delta+k_1\xi_1+k_2\xi_2. \]
          从而对应有
          \[ t_1\eta_1+t_2\eta_2-k_1\xi_1-k_2\xi_2=\gamma-\delta. \]
          将$t_1,t_2,k_1,k_2$设为未知量,对上述方程组的增广矩阵进行初等行变换,可得
          \[ (\eta_1,\eta_2,-\xi_1,-\xi_2,\delta-\gamma)=
              \begin{pmatrix}
                  -6 & -5 & -8 & -10 & -16 \\
                  5  & 4  & 1  & 2   & 6   \\
                  1  & 0  & -1 & 0   & 0   \\
                  0  & 1  & 0  & -1  & 0
              \end{pmatrix}\rightarrow
              \begin{pmatrix}
                  1 & 0 & -1 & 0  & 0  \\
                  0 & 1 & 0  & -1 & 0  \\
                  0 & 0 & 1  & 1  & 1  \\
                  0 & 0 & 0  & 1  & 2.
              \end{pmatrix} \]
          故$(t_1,t_2,k_1,k_2)'$有唯一解$(-1,2,-1,2)'$. 因此公共解为
          \[ X=\gamma-\eta_1+2\eta_2=\delta-\xi_1+2\xi_2=(1,0,-1,2)'. \]

    \item 方程组 (1)、(2) 有公共解,即方程组$\begin{cases}
                  x_1+x_2+x_3=0,     \\
                  x_1+2x_2+ax_3=0,   \\
                  x_1+4x_2+a^2x_3=0, \\
                  x_1+2x_2+x_3=a-1
              \end{cases}$有解,对其增广矩阵进行初等行变换,
          \begin{align*}
              \bar{A}= & \begin{pmatrix}
                             1 & 1 & 1   & 0   \\
                             1 & 2 & a   & 0   \\
                             1 & 4 & a^2 & 0   \\
                             1 & 2 & 1   & a-1
                         \end{pmatrix}\rightarrow
              \begin{pmatrix}
                  1 & 1 & 1     & 0   \\
                  0 & 1 & a-1   & 0   \\
                  0 & 3 & a^2-1 & 0   \\
                  0 & 1 & 0     & a-1
              \end{pmatrix}\rightarrow            \\
                       & \begin{pmatrix}
                             1 & 1 & 1        & 0   \\
                             0 & 1 & a-1      & 0   \\
                             0 & 0 & a^2-3a+2 & 0   \\
                             0 & 0 & 1-a      & a-1
                         \end{pmatrix}\rightarrow
              \begin{pmatrix}
                  1 & 1 & 1   & 0          \\
                  0 & 1 & a-1 & 0          \\
                  0 & 0 & 1-a & a-1        \\
                  0 & 0 & 0   & (a-1)(a-2)
              \end{pmatrix}.
          \end{align*}
          当$a\neq 1$且$a\neq 2$时,方程组 (1)、(2) 没有公共解.

          当$a=1$时,$\bar{A}\rightarrow\begin{pmatrix}
                  1 & 1 & 1 & 0 \\
                  0 & 1 & 0 & 0 \\
                  0 & 0 & 0 & 0 \\
                  0 & 0 & 0 & 0
              \end{pmatrix}\rightarrow\begin{pmatrix}
                  1 & 0 & 1 & 0 \\
                  0 & 1 & 0 & 0 \\
                  0 & 0 & 0 & 0 \\
                  0 & 0 & 0 & 0
              \end{pmatrix}$,因为$r(A)=r(\bar{A})=2$,所以方程组 (1)、(2) 有公共解,公共解为$X=C\begin{pmatrix} -1 \\ 0  \\ 1 \end{pmatrix}$(其中$C$为任意常数).

          当$a=2$时,$\bar{A}\rightarrow\begin{pmatrix}
                  1 & 1 & 1  & 0 \\
                  0 & 1 & 1  & 0 \\
                  0 & 0 & -1 & 1 \\
                  0 & 0 & 0  & 0
              \end{pmatrix}\rightarrow\begin{pmatrix}
                  1 & 0 & 0 & 0  \\
                  0 & 1 & 0 & 1  \\
                  0 & 0 & 1 & -1 \\
                  0 & 0 & 0 & 0
              \end{pmatrix}$,方程组 (1)、(2) 有唯一的公共解为$X=\begin{pmatrix}
                  0 \\
                  1 \\
                  -1
              \end{pmatrix}$.
\end{enumerate}

\centerline{\heiti C组}
\begin{enumerate}
    \item 令$f(x)=a_nx^n+\cdots+a_1x+a_0$为一个$n$次多项式,设$\lambda_1,\ldots,\lambda_{n+1}$为$f(x)$的$n+1$个不同的根,则有齐次线性方程组$\begin{cases} \begin{aligned}
                      \lambda_1^n a_n+\cdots+\lambda_1a_1+a_0         & =0,               \\
                      \lambda_2^n a_n+\cdots+\lambda_2a_1+a_0         & =0,               \\
                                                                      & \vdotswithin{ = } \\
                      \lambda_{n+1}^n a_n+\cdots+\lambda_{n+1}a_1+a_0 & =0.
                  \end{aligned} \end{cases}$. 其系数矩阵为$A=\begin{pmatrix}
                  \lambda_1^n     & \lambda_1^{n-1}     & \cdots & 1      \\
                  \lambda_2^n     & \lambda_2^{n-1}     & \cdots & 1      \\
                  \vdots          & \vdots              & \ddots & \vdots \\
                  \lambda_{n+1}^n & \lambda_{n+1}^{n-1} & \cdots & 1
              \end{pmatrix}$. 显然$|A|\neq 0$,从而$a_n=a_{n-1}=\cdots=a_1=a_0=0$,这与$f(x)$是$n$次多项式矛盾.

    \item 见教材P214第7题.

    \item \begin{enumerate}
              \item 由于$r(A)=n$,所以$A$可逆,由打洞原理可知$|B|=|A|(0-\beta' A^{-1}\beta)=-|A|\beta' A^{-1}\beta$,从而$B$可逆的充要条件为$\beta' A^{-1}\beta\neq 0$.

              \item 必要性. 由于
                    \[ r=r(A)\leqslant r(A,\beta)\leqslant r\begin{pmatrix} A & \beta \\ \beta' & 0 \end{pmatrix}=r(B)=r \]
                    再结合$A'=A$,可知
                    \[ r\begin{pmatrix} A & \beta \\ \beta' & 0 \end{pmatrix} =r(A,\beta)=r\begin{pmatrix} A \\ \beta' \end{pmatrix} \]
                    于是由定理15.1可知方程组$\begin{pmatrix} A \\ \beta' \end{pmatrix}X=\begin{pmatrix} \beta \\ 0 \end{pmatrix}$有解,即$\begin{cases} AX=\beta, \\ \beta' X=0. \end{cases}$有解.

                    充分性. 由于$\begin{cases} AX=\beta \\ \beta' X=0 \end{cases}$有解,从而$AX=\beta$也有解,即有$r(A)=r(A,\beta)$. 另外$\begin{cases} AX=\beta \\ \beta' X=0 \end{cases}$有解也说明$\begin{pmatrix} A \\ \beta' \end{pmatrix}X=\begin{pmatrix} \beta \\ 0 \end{pmatrix}$有解,于是结合定理15.1可知$r\begin{pmatrix} A & \beta \\ \beta' & 0 \end{pmatrix}=r\begin{pmatrix} A \\ \beta' \end{pmatrix}$. 而显然$r\begin{pmatrix} A \\ \beta' \end{pmatrix}=r(A,\beta)=r(A)$,于是$r(B)=r\begin{pmatrix} A & \beta \\ \beta' & 0 \end{pmatrix}=r$.

              \item 必要性. 若$B$可逆,则$B$的行向量组线性无关,从而$r(A,\beta)=n$,又由于$r(A)=n-1$,所以$r(A,\beta)=r(A)+1$,从而由定理15.1知$AX=\beta$无解.

                    充分性. 由于$r(A)=n-1$,若$AX=\beta$无解,则由定理15.1可知$r(A,\beta)=r(A)+1=n$,于是
                    \[ r(B)=r\begin{pmatrix}
                            A      & \beta \\
                            \beta' & 0
                        \end{pmatrix}\geqslant r(A,\beta)=n \]
                    若$r(B)=n$,则$r(B)=r\begin{pmatrix}
                            A      & \beta \\
                            \beta' & 0
                        \end{pmatrix}=r(A,\beta)$,取转置结合$A'=A$有$r\begin{pmatrix}
                            A      & \beta \\
                            \beta' & 0
                        \end{pmatrix}=r\begin{pmatrix}
                            A \\
                            \beta'
                        \end{pmatrix}$,再次结合定理15.1可知方程组$\begin{cases}
                            AX=\beta, \\
                            \beta' X=0.
                        \end{cases}$有解,特别地,$AX=\beta$也有解,这就与已知产生了矛盾. 所以$r(B)=n+1$,即$B$可逆.
          \end{enumerate}

    \item \begin{enumerate}
              \item 当$n$为偶数时,将增广矩阵$\bar{A}$的第$i\enspace(i=1,2,\ldots,n-1)$行乘以$(-1)^i$加到最后一行,得
                    \[ \bar{A}\rightarrow\begin{pmatrix}
                            1      & 1      & \cdots & 0      & 0      & b_1                                  \\
                            0      & 1      & \cdots & 0      & 0      & b_2                                  \\
                            0      & 0      & \cdots & 0      & 0      & b_3                                  \\
                            \vdots & \vdots & \ddots & \vdots & \vdots & \vdots                               \\
                            0      & 0      & \cdots & 1      & 1      & b_{n-1}                              \\
                            0      & 0      & \cdots & 0      & 0      & \displaystyle\sum_{i=1}^{n}(-1)^ib_i
                        \end{pmatrix}. \]
                    故当$\displaystyle\sum_{i=1}^{n}(-1)^ib_i=0$时,方程组有无穷多解,一般解为
                    \[\begin{cases} \begin{aligned}
                                x_1     & =\displaystyle\sum_{i=1}^{n-1}(-1)^{i-1}b_i+(-1)^1x_n, \\
                                x_2     & =\displaystyle\sum_{i=2}^{n-1}(-1)^{i-2}b_i+(-1)^2x_n, \\
                                        & \vdotswithin{ = }                                      \\
                                x_{n-2} & =b_{n-2}-b_{n-1}+(-1)^{n-2}x_n,                        \\
                                x_{n-1} & =b_{n-1}+(-1)^{n-1}x_n,                                \\
                            \end{aligned} \end{cases}\]
                    其中$x_n$为自由未知量.

              \item 当$n$为奇数时,有
                    \[ \bar{A}\rightarrow\begin{pmatrix}
                            1      & 1      & \cdots & 0      & 0      & b_1                                        \\
                            0      & 1      & \cdots & 0      & 0      & b_2                                        \\
                            0      & 0      & \cdots & 0      & 0      & b_3                                        \\
                            \vdots & \vdots & \ddots & \vdots & \vdots & \vdots                                     \\
                            0      & 0      & \cdots & 1      & 1      & b_{n-1}                                    \\
                            0      & 0      & \cdots & 0      & 2      & b_n+\displaystyle\sum_{i=1}^{n-1}(-1)^ib_i
                        \end{pmatrix}. \]
                    此时无论$b_1,b_2,\ldots,b_n\enspace(n\geqslant 2)$取何值,方程组都有唯一解为
                    \[\begin{cases} \begin{aligned}
                                x_1     & =\displaystyle\sum_{i=1}^{n-1}(-1)^{i-1}b_i+(-1)^{n-1}x_n, \\
                                x_2     & =\displaystyle\sum_{i=2}^{n-1}(-1)^{i-2}b_i+(-1)^{n-2}x_n, \\
                                        & \vdotswithin{ = }                                          \\
                                x_{n-2} & =b_{n-2}-b_{n-1}+(-1)^2x_n,                                \\
                                x_{n-1} & =b_{n-1}+(-1)^1x_n,                                        \\
                                x_n     & =\frac 12b_n+\displaystyle\sum_{i=1}^{n-1}(-1)^ib_i.
                            \end{aligned} \end{cases}\]
          \end{enumerate}

    \item \begin{enumerate}
              \item 必要性. 设$A$的行向量为$\alpha_1,\alpha_2,\ldots,\alpha_s$,$B$的行向量为$\beta_1,\beta_2,\ldots,\beta_m$,由$AX=0$与$BX=0$同解得$AX=0$与$\begin{pmatrix}A \\ B\end{pmatrix}=0$同解,从而系数矩阵的秩相同,即行向量$\alpha_1,\alpha_2,\ldots,\alpha_s$与$\alpha_1,\alpha_2,\ldots,\alpha_s,\beta_1,\beta_2,\ldots,\beta_m$秩相同. %“由命题3.2.5知道”?因此$\alpha_1,\alpha_2,\ldots,\alpha_s$与$\alpha_1,\alpha_2,\ldots,\alpha_s,\beta_1,\beta_2,\ldots,\beta_m$等价,即有$\beta_1,\beta_2,\ldots,\beta_m$可由$\alpha_1,\alpha_2,\ldots,\alpha_s$线性表出. 同理,$\alpha_1,\alpha_2,\ldots,\alpha_s$可由$\beta_1,\beta_2,\ldots,\beta_m$线性表出,即$A$与$B$的行向量等价.

                    充分性可由必要性的证明得到. 但是由于$A,B$列向量的维数都可能不同,所以不存在等价关系.

              \item 提示. $V_1\subseteq V_2$等价于$AX=0$与$\begin{pmatrix}
                            A \\
                            B
                        \end{pmatrix}X=0$同解.

              \item 必要性. 由题意可知$AX=a$与$\begin{pmatrix} A \\ B \end{pmatrix}X=\begin{pmatrix} a \\ b \end{pmatrix}$同解,从而它们的增广矩阵秩相同,即$r(A,a)=r\begin{pmatrix}
                            A & a \\
                            B & b
                        \end{pmatrix}$,可知$(A,a)$与$\begin{pmatrix}
                            A & a \\
                            B & b
                        \end{pmatrix}$的行向量组等价,从而$(B,b)$的每一个行向量都可以由$(A,a)$的行向量线性表出.\\
                    充分性可由必要性得到.

              \item 证明留给读者.
          \end{enumerate}

    \item \begin{enumerate}
              \item \label{item:15:B:5:1}
                    采用反证法. 设 $\lvert A \rvert = 0$,则线性方程组 $AX = 0$ 有非零解,设为 $X_0 = (x_1, x_2, \ldots, x_n)^{\mathrm{T}}$,记
                    \[\lvert x_k \rvert = \max \{\lvert x_1 \rvert, \lvert x_2 \rvert, \ldots, \lvert x_n \rvert\}.\]
                    由 $X_0 \neq 0$ 可知 $\lvert x_k \rvert > 0$,考虑 $AX = 0$ 的第 $k$ 个方程,有 $\displaystyle\sum_{j=1}^n a_{kj}x_j = 0$,于是
                    \[\lvert a_{kk} \rvert \lvert x_k \rvert = \lvert -\displaystyle\sum_{j \neq k}a_{kj}x_j \rvert \leqslant \displaystyle\sum_{j \neq k}\lvert a_{kj} \rvert \lvert x_k \rvert.\]
                    约去 $\lvert x_k \rvert$ 后可得 $\lvert a_{kk} \rvert \leqslant \displaystyle\sum_{j \neq k} \lvert a_{kj} \rvert$,这与条件矛盾. 所以 $\lvert A \rvert \neq 0$.

              \item 构造实函数
                    \[f(t) = \begin{vmatrix}
                            a_{11}  & ta_{12} & ta_{13} & \cdots & ta_{1n} \\
                            ta_{21} & a_{22}  & ta_{23} & \cdots & ta_{2n} \\
                            ta_{31} & ta_{32} & a_{33}  & \cdots & ta_{3n} \\
                            \vdots  & \vdots  & \vdots  & \ddots & \vdots  \\
                            ta_{n1} & ta_{n2} & ta_{n3} & \cdots & a_{nn}  \\
                        \end{vmatrix}\]
                    由于 $A$ 是实矩阵,所以 $f(t)$ 是关于 $t$ 的一个实系数多项式(连续)函数,同时
                    \[f(0) = a_{11}a_{22}\cdots a_{nn} > 0.\] 当$t \in [0, 1]$ 时,还有
                    \[a_{ii} > \sum_{j \neq i} \lvert a_{ij} \rvert \leqslant \sum_{j \neq i} \lvert ta_{ij} \rvert.\]
                    由 \ref*{item:15:B:5:1} 可知 $f(t)$ 在 $[0, 1]$ 上非零,由连续函数的介值定理可知 $f(1) > 0$,即 $\lvert A \rvert > 0$.

              \item 此为直接推论不再赘述.
          \end{enumerate}
\end{enumerate}

\clearpage
