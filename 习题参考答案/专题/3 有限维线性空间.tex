\section*{3 有限维线性空间}
\addcontentsline{toc}{section}{3 有限维线性空间}

\vspace{2ex}

\centerline{\heiti A组}
\begin{enumerate}
    \item \begin{enumerate}
        \item 错. 反例:$\alpha_1=(1,0),\alpha_2=(2,0),\alpha_3=(0,1)$,则 $\alpha_1,\alpha_2,\alpha_3$ 线性相关而 $\alpha_3$ 不是 $\alpha_1.\alpha_2$ 的线性组合.
        \item 对. 该命题的等价命题(逆否命题)是:若存在一个向量是其余向量的线性组合,则 $\alpha_1,\cdots,\alpha_m$ 线性相关. 这正是定理 $3.1$ 的内容,因而成立.
        \item 错. 反例:$\alpha_1=(1,0),\alpha_2=(0,1),\alpha_3=(1,1)$,则 $\alpha_1,\alpha_2,\alpha_3$ 两两无关,而三者线性相关. 可证两两无关是向量组无关的必要条件.
        \item 错. 反例:$\alpha_1=(1,0),\alpha_2=(0,0),\beta_1=(0,0),\beta_2=(0,1)$,有 $\alpha_1,\alpha_2$ 相关,$\beta_1,\beta_2$ 相关,而 $\alpha_1+\beta_1$ 与 $\alpha_2+\beta_2$ 线性无关.
        \item 错. 若 $\alpha_1,\cdots,\alpha_n$ 线性无关,有
        \[\lambda_1\alpha_1+\cdots+\lambda_n\alpha_n\Rightarrow\lambda_1=\lambda_2=\cdots=\lambda_n=0,\]
        判断 $\alpha_1+\alpha_2,\alpha_2+\alpha_3,\cdots,\alpha_n+\alpha_1$ 是否无关. 设
        \[\lambda_1'(\alpha_1+\alpha_2)+\cdots+\lambda_n'(\alpha_n+\alpha_1)=0,\]
        则
        \[(\lambda_n'+\lambda_1')\alpha_1+(\lambda_1'+\lambda_2')\alpha_2+\cdots+(\lambda_{n-1}'+\lambda_{n}')\alpha_n=0,\]
        则
        \[\Rightarrow \begin{cases}
		\lambda_n'+\lambda_1'=0 \\
		\cdots \\
		\lambda_{n-1}'+\lambda_{n}' =0\\
	    \end{cases}.\]
        解该方程可得 $\lambda_n'=(-1)^n\lambda_1'$,因此当 $n$ 为偶数时,上述方程组有非零解,则向量组相关,而当 $n$ 为奇数时,向量组无关. 综上,该命题不成立.
        \item 对. 由定理 $3.1$,不妨设 $\alpha_3$ 可由 $\alpha_1,\alpha_2$ 线性表示,则 $\alpha_1+\alpha_2,\alpha_2+\alpha_3,\alpha_3+\alpha_1$ 均可由 $\alpha_1,\alpha_2$ 线性表示,再由定理 $3.3$ 可知,$\alpha_1+\alpha_2,\alpha_2+\alpha_3,\alpha_3+\alpha_1$ 线性相关.
        \item 错. 反例:取 $\alpha_0=\alpha_1-\alpha_2-\alpha_3$,则 $\alpha_0+\alpha_1=(\alpha_0+\alpha_2)+(\alpha_0+\alpha_3)$,三者线性相关,不是 $\mathbb R^3$ 的基.
        \item 对. 判断 $\alpha_1+\alpha_2$ 与 $\alpha_1-\alpha_2$ 是否无关.
        \[\lambda_1(\alpha_1+\alpha_2)+\lambda_2(\alpha_1-\alpha_2)=0\]
        则有$(\lambda_1+\lambda_2)\alpha_1+(\lambda_1-\lambda_2)\alpha_2=0$,则 $\lambda_1+\lambda_2=0,\lambda_1-\lambda_2=0\Rightarrow \lambda_1=\lambda_2=0$,因此线性无关且个数等于维数,是一组基.
        \item 错. 反例:$\mathbb R^2$ 中过原点的直线 $L_0$ 是 $\mathbb R^2$ 的一个子空间. 显然这样的直线有无数条.
        \item 错. 反例:$\mathbb R^3$ 中,子空间 $W_1=\spa(e_1,e_2)$,$W_2=\spa(e_1+e_2,e_3)$,则 $B_1\cup B_2=\{e_1,e_2,e_3,e_1+e_2\}$,显然 $\mathbb R^3$ 中的任一组基都不可能包含四个元素.
    \end{enumerate}
    \item 设原向量组为 $\{\alpha_1,\alpha_2,\cdots,\alpha_n\}$. 每个向量有 $s$ 个分量 $(s>m)$. 对于齐次线性方程组:$x_1\alpha_1+x_2\alpha_2+\cdots+x_n\alpha_n=0$.
	\begin{enumerate}
        \item 若向量组线性相关,则对应该方程组有无穷多解. 去掉 $m$ 个分量,相当于删去该方程组中的任意 $m$ 行方程,依然有无穷多解. 这是因为对于原方程组的任意一个解,将其带入被削减后的方程组也依然成立. 故线性相关得证.
        \item 若向量组线性无关,对应原方程组仅有唯一解,也就是全零解. 增加 $m$ 个分量相当于增加 $m$ 个方程,依然只有唯一解,因为若出现非零解,代入原方程组对应的方程中不会成立,矛盾. 故线性无关得证.
    \end{enumerate}
	\item 方程组:$x_1\beta_1+x_2\beta_2+x_3\beta_3=0$ 系数矩阵
    \[\begin{pmatrix}
		1 & 2 & 1\\
		3 & 1 & -1\\
		6 & 2 & a\\
		2 & -1 &
		-2\end{pmatrix}\rightarrow\begin{pmatrix}
		1 & 2 & 1\\
		0 & -5 & -4\\
		0 & -10 & a-6\\
		0 & -6 &-4
	\end{pmatrix}\rightarrow\begin{pmatrix}
		1 & 2 & 1\\
		0 & -5 & -4\\
		0 & 0 & a+2\\
		0 & 0 & 0
	\end{pmatrix},\]
	仅全零解的条件是 $a\ne-2$,此时向量组线性无关.
    \item \begin{enumerate}
        \item 必要性:$\alpha_1,\cdots,\alpha_n$ 线性无关,对于 $F^n$ 中的任一向量 $\beta$, $\alpha_1,\cdots,\alpha_n,\beta$ 的向量个数大于维数 $n$,则线性相关. 由定理 3.2,$\beta$ 可被 $\alpha_1,\cdots,\alpha_n$ 唯一表示.
        \item 充分性:由于 $F^n$ 中任意向量均可被 $\alpha_1,\cdots,\alpha_n$ 线性表示,并且向量个数等于维数. 则 $\alpha_1,\cdots,\alpha_n$ 是 $F^n$ 的一组基. 则 $\alpha_1,\cdots,\alpha_n$ 线性无关.

        * 更详细的证明:对于 $F^n$ 的一组基 $e_1,\cdots,e_n$,其可被 $\alpha_1,\cdots,\alpha_n$ 表示. 若 $\alpha_1,\cdots,\alpha_n$ 线性相关,不妨设 $\alpha_n$ 可被 $\alpha_1,\cdots,\alpha_{n-1}$ 表示,则有 $e_1,\cdots,e_n$ 可被 $\alpha_1,\cdots,\alpha_{n-1}$ 表示. 由于 $e_1,\cdots,e_n$ 线性无关. 根据定理 3.3,$n\le n-1$,矛盾. 因此得证.
    \end{enumerate}
    \item \begin{enumerate}
        \item 必要性:对于 $\forall v\in\spa(S_1)\cap\spa(S_2)$ 有 $v=a_1\alpha_1+\cdots+a_s\alpha_s=b_1\beta_1+\cdots+b_t\beta_t$. 由于 $\alpha_1,\cdots,\alpha_s,\beta_1,\cdots,\beta_t$ 线性无关 $\Rightarrow a_1=\cdots=a_s=b_1=\cdots=b_t=0$,则 $v=0$,即 $\spa(S_1)\cap\spa(S_2)=\{0\}$.
        \item 充分性:考虑反证法. 如果 $\alpha_1,\cdots,\alpha_s,\beta_1,\cdots,\beta_t$ 线性相关,则存在不全为零的系数使得 $a_1\alpha_1+\cdots+a_s\alpha_s+b_1\beta_1\cdots+b_t\beta_t=0$. 因此存在一个向量 $v=a_1\alpha_1+\cdots+a_s\alpha_s=-(b_1\beta_1\cdots+b_t\beta_t)\ne 0$ 且 $v\in\spa(S_1),v\in\spa(S_2)$. 即存在非零向量 $v$ 属于 $\spa(S_1),v\in\spa(S_2)$,矛盾!则充分性得证.
    \end{enumerate}
    \item \begin{enumerate}
        \item 也就是求 $\{\alpha_1,\alpha_2,\alpha_3,\alpha_4\}$ 的极大线性无关组. 利用讲义中所述求法:方程组 $a_1\alpha_1+\cdots+a_4\alpha_4=0$ 对应系数矩阵 $\begin{pmatrix}
            1 & 1 & -2 & 1\\
            2 & -1 & -1& 1\\
            4 & -6 & 2 & -2\\
            3 & 6 & -9&7\end{pmatrix}$,化简为行阶梯型:$\begin{pmatrix}
            1 & 1 & -2 & 1\\
            0 & -3 & 3& -1\\
            0 & 0 & 0 & 3\\
            0 & 0 & 0&0\end{pmatrix}$,因此 $\alpha_1,\alpha_2,\alpha_3,\alpha_4$ 有非零解,这四个向量线性相关. (其实此处已知矩阵秩为 $3$,即维数是 $3$).

        再选取 $\{\alpha_1,\alpha_2,\alpha_4\}$ 来求解方程 $a_1\alpha_1+a_2\alpha_2+a_4\alpha_4=0$:
        \[\begin{pmatrix}
            1 & 1 & 1\\
            2 & -1 & 1\\
            4 & -6 & -2\\
            3 & 6 & 7\end{pmatrix}\rightarrow\begin{pmatrix}
            1 & 1 & 1\\
            0 & -3 & -2\\
            0 & 0 & 2\\
            0 & 0 & 0\end{pmatrix},\]
        因此该方程组只有全零解,即 $\{\alpha_1,\alpha_2,\alpha_4\}$ 是$\{\alpha_1,\alpha_2,\alpha_3,\alpha_4\}$  的极大线性无关组. 则 $\spa(\alpha_1,\alpha_2,\alpha_4)$ 的维数是 $3$. 一组基是 $\{\alpha_1,\alpha_2,\alpha_4\}$ .
        \item 也就是 $a_1\alpha_1+\cdots+a_4\alpha_4=\beta$有解:增广矩阵 $\begin{pmatrix}
            1 & 1 & -2&1&4\\
            2 & -1 & -1&1&2\\
            4 & -6 & 2&-2&4\\
            3 & 6 & -9&7&a\end{pmatrix}$化为 $\begin{pmatrix}
            1 & 1 & -2&1&4\\
            0 & -3 & 3&-1&-6\\
            0 & 0 & 0&-\frac 83&8\\
            0 & 0 & 0&0&a-9\end{pmatrix}$,若方程有解,则 $a=9$. 求坐标,取 $x_3=0$ 代入得 $\beta=4\alpha_1+3\alpha_2-3\alpha_4$.
    \end{enumerate}
    \item 只需证明 $B$ 线性无关即可. $\lambda_1+\lambda_2(x_a)+\lambda_3(x-a)^2$求导,增加方程数得到
	\[\lambda_2+2\lambda_3x=0,\]
	\[2\lambda_3=0,\]
	则 $\lambda_1=\lambda_2=\lambda_3$,线性无关得证. 又 $B$ 中向量个数等于 $R[x]_3$ 维数. 则 $B$ 是一组基. $1=1\times 1+0\times (x-a)+0\times(x-a)^2$ ,即 $(1,0,0)$;$x=a\times 1+1\times (x-a)+0\times(x-a)^2$ ,即 $(a,1,0)$;$x^2=a^2\times 1+2a\times (x-a)+1\times(x-a)^2$ ,即 $(a^2,2a,1)$.
    \item 等价于证明 $\alpha_1,\alpha_2,\alpha_3,\alpha_5-\alpha_4$ 线性无关. 即求解 $\lambda_1\alpha_1+\lambda_2\alpha_2+\lambda_3\alpha_3+\lambda_4(\alpha_5-\alpha_4)=0$ (*)

	由于 $r(A)=r(B)=3$ 可得 $A$ 线性无关. $B$ 线性相关. 由定理 $3.2$ 得 $\alpha_4$ 可由 $\alpha_1,\alpha_2,\alpha_3$ 唯一表示:$\alpha_4=\mu_1\alpha_1+\mu_2\alpha_2+\mu_3\alpha_3$.则代入 (*) 式. 有
	\[(\lambda_1-\mu_1\lambda_4)\alpha_1+(\lambda_2-\mu_2\lambda_4)\alpha_2+(\lambda_3-\mu_3\lambda_4)\alpha_3+\lambda_4\alpha_5=0,\]
	因为 $r(C)=4$ . $\{\alpha_1,\alpha_2,\alpha_3,\alpha_5\}$ 无关. 有 $\lambda_4=0,\lambda_1=\mu_1\lambda_4=0,\lambda_2=\mu_2\lambda_4=0,\lambda_3=\mu_3\lambda_4=0$. 故 $\alpha_1,\alpha_2,\alpha_3,\alpha_5-\alpha_4$ 线性无关. 原题得证.
    \item 相当于从 $\alpha_1,\cdots,\alpha_s$ 向量中选取 $s-m$ 个向量丢弃,剩余向量的秩:
	\[r(\{\alpha_{i1},\cdots,\alpha_{im}\})\ge r-(s-m) =r+m-s.\]
    \item 方程:$\lambda_1\alpha_1+\cdots+\lambda_n\alpha_n+\lambda_{n+1}\beta+\lambda_{n+2}\gamma=0$,显然 $\lambda_{n+1},\lambda_{n+2}$ 不全为零. 否则与 $\alpha_1,\cdots,\alpha_n$ 线性无关矛盾.
	\begin{enumerate}
        \item 若 $\lambda_{n+1}=0,\lambda_{n+2}\ne0$,则 $\gamma$ 可被 $\alpha_1,\cdots,\alpha_n$ 表示. 若 $\lambda_{n+1}\ne0,\lambda_{n+2}=0$, 则 $\beta$ 可被 $\alpha_1,\cdots,\alpha_n$ 表示.
        \item 若 $\lambda_{n+1}\lambda_{n+2}\ne 0$ ,则有
        \[\beta=-\frac 1{\lambda_{n+1}}(\lambda_1\alpha_1+\cdots+\lambda_n\alpha_n+\lambda_{n+2}\gamma),\]
        \[\gamma=-\frac 1{\lambda_{n+2}}(\lambda_1\alpha_1+\cdots+\lambda_n\alpha_n+\lambda_{n+1}\beta).\]
        两组向量可以相互表示. 两者等价. 综上原题得证.
    \end{enumerate}
\end{enumerate}

\centerline{\heiti B组}
\begin{enumerate}
    \item 充分性显然成立,下证必要性:由于 $\alpha_1,\cdots,\alpha_n$ 线性相关,则存在 $m$,其能使得 $\alpha_1,\cdots,\alpha_m$ 线性无关的最大下标,有 $1\le m<n$. 因此 $i=m+1$,$\alpha_1,\cdots,\alpha_{i-1}$ 线性无关,$\alpha_1,\cdots,\alpha_i$ 线性相关. 可得 $\alpha_i$ 可被 $\alpha_1,\cdots,\alpha_{i-1}$ 唯一表示.

    \item \begin{enumerate}
        \item 设 $U$ 的一组基为 $\{u_1,\cdots,u_m\}$,$W$ 的一组基为 $\{w_1,\cdots,w_n\}$. 由于 $U\subseteq W$,则 $\{u_1,\cdots,u_m\}$ 可由 $\{w_1,\cdots,w_n\}$ 线性表示,且 $\{u_1,\cdots,u_m\}$ 线性无关. 由定理 $3.3$ 的等价命题可得 $m\le n$,则 $m=\dim U\le\dim W=n$ 的得证.
        \item 因为 $\dim U=\dim W$,则 $\{u_1,\cdots,u_m\}$ 也是 $W$ 的一组基. 则 $W$ 的任意向量均可由 $\{u_1,\cdots,u_m\}$ 表示,可得 $W\subseteq U$,而 $U\subseteq W$,故有 $U=W$ 得证.
    \end{enumerate}

    \item 反证法. 若存在两个向量 $\alpha_i,\alpha_j$ 可被前面的向量表示.
	\[\alpha_i=\lambda_0\beta+\lambda_1\alpha_1+\cdots+\lambda_{i-1}\alpha_{i-1},\]
	\[\alpha_j=\mu_0\beta+\mu_1\alpha_1+\cdots+\mu_{j-1}\alpha_{j-1}.\]
	如果 $\lambda_0$ 或者 $\mu_0$ 为 $0$,则有向量组中的 $\alpha_i$ 或 $\alpha_j$ 可被其他向量线性表示,则该向量组相关,这与条件矛盾. 若 $\lambda_0$ 与 $\mu_0$ 均不为 $0$,等式可化为
	\[\frac 1{\lambda_0}\alpha_i=\beta+\frac{\lambda_1}{\lambda_0}\alpha_1+\cdots+\frac{\lambda_{i-1}}{\lambda_0}\alpha_{i-1},\]
	\[\frac 1{\mu_0}\alpha_j=\beta+\frac{\mu_1}{\mu_0}\alpha_1+\cdots+\frac{\mu_{j-1}}{\mu_0}\alpha_{j-1}.\]
	不妨设 $i>j$. 相减得
	\[\frac 1{\lambda_0}\alpha_i=(\frac{\lambda_1}{\lambda_0}-\frac{\mu_1}{\mu_0})\alpha_1+\cdots+(\frac{\lambda_i}{\lambda_0}+\frac 1{\mu_0})\alpha_j+\frac{\lambda_{i-1}}{\lambda_0}\alpha_{i-1},\]
	则 $\alpha_i$ 可被其他向量线性表示,因此向量组线性相关,与条件矛盾. 综上,至多有一个向量 $\alpha_i$ 可被前面的相邻线性表示.

    \item 考虑使用求导构造更多方程.
	\[\begin{cases}
		k_0+k_1\cdot e^{\lambda_1 x}+k_2\cdot e^{\lambda_2 x}=0 \\
		k_1\lambda_1\cdot e^{\lambda_1 x}+k_2\lambda_2\cdot e^{\lambda_2 x}=0 \\
		k_1\lambda_1^2\cdot e^{\lambda_1 x}+k_2\lambda_2^2\cdot e^{\lambda_2 x}=0
	\end{cases},\]
	由上述方程后两行可知$k_1k_2(\lambda_1-\lambda_0)=0$. 又 $\lambda_1\ne\lambda_2$,故$k_1=k_2=0$,代回$\textcircled{1}$ 得 $k_0=0$,则 $1,e^{\lambda_1x},e^{\lambda_2x}$ 线性无关,得证.

    \item 只需证明 $\alpha_r$ 可以被 $\alpha_1,\cdots,\alpha_{r-1},\beta$ 表示即可. 由于 $\beta$ 是 $\alpha_1,\cdots,\alpha_{r-1}$ 的线性组合,若 $\lambda_r=0$,则 $\beta$ 是 $\alpha_1,\cdots,\alpha_{r-1}$ 的线性组合. 这与条件矛盾. 因此 $\alpha_r=-\frac 1{\lambda_r}(\lambda_1\alpha_1+\cdots+\lambda_{r-1}\alpha_{r-1}-\beta)$,则这两组向量等价. $\spa(\alpha_1,\cdots,\alpha_{r-1},\alpha_r)=\spa(\alpha_1,\cdots,\alpha_{r-1},\beta)$ 得证.

    \item 分析该实线性空间,可以看出加法单位元为 $1$,数乘单位元为 $1$.我们给出一组基 $\{e\}$,其中 $e$ 为自然对数的底数. 当然, $2,3$ 或者 $10$ 都可以作为一组基. 接下来我们验证 $\{e\}$ 是 $\mathbb R^+$ 的基:$\forall a\in \mathbb R^+,\exists k=\ln a\in\mathbb R$,满足 $k\odot e=e^k=a$,则 $\spa(\{e\})=\mathbb R^+$ 成立.由于该向量组只有一个元素,且并非设该空间的零元 $1$,则 $\{e\}$ 是线性无关的. 得证.
\end{enumerate}

\centerline{\heiti C组}
\begin{enumerate}
    \item \begin{enumerate}
        \item 若 不全为 $0$. 不妨设设至少有 $k_i=0$,则有 $k_1\alpha_1+\cdots+k_{i-1}\alpha_{i-1}+k_{i+1}\alpha_{i+1}+\cdots+k_m\alpha_m=0$,并且系数不全为 $0$. 因此 $\alpha_1,\cdots,\alpha_{i-1},\alpha_{i+1},\cdots,\alpha_m$ 这 $m-1$ 个向量相关,与题设矛盾. 则原题得证.
        \item $l_1\ne 0$,则 $l_2,\cdots,l_m$ 均不为 $0$.
        \begin{enumerate}
            \item 若 $k_1=\cdots=k_m=0$. 原式显然成立.
            \item 若 $k_1=\cdots=k_m=0$ 不为 $0$. 则
            \begin{equation}\label{eq:3:C.1.1}
                k_1\alpha_1+\cdots+k_m\alpha_m=0,
            \end{equation}
            \begin{equation}\label{eq:3:C.1.2}
                l_1\alpha_1+\cdots+l_m\alpha_m=0.
            \end{equation}
            我们有\autoref{eq:3:C.1.1}$\cdot l_1-$\autoref{eq:3:C.1.2}$\cdot k_1$ 可得
            \[(k_2l_1-k_1l_2)\alpha_2+\cdots+(k_ml_1-k_1l_m)\alpha_m=0,\]
            因为 $\alpha_2,\cdots,\alpha_m$ 线性无关. 则以上系数均为 $0$. 故$\dfrac {k_2}{l_2}=\dfrac {k_1}{l_1},\cdots,\dfrac {k_m}{l_m}=\dfrac {k_1}{l_1}$. 得证.
        \end{enumerate}
    \end{enumerate}

    \item 利用递推法:当 $r=1$ 时,由于 $\alpha_1$ 线性无关,可得 $\alpha_1\ne 0$. 设 $\alpha_1=\lambda_1\beta_1+\cdots+\lambda_n\beta_n$,则至少存在一个 $\lambda_i\ne 0$,不妨设 $\lambda_1\ne 0$,因此有 $\beta_1=-\frac 1{\lambda_1}(-\alpha_1+\lambda_2\beta_2+\cdots+\lambda_n\beta_n)$,故 $\{\beta_1,\cdots,\beta_n\}$ 与 $\{\alpha_1,\beta_2,\cdots,\beta_n\}$ 等价.

	当 $r=2$ 时,由于 $\alpha_1,\alpha_2$ 无关. 有 $\alpha_1\ne 0,\alpha_2\ne 0$. 根据 $r=1$ 的情况,不妨设 $\{\beta_1,\cdots,\beta_n\}$ 与 $\{\alpha_1,\beta_2,\cdots,\beta_n\}$ 等价. 因此 $\alpha_2$ 可由 $\{\alpha_1,\beta_2,\cdots,\beta_n\}$ 表出:
	\[\alpha_2=\mu_1\alpha_1+\mu_2\beta_2+\cdots+\mu_n\beta_n.\]
	由于 $\alpha_1,\alpha_2$ 无关,故 $\mu_2,\cdots,\mu_n$ 至少有一个非零的数. 不妨设 $\mu_2\ne 0$,同上可得$\{\alpha_1,\alpha_2,\beta_3,\cdots,\beta_n\}$ 就与 $\{\alpha_1,\beta_2,\cdots,\beta_n\}$ 等价,也与$\{\beta_1,\beta_2,\cdots,\beta_n\}$ 等价. 综上,通过递推可知,对正整数 $r$,上述结论依然成立.

    \item \begin{enumerate}
        \item $\alpha=(1,1,\cdots,1)^T, W=\spa(\alpha)$,显然 $W$ 是满足条件的一维子空间.
        \item 考虑反证法:若 $\dim W>1$,则 $W$ 中存在线性无关的两向量. 由条件,
        \[a_1,\cdots,a_n,b_1,\cdots,b_n\ne 0,\]
        可设 $a_1=kb_1,k\in F$,因此 $\alpha-k\beta=(0,a_2-kb-2,\cdots,a_n-kb_n)^T\in W$. 且由于 $\alpha,\beta$ 无关,$\alpha-k\beta\ne 0$ 但存在分量为 $0$,这与条件矛盾. 故 $\dim W=1$.
        \item 考虑反证法:若 $\dim W>r+1$,则存在 $r+2$ 个线性无关的向量,设为
        \[\alpha_1=(a_{11},a_{12},\cdots,a_{1(r+1)},\cdots,a_{1n})^T,\]
        \[\vdots\]
        \[\alpha_{r+2}=(a_{(r+2)1},a_{(r+2)2},\cdots,a_{(r+2)(r+1)},\cdots,a_{(r+2)n})^T.\]
        取这些向量的前 $r+1$ 个分量组成新的向量组:
        \[\beta_1=(a_{11},a_{12},\cdots,a_{1(r+1)})^T,\]
        \[\vdots\]
        \[\beta_{r+2}=(a_{(r+2)1},a_{(r+2)2},\cdots,a_{(r+2)(r+1)})^T.\]
        由于 $\beta_1,\cdots,\beta_{r+2}$ 是 $r+2$ 个 $r+1$ 维向量,其必然线性相关,则存在不全为 $0$ 的系数 $\lambda_1,\cdots,\lambda_{r+2}$,$\lambda_1\beta_1+\cdots+\lambda_{r+2}\beta_{r+2}=0.$ 由于 $\alpha_1,\cdots,\alpha_{r+2}$ 线性无关知:$\lambda_1\alpha_1+\cdots+\lambda_{r+2}\alpha_{r+2}\ne 0$,且其属于 $W$. 但其前 $r+1$ 个分量均为 $0$,这与条件矛盾. 故 $\dim W\le r+1$ 得证.
    \end{enumerate}
\end{enumerate}

\clearpage
