\phantomsection
\section*{3 有限维线性空间}
\addcontentsline{toc}{section}{3 有限维线性空间}

\vspace{2ex}

\centerline{\heiti A组}
\begin{enumerate}
    \item \begin{enumerate}
              \item 错. 反例:$\alpha_1=(1,0),\alpha_2=(2,0),\alpha_3=(0,1)$,则 $\alpha_1,\alpha_2,\alpha_3$ 线性相关而 $\alpha_3$ 不是 $\alpha_1.\alpha_2$ 的线性组合.

              \item 对. 该命题的等价命题(逆否命题)是:若存在一个向量是其余向量的线性组合,则 $\alpha_1,\ldots,\alpha_m$ 线性相关. 这正是定理 3.1 的内容,因而成立.

              \item 错. 反例:$\alpha_1=(1,0),\alpha_2=(0,1),\alpha_3=(1,1)$,则 $\alpha_1,\alpha_2,\alpha_3$ 两两无关,而三者线性相关. 可证两两无关是向量组无关的必要条件.

              \item 错. 反例:$\alpha_1=(1,0),\alpha_2=(0,0),\beta_1=(0,0),\beta_2=(0,1)$,有 $\alpha_1,\alpha_2$ 相关,$\beta_1,\beta_2$ 相关,而 $\alpha_1+\beta_1$ 与 $\alpha_2+\beta_2$ 线性无关.

              \item 错. 若 $\alpha_1,\ldots,\alpha_n$ 线性无关,有
                    \[\lambda_1\alpha_1+\cdots+\lambda_n\alpha_n\implies\lambda_1=\lambda_2=\cdots=\lambda_n=0,\]
                    判断 $\alpha_1+\alpha_2,\alpha_2+\alpha_3,\ldots,\alpha_n+\alpha_1$ 是否无关. 设
                    \[\lambda_1'(\alpha_1+\alpha_2)+\cdots+\lambda_n'(\alpha_n+\alpha_1)=0,\]
                    则
                    \[(\lambda_n'+\lambda_1')\alpha_1+(\lambda_1'+\lambda_2')\alpha_2+\cdots+(\lambda_{n-1}'+\lambda_{n}')\alpha_n=0,\]
                    则
                    \[\implies \begin{cases} \begin{aligned}
                                \lambda_n'+\lambda_1'       & = 0               \\
                                                            & \vdotswithin{ = } \\
                                \lambda_{n-1}'+\lambda_{n}' & = 0               \\
                            \end{aligned} \end{cases}.\]
                    解该方程可得 $\lambda_n'=(-1)^n\lambda_1'$,因此当 $n$ 为偶数时,上述方程组有非零解,则向量组相关,而当 $n$ 为奇数时,向量组无关. 综上,该命题不成立.

              \item 对. 由定理 3.1,不妨设 $\alpha_3$ 可由 $\alpha_1,\alpha_2$ 线性表示,则 $\alpha_1+\alpha_2,\alpha_2+\alpha_3,\alpha_3+\alpha_1$ 均可由 $\alpha_1,\alpha_2$ 线性表示,再由定理 3.3 可知,$\alpha_1+\alpha_2,\alpha_2+\alpha_3,\alpha_3+\alpha_1$ 线性相关.

              \item 错. 反例:取 $\alpha_0=\alpha_1-\alpha_2-\alpha_3$,则 $\alpha_0+\alpha_1=(\alpha_0+\alpha_2)+(\alpha_0+\alpha_3)$,三者线性相关,不是 $\mathbf{R}^3$ 的基.

              \item 对. 判断 $\alpha_1+\alpha_2$ 与 $\alpha_1-\alpha_2$ 是否无关.
                    \[\lambda_1(\alpha_1+\alpha_2)+\lambda_2(\alpha_1-\alpha_2)=0\]
                    则有$(\lambda_1+\lambda_2)\alpha_1+(\lambda_1-\lambda_2)\alpha_2=0$,则 $\lambda_1+\lambda_2=0,\lambda_1-\lambda_2=0\implies \lambda_1=\lambda_2=0$,因此线性无关且个数等于维数,是一组基.

              \item 错. 反例:$\mathbf{R}^2$ 中过原点的直线 $L_0$ 是 $\mathbf{R}^2$ 的一个子空间. 显然这样的直线有无数条.

              \item 错. 反例:$\mathbf{R}^3$ 中,子空间 $W_1=\spa(e_1,e_2)$,$W_2=\spa(e_1+e_2,e_3)$,则 $B_1\cup B_2=\{e_1,e_2,e_3,e_1+e_2\}$,显然 $\mathbf{R}^3$ 中的任一组基都不可能包含四个元素.
          \end{enumerate}

    \item 设原向量组为 $\alpha_1,\alpha_2,\ldots,\alpha_n$. 每个向量有 $s$ 个分量 $(s>m)$. 对于齐次线性方程组:$x_1\alpha_1+x_2\alpha_2+\cdots+x_n\alpha_n=0$.
          \begin{enumerate}
              \item 若向量组线性相关,则对应该方程组有无穷多解. 去掉 $m$ 个分量,相当于删去该方程组中的任意 $m$ 行方程,依然有无穷多解. 这是因为对于原方程组的任意一个解,将其带入被削减后的方程组也依然成立. 故线性相关得证.

              \item 若向量组线性无关,对应原方程组仅有唯一解,也就是全零解. 增加 $m$ 个分量相当于增加 $m$ 个方程,依然只有唯一解,因为若出现非零解,代入原方程组对应的方程中不会成立,矛盾. 故线性无关得证.
          \end{enumerate}

    \item 方程组:$x_1\beta_1+x_2\beta_2+x_3\beta_3=0$ 系数矩阵
          \[\begin{pmatrix}
                  1 & 2  & 1  \\
                  3 & 1  & -1 \\
                  6 & 2  & a  \\
                  2 & -1 &
                  -2\end{pmatrix}\rightarrow\begin{pmatrix}
                  1 & 2   & 1   \\
                  0 & -5  & -4  \\
                  0 & -10 & a-6 \\
                  0 & -6  & -4
              \end{pmatrix}\rightarrow\begin{pmatrix}
                  1 & 2  & 1   \\
                  0 & -5 & -4  \\
                  0 & 0  & a+2 \\
                  0 & 0  & 0
              \end{pmatrix},\]
          仅全零解的条件是 $a\neq-2$,此时向量组线性无关.

    \item \begin{enumerate}
              \item 必要性:$\alpha_1,\ldots,\alpha_n$ 线性无关,对于 $F^n$ 中的任一向量 $\beta$, $\alpha_1,\ldots,\alpha_n,\beta$ 的向量个数大于维数 $n$,则线性相关. 由定理 3.2,$\beta$ 可被 $\alpha_1,\ldots,\alpha_n$ 唯一表示.

              \item 充分性:由于 $F^n$ 中任意向量均可被 $\alpha_1,\ldots,\alpha_n$ 线性表示,并且向量个数等于维数. 则 $\alpha_1,\ldots,\alpha_n$ 是 $F^n$ 的一组基. 则 $\alpha_1,\ldots,\alpha_n$ 线性无关.

                    $^*$ 更详细的证明:对于 $F^n$ 的一组基 $e_1,\ldots,e_n$,其可被 $\alpha_1,\ldots,\alpha_n$ 表示. 若 $\alpha_1,\ldots,\alpha_n$ 线性相关,不妨设 $\alpha_n$ 可被 $\alpha_1,\ldots,\alpha_{n-1}$ 表示,则有 $e_1,\ldots,e_n$ 可被 $\alpha_1,\ldots,\alpha_{n-1}$ 表示. 由于 $e_1,\ldots,e_n$ 线性无关. 根据定理 3.3,$n\leqslant n-1$,矛盾. 因此得证.
          \end{enumerate}

    \item \begin{enumerate}
              \item 必要性:对于 $\forall v\in\spa(S_1)\cap\spa(S_2)$ 有 $v=a_1\alpha_1+\cdots+a_s\alpha_s=b_1\beta_1+\cdots+b_t\beta_t$. 由于 $\alpha_1,\ldots,\alpha_s,\beta_1,\ldots,\beta_t$ 线性无关 $\implies a_1=\cdots=a_s=b_1=\cdots=b_t=0$,则 $v=0$,即 $\spa(S_1)\cap\spa(S_2)=\{0\}$.

              \item 充分性:考虑反证法. 如果 $\alpha_1,\ldots,\alpha_s,\beta_1,\ldots,\beta_t$ 线性相关,则存在不全为零的系数使得 $a_1\alpha_1+\cdots+a_s\alpha_s+b_1\beta_1\cdots+b_t\beta_t=0$. 因此存在一个向量 $v=a_1\alpha_1+\cdots+a_s\alpha_s=-(b_1\beta_1\cdots+b_t\beta_t)\neq 0$ 且 $v\in\spa(S_1),v\in\spa(S_2)$. 即存在非零向量 $v$ 属于 $\spa(S_1),v\in\spa(S_2)$,矛盾!则充分性得证.
          \end{enumerate}

    \item \begin{enumerate}
              \item 也就是求 $\alpha_1,\alpha_2,\alpha_3,\alpha_4$ 的极大线性无关组. 利用讲义中所述求法:方程组
                    \[ a_1\alpha_1+\cdots+a_4\alpha_4=0 \]
                    对应系数矩阵 $\begin{pmatrix}
                            1 & 1  & -2 & 1  \\
                            2 & -1 & -1 & 1  \\
                            4 & -6 & 2  & -2 \\
                            3 & 6  & -9 & 7\end{pmatrix}$. 化简为行阶梯型:$\begin{pmatrix}
                            1 & 1  & -2 & 1  \\
                            0 & -3 & 3  & -1 \\
                            0 & 0  & 0  & 3  \\
                            0 & 0  & 0  & 0\end{pmatrix}$,因此 $\alpha_1,\alpha_2,\alpha_3,\alpha_4$ 有非零解,这四个向量线性相关. (其实此处已知矩阵秩为 3,即维数是 3).

                    再选取 $\alpha_1,\alpha_2,\alpha_4$ 来求解方程 $a_1\alpha_1+a_2\alpha_2+a_4\alpha_4=0$:
                    \[\begin{pmatrix}
                            1 & 1  & 1  \\
                            2 & -1 & 1  \\
                            4 & -6 & -2 \\
                            3 & 6  & 7\end{pmatrix}\rightarrow\begin{pmatrix}
                            1 & 1  & 1  \\
                            0 & -3 & -2 \\
                            0 & 0  & 2  \\
                            0 & 0  & 0\end{pmatrix},\]
                    因此该方程组只有全零解,即 $\alpha_1,\alpha_2,\alpha_4$ 是$\alpha_1,\alpha_2,\alpha_3,\alpha_4$的极大线性无关组. 则 $\spa(\alpha_1,\alpha_2,\alpha_4)$ 的维数是 3. 一组基是 $\alpha_1,\alpha_2,\alpha_4$ .

              \item 也就是 $a_1\alpha_1+\cdots+a_4\alpha_4=\beta$有解:增广矩阵 $\begin{pmatrix}
                            1 & 1  & -2 & 1  & 4 \\
                            2 & -1 & -1 & 1  & 2 \\
                            4 & -6 & 2  & -2 & 4 \\
                            3 & 6  & -9 & 7  & a\end{pmatrix}$化为 $\begin{pmatrix}
                            1 & 1  & -2 & 1         & 4   \\
                            0 & -3 & 3  & -1        & -6  \\
                            0 & 0  & 0  & -\frac 83 & 8   \\
                            0 & 0  & 0  & 0         & a-9\end{pmatrix}$,若方程有解,则 $a=9$. 求坐标,取 $x_3=0$ 代入得 $\beta=4\alpha_1+3\alpha_2-3\alpha_4$.
          \end{enumerate}

    \item 只需证明 $B$ 线性无关即可. $\lambda_1+\lambda_2(x_a)+\lambda_3(x-a)^2$求导,增加方程数得到
          \begin{gather*}
              \lambda_2+2\lambda_3x=0 \\
              2\lambda_3=0
          \end{gather*}
          则 $\lambda_1=\lambda_2=\lambda_3$,线性无关得证. 又 $B$ 中向量个数等于 $R[x]_3$ 维数. 则 $B$ 是一组基. $1=1 \dot 1+0 \dot (x-a)+0\times(x-a)^2$ ,即 $(1,0,0)$;$x=a \dot 1+1 \dot (x-a)+0\times(x-a)^2$ ,即 $(a,1,0)$;$x^2=a^2 \dot 1+2a \dot (x-a)+1\times(x-a)^2$ ,即 $(a^2,2a,1)$.

    \item 等价于证明 $\alpha_1,\alpha_2,\alpha_3,\alpha_5-\alpha_4$ 线性无关. 即求解
          \begin{equation}
              \lambda_1\alpha_1+\lambda_2\alpha_2+\lambda_3\alpha_3+\lambda_4(\alpha_5-\alpha_4)=0 \tag{*} \label{eq:3:A.10}
          \end{equation}

          由于 $r(A)=r(B)=3$ 可得 $A$ 线性无关. $B$ 线性相关. 由定理 3.2 得 $\alpha_4$ 可由 $\alpha_1,\alpha_2,\alpha_3$ 唯一表示:$\alpha_4=\mu_1\alpha_1+\mu_2\alpha_2+\mu_3\alpha_3$. 则代入 (\ref*{eq:3:A.10}) 式. 有
          \[(\lambda_1-\mu_1\lambda_4)\alpha_1+(\lambda_2-\mu_2\lambda_4)\alpha_2+(\lambda_3-\mu_3\lambda_4)\alpha_3+\lambda_4\alpha_5=0,\]
          因为 $r(C)=4$,$\alpha_1,\alpha_2,\alpha_3,\alpha_5$ 线性无关. 有 $\lambda_4=0,\lambda_1=\mu_1\lambda_4=0,\lambda_2=\mu_2\lambda_4=0,\lambda_3=\mu_3\lambda_4=0$. 故 $\alpha_1,\alpha_2,\alpha_3,\alpha_5-\alpha_4$ 线性无关. 原题得证.

    \item 相当于从 $\alpha_1,\ldots,\alpha_s$ 向量中选取 $s-m$ 个向量丢弃,剩余向量的秩:
          \[r(\alpha_{i1},\ldots,\alpha_{im})\geqslant r-(s-m) =r+m-s.\]

    \item 方程:$\lambda_1\alpha_1+\cdots+\lambda_n\alpha_n+\lambda_{n+1}\beta+\lambda_{n+2}\gamma=0$,显然 $\lambda_{n+1},\lambda_{n+2}$ 不全为零. 否则与 $\alpha_1,\ldots,\alpha_n$ 线性无关矛盾.
          \begin{enumerate}
              \item 若 $\lambda_{n+1}=0,\lambda_{n+2}\neq 0$,则 $\gamma$ 可被 $\alpha_1,\ldots,\alpha_n$ 表示. 若 $\lambda_{n+1}\neq0,\lambda_{n+2}=0$, 则 $\beta$ 可被 $\alpha_1,\ldots,\alpha_n$ 表示.

              \item 若 $\lambda_{n+1}\lambda_{n+2}\neq 0$ ,则有
                    \begin{gather*}
                        \beta=-\frac 1{\lambda_{n+1}}(\lambda_1\alpha_1+\cdots+\lambda_n\alpha_n+\lambda_{n+2}\gamma) \\
                        \gamma=-\frac 1{\lambda_{n+2}}(\lambda_1\alpha_1+\cdots+\lambda_n\alpha_n+\lambda_{n+1}\beta)
                    \end{gather*}
                    两组向量可以相互表示. 两者等价. 综上原题得证.
          \end{enumerate}
\end{enumerate}

\centerline{\heiti B组}
\begin{enumerate}
    \item 充分性显然成立,下证必要性:由于 $\alpha_1,\ldots,\alpha_n$ 线性相关,则存在 $m$,其能使得 $\alpha_1,\ldots,\alpha_m$ 线性无关的最大下标,有 $1\leqslant m<n$. 因此 $i=m+1$,$\alpha_1,\ldots,\alpha_{i-1}$ 线性无关,$\alpha_1,\ldots,\alpha_i$ 线性相关. 可得 $\alpha_i$ 可被 $\alpha_1,\ldots,\alpha_{i-1}$ 唯一表示.

    \item \begin{enumerate}
              \item 设 $U$ 的一组基为 $u_1,\ldots,u_m$,$W$ 的一组基为 $w_1,\ldots,w_n$. 由于 $U\subseteq W$,则 $u_1,\ldots,u_m$ 可由 $w_1,\ldots,w_n$ 线性表示,且 $u_1,\ldots,u_m$ 线性无关. 由定理 3.3 的等价命题可得 $m\leqslant n$,则 $m=\dim U\leqslant\dim W=n$ 的得证.

              \item 因为 $\dim U=\dim W$,则 $u_1,\ldots,u_m$ 也是 $W$ 的一组基. 则 $W$ 的任意向量均可由 $u_1,\ldots,u_m$ 表示,可得 $W\subseteq U$,而 $U\subseteq W$,故有 $U=W$ 得证.
          \end{enumerate}

    \item 反证法. 若存在两个向量 $\alpha_i,\alpha_j$ 可被前面的向量表示.
          \begin{gather*}
              \alpha_i=\lambda_0\beta+\lambda_1\alpha_1+\cdots+\lambda_{i-1}\alpha_{i-1} \\
              \alpha_j=\mu_0\beta+\mu_1\alpha_1+\cdots+\mu_{j-1}\alpha_{j-1}
          \end{gather*}
          如果 $\lambda_0$ 或者 $\mu_0$ 为 0,则有向量组中的 $\alpha_i$ 或 $\alpha_j$ 可被其他向量线性表示,则该向量组相关,这与条件矛盾. 若 $\lambda_0$ 与 $\mu_0$ 均不为 0,等式可化为
          \begin{gather*}
              \frac 1{\lambda_0}\alpha_i=\beta+\frac{\lambda_1}{\lambda_0}\alpha_1+\cdots+\frac{\lambda_{i-1}}{\lambda_0}\alpha_{i-1} \\
              \frac 1{\mu_0}\alpha_j=\beta+\frac{\mu_1}{\mu_0}\alpha_1+\cdots+\frac{\mu_{j-1}}{\mu_0}\alpha_{j-1}
          \end{gather*}
          不妨设 $i>j$. 相减得
          \[\frac 1{\lambda_0}\alpha_i=\left(\frac{\lambda_1}{\lambda_0}-\frac{\mu_1}{\mu_0}\right)\alpha_1+\cdots+\left(\frac{\lambda_i}{\lambda_0}+\frac 1{\mu_0}\right)\alpha_j+\frac{\lambda_{i-1}}{\lambda_0}\alpha_{i-1},\]
          则 $\alpha_i$ 可被其他向量线性表示,因此向量组线性相关,与条件矛盾. 综上,至多有一个向量 $\alpha_i$ 可被前面的相邻线性表示.

    \item 考虑使用求导构造更多方程.
          \[\begin{cases}
                  k_0+k_1\cdot e^{\lambda_1 x}+k_2\cdot e^{\lambda_2 x}=0               \\
                  k_1\lambda_1\cdot e^{\lambda_1 x}+k_2\lambda_2\cdot e^{\lambda_2 x}=0 \\
                  k_1\lambda_1^2\cdot e^{\lambda_1 x}+k_2\lambda_2^2\cdot e^{\lambda_2 x}=0
              \end{cases},\]
          由后两式可知$k_1k_2(\lambda_1-\lambda_0)=0$. 又 $\lambda_1\neq\lambda_2$,故$k_1=k_2=0$,代回第一式得 $k_0=0$,则 $1,e^{\lambda_1x},e^{\lambda_2x}$ 线性无关,得证.

    \item 只需证明 $\alpha_r$ 可以被 $\alpha_1,\ldots,\alpha_{r-1},\beta$ 表示即可. 由于 $\beta$ 是 $\alpha_1,\ldots,\alpha_{r-1}$ 的线性组合,若 $\lambda_r=0$,则 $\beta$ 是 $\alpha_1,\ldots,\alpha_{r-1}$ 的线性组合. 这与条件矛盾. 因此 $\alpha_r=-\vspace{1ex}\dfrac 1{\lambda_r}(\lambda_1\alpha_1+\cdots+\lambda_{r-1}\alpha_{r-1}-\beta)$,则这两组向量等价. $\spa(\alpha_1,\ldots,\alpha_{r-1},\alpha_r)=\spa(\alpha_1,\ldots,\alpha_{r-1},\beta)$ 得证.

    \item 分析该实线性空间,可以看出加法单位元为 1,数乘单位元为 1. 我们给出一组基:$e$,其中 $e$ 为自然对数的底数. 当然, 2,3 或者 10 都可以作为一组基. 接下来我们验证 $e$ 是 $\mathbf{R}^+$ 的基:$\forall a\in \mathbf{R}^+,\exists k=\ln a\in\mathbf{R}$,满足 $k\odot e=e^k=a$,则 $\spa(e)=\mathbf{R}^+$ 成立. 由于该向量组只有一个元素,且并非设该空间的零元 1,则 $e$ 是线性无关的. 得证.
\end{enumerate}

\centerline{\heiti C组}
\begin{enumerate}
    \item \begin{enumerate}
              \item 若 不全为 0. 不妨设设至少有 $k_i=0$,则有 $k_1\alpha_1+\cdots+k_{i-1}\alpha_{i-1}+k_{i+1}\alpha_{i+1}+\cdots+k_m\alpha_m=0$,并且系数不全为 0. 因此 $\alpha_1,\ldots,\alpha_{i-1},\alpha_{i+1},\ldots,\alpha_m$ 这 $m-1$ 个向量相关,与题设矛盾. 则原题得证.

              \item $l_1\neq 0$,则 $l_2,\ldots,l_m$ 均不为 0.
                    \begin{enumerate}
                        \item 若 $k_1=\cdots=k_m=0$. 原式显然成立.

                        \item 若 $k_1,\ldots,k_m$ 不全为 0. 则
                              \begin{gather*}
                                  l_1(k_1\alpha_1+\cdots+k_m\alpha_m)=0 \\
                                  k_1(l_1\alpha_1+\cdots+l_m\alpha_m)=0
                              \end{gather*}
                              两式相减,得
                              \[(k_2l_1-k_1l_2)\alpha_2+\cdots+(k_ml_1-k_1l_m)\alpha_m=0,\]
                              因为 $\alpha_2,\ldots,\alpha_m$ 线性无关. 则以上系数均为 0. 故$\dfrac {k_2}{l_2}=\dfrac {k_1}{l_1},\ldots,\dfrac {k_m}{l_m}=\dfrac {k_1}{l_1}$. 得证.
                    \end{enumerate}
          \end{enumerate}

    \item 利用递推法:当 $r=1$ 时,由于 $\alpha_1$ 线性无关,可得 $\alpha_1\neq 0$. 设 $\alpha_1=\lambda_1\beta_1+\cdots+\lambda_n\beta_n$,则至少存在一个 $\lambda_i\neq 0$,不妨设 $\lambda_1\neq 0$,因此有 $\beta_1=-\vspace{1ex}\dfrac 1{\lambda_1}(-\alpha_1+\lambda_2\beta_2+\cdots+\lambda_n\beta_n)$,故 $\beta_1,\ldots,\beta_n$ 与 $\alpha_1,\beta_2,\ldots,\beta_n$ 等价.

          当 $r=2$ 时,由于 $\alpha_1,\alpha_2$ 无关. 有 $\alpha_1\alpha_2\neq 0$. 根据 $r=1$ 的情况,不妨设 $\beta_1,\ldots,\beta_n$ 与 $\alpha_1,\beta_2,\ldots,\beta_n$ 等价. 因此 $\alpha_2$ 可由 $\alpha_1,\beta_2,\ldots,\beta_n$ 表出:
          \[\alpha_2=\mu_1\alpha_1+\mu_2\beta_2+\cdots+\mu_n\beta_n.\]
          由于 $\alpha_1,\alpha_2$ 无关,故 $\mu_2,\ldots,\mu_n$ 至少有一个非零的数. 不妨设 $\mu_2\neq 0$,同上可得$\alpha_1,\alpha_2,\beta_3,\ldots,\beta_n$ 就与 $\alpha_1,\beta_2,\ldots,\beta_n$ 等价,也与$\beta_1,\beta_2,\ldots,\beta_n$ 等价. 综上,通过递推可知,对正整数 $r$,上述结论依然成立.

    \item \begin{enumerate}
              \item $\alpha=(1,1,\ldots,1)^{\mathrm{T}}, W=\spa(\alpha)$,显然 $W$ 是满足条件的一维子空间.

              \item 考虑反证法:若 $\dim W>1$,则 $W$ 中存在线性无关的两向量. 由条件,
                    \[a_1,\ldots,a_n,b_1,\ldots,b_n\neq 0,\]
                    可设 $a_1=kb_1,k\in F$,因此 $\alpha-k\beta=(0,a_2-kb-2,\ldots,a_n-kb_n)^{\mathrm{T}}\in W$. 且由于 $\alpha,\beta$ 无关,$\alpha-k\beta\neq 0$ 但存在分量为 0,这与条件矛盾. 故 $\dim W=1$.

              \item 考虑反证法:若 $\dim W>r+1$,则存在 $r+2$ 个线性无关的向量,设为
                    \begin{align*}
                        \alpha_1     & = (a_{11},a_{12},\ldots,a_{1(r+1)},\ldots,a_{1n})^{\mathrm{T}}                 \\
                                     & \vdotswithin{ = }                                                              \\
                        \alpha_{r+2} & = (a_{(r+2)1},a_{(r+2)2},\ldots,a_{(r+2)(r+1)},\ldots,a_{(r+2)n})^{\mathrm{T}}
                    \end{align*}
                    取这些向量的前 $r+1$ 个分量组成新的向量组:
                    \begin{align*}
                        \beta_1     & = (a_{11},a_{12},\ldots,a_{1(r+1)})^{\mathrm{T}}             \\
                                    & \vdotswithin{ = }                                            \\
                        \beta_{r+2} & = (a_{(r+2)1},a_{(r+2)2},\ldots,a_{(r+2)(r+1)})^{\mathrm{T}}
                    \end{align*}
                    由于 $\beta_1,\ldots,\beta_{r+2}$ 是 $r+2$ 个 $r+1$ 维向量,其必然线性相关,则存在不全为 0 的系数 $\lambda_1,\ldots,\lambda_{r+2}$,$\lambda_1\beta_1+\cdots+\lambda_{r+2}\beta_{r+2}=0$.  由于 $\alpha_1,\ldots,\alpha_{r+2}$ 线性无关知:$\lambda_1\alpha_1+\cdots+\lambda_{r+2}\alpha_{r+2}\neq 0$,且其属于 $W$. 但其前 $r+1$ 个分量均为 0,这与条件矛盾. 故 $\dim W\leqslant r+1$ 得证.
          \end{enumerate}
\end{enumerate}

\clearpage
