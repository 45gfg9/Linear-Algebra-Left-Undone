\phantomsection
\section*{2 线性空间}
\addcontentsline{toc}{section}{2 线性空间}

\vspace{2ex}

\centerline{\heiti A组}
\begin{enumerate}
    \item \begin{enumerate}
              \item 有理数集 $Q$ 关于实数乘法不封闭,不构成实数域上的线性空间.

              \item $\mathbf{R}^2$ 关于通常向量加法构成交换群,封闭性也显然成立. 再看数乘.
                    \begin{enumerate}
                        \item $\exists \lambda=1$ 使得 $\lambda\cdot(x,y)=(\lambda x,y)=(x,y)$.

                        \item $\lambda(\mu\cdot(x,y))=\lambda\cdot(\mu x,y)=((\lambda\mu)x,y)=(\lambda\mu)\cdot(x,y)$.

                        \item $(\lambda+\mu)\cdot(x,y)=((\lambda+\mu)x,y)=(\lambda x,y)+(\mu x,y)$. 因此$(\lambda+\mu)\cdot(x,y)=\lambda\cdot(x,y)+\mu\cdot(x,y)$ 成立.

                        \item $\lambda((x_1,y_1)+(x_2,y_2))=\lambda\cdot(x_1+x_2,y_1+y_2)=(\lambda x_1+\lambda x_2,y_1+y_2)=(\lambda x_1,y_1)+(\lambda x_2,y_2)$,因此$\lambda((x_1,y_1)+(x_2,y_2))=(\lambda x_1,y_1)+(\lambda x_2,y_2)$.

                        \item (封闭性)$\forall \lambda \in \mathbf{R},\lambda\cdot(x,y)=(\lambda x,y)\in \mathbf{R}^2$,封闭性满足.

                              综上,$\mathbf{R}^2$ 关于通常向量加法与该数乘构成实数域上的向量空间.
                    \end{enumerate}

              \item \begin{enumerate}
                        \item 对于加法,显然,封闭性,结合律,交换律成立. 存在加法单位元 $(1,1,\ldots,1)$ 有
                              \begin{align*}
                                  (1,1,\ldots,1)+(a_1,a_2,\ldots,a_n) & = (a_1,a_2,\ldots,a_n)+(1,1,\ldots,1) \\
                                                                      & = (a_1,a_2,\ldots,a_n)
                              \end{align*}
                              由于为正实数向量,则对于 $(a_1,\ldots,a_n)$,存在唯一的逆元 $\left(\dfrac 1{a_1},\ldots,\dfrac 1{a_n}\right)$,使得 $(a_1,\ldots,a_n)+\left(\dfrac 1{a_1},\ldots,\dfrac 1{a_n}\right)=(1,\ldots,1)$.

                        \item 对于数乘,显然有封闭性成立,乘法单位元为 $\lambda_0=1$. 又有
                              \begin{enumerate}
                                  \item \begin{align*}
                                                & \lambda(\mu\cdot(a_1,\ldots,a_n))            \\
                                            ={} & \lambda\cdot(a_1^\mu,\ldots,a_n^\mu)         \\
                                            ={} & ((a_1^\mu)^\lambda,\ldots,(a_n^\mu)^\lambda) \\
                                            ={} & (a_1^{\lambda\mu},\ldots,a_n^{\lambda\mu})
                                        \end{align*}
                                        因此 $\lambda(\mu\cdot(a_1,\ldots,a_n))=(\lambda\mu)\cdot(a_1,\ldots,a_n)$ 成立.

                                  \item \begin{align*}
                                                & (\lambda+\mu)\cdot(a_1,\ldots,a_n)                         \\
                                            ={} & (a_1^{\mu+\lambda},\ldots,a_n^{\mu+\lambda})               \\
                                            ={} & (a_1^\lambda a_1^\mu,\ldots,a_n^\lambda a_n^\mu)           \\
                                            ={} & (a_1^\lambda,\ldots,a_n^\lambda)+(a_1^\mu,\ldots,a_n^\mu),
                                        \end{align*}
                                        因此 $(\lambda+\mu)\cdot(a_1,\ldots,a_n)=\lambda\cdot(a_1,\ldots,a_n)+\mu(a_1,\ldots,a_n)$,第一个加号为数的加法,第二个加号为定义的向量加法.

                                  \item $\lambda\cdot((a_1,\ldots,a_n)+(b_1,\ldots,b_n))=\lambda\cdot(a_1b_1,\ldots,a_nb_n)=(a_1^\lambda b_1^\lambda,\ldots,a_n^\lambda b_n^\lambda)=(a_1^\lambda,\ldots,a_n^\lambda)+(b_1^\lambda,\ldots,b_n^\lambda)$,因此 $\lambda\cdot((a_1,\ldots,a_n)+(b_1,\ldots,b_n))=\lambda\cdot(a_1,\ldots,a_n)+\lambda\cdot(b_1,\ldots,b_n)$.
                              \end{enumerate}
                              综上有 $\mathbf{R}_+^n$ 对如下加法和数乘构成实数域线性空间.
                    \end{enumerate}

              \item (教材第二章,习题第一题 9--11 小题,仅验证部分性质,其余请读者自行完成,或对照《大学数学·代数与几何课后习题解答》进行求解)
                    \begin{enumerate}[start=9,label={(\arabic*)}]
                        \item 当 $\lambda<0$ 时,$(\lambda\circ f)(x)=\lambda f(x)\leqslant 0$,是函数值 $\le0$ 的实变量函数,则 $\lambda f(x)\not\in V$,即关于数乘不封闭,不构成线性空间.

                        \item $V_1$ 是奇函数集合,只需验证 $V_1$ 对加法和数乘封闭即可. 这显然成立. 则 $V_1$ 构成线性空间. 对于 $V_2$:当 $\lambda\neq 1$,有 $(\lambda\circ f)(0)=\lambda f(0)=\lambda\neq 1$. 则 $(\lambda\circ f)(x)\in V_2$,$V_2$ 不封闭,不构成线性空间.

                        \item 先验证 $V$ 非空:有 $f(x)=0,\forall x\in \mathbf{R}$,则 $f(x)\in V$,即 $V$ 非空. 再验证封闭性:对于 $(f\oplus g)(x)$,有 $(f\oplus g)(-x)=f(-x)+g(-x)=\overline{f(x)}+\overline{g(x)}=\overline{f(x)+g(x)}=\overline{(f\oplus g)(x)}$. 对于 $(\lambda\circ f)(x)$,有 $(\lambda\circ f)(-x)=\lambda f(-x)=\lambda\overline{f(x)}$. 由于 $\lambda \in \mathbf{R}$,则 $\lambda \overline{f(x)}=\overline{\lambda f(x)}=\overline{(\lambda\circ f)(x)}$. 因此 $V$ 关于 $\mathbf{R}$ 的函数加法和数乘封闭. 再给出加法零元 $f(x)=0$,数乘单位元 $\lambda=1$. 其余性质还请读者自行验证. 总之,$V$ 构成 $\mathbf{R}$ 上线性空间.
                    \end{enumerate}
          \end{enumerate}

    \item \begin{enumerate}
              \item $\forall (x_1,\ldots,x_n),(y_1,\ldots,y_n)\in W,\lambda,\mu\in \mathbf{F}$,我们有$\lambda(x_1,\ldots,x_n)+\mu(y_1,\ldots,y_n)=(\lambda x_1+\mu y_1,\ldots,\lambda x_n+\mu y_n)$. 则 $a_1(\lambda x_1+\mu y_1)+\cdots+a_n(\lambda x_n+\mu y_n)=\lambda(a_1x_1+\cdots+a_nx_n)+\mu(a_1y_1+\cdots+a_ny_n)=0$. 因此 $\lambda(x_1,\ldots,x_n)+\mu(y_1,\ldots,y_n)\in W$,故 $W$ 关于 $\mathbf{F}^n$ 的加法与数乘封闭,是其子空间.

              \item \begin{enumerate}
                        \item $(x,1,0)+(x,1,0)=(2x,2,0)\not\in W_1$,则 $W_1$ 不封闭,不是子空间. 其对应几何图形直线为 $\begin{cases}
                                      y=1 \\
                                      z=0
                                  \end{cases}$.

                        \item $\lambda\cdot(x_1,y_1,0)+\mu(x_2,y_2,0)=(\lambda x_1+\mu x_2,\lambda y_1+\mu y_2,0)\in W_2$,则 $W_2$ 封闭,是 $\mathbf{R}^3$ 的子空间,其对应几何图形为平面 $z=0$.
                    \end{enumerate}

              \item \begin{enumerate}
                        \item 这是题 (1) 的一个实例,根据题 (1) 可知 $W_1$ 是 $\mathbf{R}^3$ 的子空间,这是一个过原点的平面 $x-3y+z=0$.

                        \item 对于 $(x,y,z)\in W_2$,有 $(x,y,z)+(x,y,z)=(2x,2y,2z)$,而 $2x-3\cdots 2y+2z=2(x-3y+z)=2$,因此 $(2x,2y,2z)\not \in W_2$,$W_2$ 不封闭,不是子空间. 这是一个不过原点的平面 $x-3y+z=1$.
                    \end{enumerate}

              \item \begin{enumerate}
                        \item 对于 $(x,y,z)\in W_1$,有 $\vspace{1ex}\dfrac x2=\dfrac{y-4}1=\dfrac{z-1}3$. 取 $\lambda \in \mathbf{R}$ 且 $\lambda\neq 1$,则对 $\lambda(x,y,z)$,明显有 $\vspace{1ex}\dfrac{\lambda x}2\neq\dfrac{\lambda y-4}1$,因此 $W_1$ 不封闭,不是子空间,这是一条不过原点的直线.

                        \item 根据题 (1) 推出,$W_2$ 封闭,是子空间. 这是一条过原点的直线 $\vspace{1ex}\dfrac x1=\dfrac y1=\dfrac z{-2}$.
                    \end{enumerate}

              \item \begin{enumerate}
                        \item $p(x)=0,\forall x\in \mathbf{R}$,则有 $p(x)\in W_1$,$W_1$ 非空. $\forall p(x),q(x)\in W_1,\lambda \in\mathbf{R}$,有 $(\lambda\cdot(p+q))(1)=\lambda\cdot(p(1)+q(1))=0$,故$(\lambda\cdot(p+q))(x)\in W_1$,即 $W_1$ 封闭,构成 $\mathbf{R}[x]$ 上子空间.

                        \item $W_2$ 非空也是显然的,$\forall p(x),q(x)\in W_2, \lambda \in \mathbf{R}$ 有 $(\lambda\cdot(p+q))(1)=\lambda\cdot(p(1)+q(1))=0=\lambda\cdot(p(0)+q(0))=(\lambda\cdot(p+q))(0)$,因此 $(\lambda\cdot(p+q))(x)\in W_2$,即$W_2$ 封闭. 构成 $\mathbf{R}[x]$ 上子空间.
                    \end{enumerate}

              \item 也就是问所有偶函数构成集合是否封闭,显然是封闭的. 因为 $(\lambda\cdot(f+g))(-x)=\lambda\cdot(f(-x)+g(-x))=\lambda\cdot(f(x)+g(x))=(\lambda\cdot(f+g))(x)$.
          \end{enumerate}
\end{enumerate}

\centerline{\heiti B组}
\begin{enumerate}
    \item 暂略.

    \item $W$ 是 $V$ 的子空间等价于 $\forall \alpha_1,\ldots,\alpha_k\in W$,$\forall \lambda_1,\ldots,\lambda_k\in \mathbf{F}$($\mathbf{F}$ 是 $V$ 对应数域)有 $\lambda_1\alpha_1+\cdots+\lambda_k\alpha_k\in W$. 根据线性扩张的定义,以上描述等价于 $\spa(W)\subseteq W$,又 $\spa(W)$ 是 $W$ 的线性扩张,即 $W\subseteq\spa(W)$,故$\spa(W)\subseteq W \iff \spa(W)=W$. 综上 $W$ 是 $V$ 的子空间,得证.
\end{enumerate}

\centerline{\heiti C组}
\begin{enumerate}
    \item \begin{enumerate}
              \item 此处仅验证数乘封闭性,其余性质留给读者. $\forall \alpha\in \mathbf{F},\lambda\in \mathbf{E}$. 由于$\mathbf{E}\subseteq \mathbf{F}$,因此$\lambda\in \mathbf{F}$. 由于 $\mathbf{F}$ 本身是封闭的,故$\lambda\alpha\in F$,则 $F$ 构成 $\mathbf{E}$ 上的线性空间. 例: $\mathbf{C}$ 构成 $\mathbf{R}$ 上的线性空间.

              \item 例如:$\mathbf{R}$ 不是 $\mathbf{C}$ 上的线性空间. 因为 $\forall a\in\mathbf{R}$,有 $i\cdot a=a\cdot i\not\in\mathbf{R}$. 故数乘不封闭,不构成线性空间.

              \item $\forall\lambda\in \mathbf{E}$ 有 $\lambda\in \mathbf{F}$,则 $V$ 关于 $\mathbf{E}$ 的数乘运算肯定是封闭的,其余性质与在 $\mathbf{F}$ 上一致. 又 $\mathbf{E}$ 本身也封闭,则 $V$ 也是 $\mathbf{E}$ 上的一个线性空间,得证.
          \end{enumerate}
\end{enumerate}

\clearpage
