\phantomsection
\section*{2022-2023学年线性代数I(H)期中}
\addcontentsline{toc}{section}{2022-2023学年线性代数I(H)期中(刘康生老师)}

\begin{center}
    任课老师:刘康生\hspace{4em} 考试时长:45分钟
\end{center}
\begin{enumerate}
	\item[一、] 设矩阵$A=\begin{pmatrix}
        a & -1 & 1 \\ -1 & a & -1 \\ 1 & -1 & a
    \end{pmatrix}$,$\beta=\begin{pmatrix}
        0 \\ 1 \\ 1
    \end{pmatrix}$. 假设线性方程组$Ax=\beta$有解但解不唯一.
    \begin{enumerate}[label=(\arabic*)]
        \item 求$a$的值;

        \item 给出$Ax=\beta$的所有解.
    \end{enumerate}
	\item[二、]设
	\[P=\begin{pmatrix}
        1 & 1 & 0 \\ 0 & 1 & 0 \\ 0 & 0 & 0
    \end{pmatrix},\enspace Q=\begin{pmatrix}
        0 & 0 \\ 1 & 0
    \end{pmatrix},\]
    定义$\mathbf{R}^{3\times 2}$上映射$\sigma$:
    \[\sigma(A)=PAQ.\]
    \begin{enumerate}[label=(\arabic*)]
        \item 验证$\sigma$是线性映射;

        \item 求$\im\sigma$和$\ker\sigma$;

        \item 求$\mathbf{R}^{3\times 2}$的两组基$B_1,B_2$,使得$\sigma$在$B_1,B_2$下的矩阵为对角矩阵.
    \end{enumerate}
	\item[三、]设$B=\{\beta_1,\beta_2,\ldots,\beta_n\}$是实数域$\mathbf{R}$上线性空间$V$的一组基,$T\in\mathcal{L}(V)$,$T(\beta_1)=\beta_2$,$T(\beta_2)=\beta_3$,$\cdots$,$T(\beta_{n-1})=\beta_n$,$T(\beta_n)=\sum\limits_{i=1}^{n}a_i\beta_i(a_i\in\mathbf{R})$. 求$T$在$B$下的表示矩阵. 在什么条件下$T$是同构映射?
	\item[四、]判断下列命题的真伪,若它是真命题,请给出简单的证明;若它是伪命题,给出理由或举反例将它否定.
	\begin{enumerate}[label=(\arabic*)]
        \item 若$W$是线性空间$V$的子空间,$\alpha\in V$,则$\alpha+W$是$V$的子空间;

        \item 若$W$是线性空间$V$的子空间,对任何的$\alpha\in V$,定义$\overline{\alpha}=\alpha+W$,则
        \[\overline{\alpha}=\overline{\beta}\text{ 或者 }\overline{\alpha}\cap\overline{\beta}=\varnothing;\]

        \item 若方阵$A^3=0$,则$E+A$和$E-A$都是可逆矩阵;

        \item 若方阵$A^2=A$,则$E+A$和$E-A$都是可逆矩阵.
    \end{enumerate}
\end{enumerate}

\clearpage
